\documentclass{inVerba-notes}

\definecolor{title-color}{HTML}{5f89f5}
\newcommand{\theTitle}{\href{https://github.com/cullyn-inverba/notes/tree/master/ch-335} {Organic Chemistry II}}

\begin{document}
\hypertarget{ToC}{\tableofcontents}
% chktex-file 36

%%%%%%%%%%% Chapter 14: Alkenes %%%%%%%%%%%
%\begingroup
\chapter{Chapter 14: Alkenes}\label{Chapter 14: Alkenes}


\section{Alkenes Basics/Review}\label{Alkenes Basics/Review}
\begin{itemize}
  \item Alkanes vs alkenes: 
\medskip
\begin{center}
  \schemestart{}
    \chemname{\chemfig{H-C(-[::+90]H)(-[::-90]H)-C(-[::+90]H)(-[::-90]H)-H}}{Saturated\\alk\emph{anes} eth\emph{ane}}
    \qquad
    \chemname{\chemfig{C(-[::150]H)(-[::-150]H)=C(-[::+30]H)(-[::-30]H)}}{Unsaturateed\\alk\emph{enes} eth\emph{ene}}
    \qquad
    \chemname{\chemfig{!\cu!\cd}}{prop\emph{ane}}
    \qquad
    \chemname{\chemfig{!\cu=[:-30]}}{prop\emph{ene}}
  \schemestop{}
\end{center}
\bigskip

  \subsection{Practice and Review}
  \begin{itemize}
    \item \textbf{Electronegativity}: negative charges on atoms with lower hybridization result in greater stability due to proximity (overlap) to positive nucleus. More s character results in greater stability.
      \begin{itemize}
        \item I.e., \(\bbb{sp~(50\%~s)} > sp^2~(33\%~s) > \rrr{sp^3~(25\%~s)}\)
        \item E.g., ethene has two carbons that are both \(sp^2\) due to one unhybridized p-orbital. This gives ethene a trigonal planar geometry.
      \end{itemize}

      \bigskip
      \begin{center}
      \hspace{-20pt}
      \schemestart{}
        \chemfig{C(-[::150]H)(-[::-150]H)=C(-[::+30]H)(-[::-30]H)}
        \arrow{<->}
        \quad
        \chemfig{
          \orbital{s}
            -[:-30,.85]
              {\orbital[scale=1.8]{p}}
              {\orbital[angle=0,scale=1.6,half,color=dark]{p}}
              {\orbital[angle=150,half]{p}}
              {\orbital[angle=-150,half]{p}}
              (-[:-150,.85]\orbital{s})
            =[,.75]
            =[,.75]
            {\orbital[angle=180,scale=1.6, half, color=dark]{p}}
            {\orbital[scale=1.8]{p}}
            {\orbital[angle=30,half]{p}}
            {\orbital[angle=-30,half]{p}}
            (-[::30,.85]{\orbital{s}})(-[::-30,.85]{\orbital{s}})
        }
        \arrow{0}
        \hspace{-20pt}
        \orbital[scale=1.6]{p}
        \quad (unhybridized p-orbital)
      \schemestop{}
      \end{center}
      \bigskip
      \bigskip

    \item \textbf{Hydrogen deficiency index (HDI)}: the measure of degrees of unsaturation. 
      \begin{itemize}
        \item E.g., two degrees of unsaturation results in a HDI of 2.
        \item Degrees of freedom help represent possible structures, indicating possible double bounds, triple bounds, rings, or various combinations of each.
        \item Only helpful when molecular formula is known for certainty.
        \item Formula: \emph{HDI = \(\frac{1}{2}(2C + 2 + N - H - X)\)}
        \begin{itemize}
            \item \(X\): halogen atoms.
        \end{itemize}
      \end{itemize}
    \item What is the HDI for the following molecules?
    
    \begin{multicols}{2}  
    \begin{itemize}
        \item[i] \chemfig{!\cu=[:-30]!\cu=[:-30]!\cu=[:-30]}
        \item[ii]~~{\footnotesize\chemfig{*6(-=-=-=)}}
        \item[i] \(\frac{1}{2}(2(7) + 2 + (0) - (3+(5(1))+2) - 0) = \emph{3}\)
        \smallskip 
        \item[ii] \(\frac{1}{2}(2(6) + 2 + 0 - (6(1)) - 0) = \emph{4}\)
      \end{itemize}
    \end{multicols}

    \item \textbf{Degree of substitution}: not a substitution reaction, but the \emph{number of groups} connected to the double bond.
    \begin{itemize}
      \item 
      
        \chemname{\chemfig{R-[:45]=}}{Monosubstituted}
        \qquad
        \chemname{\chemfig{R-[:45]=-[:45]R}}{Disubstituted}
        \qquad
        \chemname{\chemfig{R-[:45]=(-[:-45]R)-[:45]R}}{Trisubstituted}
        \qquad
        \chemname{\chemfig{R-[:45](-[:135]R)=(-[:-45]R)-[:45]R}}{Tetrasubstituted}
        \bigskip
        
    \end{itemize}
  \end{itemize}
  
  \subsection{Types of Alkenes}\label{Types of Alkenes}
  \begin{itemize}
    \item Basic types of alkenes:
    
    \schemestart{}
      \chemname{\chemfig{=[:30,,,,emph]!\cd!\cu}}{Terminal Alkene}
      \qquad
      \chemname{\chemfig{!\cu=[:-30,,,,emph]!\cu}}{Internal Alkene}
      \qquad
      \chemname{{\footnotesize\chemfig{*6(---=[,,,,emph]--)}}}{Cyloalkene}
    \schemestop{}
    \bigskip

    \item Types of terminal alkenes:  
    
    \schemestart{}
      \chemname{\chemfig{R=(-[:45]H)(-[:-45]H)}}{Methylene}
      \qquad
      \chemname{\chemfig{R-=[:-45]}}{Vinyl}
      \qquad
      \chemname{\chemfig{R--[:-45]=}}{Allyl}
    \schemestop{}
    \bigskip

    \begin{itemize}
      \item ``R'' always tells you it's a carbon containing functional group, or hydrogen. 
      \item ``A'' can be used to represent any functional group. 
    \end{itemize}
  \end{itemize}

  \subsection{Chirality}
  \begin{itemize}
    \item \ddd{Achiral (nonsuperimposable)}: when an object's mirrored version is identical to the actual object.
    \item \ddd{Chiral}: objects that are not superimposable.
      \begin{itemize}
        \item The most common source of molecular chirality is the presence of a \emph{carbon bearing four different groups}.
      \end{itemize}
    \item All three-dimensional objects can be classified as either chiral or achiral.
    \item \ddd{Enantiomer}: the nonsuperimposable mirror image of a chiral compound.
      \begin{itemize}
          \item Can be used in speech the same way the word \emph{twin} is used
          \item The easiest way to draw enantiomers is to simply change wedges and dashes, but there are multiples ways to mirror a molecule, so it can be more complex.
      \end{itemize}
    \item \ddd{Diastereomers}: non-identical stereoisomers (nonsuperimposable) that are \emph{not mirror images} of one another. 
      \begin{itemize}
          \item  Enantiomers have the same physical properties, while diastereomers have \emph{different physical properties}.
          \item Differences between enantiomers and diastereomers are especially relevant when comparing compounds with \emph{more than one chiral center}.
          \item \emph{Maximum} (could be less) number of stereoisomers: \emph{\(2^n\)}
        \begin{itemize}
          \item \(n\): number of chiral centers
          \item \(\dfrac{2^n}{2}\): max pairs of enantiomers.
        \end{itemize}
      \end{itemize}
    \end{itemize}

    \medskip

    \subsection{Cahn-Ingold-Prelog System}
    \begin{itemize}
    \item \ddd{Chan-Ingold-Prelog system}: a system of nomenclature for Identifying each enantiomer individually.
    \begin{enumerate}
      \item Assign priorities to each of the four groups based on atomic number; the highest atomic number has the highest priority.
      \item Rotate the molecule so that the fourth priority group is on a dash (behind)
      \item Determine the configuration, i.e., sequence of 1--2--3 groups.
        \begin{itemize}
          \item \ttt{clockwise (R)} or \fff{counterclockwise (S)}.
        \end{itemize}
    \end{enumerate}
    \item If there is a tie between the atoms connected, then continue outward until a difference is found.
      \begin{itemize}
        \item Do not add the sum all atomic numbers attached to each atom, just the first in which the atoms differ.
        \item Any multiple bonded atom, (2 or 3) is treated as if connected to multiple atoms equal to number of bonds.
      \end{itemize}
    \item Switching any two groups on a chiral center will invert the configuration, e.g.,
    
      \medskip
      \schemestart{}
        \fff{\chemfig{2-[:30](-[:-30]3)(<[:130]1)(<:[:50]4)}}
        \arrow{->}
        \ttt{\chemfig{2-[:30](-[:-30]3)(<[:130]4)(<:[:50]1)}}
      \schemestop{}
      \bigskip 

    \item Switching twice results in a change without changing configuration, e.g.,
            
      \medskip
      \schemestart{}
        \ttt{\chemfig{
            2-[:30](<[:130]1)(<:[:50]3)-[:-30]4
            }}
        \arrow{->}
        \fff{\chemfig{
            2-[:30](<[:130]1)(<:[:50]4)-[:-30]3
            }}
        \arrow{->}
        \ttt{\chemfig{
            1-[:30](<[:130]2)(<:[:50]4)-[:-30]3
            }}
      \schemestop{}
      \bigskip
    
    \bigskip

    \item \textbf{Configuration in IUPAC nomenclature}:
    \begin{itemize}
        \item The configuration of the chiral center is indicated at the beginning of the name, italicized, and surrounded by parentheses.
        \item When multiple centers are present, then each must be preceded by a locant.
    \end{itemize}
  \end{itemize}
\end{itemize}

\section{Addition Reactions of Alkenes}\label{Addition Reactions of Alkenes}
\begin{itemize}
  \item[]
  \subsection{Syn Addition}\label{Syn Addition}
  \begin{itemize}
    \item Syn addition means that both components (A \& B) came in from the same side.

    \medskip
    \schemestart{}
      \chemfig{C(-[::150])(-[::-150])=C(-[::+30])(-[::-30])}
      \+
      \chemfig{A-B}
      \arrow{}
      \chemfig{-C(-[:90])(-[:-90]A)-C(-[:90])(-[:-90]B)-}
    \schemestop{}
    \bigskip
        
    \item When there is mixture of products, then the stereochemistry is important---A \& B could come from different directions.
    \item Switching to a 3D perspective, with groups pointing towards and away from the viewer, rather than a flat view:
        
    \medskip
    \schemestart{}
      \chemfig{C(-[::150])(-[::-150])=C(-[::+30])(-[::-30])}
      \arrow{<->}
      \chemfig{C(<:[::150])(<[::-150])=C(<:[::+30])(<[::-30])}
      \arrow{->[A---B][\ttt{Syn}]}
      \chemfig{C(<:[:190])(<[:-110])(-[:90]A)-C(-[:90]B)(<[:-70])(<:[:-10])}
    \schemestop{}
    \bigskip
        
    \item The final product above results in no chiral compounds (no carbons contain four different groups). If a product does have chiral compounds, you must show the other possible products.

    \medskip
    \schemestart{}
      \chemfig{C(<:[::150])(<[::-150]-[:180,.6])=C(<:[::30]-[,.6])(<[::-30])}
      \arrow{->[A---B][\ttt{Syn}]}
      \chemfig{C(<:[:190])(<[:-110]-[:180,.6])(-[:90]A)-C(-[:90]B)(<[:-70])(<:[:-10]-[,.6])}
    \schemestop{}
    \bigskip
        
    \item The product in this case has chiral centers, so enantiomers must be considered.
        
    \medskip
    \schemestart{}
      \dots
      \arrow(--[braces]){->[A---B][\ttt{Syn}]}
      \chemfig{C(<:[:190])(<[:-110]-[:180,.6])(-[:90]A)-C(-[:90]B)(<[:-70])(<:[:-10]-[,.6])}
      \+
      \chemfig{C(<:[:-190])(<[:110]-[:180,.6])(-[:-90]A)-C(-[:-90]B)(<:[:70]-[,.6])(<[:10])}
    \schemestop{}
    \bigskip

  \end{itemize}

  \subsection{Anti Addition}\label{Anti Addition}
  \begin{itemize}
    \item Anti addition means that the added components come in from different sides.
    
    \medskip
    \hspace{-28pt}
    \schemestart{}
      \chemfig{C_1(<:[::150])(<[::-150]-[:180,.6])=C_2(<:[::30]-[,.6])(<[::-30])}
      \arrow(--[braces]){->[A---B][\fff{Anti}]}
      \chemfig{C_1(<:[:190])(<[:-110]-[:180,.6])(-[:90]A)-C_2(-[:-90]B)(<:[:70]-[,.6])(<[:10])}
      \+
      \chemfig{C_1(<:[:90])(<[:140]-[:180,.6])(-[:-90]A)-C_2(-[:90]B)(<[:-90])(<:[:-40]-[,.6])}
    \schemestop{}
    \bigskip
  \end{itemize}

  \subsection{Types of Addition Reactions}\label{Types of Addition Reactions}
  \begin{itemize}
    \item \ddd{Hydrogenation}: a chemical reaction between molecular hydrogen \ch{H2} and another compound or element, usually in the presence of a catalyst such as nickel, palladium (typically pd, due to cost) or platinum.
      \begin{itemize}
        \item Commonly employed to reduce or saturate organic compounds.
        \item Usually constitutes the addition of pairs of hydrogen atoms to a molecule, often an alkene.
        \item Catalysts are required for the reaction to be usable; non-catalytic hydrogenation takes place only at very high temperatures.
        \item \emph{Reduces} double and triple bonds in hydrocarbons.
        \item Always occurs with \emph{syn addition}, with hydrogen entering from the least hindered side.
      \end{itemize}
    
    \item Hydrogenation of 2-methyl-3-ethyl-2-pentene:
      \begin{itemize}
        \item First convert to 3D view:
        
        \medskip
        \schemestart{}
          \chemfig{(-[::150])(-[::-150])=(-[::30]-[,.6])(-[::-30]-[,.6])}
          \arrow{<->}
          \chemfig{(<:[::150])(<[::-150])=(<:[::30]-[,.6])(<[::-30]-[,.6])}
        \schemestop{}
        \bigskip

        \item Then apply \ch{H2} with catalyst to trigger syn addition:
        
        \medskip
        \schemestart{}
          \chemfig{(<:[::150])(<[::-150])=(<:[::30]-[,.6])(<[::-30]-[,.6])}
          \arrow{->[\ch{H2}][Pd-c]}
          \chemfig{(<:[:200])(<[:240])(-[:90]H)-(-[:90]H)(<:[:-20]-[,.6])(<[:-60]-[,.6])}
        \schemestop{}
        \bigskip

        \item Product is achiral; no enantiomers. Bottom would attack just flips the orientation; not a different product. 
      \end{itemize}
  \end{itemize}

  \subsubsection{Hydrogenation Practice Problems}\label{Hydrogenation Practice Problems}
  \begin{enumerate}\footnotesize
      \item

      \medskip
      \schemestart{}
        \chemfig{(<:[:150])(<[:-150]-[180,.6])=(<:[:30](-[:60,.6])(-[:-10,.6]))(<[:-30])}
        \arrow{->[\ch{H2}][Pd-c]}
        \chemfig{(<:[:200])(<[:-100]-[:180,.6])(-[:90])-(-[:90])(<:[::-20](-[::60,.6])(-[::-10,.6]))(<[:-80])}
      \schemestop{}
      \bigskip

      \item

      \medskip
      \schemestart{}
        \chemfig{*6(---(-)=(-)--)}
        \arrow{->[\ch{H2}][Pd-c]}
        \chemfig{*6(---(<)-(<)--)}
      \schemestop{}
      \bigskip

      \item

      \medskip
      \schemestart{}
        \chemfig{*6(---(--[:-30])=(-)--)}
        \arrow(--[braces]){->[\ch{H2}][Pd-c]}
        \chemfig{*6(---(<-[:-30])=(<)--)}
        \+
        \chemfig{*6(---(<:-[:-30])=(<:)--)}
      \schemestop{}
      \bigskip
  \end{enumerate}

  \begin{itemize}
    \item \ddd{Hydrohalogenation}: the electrophilic addition (\(\pi \to 2\sigma \) bonds) of hydrohalic acids (\ch{HX}, e.g., \ch{HCl} or \ch{HBr}) to alkenes to yield the corresponding haloalkanes.
    \item \ddd{Markovnikov's rule}: an addition of a protic acid (HX) or other polar reagent to an asymmetric alkene results in the \rrr{electropositive part (usually H)} gets attached to the carbon with \rrr{more hydrogen substituents}, and the \bbb{electronegative part (usually the halide)} attaches to the carbon with \bbb{more alkyl substituents}. 
      \begin{itemize}
        \item Alternatively: \rrr{\(H^+\)} is added to the carbon with the \rrr{greatest} number of hydrogen atoms while the \bbb{\(X^-\) component} is added to the carbon with the \bbb{fewest} hydrogen atoms.
      \end{itemize}

    \medskip
    \schemestart{}
      \chemfig{C_1(<:[:150])(<[:-150](-[:190,.6])(-[:250,.6]))=[@{db}]C_2(<:[:30]H)(<[:-30])}
      \arrow{->[\chemfig{@{hx}\rrr{H}@{xx}\bbb{X}}]}
      \chemname{\chemfig{
        \chemabove[3pt]{C_1}{\cat}
        (<:[:150])(<[:-150](-[:190,.6])(-[:250,.6]))
        -C_2
        (-[:90]@{h}\rrr{H})(<:[:250]H)(<[:-50,.9])}
        }{most stable carbocation intermediate}
      \arrow{->}
      \chemname{\chemfig{
        C_1
        (-[:90]@{x}\bbb{X})(<:[:150])(<[:-150](-[:190,.6])(-[:250,.6]))
        -C_2
        (-[:90]\rrr{H})(<:[:250]H)(<[:-50,.9])}
      }{final product} 
    \schemestop{}
    \chemmove[shorten <=2pt,dash pattern= on 1pt off 1pt]{
      \draw(db)..controls +(90:1cm)and+(north:1.5cm)..(hx);
      \draw(hx)..controls +(90:1cm)and+(north:1.5cm)..(h);
      \draw(xx)..controls +(90:1.5cm)and+(north:1.8cm)..(x);
    }
    \bigskip

    \item However, we have a chiral carbon, so we have to take enantiomers into account:
    
    \medskip
    \schemestart{}
      \chemfig{(<:[:150])(<[:-150](-[:190,.6])(-[:250,.6]))=[@{db}](<:[:30]H)(<[:-30])}
      \arrow(--[braces]){->[\rrr{H}\bbb{X}]}
      \chemname{\chemfig{
        (-[:90]\bbb{X})(<:[:150])(<[:-150](-[:190,.6])(-[:250,.6]))-
        (-[:90]\rrr{H})(<:[:250]H)(<[:-50,.9])}}
        {\ttt{Syn}}
      \+
      \chemname{\chemfig{
        (-[:-90]\bbb{X})(<:[:80])(<[:120](-[:80,.6])(-[:160,.6]))-
        (-[:90]\rrr{H})(<:[:250]H)(<[:-50,.9])}}
        {\fff{Anti}}
    \schemestop{}
    \bigskip
    
    \item The previous example is an example of a regioselective reaction, i.e., it's a chemical reaction where one reaction site is preferred over another.
      \begin{itemize}
        \item Markovnikov additions are common examples of regioselective reactions since there is a specific region in where the hydrogen is supposed to go.
      \end{itemize}
    
    \item Another example that contains multiple products:
    
    \medskip
    \schemestart{}
      \chemfig{*6(---(-)=(-)--)}
      \arrow(--[braces]){->[\rrr{H}-\bbb{I}]}
      \chemfig{*6(---(<)(<:[:-30]\rrr{H})-(-[:30,0.3,,,draw=none]\cat)(-)--)}
      \+
      \chemfig{*6(---(<\rrr{H})(<:[:-30])-(-[:30,0.3,,,draw=none]\cat)(-)--)}
    \schemestop{}
    \bigskip

    \item Again, this first step shows the most stable carbocation intermediate, which helps determine where the \rrr{\ch{H+}} will attach to. In this case either location works, so four products are formed, each with syn and anti enantiomers.
   
    \medskip
    \hspace{-20pt}
    \schemestart{}
      \dots
      \arrow(--[braces]){->[\rrr{H}-\bbb{I}]}
      {\footnotesize
      \chemname{\chemfig{*6(---(<)(<:[:-30]\rrr{H})-(<:[:30])(<\bbb{I})--)}}{\fff{Anti}}
      \+
      \chemname{\chemfig{*6(---(<)(<:[:-30]\rrr{H})-(<:[:30]\bbb{I})(<)--)}}{\ttt{Syn}}
      \+ 
      \chemname{\chemfig{*6(---(<\rrr{H})(<:[:-30])-(<:[:30])(<\bbb{I})--)}}{\ttt{Syn}}
      \+
      \chemname{\chemfig{*6(---(<\rrr{H})(<:[:-30])-(<:[:30]\bbb{I})(<)--)}}
      {\fff{Anti}}
      }
    \schemestop{}
    \bigskip

    \item The above products are all \emph{stereoisomers} of each other. Note: not all chemical reactions produce all possible stereoisomers.
  \end{itemize}

  \subsubsection{Hydrohalogenation Practice Examples}

    {\footnotesize
    \medskip
    \schemestart{}
      \chemfig{*6(---=(-)--)}
      \arrow{->[\rrr{H}\bbb{Br}]}
      \chemfig{*6(---(-\rrr{H})-(-)(-[:30,0.3,,,draw=none]\cat)--)}
      \arrow(--[braces]){}[,,shorten >=6pt]
      \chemfig{*6(---(-\rrr{H})-(<:[:120]\bbb{Br})(<[:50])--)}
      \+
      \chemfig{*6(---(-\rrr{H})-(<[:120]\bbb{Br})(<:[:50])--)}
    \schemestop{}
    \bigskip
    }

  \ddd{Addition of Water or Alcohol (ROH)}: analogous to that of hydrohalogenation (HX); both of which involve a carbocation intermediate, syn/anti stereochemistry, and Markovnikov regioselectivity.
    \begin{itemize}
      \item Note: the hydronium ion (\rrr{\ch{H3O+}}) forms via the reaction of \rrr{\ch{H2SO4}} with \bbb{\ch{H2O}}, and is the source of the proton that reacts with the starting alkene.
      \item \ch{H-OH} and \ch{RO-H} helps to visualize where the proton is coming from. What ever works with water also works with alcohol and vice versa. 
      \item An example that is very similar to the example from \hyperref[Hydrohalogenation]{\ulink{hydrohalogenation}}:
        
        \hspace{-30pt}
        \bigskip
        \schemestart{}
         \chemfig{(<:[::150])(<[::-150]-[:180,.6])=[,.8](<:[::30]H)(<[::-30]-[,.6])}
         \arrow{->[\ch{H2O}][dil \ch{H2SO4}]}
         \chemfig{
           \chemabove[3pt]{C_1}{\cat}
           (<:[:150])(<[:-150](-[:190,.6]))-
           (-[:90]\rrr{H})(<:[:250]H)(<[:-50,.9](-[,.6]))}
         \arrow{->[\ch{H2O}]}[,,shorten <=8pt]
         \chemfig{
           (-[:90]\bbb{OH})(<:[:200])(<[:-110](-[:190,.6]))-
           (-[:90]\rrr{H})(<:[:250]H)(<[:-50,.9](-[,.6]))}
        \schemestop{}
        
        \begin{itemize}
          \item Note: \rrr{\ch{H2SO4}} is a \rrr{strong acid} and can be generalized to \rrr{\ch{H+}} or \rrr{HA}. 
          \item Note: I occasionally use \rrr{red} for both \rrr{cations} and \rrr{acids}. I also use \bbb{blue} for \bbb{anions} and \bbb{bases} to help easily keep track of things. This is an arbitrary color choice, but it holds weight in these notes. 
        \end{itemize}

      \item Again, this reaction is stereoselective, so there is actually more than one product:
        
        \medskip
        \schemestart{}
          \dots
          \arrow(--[braces]){->[\ch{H2O}]}[,,shorten <=8pt]
          \chemname{\chemfig{
           (-[:-90]\bbb{OH})(<[:-200](-[:190,.6]))(<:[:110])-
           (-[:90]\rrr{H})(<:[:260]H)(<[:-40,.9](-[,.6]))
           }}{\fff{Anti}}
          \+
          \chemname{\chemfig{
           (-[:90]\bbb{OH})(<:[:200])(<[:-110](-[:190,.6]))-
           (-[:90]\rrr{H})(<:[:250]H)(<[:-50,.9](-[,.6]))
           }}{\ttt{Syn}}
        \schemestop{}
        \bigskip
        
      \item Example 2:
      
      \medskip
      \schemestart{}
        \chemfig{=[:-30]-[:30](<[:90,.8])-[:-30]-[:30]}
        \arrow{->[\bbb{\ch{H2O}}][\rrr{\ch{H+}}]}
        \chemfig{\rrr{H}-[:-90,.8]-[:-30]\chemabove[6pt]{}{\cat}-[:30](<[:90,.8])-[:-30]-[:30]}
      \schemestop{}
      \bigskip
        
      \item Rearrangement is always a possibility to be considered when cation are generated. In this case, a \ang{2} carbocation was generated, but we can do better:
      
      \medskip
      \schemestart{}
        \chemfig{\rrr{H}-[:-90,.8]-[:-30]@{c}\chemabove[6pt]{}{\cat}-[:30](<:[:30]@{h}H)(<[:90,.8])-[:-30]-[:30]}
      \schemestop{}
      \chemmove[dash pattern= on 1pt off 1pt]{
        \draw(h)..controls +(south:1cm) and +(south:1cm).. (c);}
      \bigskip
      
    \end{itemize}
\end{itemize}



\section{Nomenclature of Alkenes}\label{Nomenclature of Alkenes}
\begin{itemize}
  \item Generally prepared through beta elimination, which results in the formation of alkenes (series of unsaturated hydrocarbons contain that a \(\pi \) bond).
  \item Alkenes are named using the same four steps in the previously used nomenclature, though the suffix of \minimal{``ane''} is replaced with \emph{``ene.''}
  \item When choosing the parent chain, choose the parent chain that \emph{includes} the \(\pi \) bond.
  \item When numbering the parent chain, the \(\pi \) bond should receive the \emph{lowest} number possible.
  \item The locant of the \(\pi \) bond should be place right before the suffix of ``ene,'' though, it was previously recommended before the parent (both are acceptable).
  \item Commonly recognized alternative names:

    \medskip
    \begin{center}
    \schemestart{}
      \chemname{\chemfig{=[:30]}}{Ethylene}
      \qquad
      \chemname{\chemfig{=[:30]-[:-30]}}{Propylene}
      \qquad
      \chemname{{\tiny\chemfig{*6(-=-(-=[:-30])=-=)}}}{Styrene}
    \schemestop{}
    \end{center}
    \bigskip

\end{itemize}
%\endgroup
%%%%%%%%%%% Chapter 14: Alkenes %%%%%%%%%%%
\end{document}