\documentclass[12pt,a4paper]{article}
\usepackage{inverba}
\newcommand{\userName}{Cullyn Newman} 
\newcommand{\class}{BI 216} 
\newcommand{\institution}{Portland State University} 
\newcommand{\thetitle}{\hypertarget{home}{Lab 8: Scientific Writing/Reading}}
\rfoot{\hyperlink{home}{\thepage}}

\begin{document}
\section*{Introduction}
\begin{enumerate}[font=\bfseries, wide]
    {\color{under}\item Your TA will assign your group a topic to learn more about and present to the class. Record notes on your team’s findings/discussion of your specifically assigned topic below. (1 pt)}
    
    \textbf{Assigned Topic:}

    \textbf{Notes:}
    
    {\color{under}\item Insert a picture of your team’s concept map of the introduction below. Use the words in the word bank below. Don’t forget to include linking words between each connection!   (1 pt)}
    
    
    {\color{under}\item What is the overarching hypothesis that the authors of this study want to test? (1 pt)}
    
    
\end{enumerate}

\section*{Methods}
\begin{enumerate}[font=\bfseries, wide, resume]
    {\color{under}\item What is the specific question the authors are attempting to answer with “Experiment 1?” (1 pt)}
    
    
    {\color{under}\item What is the specific question the authors are attempting to answer with “Experiment 2?” (1 pt)}
    

    {\color{under}\item Insert a picture of your team’s sketch of the methods used to produce Figure 3 below. (1 pt)}
    

    {\color{under}\item Insert a picture of your team’s sketch of the methods that would be used to produce Figure 4 below. (1 pt)}
    
    \subsection*{Figure 3 Analysis Questions}
    {\color{under}\item What is the “control” group and what is the “treated” group? (1 pt)}
    
    
    {\color{under}\item What do you learn when you compare Panel A to Panel B? (1 pt)}


    {\color{under}\item Overall, what do you learn from Figure 3? (1 pt)}
    
    \subsection*{Figure 4 Analysis Questions}
    {\color{under}\item  What do the two sets of red and blue lines in each of the graphs in this figure represent? (1 pt)}
    
    
    {\color{under}\item What do the grey shaded areas represent? (1 pt)}
    
    
    {\color{under}\item What do you learn from comparing panels (A and D) to panels (B and E) to panels (C and F)? (hint: what condition is changed between these three sets of panels?) (1 pt)}


    {\color{under}\item What do you learn from comparing panels (A, B, and C) to panels (D, E, and F)? (hint: what condition is changed between these two sets of panels?) (1 pt)}
    
    
    {\color{under}\item  Overall, what did you learn from Figure 4? (1 pt)}
\end{enumerate}
    
\section*{Future Directions/Paper Evaluation}
\begin{enumerate}[font=\bfseries, wide, resume]
    {\color{under}\item  Within your team, discuss future experiments that would build off of this research. Individually, summarize an idea for one potential future research direction below. Make sure you explain why your proposed project would be an interesting contribution to this field of research (2 pts)}
    
    
    {\color{under}\item What concerns do you have about this research study (these could be things you don’t understand, criticisms of the methods, questions for the authors, or anything else that comes to mind)? (1 pt)}
    
    
    {\color{under}\item  What are some potential behavioral ecology questions that you could answer with an experimental design based off of observing animals in your backyard or on webcams? What’s your group's plan of action? (2 pts)}  
\end{enumerate}

\end{document}