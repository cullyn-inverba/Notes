\documentclass[12pt,a4paper]{article}
\usepackage{inverba}
\newcommand{\userName}{Cullyn Newman} 
\newcommand{\class}{} 
\newcommand{\institution}{Portland State University} 
\newcommand{\thetitle}{\hypertarget{home}{Principles of Neural Science}}
\rfoot{\hyperlink{home}{\thepage}}

\begin{document}
%%%%%%%%%%%%%%%%%%%%%%%%%%%%%%%%%%%%%%%%%%%%%%%%%%%%%%%%%%%%%%%%%%%%%
\tableofcontents
\cleardoublepage
\fancyhead{}
\fancyhead[R]{\hyperlink{home}{\nouppercase\leftmark}}
%%%%%%%%%%%%%%%%%%%%%%%%%%%%%%%%%%%%%%%%%%%%%%%%%%%%%%%%%%%%%%%%%%%%%

\fancyhead[L]{Part I Overall Perspective}
%\begingroup
%%%%%%%%%%%%%%%%%%%%%%%%%%%%% Chapter 1 %%%%%%%%%%%%%%%%%%%%%%%%%%%%%
%\begingroup
\clearpage
\section{The Brain and Behavior}

\subsection{The Brain Has Distinct Functional Regions}

    \textbf{The Central Nervous System Has Seven Main Parts}
    \begin{itemize}
        \item \textbf{Spinal cord}: most caudal part of the central nervous system. It is subdivided into cervical, thoracic, lumbar, and sacral regions.
        \item \textbf{Brain stem}: consists of the medulla oblongata, pons, and midrain. Relays input from the spinal cord and back, and controls input to and from the head.
        \item \textbf{Medulla oblongata}: rostral to spinal cord and includes several centers responsible for vital autonomic functions. 
        \item \textbf{Pons}: rostral to medulla and conveys information about movement.
        \item \textbf{Cerebellum}: lies behind pons, modulates force and range of movement, and involved in learning motor skills.
        \item \textbf{Diencephalon}: lies rostral to midrain and contains two structures, thalamus (processes information reaching cerebral cortex) and hypthalamus (regulates autonomic, endocrine, and visceral functions).
        \item \textbf{Cerebrum}: comprises two cerebral hemispheres, each consisting of wrinkled outer layer (the cerebral cortex), and three deep lying structures (basal ganglia, the hippocampus, and the amygdaloid nuclei).
        \item \textbf{Cerebral cortex}: divided into four distinct lobes--- frontal, parietal, occipital, and temporal. The frontal lobe is largely concerned with short-term memory and planning, as well as movement; the parietal lobe with somatic sensation, forming a body image, and relating it to extrapersonal space; the occipital lobe with vision; and the temporal lobe with hearing---combined with deeper structures---with learing, memory, and emotion.
    \end{itemize}
%\endgroup
%%%%%%%%%%%%%%%%%%%%%%%%%%%%% Chapter 1 %%%%%%%%%%%%%%%%%%%%%%%%%%%%%

%%%%%%%%%%%%%%%%%%%%%%%%%%%%% Chapter 2 %%%%%%%%%%%%%%%%%%%%%%%%%%%%%
%\begingroup
\clearpage
\section{Nerve Cells, Neural Circuitry, and Behavior}
\subsection{The Nervous System Has Two Classes of Cells}
\begin{itemize}
    \item There are two main classes of cells in the nervous system: nerve cells, or neurons, and glial cells, or glia.
    \item A neuron has four defined regions:
        \begin{itemize}
            \item \textbf{Cell body}: or \textbf{soma}, is the metabolic center of the cell, containing normal cell organelles.
            \item \textbf{Dendrites}: branch out in tree-like fashion and are main apparatue for receiving signals.
            \item \textbf{Axon}: extends some distance from a cell and carries signals to other neurons.
            \item \textbf{Presynaptic terminals}: specialized enlarged regions of it's axon's branches and is responsible for tranfer of signals.
        \end{itemize}
    \item \textbf{Principle of dynamic polarization}: electrical signals only forlow in one direction in neurons.
    \item \textbf{Connectional specificity}: nerve cells do not connect randomly with one another in these formation of networks.
    \item Neurons are classified into three groups:
        \begin{itemize}
            \item \textbf{Unipolar}: simpiliest due to single primary process, which gives rise to many branches. One branch as axon and others as receiving structures. These cells predominate invertebrates; they orccur in the autonomic nervous system in vertebrates. 
            \item \textbf{Bipolar}: oval soma that gives rise to two processes: a dendritic structure that receives signals and an axon that carries information towards the central nervous system. Many sensory cells are bipolar, and pain receptors are pseudo-unipolar. 
            \item \textbf{Multipolar}: predominate nervous system of vertebrates and vary greatly in shape; typically containing a single neuron and many dendritic points emerging from various points around the cell body.
        \end{itemize}
    \item Glial cells support nerve cells and greatly outnumber neurons. 
    \item Glial cells surround the cell bodies, axons, and neurons and can be divided into two major classes:
        \begin{itemize}
            \item \textbf{Microglia}: immune system cells that become phagocytes during injury, infection, or degenerative diseases. 
            \item There are three main types of \textbf{macroglia}: oligodendrocytes, Schwann cells, and astrocytes. About 80\% of all brain cells are macrogalia.
        \end{itemize}
\end{itemize}
%\endgroup
%%%%%%%%%%%%%%%%%%%%%%%%%%%%% Chapter 2 %%%%%%%%%%%%%%%%%%%%%%%%%%%%%

%%%%%%%%%%%%%%%%%%%%%%%%%%%% Chapter 3 %%%%%%%%%%%%%%%%%%%%%%%%%%%%%%
%\begingroup
\clearpage
\section{Genes and Behavior}

\begin{center}
    This chapter has been intentionally left blank, see genetic notes for more information.
\end{center}
%\endgroup
%%%%%%%%%%%%%%%%%%%%%%%%%%%%% Chapter 3 %%%%%%%%%%%%%%%%%%%%%%%%%%%%%
%\endgroup

\clearpage
\fancyhead[L]{Part II Molecular Biology of the Neuron}
%\begingroup
%%%%%%%%%%%%%%%%%%%%%%%%%%%%% Chapter 4 %%%%%%%%%%%%%%%%%%%%%%%%%%%%%
%\begingroup
\clearpage
\section{The Cells of the Nervous System}
\begin{center}
    This chapter was intentionally left blank, see cell biology notes for more information. 
\end{center}
%\endgroup
%%%%%%%%%%%%%%%%%%%%%%%%%%%%% Chapter 4 %%%%%%%%%%%%%%%%%%%%%%%%%%%%%

%%%%%%%%%%%%%%%%%%%%%%%%%%%%% Chapter 5 %%%%%%%%%%%%%%%%%%%%%%%%%%%%%
%\begingroup
\clearpage
\section{Ion Channels}
\subsection{Rapid Signaling in the Nervous System Depends on Ion Channels}
\begin{itemize}
    \item Up to 100 million ions can pass through a single channel each second, comaprable to the turnover rate of the fastest enzymes, catalase and carbonic anhydrase.
    \item Each channel allows only one or a few types on ions to pass.
    \item Many open and close, however, some remain open resulting in significant contribution to resting potential. 
    \item Ions pumps maintained gradients and are 100 to 100,000 times slower than channels.
    \item Questions for this chapter:
        \begin{itemize}
            \item Why do nerve cells have channels?
            \item How can channels conduct ions at such high rates adn still be selective?
            \item How are channels gated?
            \item How are properties of theses channels modified by various intrinsic and extrinsic conditions?
        \end{itemize}
\end{itemize}

\subsection{Ion Channels are Proteins That Span the Cell Membrane}
\begin{itemize}
    \item Cells have channels in order the transport ions across lipid bilayer easily and eliminate the need to be stripped of waters of hydration.
    \item The smaller the ion, the greater attraction to water, and the lower its mobility. This partially explains selection, but does how does the inverse selection, that selecting of lower mobility, occur?
    \item Some ions bind to proteins that can transport them, but this is far to slow for some cases.
    \item An extension of pore theory says that channels have narrow regions that act as molecular sieves, where the ion sheds most of it's water and only is let through by a binding to a specifically charged selectivity filter. 
\end{itemize}

\subsection{Ion Channels in ALl Cells Share Several Characteristics}
\begin{itemize}
    \item The opening and closing of a channel invole conformational changes.
    \item \textbf{Gating}: the transition of a channel between theses stable functional states.
    \item Three major gating mechanisms:
        \begin{itemize}
            \item Ligand: binding of chemical ligands known as agonists at either cellular site; transmitters on the extracelluar; others that activate signaling cascades; and more. 
            \item Voltage-gated: changes in electrochemical changes as often as temperature sensors.
            \item Mechanical stretch or physical changes in the membrane.
        \end{itemize}
\end{itemize}
%\endgroup
%%%%%%%%%%%%%%%%%%%%%%%%%%%%% Chapter 5 %%%%%%%%%%%%%%%%%%%%%%%%%%%%%

%%%%%%%%%%%%%%%%%%%%%%%%%%%%% Chapter 6 %%%%%%%%%%%%%%%%%%%%%%%%%%%%%
%\begingroup
\clearpage
\section{Electrical Properties of the Neuron}
\begin{center}
    This chapter was intentionally left blank. No alternatvie notes, but may need to review chemistry and physics if this chapter is needed.
\end{center}
%\endgroup
%%%%%%%%%%%%%%%%%%%%%%%%%%%%% Chapter 6 %%%%%%%%%%%%%%%%%%%%%%%%%%%%%

%%%%%%%%%%%%%%%%%%%%%%%%%%%%% Chapter 7 %%%%%%%%%%%%%%%%%%%%%%%%%%%%%
%\begingroup
\clearpage
\section{Propagated Signaling}
\subsection{Personal Notes}
\begin{itemize}
    \item The variety of voltage-sensitive ion channels and the influence of cytoplasmic factors may be analogous to bias/weights or other hyperperamters in neural networks.
    \item Genetic changes thus change change these networks and may be how transfer learning takes place instead of starting from complete scratch for every organism. 
    \item Innate abilities could represent earlier layers in the network and heavily genetically promgrammed, while later layers are given more time to develop based on environment. 
    \item Can epigenetics have a relatively fast acting change on inherited intelligence? 
\end{itemize}

\begin{center}
    The rest of this chapter has been left blank, as it is more of an extension of molecular gentics, which I may need to return to later to answer questions like those listed above.
\end{center}
%\endgroup
%%%%%%%%%%%%%%%%%%%%%%%%%%%%% Chapter 7 %%%%%%%%%%%%%%%%%%%%%%%%%%%%%
%\endgroup

\clearpage
\fancyhead[L]{Part III Synaptic Transimission}
%\begingroup
%%%%%%%%%%%%%%%%%%%%%%%%%%%%% Chapter 8 %%%%%%%%%%%%%%%%%%%%%%%%%%%%%
%\begingroup
\clearpage
\section{Overview of Synaptic Transmission}
\subsection{Synapses Are Either Electrical or Chemical}
\begin{itemize}
    \item Average neuron forms and receive several thousand synaptic connections each, with the Purkinje cell of the cerebellum receiving up to 100,000.
    \item Both forms of transmission can be enhanced or diminished by cellular activity.
    \item Electrial synapses are used to send rapid stereotyped depolarizing signals.
    \item Chemical synapses are capable of more complex behaviors due to vairable signaling. 
    \item Most synapses are chemical.
\end{itemize}
\subsection{Electrical Synapses Provide Instantaneous Signal Transmission}
\begin{itemize}
    \item Presynaptic terminals must be big enough for its membrane to contain many ion channels to trigger initial depolarization.
    \item Postsynaptic terminals must be relatively to small in due to Ohm's law.
    \item Even weak subthreshold depolarizing currents can be carried to the postsynaptic neuron and depolarize it.
    \item Electrical synapses have a specialized region of contact called the gap junction, with seperation of only 4 nm, bridged by gap-junction channels specialized to conduct ionic current.
    \item Electrical transmission can be used to orchestrate actions or large groups of neurons.
    \item Groups of electrically coupled cells allows for explosive reactions.
    \item Gap junctions are formed between glial cells as well as neurons.
\end{itemize}
\subsection{Chemical Synapses Can Amplify Signals}
\begin{itemize}
    \item Chemical synapses are used to amplify or inhibit signals. 
    \item The synaptic cleft is 20-40 nm wide and depend on the diffusion of neurotransmitters to carry out signaling.
    \item Neurotransmitters are clustered at speclizied regions called \textbf{active zones}, which allow for selective activate of nearby postsynaptic receptors, which lead to the opening or closing of ion channels.
    \item Chemical synapses can be as short as 0.3 ms but often last several ms.
    \item Weak activations of chemical synapses can activate larger electrial synapses. 
    \item The action of a transmitter depends on the properties of the postsynaptic receptor, not the chemical properties of itself.
    \item Neurotransmitters control the opening of ion channels in the postsynaptic cell either directly or indirectly.
    \item Indirect effects tend to last seconds to minutes and often modulate behavior due to alterations in the excitability of neurons and their synaptic connections.
\end{itemize}
%\endgroup
%%%%%%%%%%%%%%%%%%%%%%%%%%%%% Chapter 8 %%%%%%%%%%%%%%%%%%%%%%%%%%%%%

%%%%%%%%%%%%%%%%%%%%%%%%%%%%% Chapter 9 %%%%%%%%%%%%%%%%%%%%%%%%%%%%%
%\begingroup
\clearpage
\section{Directly Gated Transmission}   
\begin{center}
    This chapter has been intentionally left blank. Unlikely to return to this chatper.
\end{center}
%\endgroup
%%%%%%%%%%%%%%%%%%%%%%%%%%%%% Chapter 9 %%%%%%%%%%%%%%%%%%%%%%%%%%%%%

%%%%%%%%%%%%%%%%%%%%%%%%%%%%% Chapter 10 %%%%%%%%%%%%%%%%%%%%%%%%%%%%%
%\begingroup
\clearpage
\section{Synaptic Integration in CNS}
\subsection{Central Neurons Receive Excitatory and Inhibitory Inputs}
\begin{itemize}
    \item Generation of an action potential often requires the near-synchronous firing of a number of sensor neurons.
    \item Small inhibitory postsynaptic potential (ISPS), if strong enough, can counteract the sum of the excitatory actions and prevent membrane potential from reaching threshold potential.
    \item \textbf{Sculpting} function of synaptic inhibition that exerts control over action potentials in neurons that are spontaneously active due to intrinsic pacemaker channels, often completely shaping the firing patterns of cells.
    \item Most transmitters usually are inhibitory or excitatory despite being able to be either type. 
\end{itemize}

\subsection{Inhibitory Synaptic Action}
\begin{itemize}
    \item Inhibitory synapses play an essential role in the nervous system both by preventing too much excitation and by helping coordinate activity among networks of neurons. 
    \item Inhibitory inputs that hyperpolarize the cell perform subtraction on the excitatory inputs, where those that \textbf{shunt} perform division. 
    \item Adding or removing nonshunting inhibitory inputs results in summation, while the combination or excitatory with the removal of inhibitory shunt produces a multiplication.
\end{itemize}

%\endgroup
%%%%%%%%%%%%%%%%%%%%%%%%%%%%% Chapter 10 %%%%%%%%%%%%%%%%%%%%%%%%%%%%%



%%%%%%%%%%%%%%%%%%%%%%%%%%%%% Chapter 11 %%%%%%%%%%%%%%%%%%%%%%%%%%%%%
%\begingroup
\clearpage
\section{Second Messengers}
\begin{center}
    This chapter has been intentionally left blank. Molecular focused chapters may be reviewed later when a more narrow question needs to be ansewered. 
\end{center}
%\endgroup
%%%%%%%%%%%%%%%%%%%%%%%%%%%%% Chapter 11 %%%%%%%%%%%%%%%%%%%%%%%%%%%%%

%%%%%%%%%%%%%%%%%%%%%%%%%%%%% Chapter 12 %%%%%%%%%%%%%%%%%%%%%%%%%%%%%
%\begingroup
\clearpage
\section{Transmitter Release}
\begin{center}
    This chapter has been intentionally left blank. More on the chemistry and molecular function of transmission. Such information May answer questions in the future, but it's not the focus of inquiry at the moment. 
\end{center}
%\endgroup
%%%%%%%%%%%%%%%%%%%%%%%%%%%%% Chapter 12 %%%%%%%%%%%%%%%%%%%%%%%%%%%%%

%%%%%%%%%%%%%%%%%%%%%%%%%%%%% Chapter 13 %%%%%%%%%%%%%%%%%%%%%%%%%%%%%
%\begingroup
\clearpage
\section{Neurotransmitters}
\subsection{Four Criteria of a Neurotransmitter}   
\begin{itemize}
    \item Four steps of synaptic transmission: 
        \begin{itemize}
            \item[1.] Synthesis and storage of a transmitter.
            \item[2.] Release of the transmitter.
            \item[3.] Interaction of the transmitter with receptors and postsynaptic membrane.
            \item[4.] removal of the tranmitter from the synaptic cleft.
        \end{itemize}
    \item First approximation of a neurotransmitter can be defined as a substance released by a neuron that affects a specific target in a specific manner.
    \item Neurotransmitters typically act on targets other thanthe releasing neuron itself, unlike autacoids.
    \item Neurotransmitter interaction with receptors is typically transient, lasting from milliseconds to minutes.
    \item General criteria for neurotransmitters:
        \begin{itemize}
            \item It is synthesized in the presynaptic neuron.
            \item It is present in the presynaptic terminal and is released in amounts sufficient to exert a defined action on the postsynaptic neuron or effector organ.
            \item When administered exogenously in a reasonable concentrations it mimics the action of the endogenous transmitter.
            \item A specific mechanism usually exists for removing the substance from the synaptic cleft.
        \end{itemize}
\end{itemize}

\subsection{Only a Few Small-Molecule Substances Act as Transmitters}
\begin{itemize}
    \item \textbf{Acetylcholine (ACh)}: 
        \begin{itemize}
            \item Only low weight amine transmitter substance that is not an amino acid or derived directly from one.
            \item Nervous tissue cannot synthesize choline, which limits ACh biosynthesis due to choline acetyltransferase being the only enzymatic reaction. 
            \item ACh is released by spinal motor neurons. 
            \item In the autacoids nervous system it is the transmitter for all preganglionic neurons and for parasympathetic postganglionic neurons as well.
            \item ACh is the principle neurotransmitter of the reticular activating system, which modulates arousal, sleep, wakefulness, and other critical aspects of human consciousness.
        \end{itemize}
    \item \textbf{Biogenic Amines}:
            \begin{itemize}
                \item \textbf{Catecholamine Transmitters}:
                    \begin{itemize}
                        \item \textbf{Dopamine} --- Tyrosine
                        \item \textbf{Norepinephrine} --- Tyrosine
                        \item \textbf{Epinephrine} --- Tyrosine
                        \item Tyrosine hydroxylase is the rate-limiting for synthesis of bothdopamine and norepinephrine.
                        \item $\beta$-hydroxylase converts dopamine to norepinephrine and is membrane-associated.
                        \item Norepinephrine is the only transmitter synthesized within vesicles.
                        \item In order for epinephrine to be formed, then its immediate precursor, norepinephrine, must exit from vesicles into the cytoplasm.
                        \item In order to be released, epinephrine must be taken up into vesicles.
                        \item Three of four dopaminergic nerve tracts arise in the midrain, with the last arising in the arcuate nucleus of the hypothalamus.
                        \item Synthesis of biogenic amines  is highly regulated and ca be rapidly increased.
                    \end{itemize}
                \item \textbf{Serotonin} --- Tryptophan
                \item \textbf{Melatonin} --- Serotonin
                    \begin{itemize}
                        \item Typtophan hydroxylase is the limiting reaction and the first enzyme in the pathway.
                        \item Cell bodies with serotonergic neurons are found around the midline raphe nuclei of the brain stem and are involved in regulating attention.
                        \item Productions of serotonergic cells are widely distrubted throughout the brain and spinal cord.
                        \item Antidepressant mediacation inhibit the uptake of serotonin, norepinephrine, and dopamine. 
                    \end{itemize}
                \item \textbf{Histamine} --- Histidine
                    \begin{itemize}
                        \item Long been recognized as a autacoid, active when released from mast calls in the inflammatory reaction.
                        \item Concentrated in the hypothalamus.
                    \end{itemize}
            \end{itemize}
        \item \textbf{Amino Acid Transmitters}:
            \begin{itemize}
                \item \textbf{Apartate} --- Oxaloacetate
                \item \textbf{$\gamma$-Aminobutyric acid (GABA)} --- Glutamine:
                    \begin{itemize}
                        \item Presnet at high concentrations thoughout the central nervous system and detectable in other tissues.
                    \end{itemize}
                \item \textbf{Glutamate} --- Glutamine:
                \begin{itemize}
                    \item Most frequently used at excitatory synapses throughout the central nervous system.
                \end{itemize}
                \item \textbf{Glycine} --- Serine:
                    \begin{itemize}
                        \item Major transmitter used by inhibitory interneurons of the spinal cord.
                    \end{itemize}
            \end{itemize}
        \item \textbf{ATP and Adenosine}
            \begin{itemize}
                \item Can act as transmitters at some synapses.
                \item Adenosine has an inhibitory effect in the central nervous system.
                \item Caffeine's stimulatory effect depends on inhibition of adenosine binding to its receptors.
                \item ATP released by tissue damage acts to transmit pain sensation in some cases.
            \end{itemize}
\end{itemize}

\subsection{Small-Molecule Transmitters Are Actively Taken up into Vesicles}
\begin{itemize}
    \item \textbf{Tranmitter} glutamate must be kept separate from \textbf{metabolic} glutamate; this is done through compartmentalization in synaptic vesicles.
    \item Drugs that are sufficiently similar to the normal transmitter substance can act as false transmitters, tho they often bind weakly decreasing efficacy of of transmission.
\end{itemize}

\subsection{Removal of Transmitter from the Synaptic Cleft Terminates Synaptic Transimission}
\begin{itemize}
    \item If transmitter molecules released in one synaptic action were allowed to remain in the cleft after release, then they would prevent ner signlas from getting through, the synapse would become refractory due to desensitization.
    \item Transmitters are removed by three mechanisms: diffusion, enzymatic degradation, and reuptake.
    \item Degradation is only used by cholinergic synapses.
    \item Degradation allows for single use signaling and for the lingering choline to be reused.
\end{itemize}
%\endgroup
%%%%%%%%%%%%%%%%%%%%%%%%%%%%% Chapter 13 %%%%%%%%%%%%%%%%%%%%%%%%%%%%%

%%%%%%%%%%%%%%%%%%%%%%%%%%%%% Chapter 14 %%%%%%%%%%%%%%%%%%%%%%%%%%%%%
%\begingroup
\clearpage
\section{Diseases of the Nerve and Motor Unit}
\begin{center}
    This chapter has been intentionally left blank. Might revisit, reading comprehension was low.
\end{center}
%\endgroup
%%%%%%%%%%%%%%%%%%%%%%%%%%%%% Chapter 14 %%%%%%%%%%%%%%%%%%%%%%%%%%%%%
%\endgroup


\clearpage
\fancyhead[L]{Part IV Basis of Cognition}
%\begingroup
%%%%%%%%%%%%%%%%%%%%%%%%%%%%% Chapter 15 %%%%%%%%%%%%%%%%%%%%%%%%%%%%%
%\begingroup
\clearpage
\section{Organization of the Central Nervous System}
\subsection{The Central Nervous System Consists of the Spinal Cord and the Brain}
\begin{itemize}
    \item The spinal cord is divided into a core of central gray matter and surrounding white matter.
    \item The gray matter is divied into \textbf{dorsal} and \textbf{ventral} horns.
    \item \textbf{Dorsal horn}: contains orderly sensory relay neurons that receive input from periphery.
    \item \textbf{Ventral horn}: contains group of motor neurons and interneurons that regulate motor neural firing patterns.
    \item The brain stem(\textbf{Medulla, pons, and midbrain}) has five distinct functions:
        \begin{itemize}
            \item[1.] Spinal cord mediate sensation and motor control of trunk and limbs, but the brain stem control the head, neck and face.
            \item[2.] Site of entry for information from several specialized sites such as hearing, balance, and taste.
            \item[3.] Mediation of parasympathetic reflexes, such out cardiac output, pupil constriction, and more.
            \item[4.] Contains ascending and descending pathways that carry sensory and motor information to other parts of the CNS.
            \item[5.] Contains the \textbf{reticular formation}, which receives a summary of incoming sensory information and regulates alertness and arousal.    
        \end{itemize}
\end{itemize}
\subsection{The Major Functional Systems Are Similarly Organized}
\begin{itemize}
    \item The central nervous system consists of several functional systems that are relatively autonomous and much work together using numerous interconnected anatomical sites throughout the brain.
    \item Information is transformed at each synaptic relay, with the output rarely being the same as the input.
    \item Neurons at each synaptic relay are organized into a neural map of the body.
    \item Most sensory systems inputs are arranged topographically through out successive stages of processing.
    \item Each functional system is hierarchically organized.
    \item \textbf{Decussations}: crossing of second order fiber from the brain stem and the spinal cord.
\end{itemize}

\subsection{The Cerebral Cortex is Concerned with Cognition}
\begin{itemize}
    \item Increasing the surface area due to sulci and gyri allow for greater number of cortical neurons which provide a greater capacity for information processing.
    \item Neurons in the cerebral cortex are organized in layers and columns which helps computational efficiency.
    \item The neocortex receives inputs from the thalamus, other cortical regions on both sides of the brain, and other structures then output to other various regions.
    \item The input-output relation is organized into orderly layering of cortical neurons, with most containing six layers.
        \begin{itemize}
            \item Layer I: the \textbf{molecular layer}, is occupied by dendrites of cells located in deeper layers and axons that make connections to other areas of the cortex.
            \item Layer II: the \textbf{external granule cell layer}, one of two layers that contain small spherical neurons.
            \item Layer III: the \textbf{external pyramidal cell layer}, second layer of small spherical neurons, typically larger that layer II.
            \item Layer IV: the \textbf{internal granule cell layer}, contains much larger number of spherical neurons and is main recipient of sensory input from the thalamus.
            \item Layer V: the \textbf{internal pyramidal cell layer}, contains pyramidal neurons that are also larger than it's external layer. Theses neurons give rise to major output pathways of the cortex.
            \item Layer VI: the \textbf{multiform layer}, a blend neurons into white matter that forms the deep limit of the cortex and carries axons to and from areas of the cortex.
        \end{itemize}
    \item Thickness of the layers vary throughout the cortex.
    \item The cerebral cortex has a large variety of neurons, more than 40 different types based only on the distribution of their dentrites and axons.
    \item Most neurons are either principal (projection) neurons or local interneurons.
\end{itemize}

\subsection{Subcortical Regions of the Brain are Functionally Organized into Nuclei}
\begin{itemize}
    \item Three major structures lie deep within the cerebral hemisphere: the basal ganglia, the hippocampal foramtion, the amygdala, and the basal ganglia. These subcortical structures act to regulate the cortical activity. 
    \item Basal ganglia regulates movement and certain cognitive functions such as learning of motor skills.
    \item THe basal ganglia has five major functional subcomponents: the caudate nucleus, putamen, globus pallidus, subthalamic nucleus, and substantia nigra.
    \item The hippocampal formation includes the hippocampus, dentate gyrus, and subiculum. Together theses structures are responsible for the formation of long-term memories episodic memories, but not responsible for storage.
    \item The amygdala is involved in analying the emotional significance of sensory stimuli.
\end{itemize}

\subsection{Modulatory Systems in the Brain Influence Motivation, Emotion, and Memory}
\begin{itemize}
    \item Some brain areas are neither sensory nor motor, but instead modify specific functions.
    \item Distinct modulatory systems within the brain stem modulate attention and arousal. 
\end{itemize}

\subsection{The Peripheral Nervous System is Anatomically Distinct from the Central Nervous System}
\begin{itemize}
    \item The peripheral nervous system supplies the central nervous system with a continuous stream of information about both externala and internal environments. It is split into two divisions.
    \item The \textbf{somotic division} includes the sensory neurons that receive information from skin, muscles, and joints and provide information about position and pressure. 
    \item The \textbf{autonomic division} mediates visceral sensation as well as motor control of the viscera, vascular, and exocrine glands. It consists of sympathetic (response to stress), parasympathetic (restores homeostasis), and enteric (controls smooth muscle of the gut) systems.
\end{itemize}
%\endgroup
%%%%%%%%%%%%%%%%%%%%%%%%%%%%% Chapter 15 %%%%%%%%%%%%%%%%%%%%%%%%%%%%%

%%%%%%%%%%%%%%%%%%%%%%%%%%%%% Chapter 16 %%%%%%%%%%%%%%%%%%%%%%%%%%%%%
%\begingroup
\clearpage
\section{Organization of Perception and Movement}
\subsection{Sensory Information Processing is Illustrated in the Somatosensory System}
\begin{itemize}
    \item Complex behaviors require the integrated action of several nuclei and cortical regions, processed in a hierarchical fashion, and becomes increasingly complex.
    \item Complex processing results in a light touch or painful prick in the skin being mediated by often very different pathways.
    \item Somatosensory information from the trunk and limbs is conveyed to the spinal cord.
    \item The spinal cord is divied into four major regions: cervical, thoracic, lumbar, and sacral.
    \item Spinal nevres at the cervical level are involved with sensory perceptions and motor function of the back of the head, neck, and arms.
    \item Thoracic nerves innervate the upper trunk.
    \item Lumbar and sacral nevres innervate the lower trunk, back, and legs.
    \item Each of the four regions of the spinal cord contains several segments, depsite the lack of appearance of segmentation of mature spinal cords.  
    \item The spinal cord varies in size and shape due to two organizational features.
    \item First, the relatively few sensory axons enter the cord at the sacral level, with number of entering axons increasing progressively at higher levels.
    \item Most descending axons from the brain terminate at cervical levels.
    \item Second, the variation in the size of the ventral and dorsal hons.
    \item The number of ventral motor neurons dedicated to the body region roughly parallels the dexterity of movements of that region.
    \item \textbf{lumbosaral} and \textbf{cervical enlargements}: regions of the spinal cord where fibers enter the cord due demends of sensory neurons for finer tactile discrimination in limbs.
    \item The primary sensory neurons of the trunk and limbs are clustered in the dorsal root ganglia. These neurons are pseudo-unipolar in shape and have bifurcated axon with central and peripheral branches.
    \item Local branches activate local reflex circuits while ascending branches carrying information to the brian that give rise the perception.
    \item The central axons of dorsal root ganglion neurons are arranged to produce a map of the body surface.
    \item Each somatic submodality if processed in a distinct subsystem from the periphery to the brain.
\end{itemize}

\subsection{The Thalamus is an Essential Link Between Sensory Receptors and the Cerebral Cortex for All Modalities Except Olfaction}
\begin{itemize}
    \item The thalamus conveys sensory input to the primary sensory areas of the cerebral cortex and additionally acts as a gatekeeper depending of behavioral state of the animal.
    \item The thalamus is a good example of a brain region made up of several well-defined nuclei.
    \item Some nuclei receive information specific to a sensory modality and projet to a specific area of the neocortex.
    \item The nuclei of the thalamus are most commonly classified into four groups:
        \begin{itemize}
            \item \textbf{anterior group}: recvies most input form the mammillary nuclei of the hypothalamus and presubiculum of the hippocampal formation. The role of this region is uncertain, but thought to be related to memory and emotion.
            \item \textbf{Medial group}: consists mostly of the mediodorsal nucleus. It receives input from the basal ganglia, amygdala, and midrain and is been implicated in memory.
            \item \textbf{ventral group}: important for motor control and carry information from basal ganglia and cerebellum to the motor cortex.
            \item \textbf{posterior group}: includes the medial and lateral geniculate nucleus (component of auditory system), lateral posterior nucleus (componenet of the retina and visual cortex in the occipital lobe), and the pulmonary(involved in the parietal-occipital-temporal cortex).
        \end{itemize}
    \item \textbf{Reticular nucleus}: a unique sheet-like structure covering the thalamus.
    \item Neurons ot the reticular nucleus are not interconnected with the neocortex, instead the axons terminate on the other nuclei of the thalamus.
    \item Thus, the reticular nucleus modulates activity in other thalamic nuclei based on its moitoring of the entirety of the thalamocortical stream.
    \item The thalamus not only relays information but is a crucial step and adds substantial degree of information processing.
\end{itemize}

\subsection{Sensory Information Processing Culminates in the Cerebral Cortex}
\begin{itemize}
    \item  Parts of the body are represented in the cortex somatotopically, but the area of the cortex is not proportional to it mass. Instad, it proportional t the density of innervation.
    \item The cortical areas involved in the early stages of sensory processing are concerned primarily with a single modality. 
    \item Unimodal association areas converge on multimodal association areas of the cortex concerned with combining sensory modalities.
    \item Multimodal associational areas are heavily interconnected with the hippocampus and appear to be important for unified percept and representation of the percept in memory.
    \item There is a close linkage between the somatosensory and motor functions of the cortex.
\end{itemize}

\subsection{Voluntary Movement is Mediated by Direct Connections Between the Cortex and Spinal Cord}
\begin{itemize}
    \item The human corticospinal tract consists of approximately one million axons, with 40\% originating from the motor cortex.
    \item Most of the corticospinal fibers cross the midline in the medulla at a location known as the pyramidal decussation.
    \item 10\% of those fibers do not cross until they reach the local where they terminate.
    \item THe motor information carried in the corticospinal tract is significantly modulated by the sensory information and information from other motor regions.
    \item The cerebellum is thought to be part of an error-correcting mechanism for movements because it can compare movement commands from the cortex with somatic sensory information about what actually happened.
\end{itemize}
%\endgroup
%%%%%%%%%%%%%%%%%%%%%%%%%%%%% Chapter 16 %%%%%%%%%%%%%%%%%%%%%%%%%%%%%

%%%%%%%%%%%%%%%%%%%%%%%%%%%%% Chapter 17 %%%%%%%%%%%%%%%%%%%%%%%%%%%%%
%\begingroup
\clearpage
\section{Representation of Space and Action}
\subsection{The Brain has an Orderly Representation of Personal Space}
\begin{itemize}
    \item Internal representation can be thought of as a certain pattern of neural activity that has at least two aspects:
        \begin{itemize}
            \item The pattern of activation within a particular population of neurons.
            \item the pattern of firing in individual cells.
        \end{itemize}
    \item The cortex has a map of the sensory receptive surface for each sensory modality.
    \item Cortical maps of the body are the basis of accurate clinical neurological examinations.
    \item The is a direct relationship between the anatomical organization of the functionalpathwaysin the brain and specific perceptual and motor behaviors.
\end{itemize}

\subsection{The Internal Representation of Personal Space can be Modified by Experience}
\begin{itemize}
    \item Details of sensory maps vary considerably from one individual to another.
    \item Lost connections can be taken over by existing nearby connections.
\end{itemize}

\subsection{Is Consciousness Accessible to Neurobiological Analysis?}
\begin{itemize}
    \item John Searle and Thomas Nagel have defined three essential features of self-awareness:
        \begin{itemize}
            \item Subjectivity, or the awareness of a self that is the center of experience.
            \item Unity, or the fact that our experience of the world at any given moment is felt as a single unified experience.
            \item Intentionality, or the the experience that connets successive moments and the sense that the successive moments are directed to some goal.
        \end{itemize}
    \item Crick and Koch argue that our efforts should be focused on visual perceptionand in particular on two phenomena: binocular rivalry and selective attention.
    \item Sensory input alone does not give rise to consciousness; higher-level interpretation of that input is needed.
\end{itemize}
%\endgroup
%%%%%%%%%%%%%%%%%%%%%%%%%%%%% Chapter 17 %%%%%%%%%%%%%%%%%%%%%%%%%%%%%

%%%%%%%%%%%%%%%%%%%%%%%%%%%%% Chapter 18 %%%%%%%%%%%%%%%%%%%%%%%%%%%%%
%\begingroup
\clearpage
\section{Organization of Cognition}
\subsection{Functionally Related Areas of Cortex Lie Close Together}
\begin{itemize}
    \item The cortex of each cerebral hemisphere is a continuous sheet of gray matter.
    \item At the coarsest level, it consists of five lobes, with each lobe further subdivided.
    \item Functional areas are distinguished by cellular structure, connectivity, and the physiological response properties of neurons.
    \item Precepts that govern the organization of functional areas in the macaque (old world monkey) cerebral cortex:
        \begin{itemize}
            \item[1.] All areas fall into a few major functional groups.
            \item[2.] Areas in a given category occupy a discrete, continuous portion of the cortical sheet.
            \item[3.] Functionally related areas occupy neighboring sites. 
        \end{itemize}
\end{itemize}

\subsection{Sensory Information is Processed in the Cortex in Serial Pathways}
\begin{itemize}
    \item Cortical areas communicate with each other through bundles of axons traveling together in identifiable tracts.
    \item \textbf{Primary sensory areas} posses four properties characteristic of their role in the early stages of information processing:
        \begin{itemize}
            \item Inputs from thalamic sensory relay nuclei.
            \item Neurons in a primary sensory area have small receptive fields adn are arranged to form a precise somatotopic map of the sensory receptor surface.
            \item Injury to a part of the map causes a simple sensory loss confined to the corresponding part of the contralateral sensory receptor surface.
            \item Connections to other cortical areas are limited, mostly to nearby areas that process information in the same modality.
        \end{itemize}
    \item Higher-order sensory areas have a different set of properties important to their role in the later stages of information processing:
        \begin{itemize}
            \item Inputs arise from other thalamic nuclei and lower-order areas of sensory cortex instead of sensory relay nuclei.
            \item Large receptive fields and imprecise maps of the array of receptors in the periphery.
            \item Injury results in abnormalities of perception, but does not impair ability to detect sensory stimuli.
            \item Connected to distant areas in the frontal and limbid nodes as well as nearby unimodal areas.
        \end{itemize}
    \item Sensory information is processed serially, but not exclusively; higher-order areas project back to lower-order areas which can modulate the activity of neurons in lower-order areas.
    \item \textbf{Association cortex}: regions of the cortex where injury causes cognitive deficits that cannot be explained by impairment of sensory or motor function alone.
    \item Large regions of association cortex are contained within each of the four lobes:
        \begin{itemize}
            \item \textbf{parietal}: critical for sensory guidance of motor behavior and spatial awareness.
            \item \textbf{temporal}: recognition of sensory stimuli and for storage of semantic (factual) knowledge.
            \item \textbf{frontal}: key role in organizing behavior in working memory.
            \item \textbf{limbic}: complex functions related to emotion and episodic (autobiographical) memory.
        \end{itemize}
    \item Association areas have much more extensive input and output connections than do lower-order sensory and motor areas.
    \item All association lobes are densley interconnected network of pathways.
\end{itemize}

\subsection{Goal-Directed Motor Behavior Is Controlled in the Frontal Lobe}
\begin{itemize}
    \item All areas of th frontal lobe participate in the control of motor behavior and are connected in a series of functional hierarchy.
    \item Neuronal activity in the premotor cortex, adjacent to the primary motor cortex, reflect global aspects of motor behavior.
    \item Dorsolateral prefrontal cortex contributes to cognitive control of behavior.
    \item The orbital-ventromedial prefrontal corftex, connected to the dorsolateral prefrontal cortex (then premotor), is involved with emotional processes associated with executive control of behavior.
    \item Information flows from higher-order areas in the frontal lobe to primary order cortex, contrasting sensory's periphery first flow.
    \item Prefrontal cortex is important for the executive control of behavior.
    \item The orbital-ventromedial prefrontal cortex is linked strongly to the hypothalamus and amygdala, receives input from every sensory system, and projects to the dorsolateral prefrontal cortex. 
    \item Thus, the above pathway allows for response to emotional and sensory inputs and allows the trigger of appropriate behavior.
\end{itemize}

\subsection{Limbic Association Cortex is a Gateway to the Hippocampal Memory System}
\begin{itemize}
    \item The limbic (limbus---edge) association cortex forms a ring that is visible in the medial view of the hemispher.
    \item Previously it was thought to make up an entire system in combinations with other areas, but some divisions of the limbic lobe have other functions, with some not yet well understood.
    \item The limbic association cortex does play an important role in long-term memory formation.
\end{itemize}

%\endgroup
%%%%%%%%%%%%%%%%%%%%%%%%%%%%% Chapter 18 %%%%%%%%%%%%%%%%%%%%%%%%%%%%%

%%%%%%%%%%%%%%%%%%%%%%%%%%%%% Chapter 19 %%%%%%%%%%%%%%%%%%%%%%%%%%%%%
%\begingroup
\clearpage
\section{Functions of the Premotor Systems}
\subsection{Direct Connections Between the Cerebral Cortex and Spinal Cord Play a Fundamental Role in the Organization of Voluntary Movements}
\begin{itemize}
    \item Individual muscles and joints are represented in the cortex multiple times in a complex mosaic.
    \item Each muscle joint is represented by a column of neurons whose axons branch and terminate in several functionally related spinal motor nuclei.
    \item Movement can also be elicited by stimulation of premotor areas.
    \item Neurons in the primary motor cortex fire in connection with a variety of goal-directed movements.
    \item There are three pathways from the premotor and motor areas to the motor neurons in the spinal cord: a direct corticospinal and two indirect, the medial and lateral brain stem systems.
    \item The pathways together make up the \textbf{corticospinal system}.
    \item \textbf{Medial brain stem system}: receives information from the cortex and other motor centers for the control of posture and locomotion.
    \item \textbf{Lateral brain stem system}: similar to medial but is involved in control of arm and hand movements.
    \item Reflex circuits can generate stereotyped movements without descending commands; new patterns can be generated through direct action on motor neurons.
    \item The cortical motor areas receive feedback from the cerebellum and basal ganglia in order for smooth execution of skilled movements in motor learning.
\end{itemize}

\subsection{The Four Premotor Areas of the Primate Brain Also Have Direct Connections in the Spinal Cord}
\begin{itemize}
    \item \textbf{Lateral ventral premotor area}: concerned the "what" in visual preception and controls mostly mouth and hand movements.
    \item \textbf{Lateral dorsal premotor area}: concerned with the "where" in visual preception and controls direction or movements.
    \item \textbf{Supplementary motor area}:
    \item \textbf{Cingulate motor areas}: a group of areas in the cingulate sulcus. 
    \item The four areas are connected to the primary motor cortex.
\end{itemize}

\subsection{Motor Circuits Involved in Voluntary Actions are Organized to Achieve Specific Goals}
\begin{itemize}
    \item The parietal lobe contains more than one representation of space and each one is dependent on motor activity.
    \item Neurons that respond to objects in peripersonal space are located mostly in the inferior parietal lobe where hand and mouth movements are represented, whereas neurons that respond to futher away objects are found where eye movements are represented.
\end{itemize}

\subsection{The Hand Has a Critical Role in Primate Behavior}
\begin{itemize}
    \item Investigation of the \textbf{anterior intraparietal area (AIP)} show that neurons fall into three main classes: motor-dominant, visual-dominant, and visual-motor combination.
    \item Futher suggestion shows that these neurons are involved in transfering sensory representations of objects into motor representations.
    \item \textbf{Canonical neurons}: neurons that fire in response to visual observation (and actual grasping) of graspable objects of certain size, shape, and orientation.
    \item Canonical neurons are thought to translate objects physical properties into \textit{potential motor acts}.
\end{itemize}

\subsection{The Join Activity of Neurons in the Parietal and Premotor Cortex Encodes Potential Motor Acts}
\begin{itemize}
    \item Studies of the parietal and premotor canonical neurons show that some neurons encode the possibilities for interaction with an object.
    \item \textbf{Mirror neurons}: discharge during specific motor acts, but also fire when the individual observation action being done by another.
    \item Mirror neurons may help us understand the intention of others.
    \item Potential motor acts are suppreseed or released by motor planing centers.
    \item Neurons in the supplementary motor area are involved in the planning, generation, and control of sequential motor actions.
    \item After long periods of practice, when the behavior becomes automatic, activity in the presupplementary motor area ceases.
\end{itemize}
%\endgroup
%%%%%%%%%%%%%%%%%%%%%%%%%%%%% Chapter 19 %%%%%%%%%%%%%%%%%%%%%%%%%%%%%

%%%%%%%%%%%%%%%%%%%%%%%%%%%%% Chapter 20 %%%%%%%%%%%%%%%%%%%%%%%%%%%%%
%\begingroup
\clearpage
\section{Functional Imaging of Cognition}
\subsection{Functional Imaging Reflects the Metabolic Demand of Neural Activity}
\begin{itemize}
    \item A large amount of neuron's total energy metabolism, about one-half, is devoted to mainting resting potential.
    \item The other half is for other biochemical processes, including all molecular reactions for normal function.
    \item fMRI responses have been highly correlated with neural spiking.
    \item Though, some fMRI measurements fail to capture subthreshold modulatory inputs.
\end{itemize}

\begin{center}
    Interesting chapter, but each section had one-two random notes. Omitted due to lack of conclusiveness.
\end{center}
%\endgroup
%%%%%%%%%%%%%%%%%%%%%%%%%%%%% Chapter 20 %%%%%%%%%%%%%%%%%%%%%%%%%%%%%
%\endgroup


\clearpage
\fancyhead[L]{Part V Perception}
%\begingroup

%%%%%%%%%%%%%%%%%%%%%%%%%%%%% Chapter 21 %%%%%%%%%%%%%%%%%%%%%%%%%%%%%
%\begingroup
\clearpage
\section{Sensory Coding}
\subsection{Psychophysics Relates the Physical Properties of Stimuli to Sensations}
\begin{itemize}
    \item \textbf{Psychophysics}: relationship between the physical characteristics of a stimulus and attributes of sensory experience.
    \item \textbf{Sensory physiology}: examination of neural consequences of a stimulus.
    \item \textbf{Sensory threshold}: the lowest stimulus strength a subject can detect.
    \item Sensory thresholds can be altered by emotional or psychological factors.
    \item Sensations are quantified using probabilistic statistics.
    \item Reaction times are correlated with cognitive processes.
\end{itemize}

\subsection{Physical Stimuli are Represented in the Nervous System by Means of the Sensory Code}
\begin{itemize}
    \item Neural coding of sensory information is better understood at the early stages than later.
    \item Sensory receptors are responsive to a single type of stimulus energy.
    \item \textbf{Stimulus transduction}: the time it takes to convert a stimulus response into an electrical signal.
    \item Multiple subclasses of sensory receptors are found in each sense organ.
    \item There are rapid and slowly adapting sensors that illustrate a major principal of decoding: \textit{contrast}.
    \item  The timing of action potentials between neurons has a profound effect on long-term potentiation and depression at synapses.
    \item The receptive field of a sensory neuron conveys spatial information.
    \item Fragmentation of a stimulus into componenets, each encoded by an individual neuron, is the initial step in sensory processing.
\end{itemize}

\subsection{Modality-Specific Pathways Extend to the Central Nervous System}
\begin{itemize}
    \item Activity of sensory neurons are more variable than that of neurons in the periphery.
    \item Central sensory neurons fire irregularly before and after stimulation, even during times of no stimulation.
    \item The variability is a result of: alertness, attention, previous experience, and recent activation by similar stimuli.
    \item The receptor surface is represented topographically in central nuclei.
    \item Feedback regulates sensory coding and top-down learing mechanisms influences sensory processing.
\end{itemize}
%\endgroup
%%%%%%%%%%%%%%%%%%%%%%%%%%%%% Chapter 21 %%%%%%%%%%%%%%%%%%%%%%%%%%%%%

%%%%%%%%%%%%%%%%%%%%%%%%%%%%% Chapter 22 %%%%%%%%%%%%%%%%%%%%%%%%%%%%%
%\begingroup
\clearpage
\section{Somatosensory System}
\begin{itemize}
    \item The somatosensory system serves three major functions:
        \begin{itemize}
            \item \textbf{Proprioception}: the sense of oneself. Skeletal muscle, joint capsules, and the skin allow for aweraness of our own body.
            \item \textbf{Exteroception}: the sense of direct interaction with the external world. Touch, contact, pressure, storking, temperature, pain, motion, vibration are used to identify objects.
            \item \textbf{Interoception}: sense of major organ systems of the body and it's internal state. Most information does not appear conscious, but plays a major role. Primarily consist of chemoreceptors.
        \end{itemize}
    \item All somatic senses are mediated by the dorsal root ganglion neurons.
\end{itemize}
\subsection{The Primary Sensory Neuron of the Somatosensory System are Clustered in the Dorsal Root Ganglia}
\begin{itemize}
    \item Dorsal root ganglion neurons are pseudo-unipolar cells.
    \item The central branches termintate in the spinal cord or brain stem, forming the first synapses in somatosensory pathways.
    \item \textbf{Primary afferent fiber}: the axon of each dorsol root ganglion cell serves as a single tranmission line from receptor to central nervous system.
    \item \textbf{Peripheral nerves}: individual primary afferent fibers group that are grouped together, and also include motor axons innervating nearby muscles, blood vessels, glands, or viscera. 
\end{itemize}

\subsection{Peripheral Somatosensory Nerve Fibers Conduct Action Potentials at Different Rates}
\begin{itemize}
    \item Difference in peripheral nerve's diameter and conduction velocity mediate somatic sensation.
    \item Larger diameter tends to relay faster, not accounting for degree myelinated fibers.
    \item Electrial stimulation of whole nerves is also used to classify peripheral nerve fibers.
    \item \textbf{Compound action potential}: summed action potential of all nerve fibers excited by a stimulus pulse and is roughly proportional to the total number of active nerve fibers.
    \item The conduction velocity throughout teh nervous system is correlated with the need to maintain synchrony.
\end{itemize}

\subsection{Many Specialized Receptors Are Employed by the Somatosensory System}
\begin{itemize}
    \item The receptor class expressed in the nerve terminal of a sensory neuron determines the type of stimulus detected.
    \item Mechanoreceptors mediate touch and proprioception.
    \item The skin has eight types of mechanoreceptors that are responsible for touch.
    \item Proprioceptors measure muscle activity and joint positions.
    \item Nociceptors mediate pain.
    \item Thermal receptors detect changes in skin temperature.
    \item Itch is a distinctive cutaneous sensation.
    \item Visceral sensations represnet the status of various interanal organs.
\end{itemize}
%\endgroup
%%%%%%%%%%%%%%%%%%%%%%%%%%%%% Chapter 22 %%%%%%%%%%%%%%%%%%%%%%%%%%%%%

%%%%%%%%%%%%%%%%%%%%%%%%%%%%% Chapter 23 %%%%%%%%%%%%%%%%%%%%%%%%%%%%%
%\begingroup
\clearpage
\section{Touch}
\subsection{Active and Passive Touch Evoke Similar Responses in Mechanoreceptors}
\begin{itemize}
    \item Active and passive modes of tactile stimulation excite the same population of receptors in the skin and evoke similar responses in afferent fibers.
    \item Passive touch is used for naming objects or describing sensations.
    \item Active touch is used when the hand manipulates objects.
\end{itemize}

\subsection{The Hand Has Four Major Types of Mechanoreceptors}
\begin{itemize}
    \item \textbf{Merkel cell}: tips of epidermal sweat ridges; detects edges, points; slow adaption to sustained indentation.
    \item \textbf{Meissener corpuscle}: close to skin surface; detects lateral motion; no adaption to sustained indentation.
    \item \textbf{Ruffini ending}: located in dermis; senses skin stretching; slow adaption to sustained indentation.
    \item \textbf{Pacinian corpuscle}: located deep in dermis; senses vibration; no adaption to sustained indentation. 
    \item Receptive fields define the zone of tactile sensitivity.
    \item There are two types of receptive fields: one with highly specialized fields; the other with broader fields with a central hotspot.
    \item Slowly adapting fibers detect object pressure and from.
    \item Rapidly adapting ribers detect motion and vibration.
    \item Combination of slow and rapidly adapting fibers contribute to grip control.
\end{itemize}

\subsection{Tactile Information is Processed in the Central Touch System}
\begin{itemize}
    \item Cortical receptive fields intergrate information from neighboring receptors.
    \item Neurons in the somatosensory cortex are organized into functionally specialized columns.
    \item All neurons within a column receive inputs from teh same local area of teh receptor sheet and respond to the same class(es) of receptors.
    \item Columns share a common center that is clearly evident in layer IV.
    \item Horizontal connections within layers II and III link neurons in neighboring columns, sharing information when activated by the same stimulus.
    \item Cortical columns are organized somatotopically.
    \item \textbf{Cortical magnification}: the amount of cortical area devoted to a unit of area of skin. This various by more than a hundredfold across differrent body surfaces.
\end{itemize}
%\endgroup
%%%%%%%%%%%%%%%%%%%%%%%%%%%%% Chapter 23 %%%%%%%%%%%%%%%%%%%%%%%%%%%%%

%%%%%%%%%%%%%%%%%%%%%%%%%%%%% Chapter 24 %%%%%%%%%%%%%%%%%%%%%%%%%%%%%
%\begingroup
\clearpage
\section{Pain}
\subsection{Noxious Insults Activate Nociceptors}
\begin{itemize}
    \item Most nociceptors are simply the free nerve endings of primary sensory neurons.
    \item There are three main classes of nociceptors:
        \begin{itemize}
            \item Thermal
            \item Mechanical
            \item Polymodal--- high-intensity mechanical, chemical, or thermal. 
        \end{itemize}
    \item There is a less understood fourth class: silent nociceptors.
    \item The three main classes are widely distributed in the skin and deep tissues and are oftening coactivated.
    \item Silent nociceptors are found in viscera; activated by inflammation and various chemical agents.
    \item \textbf{Allodynia}: pain in response to stimuli that re normally innocuous.
    \item \textbf{Hyperalgesia}: an exaggerated response to noxious stimuli, typically persistant even in absence of sensory stimulation. 
    \item \textbf{Nociceptive pain}: activation of nociceptors and normally from accompanying inflammation.
    \item \textbf{Neuropathic pain}: direct injury to nerves in peripheral or central nervous system and is accompanied by burining or electric sensation.
\end{itemize}
%\endgroup
%%%%%%%%%%%%%%%%%%%%%%%%%%%%% Chapter 24 %%%%%%%%%%%%%%%%%%%%%%%%%%%%%

%%%%%%%%%%%%%%%%%%%%%%%%%%%%% Chapter 25 %%%%%%%%%%%%%%%%%%%%%%%%%%%%%
%\begingroup
\clearpage
\section{Visual Processing}
\subsection{Visual Perception is a Constructive Processe}
\begin{itemize}
    \item The brain guesses at scene presented to the eyes based on past experience.
    \item The modern vew of perception is based on the gestalt psychology---the perceptual interpretation we make of any visual object depends not just on the properties of the stimulus, but also the context.
    \item An important step in object recognition separating figures from the background.
    \item The brain analyzes a scene at three levels:
        \begin{itemize}
            \item Low: local contrast, orientation, color, and movement.
            \item Intermediate: analysis of the layout, surfaces, parsing global contours, and depth.
            \item High: object recognition.
        \end{itemize}
    \item Motion, depth, form, and color occur in a unified percept due to interacting neural pathways.
\end{itemize}

\subsection{Form, Color, Motion, and Depth are Processed in Discrete Areas of the Cerebral Cortex}
\begin{itemize}
    \item Visual areas of the cortex can be differentiated either by a visuotopic map, of by functional properties of the neurons.
    \item Visual areas are organized into two hierarchical pathways: ventral, involved in object recognition; and dorsal, dedicated to the use of visual information guiding movements.
    \item Pathways are interconnected so that information is shared and each connection is reciprocal--- each area sends information back to areas from which it receives input.
    \item The shared connections provide information about cognitive functions, spatial attention, stimulus expectation, and emotional conetent, to earlier levels of visual processing.
\end{itemize}

\subsection{The Receptive Fields of Neurons at Successive Relays in an Afferent Pathway Provide Clues to How the Brain Analyzes Visual Form}
\begin{itemize}
    \item \textbf{On-center}: cells that fire when a spot of light is turned on within a circular central region.
    \item \textbf{Off-center}: cells that fire inversely to on-center.
    \item If both cells are stimulated with diffuse light, then there is little to no response. This allows them to distinguish borders and contours very well and leads to the encoding of contrast.
    \item \textbf{Eccentricity}: size of the retina's receptive field, which varies in relative to the fovea and the position of neurons along the visual pathway.
\end{itemize}

\subsection{The Visual Cortex is Organized into Columns of Specialized Neurons}
\begin{itemize}
    \item The dominant feature of the functional organization of the primary visual cortex is the visuotopic organization of the of its cells: the visual field is systematically represented across the surface of the cortex.
    \item Columins reflect the functional role of that area in vision.
    \item Orientation and ocular dominance columns have embedded clusters of neurons that have strong color preferences.
    \item These clusters specialize to provide information about surfaces rather than edges.
    \item \textbf{Serial processing}: processing in successive connections between cortical areas that run from the back of the brain forward.
    \item \textbf{Parallel processing}: occurs simultaneously in subsets of fibers that process different submodalities such as from, color, movement.
\end{itemize}

\subsection{Intrinsic Cortical Circuits Transform Neural Information}
\begin{itemize}
    \item Each area of the visual cortex transforms information gathered by the eyes.
    \item Principal input to the primary visual cortex comes from two parallel pathways that originate in the parvocellular and magnocellular layers of the lateral geniculate nucleus.
    \item Neurons in different layers have distinctive receptive-field properties, with superficial layers have smaller fields while deeper layers tend to have larger ones.
    \item Feedback projections are thought to provide a means where higher centers in a pathway can influence lower ones.
    \item Feedback projection is still largely unknown.
\end{itemize}
%\endgroup
%%%%%%%%%%%%%%%%%%%%%%%%%%%%% Chapter 25 %%%%%%%%%%%%%%%%%%%%%%%%%%%%%


%%%%%%%%%%%%%%%%%%%%%%%%%%%%% Chapter 26 %%%%%%%%%%%%%%%%%%%%%%%%%%%%%
%\begingroup
\clearpage
\section{Low-Level Visual Processing}
\subsection{The Photoreceptor Layer Samples the Visual Image}
\begin{itemize}
    \item Ocular optics limit the quality of the retinal image.
    \item The density of photoreceptors, bipolar cells, and ganglion cells is highest at the fovea(center of eye).
    \item There are two types of photoreceptors: rods and cones.
    \item Rods: very sensitive---low light, no color.
    \item Cones: less sensitive---for daylight, multiple types, faster response time.
    \item Central fovea has an absence of rods.
\end{itemize}

\subsection{Ganglion Cells Tranmist Neural Images to the Brain}
\begin{itemize}
    \item Optic nerve has only 1\% as many axons as there are receptor cells, so the retinal circuit must edit information before it is conveyed to the brain.
    \item The two major ganglion cells are binary, ON or OFF cells.
    \item Many ganglion cells fire regardless of current lighting condition, but ON cells fire more rapidly with increasing light, while OFF slows or stops. The inverse is also true when going form light to dark.
    \item Many ganglion cells respond strongly to edges in the image.
    \item Output produced by ganglion cells enhance regions of contrast, while reducing homogenous illumination.
    \item Ganglion output also emphasizes temporal changes in stimuli through transient (burst response) and sustained (steady) neurons.
    \item Retinal output emphasizes moving objects.
    \item Several ganglion cell types project to the brain through parallel pathways.
    \item About 20 ganglion cells have been described, which allow the optic nerve to convey about 20(?) different represnations of the world based on polarity (on/off, fine/coarse, sustained/transient, motion, spectral filtering... and more?)
\end{itemize}

\subsection{A Network of Interneurons Shapes the Retinal Output}
\begin{itemize}
    \item Parallel pathways originate in bipolar cells.
    \item Most retinal processing is accomplished with graded membrane potentials via the \textit{ribbon synapse}.
    \item Action potentials occur only in certain amacrine and ganglion cells.
    \item Stimulus represention in ganglion cell population originates in dedicated bipolar cell pathways that are differentiated by their selective connections to photoreceptors and postsynaptic targets.
    \item Spatial filtering is accomplished by lateral inhibition.
    \item Amacrine cells are axonless neurons with dendrites that ramify in the inner plexiform layer, generally producing an inhibitory network.
    \item Temporal filtering occurs in synapses and feedback circuits.
    \item Color vision begins in cone-selective circuits.
    \item Rod and cone circuits merge in the inner retina.
\end{itemize}








%\endgroup
%%%%%%%%%%%%%%%%%%%%%%%%%%%%% Chapter 26 %%%%%%%%%%%%%%%%%%%%%%%%%%%%%
%\endgroup
\end{document}