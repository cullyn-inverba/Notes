\chapter{16: Oxidation and Reduction}\label{16: Oxidation and Reduction}
\section{Redox Basics/Review}\label{Redox Reactions Basics/Review}
\begin{itemize}
    \item \ddd{Oxidation}: is the \rrr{loss of electrons} or an \rrr{increase in the oxidation state} of an atom, an ion, or of certain atoms in a molecule.
    
    \medskip
    \schemestart{}
      \rrr{\ch{Fe^{+2}}}
      \arrow{}
      \rrr{\ch{Fe^{+3}}}
      \+
      \bbb{\ch{e-}}
    \schemestop{}
    \bigskip
    
    \item \ddd{Reduction}: is the \bbb{gain of electrons} or a \bbb{decrease in the oxidation} state of an atom, an ion, or of certain atoms in a molecule (a reduction in oxidation state).
    
    \medskip
    \schemestart{}
      \rrr{\ch{Mn^{+7}}}
      \+
      \bbb{\ch{e^{-2}}}
      \arrow{}
      \rrr{\ch{Mn^{+5}}}
    \schemestop{}
    \bigskip
    
    \item \ddd{Redox (reduction-oxidation)}: a type of chemical reaction in which the oxidation states of atoms are changed. 
      \begin{itemize}
        \item Characterized by the actual or formal transfer of electrons between chemical species.
        \item Most often one species (\rrr{the reducing agent}) undergoing \rrr{oxidation} while another species (\bbb{the oxidizing agent}) undergoes \bbb{reduction}.
      \end{itemize}
    \item Many reactions in organic chemistry are redox reactions due to changes in oxidation states but without distinct electron transfer; rather changes in \emph{electron density}.
      \begin{itemize}
        \item \rrr{Oxidation [O]}: going from a \ch{C-H} \to\ \ch{C-\pcn} bond; pulls \rrr{electron density away} from the carbon.
        \item \bbb{Reduction [H]}: going from a \ch{C-\pcn} \to\ \ch{C-H} bond; pulls \bbb{electron density towards} the carbon.
      \end{itemize}
    
    \subsection{Redox Practice Examples}\label{Redox Practice Examples}
    \begin{itemize}
      \item Examples of oxidation reactions:
        
      \medskip
      \hspace{-19pt}
      \schemestart{}
        \chemfig{-[:30]-[:-30](-[:60]H)(-H)(-[:-60]H)}
        \arrow{->[\rrr{[O]}]}
        \chemfig{-[:30]-[:-30]\pcpab{C}(-[:30,1.2]\pcnab{OH})(-[:-30]H)(-[:-90]H)}
     \qquad
        \chemfig{-[:-30]-[:30](=[:90]O)-[:-30]H}
        \arrow{->[\oxi]}
        \chemfig{-[:-30]-[:30](=[:90]OH)-[:-30]OH}
      \schemestop{}
      \bigskip
      
      \item Examples of reduction reactions:
    
      \bigskip
      \hspace{-29pt}
      \schemestart{}
        \chemfig{-[:30]-[:-30]\chembelow[6pt]{}{\pcp}-[:30]\pcnab{Cl}}
        \arrow{->[\bbb{[H]}]}
        \chemfig{-[:30]-[:-30]-[:30]H}
      \qquad
        \chemfig{-[:30]-[:-30]-[:30](=[:90]O)-[:-30]H}
        \arrow{->[\reduc]}
        \chemfig{-[:30]-[:-30]-[:30](-[:90]OH)(-[:-90]H)-[:-30]H}
      \schemestop{}
      \bigskip
    
    \end{itemize}
    
\end{itemize}

\clearpage
\section{Reduction Reactions}\label{Reduction Reactions}
\begin{itemize}
  \item[]
  \subsection{Reduction Agents}\label{Reduction Reagents}
  \begin{itemize}
    \item \ddd{Hydride}: \bbb{\ch{H-}}, a negatively charged hydrogen ion (anion); commonly used as a \bbb{strong base} that react with \rrr{weak acids}, releasing \ch{H2}. 
    \item \ch{NaBH4}: sodium borohydride.
    \item \ch{LiAlH4} (LAH): lithium aluminum hydride.
    \item Both \ch{NaBH4} and \ch{LiAlH4} are \bbb{hydride donors}; they give up a hydride to become neutral.
      \begin{itemize}
        \item Boron < aluminum in terms of size; electrons are attracted to boron to a greater degree, making it harder to remove the hydride for boron vs.\ aluminum. Thus:
          \begin{itemize}
            \item \ch{LiAlH4} is a \emph{stronger} reducing agent.
            \item \ch{NaBH4} is a \emph{weaker} reducing agent.
          \end{itemize}
      \end{itemize}
  \end{itemize}
  
  \subsection{Reduction with \ch{LiAlH4}}\label{Reaction of Alkyl Halides with LAH}
  \begin{itemize}
      \item Example with an alkyl halide:

      \bigskip
      \schemestart{}
        \chemfig{-[:30]-[:-30]-[:30]\pcpab{}-[:-30,1.2]\pcnab{Br}}
        \arrow{->[\ch{LiAlH4}][\reduc]}[0,1.2]
        \chemfig{-[:30]-[:-30]-[:30]-[:-30]H} 
        \+
        \minimal{\ch{AlH3 + LiBr}}
      \schemestop{}
      \bigskip
      
      \item Example with an epoxide:
      
      \bigskip
      \schemestart{}
        \chemfig{R-[:30,0.8]?-[:60]O-[:-60]?}
        \arrow{->[LAH]}
        \chemfig{R-[:30](<[:90]\nab{O})-[:-30]}
        \+{,,20pt}
        \chemfig{R-[:30](<:[:90]\nab{O})-[:-30]}
        \arrow{->[\ch{H2O}]}
        \chemfig{R-[:30](<[:90]OH)-[:-30]}
        \+{,,20pt}
        \chemfig{R-[:30](<:[:90]OH)-[:-30]}
      \schemestop{}
      \bigskip

      \item Remember: epoxide rings \emph{no positive charge} will have the \emph{less sterically hindered} side attacked.
      \item \ddd{Quenching}: deactivation of any unreacted reagents (adding a hydrogen source using \ch{H2O} for the negatively charged oxygen in this case).
  \end{itemize}
  

  \subsection{Reduction of Aldehydes}\label{Reduction of Aldehydes}
  \begin{itemize}
      \item Example of reduction to \ang{1} alcohol:
      
      \bigskip
      \schemestart{}
        \chemfig{R-[:30]@{dc}\pcpbe{}(=[:90]\pcnab{O})-[:-30]H}
        \arrow{->[\chemfig{LiAlH_3-@{h-}\bbb{H^{-}}}][\ch{H2O}]}[0,1.2]
        \chemfig{R-[:30](-[:90]OH)(-H)(-[:-60]H)}
      \schemestop{}
      \chemmove[dash pattern= on 1pt off 1pt]{
        \draw(h-)..controls +(130:1.5cm) and +(40:1cm).. (dc);
      }
      \bigskip
      \begin{itemize}
        \item Note: the reaction is done in two steps, but the addition of water was omitted here.
        \item The partial charge on the carbon double bonded to oxygen gives a slight \rrr{electrophilic} center for the \bbb{hydride} to attack due to the \emph{inductive effect}.
      \end{itemize}
      
      \item Same example as above, but with \ch{NaBH4} and showing the intermediate step:

      \bigskip
      \schemestart{}
        \chemfig{R-[:30](=[:90]O)-[:-30]H}
        \arrow{->[\ch{NaBH4}][\ch{H2O}]}[0,1.2]
        \chemfig{R-[:30](-[:90]\nab{O})(-H)(-[:-60]H)}
        \arrow{->[\ch{H2O}]}
        \chemfig{R-[:30](-[:90]OH)(-H)(-[:-60]H)}
      \schemestop{}
      \bigskip

    \end{itemize}

      \subsection{Reduction of Ketones}\label{Reduction of Ketones}
      \begin{itemize}
      \item Examples of generating racemic mixtures of \ang{2} alcohols:

      \medskip
      \schemestart{}
        \chemfig{R-[:30](=[:90]O)-[:-30]R'}
        \arrow{->[\ch{LiAlH4}][\ch{H2O}]}[0,1.2]
        \chemfig{R-[:30](<:[:90]O)-[:-30]R'}
        \+{,,20pt}
        \chemfig{R-[:30](<[:90]O)-[:-30]R'}
      \schemestop{}
      \bigskip
      
      \begin{itemize}
        \item Note: the above can be done with \ch{NaBH4}, it's just slower.
        \item The inductive effect is slightly increased vs.\ aldehydes, since there are now two alkyl groups pushing electron density.
      \end{itemize}
  \end{itemize}
  
  \subsection{Reduction of Esters}\label{Reduction of Esters}
  \begin{itemize}
    \item First the ester is reduced to an aldehyde, then in presence of excess it will be further reduced from an aldehyde to a \ang{1} alcohol.
      
      \bigskip
      \schemestart{}
        \chemfig{R-[:30](=[:90]O)-[:-30]OR'}
        \arrow{->[\ch{LiAlH4}][(excess)]}[0,1.2]
        \chemfig{R-[:30]@{c}(-[:90]@{o}\nab{O})(-[@{h}:-60]H)-@{or}OR'}
        \arrow{}[0,0.7]
        \chemfig{R-[:30](=[:90]O)-[:-30]H} 
        \arrow{->[(excess)]}[0,1.2]
        \hyperref[Reduction of Aldehydes]{\ulink{aldehyde}}
      \schemestop{}
      \chemmove[dash pattern= on 1pt off 1pt]{
        \draw(h)..controls +(-20:0.5cm) and +(230:0.5cm).. (or);
        \draw(o)..controls +(west:0.8cm) and +(130:0.5cm).. (c);
      }
      \bigskip

    \begin{itemize}
      \item Resonance (shared electron density due to delocalization of electrons between either oxygen) is stronger than inductive effect, so \emph{only reduction using \ch{LiAlH4}} works.
    \end{itemize}
  \end{itemize}

  \subsection{Reduction of Carboxylic Acids}\label{Reduction of Carboxylic Acids}
    \begin{itemize}
      \item Very similar to Esters, which ends up producing a \ang{1} alcohol: 
        
      \medskip
      \schemestart{}
        \chemfig{R-[:30](=[:90]O)-[:-30]OH}
        \arrow{->[\ch{LiAlH4}]}[0,1.2]
        \chemfig{R-[:30](-[:90]OH)(-H)(-[:-60]H)}
      \schemestop{}
      \bigskip

      \begin{itemize}
        \item Again, the presence of resonance will only allow the use of \ch{LiAlH4}; the first step will not occur if \ch{NaBH4} is used as it is too weak to be a reagent.
      \end{itemize}
  \end{itemize}

  \subsection{Reduction Practice Problems}\label{Reduction Practice Problems}
  \begin{itemize}
      \item[1.] No reaction, needs carbonyl (\ch{C=O}) functional group to proceed:
      
      \medskip
      \schemestart{}
        {\scriptsize\chemfig{*6(-----=)}}
        \arrow{->[\ch{LiAlH4}][or \ch{NaBH4}]}[0,1.4] 
        {\scriptsize\chemfig{*6(-----=)}}
      \schemestop{}
      \bigskip

      \item[2.] Example that contains a carbonyl group:

      \medskip
      \schemestart{}
        {\scriptsize\chemfig{*6(---(-[:30](=[:90]O)(-[:-30]H))--=)}}
        \arrow{->[\ch{LiAlH4}][\ch{H2O}]}[0,1.3] 
        {\scriptsize\chemfig{*6(---(-[:30](-[:90]OH)(-[:-90]H)(-[:-30]H))--=)}}
        \arrow{<->}
        {\scriptsize\chemfig{*6(---(-[:30](-[:-30]OH))--=)}}
      \schemestop{}
      \bigskip

      \item[3.] Example with a ketone and an acyclic ester:
      
      \medskip
      \schemestart{}
        {\scriptsize\chemfig{*6(---(-[:30](=[:90]O)-[:-30]OR)-(=O)--)}}
        \arrow{->[\ch{NaBH4}][\ch{H2O}]}[0,1.2]
        {\scriptsize\chemfig{*6(---(-[:30](=[:90]O)-[:-30]OR)-
        ((<:[:120]OH)(<[:60]H))
        --)}}
        \+{,,10pt}
        {\scriptsize\chemfig{*6(---(-[:30](=[:90]O)-[:-30]OR)-
        ((<[:120]OH)(<:[:60]H))
        --)}}
      \schemestop{}
      \bigskip
     
    \item[3b.] Same example as above but with \ch{LiAlH4} (assuming excess): 

    \medskip
    \schemestart{}
      {\scriptsize\chemfig{*6(---(-[:30](=[:90]O)-[:-30]OR)-(=O)--)}}
      \arrow{->[\ch{LiAlH4}][\ch{H2O}]}[0,1.2]
      {\scriptsize\chemfig{*6(---(-[:30]-[:-30]OH)-
      ((<:[:120]OH)(<[:60]H))
      --)}}
      \+{,,10pt}
      {\scriptsize\chemfig{*6(---(-[:30]-[:-30]OH)-
      ((<[:120]OH)(<:[:60]H))
      --)}}
    \schemestop{}
    \bigskip

    \item[4.] Example with a cyclic ester (assuming excess):

    \medskip
    \schemestart{}
      {\scriptsize\chemfig{*6(--(-[:30])(-[:-30])-O-(=[:90]O)--)}}
      \arrow{->[\ch{LiAlH4}][\ch{H2O}]}
      {\scriptsize\chemfig{*6(--(-[:30])(-[:-30])-@{o}O-[@{sbo}]
      ((-[:120]O)(-[:60]H))
      --)}}
      \arrow{->}[0,0.8]
      {\scriptsize\chemfig{*6(--(-[:30])(-[:-30])-\nab{O}-[,,,,draw=none]
      ((=[:120]O)(-[:50]H))
      --)}}
      \arrow{->}[0,0.8]
      {\scriptsize\chemfig{*6(--(-[:30])(-[:-30])-\nab{O}-[,,,,draw=none]
      ((-[:120]\nab{O})(-[:50]H)(-[:-90]H))
      --)}}
      \arrow{->[\ch{H2O}]}[0,0.8]
      {\scriptsize\chemfig{*6(--(-[:30])(-[:-30])-OH-[,,,,draw=none]
      (-[:90]OH)
      --)}}
    \schemestop{}
    \chemmove[dash pattern= on 1pt off 1pt]{
      \draw(sbo)..controls +(220:0.5cm) and +(200:0.3cm).. (o);
    }
    \bigskip
  \end{itemize}
\end{itemize}

\clearpage
\section{Oxidation Reactions}\label{Oxidation Reactions}
\begin{itemize}
    \item[]
    
    \subsection{Oxidizing Agents}\label{Oxidizing Agents}
    \begin{itemize}
        \item \emph{Peroxide reagents}: reagents that contain \ch{O-O} linkage of some kind:
        
        \medskip
      \begin{center}
        \hspace{-30pt}
        \schemestart{}
          \chemname[-50pt]{\ch{H-O-O-H}}{\ch{H2O2} hydrogen peroxide }
          \hspace{60pt}
          \chemname[-50pt]{\chemfig[atom sep=2.5em]{O=[:30]\pab{O}-[:-30]\nab{O}}}{\ch{O3} ozone}
          \hspace{60pt}
          \chemname[-50pt]{\chemfig{R-[:30](=[:90]O)-[:-30]O-O-H}}{\ch{-OOH} peroxy acid (peracid)}
        \schemestop{}
      \end{center}
        \bigskip
        
      \item \ddd{Arene substitution patterns}: IUPAC nomenclature for naming substituents other than hydrogen in relation to each other on an aromatic hydrocarbon:
        
      \medskip
      \begin{center}
      \hspace{-30pt}
      \schemestart{}
        \chemfig{*6
          ((-[:195,0.7,,,draw=none]meta)
          -(-[,0.5,,,draw=none]para)
          =(-[:-15,0.7,,,draw=none]meta)
          -(-[,0.7,,,draw=none]ortho)
          =(-R)
          -(-[,0.7,,,draw=none]ortho)
          =)
          }
      \schemestop{}
      \end{center}
      \bigskip
      
      \item \ddd{meta-Chloroperoxybenzoic acid (mCPBA)}: a strong and widely used oxidant in organic synthesis due to relative ease of handling.
      
      \medskip
      \begin{center}
      \hspace{-10pt}
      \schemestart{}
      \chemfig{*6(-=(-\ch{Cl})-=(-(=[3]O)-[:30]O-[0]OH)-=)}
      \schemestop{}
      \end{center}
      \bigskip
    \end{itemize}
  
    \subsubsection{Metal Based Oxidizing Agents}\label{Metal Based Oxidizing Agents}
    \begin{itemize}
      \item[] 

      \medskip
      \begin{center}
      \hspace{-30pt}
      \schemestart{}
        \chemname[-35pt]{\chemfig[atom sep=2.5em]
        {\ch{Cr}(=[5]O)(=[2,0.9]O)(=[-1]O)}}{\ch{CrO3} chromium trioxide}
        \hspace{90pt}
        \chemname[-35pt]{\chemfig[atom sep=2.5em]
        {\ch{Mn}(=[5]O)(=[1]O)(=[-1]O)(-[3]\nab{O})}}{\ch{MnO4-} permanganate}
        \hspace{90pt}
        \chemname[-35pt]{\chemfig[atom sep=2.5em]
        {\ch{Os}(=[5]O)(=[1]O)(=[-1]O)(=[3]O)}}{\ch{OsO4} osmium tetroxide}
      \schemestop{}
      \end{center}
      \bigskip
  
        \item \ch{KMnO4} is \emph{inexpensive}, but a very \emph{strong oxidizing} agent and is \emph{not soluble} in organic solvents.
        \begin{itemize}
          \item Keeping \ch{KMnO4} cold can help reduce activation energy, leading a reduction of the more oxidized byproduct.
          \item \ch{MnO4-} comes as \ch{KMnO4}. The anion must be created in order to be used as a reagent.
        \end{itemize}
        
      \item \ch{OsO4} is \emph{very expensive}, but a \emph{mild oxidizing} agent that is \emph{soluble} in organic solvents.
          \begin{itemize}
            \item N-metyhlmopholine N-oxide (NMO): used with \ch{OsO4} to reduce cost; it oxidizes the byproduct (\ch{Os^{+6}}) back to \ch{OsO4} in solution, allowing for reuse.
          \end{itemize}
      
    \item \ddd{Pyridinium chlorochromate (PCC)}: \ch{[C5H5NH]\plus[CrO3Cl]−} --- a \emph{mild} oxidizing reagent primarily used for \emph{selective} oxidation of alcohols to aldehyde or ketones.
      
      \medskip
      \begin{center}
      \hspace{-30pt}
      \schemestart{}
        {\footnotesize\chemfig{*6(-\pab{N}(-[,0.5,,,draw=none]H)=-=-=)}}
        \qquad
        \chemfig{\nab{O}-\ch{Cr}(=[:90]O)(=[:-90]O)-\ch{Cl}}
      \schemestop{}
      \end{center}
      \bigskip
    \end{itemize}

    \subsection{Epoxidation}\label{Epoxidation}
    \begin{itemize}
      \item Example of oxidizing a \textit{cis}-alkene to a \textit{cis}-epoxide:
      
      \medskip
      \schemestart{}
      \chemfig{(<:[:150]H)(<[:-150])=(<:[:30]H)(<[:-30])}
      \arrow{->[mCPBA]}[0,1.3]
      \chemfig{?(<:[:200]H)(<[:-110])-[:60]O-[:-60]?(<:[:-20]H)(<[:-70])}
      \schemestop{}
      \bigskip
      
      \begin{itemize}
        \item Note: epoxide rings are always \minimal{(commonly?)} made when using mCPBA\@.
        \item The epoxide ring is an example of a \emph{meso compound}, where there are chiral centers, but there is a plane of symmetry making it \emph{superimposable on its mirror image}, so no other products are made. 
      \end{itemize}

      \item Similar to example of above, but with a \textit{trans}-alkene to a \textit{trans}-epoxide:
      
      \medskip
      \schemestart{}
      \chemfig{(<:[:150]H)(<[:-150])=(<:[:30])(<[:-30]H)}
      \arrow{->[mCPBA]}[0,1.3]
      \chemfig{?(<:[:200]H)(<[:-110])-[:60]O-[:-60]?(<:[:-20])(<[:-70]H)}
      \+
      \chemfig{?(<[:-200])(<:[:110]H)-[:-60]O-[:60]?(<:[:20])(<[:70]H)}
      \schemestop{}
      \bigskip
      
      \begin{itemize}
        \item The \textit{trans}-epoxide ring is no longer a meso compound, so you must show both products; one when the ring forms on the top and one where the ring forms on the bottom.
      \end{itemize}
    \item These examples show that epoxidation reactions are \emph{concerted} reactions, since there is no mixture of products between cis and trans products; it's dependent on the starting compound.
  \end{itemize}

  \subsubsection{Epoxidation Practice Problems}
  \begin{itemize}
      \item[1.] ~ % chktex 39

      \medskip
      \schemestart{}
        {\small\chemfig{*6(---=--)}}
        \arrow{->[mCPBA]}[0,1.3]
        {\small\chemfig{*6(---(-[:90]O)-(-[:30,0.75])--)}}
      \schemestop{}
      \bigskip
      
      \item[2.] ~ % chktex 39

      \medskip
      \schemestart{}
        {\small\chemfig{*6(---=(-)--)}}
        \arrow{->[mCPBA]}[0,1.3]
        {\small\chemfig{*6(---(<[:90]O)-(<[:30,0.75])(<:[:120])--)}}
        \+{,,15pt}
        {\small\chemfig{*6(---(<:[:90]O)-(<:[:30,0.75])(<[:120])--)}}
      \schemestop{}
      \bigskip
      
  \end{itemize}

  \subsection{Trans Dihydroxylation}\label{Trans Dihydroxylation}
  \begin{itemize}
      \item \rrr{Acid catalyzed}:

      \medskip
      \schemestart{}
        \chemfig{(<:[:150]H)(<[:-150])=(<:[:30]H)(<[:-30])}
        \arrow{->[mCPBA]}[0,1.3]
        \chemfig{?(<:[:200]H)(<[:-110])-[:60]O-[:-60]?(<:[:-20]H)(<[:-70])}
        \arrow{->[\rrr{\ch{H+}}][\ch{H2O}]}
        \chemfig{?(<:[:200]H)(<[:-110])-[:60]\ch{\pab{O}H}-[:-60]?(<:[:-20]H)(<[:-70])}
      \schemestop{}
      \bigskip

      \medskip
      \schemestart{}
        \chemfig{?@{elec}(<:[:200]H)(<[:-110])-[@{sb}:60]@{op}\ch{\pab{O}H}-[:-60]?(<:[:-20]H)(<[:-70])}
        \arrow{->[\chemfig{\bbb{\ch{H2O}}@{nuc}}]}
        \chemfig{(<:[:80]H)(-[:-90]\ch{OH})(<[:120])-(-[:90]\ch{OH})(<:[:-60]H)(<[:-100])}
        \+
        \chemfig{(<[:-80])(-[:90]\ch{OH})(<:[:-120]H)-(-[:-90]\ch{OH})(<:[:60]H)(<[:100])}
      \schemestop{}
      \chemmove[dash pattern= on 1pt off 1pt]{
        \draw(nuc)..controls +(10:2cm) and +(-60:3.5cm).. (elec);
        \draw(sb)..controls +(west:0.5cm) and +(west:0.5cm).. (op);
      }
      \bigskip
      
      \begin{itemize}
        \item If both carbon centers in a product have the same configuration (S---S, R---R), then means it must have as corresponding enantiomer; R---S would be meso.
        \item Our first product is S---S, so we must show the second product, where OH comes from the top vs.\ the bottom. 
      \end{itemize}

      \item \bbb{Base catalyzed}: 
      
      \medskip
      \hspace{-20pt}
      \schemestart{}
        \chemfig{(<:[:150]H)(<[:-150])=(<:[:30]H)(<[:-30])}
        \arrow{->[mCPBA][\bbb{\ch{OH}}/\ch{H2O}]}[0,1.4]
        \chemfig{?(<:[:200]H)(<[:-110])-[:60]O-[:-60]?(<:[:-20]H)(<[:-70])}
        \arrow{->[\ch{\nab{O}H}/\ch{H2O}]}[0,1.2]
        \chemfig{(<:[:80]H)(-[:-90]\ch{OH})(<[:120])-(-[:90]\ch{OH})(<:[:-60]H)(<[:-100])}
        \+
        \chemfig{(<[:-80])(-[:90]\ch{OH})(<:[:-120]H)-(-[:-90]\ch{OH})(<:[:60]H)(<[:100])}
      \schemestop{}
      \bigskip
      
      \begin{itemize}
        \item Note: the result is the same as acid catalyzed conditions when both sides of the alkene are equally substituted.
        \item However, under \rrr{acidic conditions} then the \rrr{more substituted} side will get attacked, while under \bbb{basic conditions} then the \bbb{less substituted} side will get attacked.
      \end{itemize}
    \end{itemize}

    \medskip
    \subsubsection{Trans Hydroxylation Practice Problems}\label{Trans Hydroxylation Practice Problems}
    \begin{itemize}
      \item \textit{Trans}-alkene reactant (\rrr{acid catalyzed}):
      
      \bigskip
      \schemestart{}
        \chemfig{(<:[:150]H)(<[:-150])=(<:[:30])(<[:-30]H)}
        \arrow{->[mCPBA][\rrr{\ch{H+}}/\ch{H2O}]}[0,1.3]
        \chemfig{?(<:[:200]H)(<[:-110])-[:60]\ch{\pab{O}H}-[:-60]?(<:[:-20])(<[:-70]H)}
        \arrow{->[\ch{H2O}][]}
      \schemestop{}
      \bigskip
      
      \item \textit{Trans}-alkene reactant (\bbb{base catalyzed}):
      
      \bigskip
      \schemestart{}
        \chemfig{(<:[:150]H)(<[:-150])=(<:[:30])(<[:-30]H)}
        \arrow{->[mCPBA][\bbb{\ch{OH}}/\ch{H2O}]}[0,1.3]
        \chemfig{?(<:[:200]H)(<[:-110])-[:60]\ch{\nab{O}H}-[:-60]?(<:[:-20])(<[:-70]H)}
        \arrow{->[\ch{H2O}][]}
      \schemestop{}
      \bigskip
      
      
      \item Cyclohexene reactant:
      
      \medskip
      \schemestart{}
        {\scriptsize\chemfig{*6(---=--)}}
        \arrow{->[mCPBA][\rrr{\ch{H+}}/\ch{H2O}]}[0,1.2]
        {\scriptsize\chemfig{*6(---(-[:90]O)-(-[:30,0.75])--)}}
        \arrow{->[\rrr{\ch{H^{+}}}]}[0,0.8]
        {\scriptsize\chemfig{*6(---(-[:90]\ch{\pab{O}H})-(-[:30,0.75])--)
        }}
        \arrow{->[\ch{H2O}]}
        {\scriptsize\chemfig{*6(---(<OH)-(<:OH)--)}}
        \+{,,10pt}
        {\scriptsize\chemfig{*6(---(<:OH)-(<OH)--)}}
      \schemestop{}
      \bigskip
      
      \item 1-methylcyclohexene reactant:
      
      \medskip
      \schemestart{}
        {\footnotesize\chemfig{*6(---=(-)--)}}
        \arrow{->[mCPBA][\bbb{\ch{OH}}/\ch{H2O}]}[0,1.3]
        {\footnotesize\chemfig{*6(---(<[:90]\nab{O}H)-(<[:30,0.75])(<:[:120])--)}}
        \arrow{->[\ch{H2O}][]}
        {\footnotesize\chemfig{*6(---(<:\ch{OH})-(<:[3])(<\ch{OH})--)}}
        \+{,,15pt}
        {\footnotesize\chemfig{*6(---(<\ch{OH})-(<[3])(<:\ch{OH})--)}}
      \schemestop{}
      \bigskip
  \end{itemize}
  
  \subsection{Syn Dihydroxylation}\label{Syn Dihydroxylation}
  \begin{itemize}
      \item Using potassium permanganate on a \textit{cis}-alkene: 
      
      \bigskip
      \hspace{-10pt}
      \schemestart{}
        {\scriptsize\chemfig{(<:[:150]H)(<[:-150])=(<:[:30]H)(<[:-30])}}
        \arrow{->[\ch{KMnO4} (cold)][\bbb{\ch{OH}}/\ch{H2O}]}[0,1.3]
        \chemfig{
        @{mn}\ch{Mn}(=[@{db2}5,1.3]@{o}O)(=[1,1.3]O)(=[@{db1}-1,1.3]O)(-[3,1.3]\nab{O})
        (-[:-105,2,,,draw=none]@{al}(<:[:150]H)(<[:-150])=@{ar}(<:[:30]H)(<[:-30]))}
        \arrow{->[][]}[0,0.6]
        \chemfig{(<:[:210]H)(<[:-110])(-[:110]O-[1,1.2]Mn (-[3,1.3]\nab{O})(=[1,1.3]O))-(-[:70]O-[3,.9])(<:[:-30]H)(<[:-70])}
        \arrow{->[\ch{H2O}]}[0,0.8]
        \chemfig{(<:[:210]H)(<[:-110])(-[:90]\ch{OH})-(-[:90]\ch{OH})(<:[:-30]H)(<[:-70])}
      \schemestop{}
      \chemmove[dash pattern= on 1pt off 1pt]{
        \draw(db1)..controls +(-130:0.5cm) and +(130:1cm).. (ar);
        \draw(al)..controls +(60:0.5cm) and +(-30:0.2cm).. (o);
        \draw(db2)..controls +(130:0.8cm) and +(west:0.3cm).. (mn);
      }
      \bigskip
      
      \begin{itemize}
        \item The addition of \ch{MnO4-} is a \emph{concerted} reaction, which is then replaced by hydroxyl groups due to addition of water to create a syn product.
        \item The product is a meso compound, so it is achiral, making it have no enantiomers.
      \end{itemize}

      \item Same reaction, but on a \textit{trans}-alkene:
      
      \bigskip
      \schemestart{}
        {\scriptsize\chemfig{(<:[:150])(<[:-150]H)=(<:[:30]H)(<[:-30])}}
        \arrow{->[\ch{KMnO4} (cold)][\bbb{\ch{OH}}/\ch{H2O}]}[0,1.3]
        \chemfig{(<:[:210])(<[:-110]H)(-[:110]O-[1,1.2]Mn (-[3,1.3]\nab{O})(=[1,1.3]O))-(-[:70]O-[3,.9])(<:[:-30]H)(<[:-70])}
        \arrow{->[\ch{H2O}]}[0,0.8]
        \chemfig{(<:[:210])(<[:-110]H)(-[:90]\ch{OH})-(-[:90]\ch{OH})(<:[:-30]H)(<[:-70])}
        \+
        \chemfig{(<[:-210]H)(<:[:110])(-[:-90]\ch{OH})-(-[:-90]\ch{OH})(<:[:30]H)(<[:70])}
      \schemestop{}
      \bigskip
      
      \begin{itemize}
        \item The product is no longer chiral, giving us different products depending on which side \ch{MnO4-} attached to.
      \end{itemize}
    \end{itemize}
\end{itemize}
