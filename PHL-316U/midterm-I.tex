\documentclass[12pt,a4paper]{article}
\usepackage{inverba, apacite}

\newcommand{\userName}{Cullyn Newman} 
\newcommand{\class}{PHL 316U} 
\newcommand{\institution}{\hyperlink{home}{Portland State University}}
\newcommand{\theTitle}{Analysis of Work Under Contemporary Capitalism}
\setlength\headheight{20pt}
\setlength{\parindent}{0pt}
\pagecolor{light} 
\color{lightText} 
\title{\hypertarget{home}\theTitle\vspace*{-0.2cm}}
\author{\normalsize\userName}
\date{\vspace*{-0.5cm}\small\today}
\rfoot{}
\lfoot{}
\renewcommand{\baselinestretch}{1.618} 
\begin{document}
%%%%%%%%%%%%%%%%%%%%%%%%%%%%%%%%%%%%%%%%%%%%%%%%%%%%%%%%%%%%%%%%%%%%%
\maketitle
\begin{abstract}
    How we go about solving problems we face can often say much more about our positions than we can ourselves. An interesting approach that is fantastic at highlighting underlying reasons behind claims is done by using a variety of exploratory questions that focus on the epistemology of the party making such claims. A retrospective alteration of this approach was used by me to compare and contrast the works David Graeber, Richard Sennet, and David Harvey in order to extract their views on work under contemporary capitalism. David Harvey’s \textit{Seventeen Contradictions and the End of Capitalism} garnered the greatest amount of interest from me, so Harvey’s ideas for political praxis were used in order to lay a common foundation to build upon and compare the views of all three authors. I found that Harvey does best at describing a multitude of issues that arise under contemporary capitalism; Graeber’s work appears to align with Harvey’s, though in a more limited sense that focuses more on issue of work itself rather than capitalism as a whole; and Sennett’s work leans more towards shaping culture and a valuing of communitarianism, which is is compatible at times with Harvey, though he doesn't seem to take in to account all of Harvey's contradictions and sees a future bearing a modified form of capitalism.
\end{abstract}
\fancyhead{}
\fancyhead[R]{\hyperlink{home}{\nouppercase\leftmark}}
\fancyhead[L]{\theTitle}
\rhead{\hyperlink{home}{\thepage}}%%%%%%%%%%%%%%%%%%%%%%%%%%%%%%%%%%%%%%%%%%%%%%%%%%%%%%%%



%\begingroup
%%%%%%%%%%%%%%%%%%%%%%%%%%%%% Essay %%%%%%%%%%%%%%%%%%%%%%%%%%%%%
%\begingroup
\clearpage

There’s a very interesting phenomenon that occurs in regards to the formation of opinions. More often than not, we tend to identify with and become unjustly attached to the ideas we encounter. One way expose such beliefs is through an epistemological approach lead by socratic questioning. Now, this essay is not exactly an epistemological study, though the methods of investigation I’ve developed in past have been so valuable for uncovering core arguments and sources of disagreements that I can’t seem to stop applying them to any method of inquiry or analytical review. Normally such methods are used in open conversations, free of judgment of assessment, and follow a dynamic, or more explorative, approach of continuous and thoughtful questions that ultimately lead to sources debatable observations. 
\vspace*{10pt}

The same dynamic approach doesn’t work as well with written work, as observations, ideas, and opinions can change rapidly or need a certain level of nuance to be cleared up that is not always available. Though, many of the same methods can be used retrospectively with slight alterations, allowing for increased chance at accurate understanding of true positions that the authors hold. The idea is the same: question the claims made by the authors, find the evidence of their claim using the author's own work, then continue to iterate until you can’t find sufficient evidence, or until you arrive at the core claims being discussed. I used this retrospective approach in this essay; I have taken the ideas for political praxis that David Harvey lays out in the epilogue of his book, \textit{Seventeen Contradictions and the End of Capitalism}, and compared them with potential solutions David Graeber and Richard Sennett describe in their works. An epistemological analysis of their potential solutions, in relation to Harvey’s mandates, led me to points of agreement and disagreement between the three, which allowed me to compare and contrast the compatibility of their core arguments. 
\vspace*{10pt}

The remaining portion of this essay will examine three of Harvey's seventeen suggested mandates, which are derived from his seventeen contradictions~\cite{con}. I attempted to focus on contradictions that have greater significance in regards to our relation to work under contemporary capitalism, though ultimately each contradiction interacts in an increasingly complex global system. Solutions for how we value work is most certainly not an isolated issue.

\textbf{Contradiction 2: The Social Value of Labour and Its Representation
by Money}
\begin{quotation} {\color{G-Moon}
    \noindent "A means of exchange is created that facilitates the circulation of goods and services but limits or excludes the capacity of private individuals to accumulate money as a form of social power." \cite{con}}
\end{quotation}
Here both Graeber and Sennett have less to say on means of exchange itself, though it’s important to include as it’s foundational to how we view work itself, and even Harvey calls his first seven contradictions “foundational” \cite{con}. Contradiction 1 highlights use value vs. exchange value, though it was omitted as it serves as an answer to the question raised about the difference between use and exchange, as they are meant roughly equal each other using a abstracted representation called money, but we see that, “exchange value considerations increasingly dominate the use value aspects of social life.” \cite{con}  
\vspace*{10pt}

Graeber does display several arguments that support Harvey’s position, emphasizing the disconnect between values and exchange, stating, “this is exactly what is missing in the domain of “values”---it might sometimes be possible to argue that one work of art is more beautiful, or one religious devotee more pious than another, but it would be bizarre to ask how much more.” \cite{bs}
\vspace*{10pt}

Sennett does touch on this problem briefly through an indirect analysis that focuses more on the issue of merit when he speaks of usefulness, “usefulness itself is more than a utilitarian exchange.” \cite{new} His view here isn’t exactly at odds with Harvey or Graeber, though here he shows how he leans more towards our attitude towards work as a culture being the culprit. Essentially he identifies the same problem, but in a way that Harvey revisits later in much greater depth. Unlike Harvey, Sennett seems implicate the “cult of meritocracy” \cite{new} and how the culture behind it is the issue, without much of a connection with this core contradiction between value and exchange.

More could easily be explored in regards to how each defines value. Harvey's mandate shows his foundational position quite well, questioning what alternatives to a means of exchange would quickly reveal even more. While further investigation would be fruitful, it's best to move onto other areas where the discussion overlaps. 

\textbf{Contradiction 5: Capital and Labour}
\begin{quotation} {\color{G-Moon}
    \noindent "The class opposition between capital and labour is dissolved into associated producers freely deciding on what, how and when they will produce in collaboration with other associations regarding the fulfillment of common social needs." \cite{con}}
\end{quotation}
This mandate highlights the key points a substantial problem---one in which all three authors must address when discussing work. "The commodification of labour power is the only way to solve a seemingly intractable contradiction within the circulation of capital." \cite{con} The mismatch between value and exchange from contradiction 2 amplifies this problem. "The effect is to transform social labour---the labour we do for others---into alienated social labour." \cite{con} 
\vspace*{10pt}

This conclusion is certainly echoed by Graeber, nearly his entire thesis is how work is near useless in many cases, alienated from actual productiveness. Sennett address this problem through a suggested solution of labour sharing, "A person is continually in work, long-term. This avoids the light-switch anxiety of short-term contracts---now I’m engaged, now I’m redundant." \cite{new} Here this shows how he too supports Harvey's mandate, claiming we need producers more able to freely decide what they would like be apart of. The difference again highlights how Sennett see's current issues more due to a narrative problem fixed through social interventions like unions that are in charge of solving the problem.

Both Sennett and Graeber suggest universal basic income (UBI) as means to fight this problem. "What Basic Income ultimately proposes is to detach livelihood from work." \cite{bs}  Harvey doesn't outright suggest UBI---these mandates are the only real policy ideas he encourages---but he does often referenc the needs to detach work from livelyhood, which a major goal of UBI. Even in this mandate, both Graeber and Harvey clearly sees that the opposition between capital and labour needs to be dissolved. After Sennett's introduction to a basic income, he state states, "insecurity is not just an unwanted consequence of upheavals in markets; rather, insecurity is programmed into the new institutional model." \cite{new} Here it's quite obvious Sennett too sees the need to dissolve capital from labour, though again, he seems to be implicating the institutional models rather than the forces that pushed the institutional models to adopt such behaviors.

\textbf{Contradiction 8: Technology, Work and Human Disposability}
\begin{quotation} {\color{G-Moon}
    \noindent "New technologies and organizational forms are created that lighten the load of all forms of social labour, dissolve unnecessary distinctions in technical divisions of labour, liberate time for free individual and collective activities, and diminish the ecological footprint of human activities." \cite{con}}
\end{quotation}

This contradiction and mandate is where the three authors align the most. All three mention the use of new technologies and how they continue to change our relationship to work. The question, originally posed by Brian Arthur in the \textit{The Nature of Technology}, and  referenced by Harvey, “who gains from the creation and who bears the brunt of the destruction” \cite{nature} is at the heart of nearly all ever-looming contemporary issues created though new technologies. 
\vspace*{10pt}

Sennett addresses this in regards to uselessness, where he talks of both age and technology's role in uselessness, "three forces shape the specter of uselessness as a modern threat: the global labor supply, automation, and the management of ageing." \cite{new} Again, there is still a disconnect here, his overall analysis of the issues seem culture-centric; Sennett often ties the effects into other sociopolitical problems rather than the underlying issues, such as, "the specter of uselessness here intersects with the fear of foreigners, which, beneath its crust of simple ethnic or race prejudice, is inflected with the anxiety that foreigners may be better armed for the tasks of survival. That anxiety has a certain basis in reality.” \cite{new} 
\vspace*{10pt}

Graeber's focus sides more on the lack of dignity that arises due to bullshit jobs. "The more the economy becomes a matter of the mere distribution of loot, the more inefficiency and unnecessary chains of command actually make sense, since these are the forms of organization best suited to soaking up as much of that loot as possible.” \cite{bs} Essentially, Graeber is claiming that technology changes the relationship between labor and value, and the lack of dignity and bullshit is the obvious product. An overall analysis and comparison seems reveal that he too implicates culture to a substantial degree, "the answer clearly isn't economic: it's moral and political. The ruling class has figured out that a happy and productive population with free time on their hands is a mortal danger." \cite{bs} However, an overall analysis does show a clear support for the labor theory of value, which more closely aligns with Harvey. 
\vspace*{10pt}

The similarties between the authors after this point is less well analyzed by myself, in relation to Harvey's contradictions, and seem to diverge more. Sennett continues to emphasizes the value of craftmanship, which he defines broadly as, "desire to do something well for its own sake." \cite{craft} He also cointues to implicate culture and institutions when elaborates on the current state of craftmanship, "The first trouble appears in the attempts of institutions to motivate people to work well." \cite{craft} He questions the development and measurement of quality, though again, does not seem implicate underlying forces causes the changes. More analysis of Sennett's view craftmanship needs to be done before more is discussed, though the key point is the what he is clearly emphasizing, i.e., culture. 

\textit{\textbf{On relation to my own experience}}\\
A portion of this essay prompt asked for a relation to personal experience. My own experience continues to demonstrate that relating personal experiences, especially in regrads to analytical exercises, obstructs many efforts of accurate---or at least objective---attempts at reconstruction of observed facts. Now, this is not to say I remove my personal experiences from analysis, I too am easily influenced by my own subjectivity---it's nearly impossible not to. Instead, I attempt to minimized cognitive fallacies by being aware of and avoiding if possible.

The epistemological approach I mentioned is the best way I've found to attempt to limit subjectivity from playing a dominant role. While I was not exactly explicit in how I used this method, I did use it to find what I think are close representations of the authors comparative positions relative to other. Any opportunity to exercise this method improves not just understanding of philosphy, but understanding in all of scientific pursuits I wish to be apart of. I hope the use of Harvey's mandates, which were not in the reccomended readings, demonstrates the involvement and reflection of the general approach I used, i.e., socratic questioning of solutions to navigate to core arguments made by the authors. 





%\endgroup
%%%%%%%%%%%%%%%%%%%%%%%%%%%%% Essay %%%%%%%%%%%%%%%%%%%%%%%%%%%%%

\clearpage
\bibliographystyle{apacite}
\bibliography{citations.bib}
%\endgroup
\end{document}