% chktex-file 36
% chktex-file 8

\chapter{22: Reactions of Aromatic Compounds}\label{22: Reactions of Aromatic Compounds}

\section{Electrophilic Aromatic Substitution}\label{Electrophilic Aromatic Substitution}
\begin{itemize}
  \item \ddd{Electrophilic Aromatic Substitution (EAS)}: a reaction where an atom (usually hydrogen) is replaced by an \rrr{electrophile} while the product retains aromaticity. 
  \item The electrophile typically used:
    \begin{itemize}
      \item \rrr{\ch{Br^{+}}} or \rrr{\ch{Cl^{+}}} (via Lewis acid catalyst) \smallskip
      \item Nitronium ion: \rrr{\ch{NO2^{+}}} \(\rightarrow \) \chemfig{O=\pab{N}=O}
      \item \rrr{\ch{SO3H^{+}}}
      \item \ang{2} or \ang{3} \rrr{\ch{R^{+}}} (not \ang{1})
      \item Acylium ion: \rrr{\ch{RCOX}}
      
      \medskip
      \schemestart{}
        \chemfig{R-[:30](=[:90]O)-[:-50,0.5,,,draw=none]\pab{}} 
        \arrow{<->}
        \chemfig{R-~\pab{O}}
      \schemestop{}
      \bigskip
    \end{itemize}
  
  \item \ddd{Arenium ion}: aka \(\sigma \)-complex (sigma complex); the reactive cyclohexadienyl cation that appears as a \rrr{reactive intermediate} for EAS\@.
  
  \bigskip
  \begin{center}
  \hspace{-30pt}
  \schemestart{}
    {\footnotesize\chemfig{*6(=-=-(-[1]H)(-[3]E)-(-[4,0.3,,,draw=none]\pab{})-)}}
    \arrow{<->}
    {\footnotesize\chemfig{*6(-(-[:-90,0.7,,,draw=none]\pab{})-=-(-[1]H)(-[3]E)-=)}}
    \arrow{<->}
    {\footnotesize\chemfig{*6(-=-(-[0,0.3,,,draw=none]\pab{})-(-[1]H)(-[3]E)-=)}}
  \schemestop{}
  \medskip

  \item Note: the hydrogen is then removed with a \bbb{base} in order to regain aromaticity.
  \end{center}
  
  \subsection{Halogenation of Benzene}\label{Halogenation of Benzene}
  \begin{itemize}
    \item Benzene can be chlorinated or brominated in the presence of a \rrr{Lewis acid catalyst}, typically ferric (iron III) chloride (\rrr{\ch{FeCl3}}) or ferric bromide (\rrr{\ch{FeBr3}}).
      \begin{itemize}
        \item The catalyst reacts with \ch{X_2} to form the strong electrophile; the catalyst regenerated by end of the reaction. 
        \item \ch{Fe.} may be used instead if used in the presence of \ch{Cl2} or \ch{Br2} since it will form the catalyst. 
        \item Aluminum chloride (\rrr{\ch{AlCl3}}) can be used as well, cost is only real factor.
      \end{itemize}

      \bigskip
      \begin{center}
      \hspace{-30pt}
      \schemestart{}
        {\footnotesize\chemfig{*6(-=-=-=)}}
        \arrow{->[\ch{Cl2}][\ch{FeBr3}]}
        {\footnotesize\chemfig{*6(-=-=(-Cl)-=)}}
      \schemestop{}
      \bigskip
      \end{center}

  \end{itemize}
  
  \subsection{Nitration of Benzene}\label{Nitration of Benzene}
  \begin{itemize}
    \item In benzene nitration, the electrophile is the \rrr{nitronium ion, \ch{NO2^{+}}}; it's formed by the protonation of nitric acid by the stronger sulfuric acid (\ch{H2SO4}), followed by loss of \ch{H2O}.
    \item Sulfuric acid is regenerated and recycled in the reaction.
    
    \bigskip
      \begin{center}
      \hspace{-30pt}
      \schemestart{}
        {\footnotesize\chemfig{*6(-=-=-=)}}
        \arrow{->[\ch{HNO3}][\ch{H2SO4}]}[,1.2]
        {\footnotesize\chemfig{*6(-=-=(-\ch{NO2})-=)}}
      \schemestop{}
      \end{center}
    \bigskip

    \item Once the nitro group is added, then a reduction step using Zinc (Zn) or Tin (Sn) and hydrochloric acid is done to make an aniline (benzene with \ch{NH2}).
    \item 
  \end{itemize}
  
  \subsection{Sulfonation of Benzene}\label{Sulfonation of Benzene}
  \begin{itemize}
    \item In this reaction sulfuric acid protonates sulfur trioxide to form the \rrr{electrophilic sulfur-containing compound, \ch{SO3H^{+}}}. 
    \item Sulfuric acid is regenerated and recycled in the reaction.

  \bigskip
    \begin{center}
    \hspace{-30pt}
    \schemestart{}
      {\footnotesize\chemfig{*6(-=-=-=)}}
      \arrow{->[\ch{SO3}][\ch{H2SO4}]}[,1.2]
      {\footnotesize\chemfig{*6(-=-=(-\ch{SO3H})-=)}}
    \schemestop{}
    \end{center}
  \bigskip

  \end{itemize}
  
  \subsection{Friedel-Crafts Alkylation of Benzene}\label{Friedel-Crafts Alkylation of Benzene}
  \begin{itemize}
    \item \ddd{Friedel-Crafts alkylation (F.C. alkylation)}: alkylation of an aromatic ring with an alkyl halide using a strong Lewis acid catalyst, typically \rrr{aluminum chloride}, \rrr{ferric chloride}, but ferric bromide is also possible, similar to \hyperref[Halogenation of Benzene]{\ulink{halogenation of benzene}}.
    \item F.C alkylation reactions only work with methyl or ethyl groups; will not work correctly for longer groups, e.g., addition of a propyl alkyl becomes isopropyl substituent. 
    \item Specific structure of the electrophile can vary due to carbocation stability trends.
      \begin{itemize}
        \item \emph{Case 1 (MeX or EtX)}: a methyl or \ang{1}-carbocation cannot from and no rearrangements are possible; MeCl and EtCl complex with \ch{AlCl3} is the elctrophile.
        \item \emph{Case 2 (\ang{1} RX)}: the \ang{1} RCl complex with \ch{AlCl3} loses \bbb{\ch{AlCl4^{-}}} during the rearrangement, creating a \ang{2} or \ang{3} \rrr{\ch{R^{+}}}.
        \item\emph{Case 3 (\ang{2} or \ang{3} RX)}: the \ang{2} or \ang{3} RCl complex with \ch{AlCl3} loses \bbb{\ch{AlCl4^{-}}} directly creating the \ang{2} or \ang{3} \rrr{\ch{R^{+}}} (with rearrangement of \ang{2} \rrr{\ch{R^{+}}} occurring, if feasible)
      \end{itemize}
    \item The Lewis acid catalyst is regenerated and recycled during the reaction with the addition of \ch{HCl} byproduct much like halogenation.

    \bigskip
      \begin{center}
      \hspace{-30pt}
      \schemestart{}
        {\footnotesize\chemfig{*6(-=-=-=)}}
        \arrow{->[\ch{RCl}][\ch{AlCl3}]}[,1.2]
        {\footnotesize\chemfig{*6(-=-=(-\ch{R})-=)}}
      \schemestop{}
      \end{center}
    \bigskip

  \end{itemize}
  
  \subsection{Friedel-Crafts Acylation of Benzene}\label{Friedel-Crafts Acylation of Benzene}
  \begin{itemize}
      \item \ddd{Friedel-Crafts acylation}: when a new carbon-carbon bond is formed between a benzene carbon and a carbonyl carbon to create a ketone.
      \item Acylation is similar to alkylation, except:
        \begin{itemize}
          \item the electrophile is an \rrr{acylium ion} derived from an acid chloride;
          \item rearrangements will not occur because of the stability of the resonance-stabilized acylium ion.
        \end{itemize}

      \bigskip
        \begin{center}
        \hspace{-30pt}
        \schemestart{}
          {\footnotesize\chemfig{*6(-=-=-=)}}
          \arrow{->[\ch{RCOCl}][\ch{AlCl3}]}[,1.3]
          {\footnotesize\chemfig{*6(-=-=(-(=[:140]O)-[:30]\ch{R})-=)}}
        \schemestop{}
        \end{center}
      \bigskip

    \item Can be used to introduce more carbons to the alkyl group than alkylation alone can with later reduction techniques to remove the oxygen.
        \begin{itemize}
          \item \ddd{Clemmensen reduction}: zinc in presence of mercury (Zn(Hg)) with concentrated hydrochloric acid can be used in the reduction step.
          \item \ddd{Wolf-Kishner reduction}: hydrazine (\ch{N2H4}; \ch{NH2-NH2}) in the presence of a base and heat can also be used for reduction.
          \item Differences in pathways depends on substituents and their interaction with various reduction agents.
        \end{itemize}
    \item \emph{Intramolecular} F.C. acylation reactions can also occur to create \emph{fused rings}.
  \end{itemize}

  \subsection{EAS Reactions of Disubstituted Benzenes}\label{EAS reactions of Disubstituted Benzenes}
  \begin{itemize}
      \item Agents and their products remain the same as monosubstituted benzenes, but the location of the new group can vary.
      \item When there are two (or more) substituents, then the \bbb{strongest activator} (the strongest electron donor) determines where the incoming electrophile reacts.
      \item \bbb{EDG push} electron density toward the benzene ring, making it more \bbb{electron rich}. 
        \begin{itemize}
          \item \bbb{\ch{-NH2}, \ch{-OH}, \ch{-OR}, \ch{-R}, \ch{-NHR}, \ch{-NR2}}
          \item \bbb{EDG} are \bbb{ortho/para} directing, with para preference increasing with increasing steric hindrance; small groups are approximately equal with slight ortho preference due to charge separation distance.
        \end{itemize}
      \item \rrr{EWG pull} electron density away from the benzene ring, making it \rrr{electron poor}, reducing electrophilic substitution tendency.
        \begin{itemize}
          \item \rrr{\ch{-COOR}, \ch{-CONH2}, \ch{-CONR2}, \ch{-CN}, \ch{-NO2}, \ch{-COOH}, \ch{-CHO}, \ch{-COR}}
          \item \rrr{EWG} are \rrr{meta} directing, as the ortho/para positions remain more \rrr{electron poor}.  
        \end{itemize}
      
  \end{itemize}
  
\end{itemize}

\section{Functional Group Interconversion}\label{Functional Group Interconversion}
\begin{itemize}
  \item [] 
  
  \subsection{Halogenation of Alkyl Benzenes}\label{Halogenation of Alkyl Benzenes}
  \begin{itemize}
    \item 
  \end{itemize}

  \subsection{Oxidation of Alkyl Benzenes}\label{Oxidation of Alkyl Benzenes}
  \begin{itemize}
      \item 
  \end{itemize}
  
  \subsection{Reduction of Aryl Ketones}\label{Reduction of Aryl Ketones}
  \begin{itemize}
      \item 
  \end{itemize}

  \subsection{Reduction of Aryl Nitro Groups}\label{Reduction of Aryl Nitro Groups}
  \begin{itemize}
      \item 
  \end{itemize}
\end{itemize}

\section{Nucleophilic Aromatic Substitution}\label{Nucleophilic Aromatic Substitution}
\begin{itemize}
    \item 
\end{itemize}
