\chapter{15: Alkynes}\label{15: Alkynes}
\section{Alkynes Basics/Review}\label{Alkynes Basics/Review}
\begin{itemize}
  \item \ddd{Alkynes}: an unsaturated hydrocarbon containing at least one carbon—carbon triple bond.
    \begin{itemize}
      \item Contains two unhybridized p-orbitals (sp) on each carbon.
      
      \medskip
      \schemestart{}
        \chemfig{\cadp{H}-\cadn{C}~[,,,,bbb]\cadn{C}-\cadp{H}}
      \schemestop{}

      \item Large amount of \(s\) character (\bbb{sp: 50\% \(s\)}) makes the hydrogens very acidic and the carbons very basic. 
      \item 1 mole of base takes off 1 hydrogen (1 equivalent, equiv, eq), which can be done twice to take off both (2 equivalent).  
      \item Traditionally known as acetylenes, though acetylene referrers specifically to \ch{C2H2} (ethyne)
    \end{itemize}
    
  \subsection{Acetylides}\label{Acetylides}
  \item \ddd{Acetylide}: an alkyne with a negatively charged carbon on one end, e.g.,
    \begin{itemize}
      \item[]
      
    \medskip
    \schemestart{}
      \chemfig{H-C~C-H}
      \arrow{->[\bbb{:B}]}
      \chemfig{H-\bbb{C}~[,,,,bbb]\bbb{C^\ani}}
      \+ 
      \rrr{\chemabove[3pt]{BH}{\cat}}
    \schemestop{}
    \medskip
    
    \item \bbb{:B} is a name of generic base, typical bases used:
      \begin{itemize}
        \item \bbb{\ch{NaNH2}} \to~sodamide, or \bbb{\ch{NaH}} \to~sodium hydride.
      \end{itemize}
    \item Acetylide acts as a \emph{nucleophile} when it reacts with an epoxide, \ang{1}-halide, or a methyl-halide.
    
    \begin{center}
      \hspace{-30pt}
      \medskip
      \schemestart{}
        \chemname[-50pt]{\chemfig{?-[:60]O-[:-60]?}}{epoxide}
        \qquad
        \chemname[-50pt]{\chemfig{-[:30]\chemabove[3pt]{}{\ang{1}}-[:-30]X}}{\ang{1}-halide}
        \qquad
        \chemname[-50pt]{\chemfig{CH3-X}}{metyhl-halide}
      \schemestop{}
      \bigskip
    \end{center}
      \begin{itemize}
        \item Is a \(S_N2\) reaction.
        \item Can be used for chain extension, e.g.,

        {\footnotesize
        \medskip
        \schemestart{}
          \chemfig{-[:-30]-~\bbb{C}@{h}\rrr{H}}
          \arrow{->[\chemfig{@{b}\bbb{\ch{NaH2}}}][\chemfig{-[:30,.8]@{cc}-[@{sb}:-30,.8]Br@{br}}]}[0,1.1]
          \chemfig{-[:-30]-~\bbb{C^\ani}@{c}}
          \arrow{->}
          \chemfig{-[:-30]-~--[:-30]}
        \schemestop{}
        \bigskip
        \chemmove[dash pattern= on 1pt off 1pt]{
          \draw(h)..controls +(north:1cm) and +(north:1cm).. (b);
          \draw(c)..controls +(south:0.8cm) and +(250:2cm).. (cc);
          \draw(sb)..controls +(south:0.5cm) and +(230:0.5cm).. (br);
          }
        }

        \item Example two:
        
        {\footnotesize
        \medskip
        \schemestart{}
          \chemfig{-[:-30]-~\bbb{C}\rrr{H}}
          \arrow{->[\bbb{NaH}][\chemfig{?@{b2}-[:60]O-[:-60.5]?(-[:-30,.5])}]}
          \chemfig{-[:-30]-~--[:30](-[:90]OH)-[:-30]}
        \schemestop{}
        \bigskip
        \chemmove[dash pattern= on 1pt off 1pt]{
          \draw(c)..controls +(south:1cm) and +(110:3cm).. (b2);}
        }
      \end{itemize}

      \newpage
      \item Acetylide acts as a \emph{base} when it reacts with \ang{2}-halide or a \ang{3}-halide due to steric crowding.
        \begin{itemize}
          \item Is an  E2 reaction.
        \end{itemize}
      
      \medskip
      \schemestart{}
        \chemfig{-~}
        \arrow{->[\bbb{NaH}][\tiny\chemfig{*6(---(-I)---)}]}
        \chemfig{-~@{bc}\bbb{C^\ani}}
        \arrow{->}
        {\scriptsize\chemfig{*6(---(-[@{sb}]I@{i})=(-@{kh}H)--)}}
        \arrow{->}
        {\scriptsize\chemfig{*6(---=--)}}
      \schemestop{}
      \chemmove[dash pattern= on 1pt off 1pt]{
        \draw(bc)..controls +(north:1cm) and +(west:1cm).. (kh);
        \draw(sb)..controls +(south:0.5cm) and +(south:0.5cm).. (i);
        }
      \bigskip
      
      \medskip
      \schemestart{}
        \chemfig{-~}
        \arrow{->[\bbb{NaH}][{\tiny\chemfig{*6(---(-[:45])(-[:-15]I)---)}}]}
        \chemfig{-~\bbb{C^\ani}}
        \arrow(--[braces]){->}
        \chemname[-40pt]{{\scriptsize\chemfig{*6(---(=)---)}}}{Hoffman}
        \+
        \chemname[-40pt]{{\scriptsize\chemfig{*6(---(-)=--)}}}{Zaitsev}
      \schemestop{}
      \bigskip
      
      \item The Zaitsev product is more likely to be the major product due to the thermodynamics (more energetically stable) than the Hofmann product; a bulkier base would likely increase the Hofmann product. 
      
      \subsection{Preparation of Alkynes}\label{Preparation of Alkynes}
        \item Introductory example of a geminal dihalide going through two E2 reactions to generate an alkyne:
        
        \medskip
        \schemestart{}
          \chemfig{R-(-[@{cbr}:90]@{br1}Br)(-[:-90]Br)-[@{sb}](-[:90]H)(-[:-90]@{h1}H)-R'}
          \arrow{->[\ch{NaNH2}][(2 eq.)]}[0,1.2]
          \chemfig{(-[:150]R)(-[@{cbr2}:-150]@{br2}Br)=[@{db}](-[:30]@{h2}H)(-[:-30]R')}
          \arrow{->[\ch{NaH2}]}
          \chemfig{R-~-R'}
        \schemestop{}
        \chemmove[dash pattern= on 1pt off 1pt]{
          \draw(h1)..controls +(170:0.5cm) and +(280:0.5cm).. (sb);
          \draw(cbr)..controls +(west:0.5cm) and +(west:0.5cm).. (br1);
          \draw(h2)..controls +(140:0.8cm) and +(90:0.5cm).. (db);
          \draw(cbr2)..controls +(-60:0.8cm) and +(-20:0.5cm).. (br2);
          }
        \bigskip
        
        \begin{itemize}
          \item Related example with a vicinal dihalide (halide on adjacent carbons):
          
          \medskip
          \schemestart{}
            \chemfig{R-(-[:90]Br)(-[:-90]@{h3}H)-[@{sb1}](-[@{cbr3}:90]@{br3}Br)(-[:-90]H)-R'}
            \arrow{->[\ch{NaNH2}][(2 eq.)]}[0,1.2]
            \chemfig{(-[:150]R)(-[:-150]@{h4}H)=[@{db}](-[@{cbr4}:30]@{br4}Br)(-[:-30]R')}
            \arrow{->[\ch{NaH2}]}
            \chemfig{R-~-R'}
          \schemestop{}
          \chemmove[dash pattern= on 1pt off 1pt]{
            \draw(h3)..controls +(-20:0.5cm) and +(280:0.5cm).. (sb1);
            \draw(cbr3)..controls +(west:0.5cm) and +(west:0.5cm).. (br3);
            \draw(h4)..controls +(-60:0.8cm) and +(-90:0.5cm).. (db);
            \draw(cbr4)..controls +(130:0.8cm) and +(150:0.5cm).. (br4);
            }
          \bigskip
        \end{itemize}

      \subsection{Regioselectivity and Stereospecificity of E2 Reactions}
        \item \ddd{Regiochemistry}: when a chemical reaction is said to produce two different regiochemical outcomes.
            \begin{itemize}
                \item \ddd{Regiochemical}: preference of chemical bonding or breaking \emph{direction}. 
            \end{itemize}
        \item \ddd{Regioselective}: when there is a preference in products of a regiochemical reaction.
        \item \ddd{Zaitsev product}: name of the \emph{more substituted} alkene that is generally observed to be the major product.
        \item \ddd{Hofmann product}: name of the \emph{less substituted} alkene.
        \item There ratio between the Zaitsev and Hofmann product is dependent on a number of factors and often difficult to predict.
            \begin{itemize}
                \item Steric hindrance of the base often plays a major role, often increasing the Hofmann product.
                \item The outcome of E2 reactions can often be carefully \emph{ controlled by choosing the base}, despite difficultly in overall prediction.
            \end{itemize}
        \item \ddd{Stereospecific}: when the stereoisomeric product of the E2 process depends on the configuration of the starting stereoisomeric substrate.
            \begin{itemize}
                \item The stereospecificity is only relevant when the \emph{\(\beta \) position has only one proton}.
            \end{itemize}
        \item \ddd{Stereoselective}: when the substrate itself is not necessarily stereoisomeric, but can yield two stereoisomeric products, one of which that often has a higher yield.
            \begin{itemize}
                \item Occurs when there are \emph{more than one proton in the \(\beta \) position}.
            \end{itemize}
        \item \ddd{Coplanar}: when the proton in the \(\beta \) position, the leaving group, and the two carbons atoms that form a double bond lie on the same plane.
            \begin{itemize}
                \item Often leads only one stereoisomer product being formed, thus more often stereospecific.
                    \begin{itemize}
                        \item If it stereoselective, then usually \textit{trans} conformation is favored.
                    \end{itemize}
                \item \ddd{Periplanar}: when the proton and leaving group are \emph{nearly} coplanar; often is used in place of coplanar to incorporate both situations.
            \end{itemize}
        \item \ddd{Syn-coplanar}: when the proton and the leaving group are \emph{ eclipsed} in a coplanar conformation.
            \begin{itemize}
                \item Elimination in this state involves a higher energy transition state due to eclipsed geometry, and is \emph{slower} than anti-coplanar arrangement.
            \end{itemize}
        \item \ddd{Anti-coplanar}: when the proton and the leaving group are \emph{staggered} in a coplanar conformation.
            \begin{itemize}
                \item Elimination in this state involves a lower energy transition state due to staggered geometry, leading to a \emph{faster} reaction relative to syn-coplanar.
            \end{itemize}
    \end{itemize}
\end{itemize}

\clearpage
\section{Alkyne Nomenclature}\label{Alkyne Nomenclature}
\begin{itemize}
  \item[]
  \subsection{Basic Alkyne Nomenclature}\label{Basic Alkyne Nomenclature}
  \begin{itemize}
      \item All other basic IUPAC rules still apply; the suffix is \emph{``yne''}.
      \item Find the longest carbon chain that includes both carbons of the triple bond.
      \item Number the longest chain starting at the end closest to the triple bond.
      \item Compounds with > 1 triple bond are called diynes, triynes, etc.
      \item Enynes are compounds that contain both double and triple bonds, and enynols also contain an alcohol.
      \item The functional groups have their assigned carbon numbers written directly before their name.
  \end{itemize}
  \subsection{Priority Rules}\label{Priority Rules}
  \begin{itemize}
      \item More functional groups will change the as we get introduced to more, but for now we only need to know the following (somewhat follows electronegativity):
      \begin{enumerate}
          \item Alcohol
          \item Alkenes
          \item Alkynes
          \item Alkyls
          \item Halogens
        \end{enumerate}
      \item Preference for naming order if multiple functional groups are present in the same chain: OH + DB + TB > OH + DB > OH + TB.
      \item When alkynes are substituents, then they are named as \(k\)-alkynyl. (\(k=\) locant)
      \item Example containing all three:
      
      \medskip
      \schemestart{}
        \chemname{\chemfig{9-8~7-6-[:-30]5=[:30]4(-[:90]-[:150]-[:-150]-[:150])-[:-30]3(<[:-90]OH)-[:30]2-[:-30]1}}{(\ttt{3R,\fff{4E}})-butylnon-4-en-7-yn-3-ol}
      \schemestop{}
      \bigskip
  \end{itemize}
\end{itemize}

\clearpage
\section{Addition Reactions of Alkynes}\label{Addition Reactions of Alkynes}
\begin{itemize}
  \item[]
  \subsection{Hydrohalogenation of Alkynes}\label{Hydrohalogenation of Alkynes}
  \begin{itemize}
    \item Introductory example:
    
    \medskip
    \schemestart{}
      \chemfig{R-~-H}
      \arrow{->[\rrr{H}-\bbb{Br}]}
      \chemfig{R-\chemabove[6pt]{}{\cat}=(-[:30]\rrr{H})-[:-30]H}
      \arrow{->[\bbb{\ch{Br-}}]}
      \chemname[-30pt]{\chemfig{R-[:-30](-[:210]\bbb{Br})=(-[:30]H)-[:-30]H}}{vinyl halide}
    \schemestop{}

    \begin{itemize}
      \item Is \hyperref[Regioselective]{\ulink{Regioselective}}
      \item Follows \hyperref[Markovnikov's Rule]{\ulink{Markovnikov's Rule}}
      \item This reaction can be done again (2 equiv):
      
      \hspace{-20pt}
      \medskip
      \schemestart{}
        \chemfig{R-[:-30](-[:210]Br)=(-[:30]H)-[:-30]H}
        \arrow{->[\rrr{H}-\bbb{Br}]}
        \chemfig{R-[:-30]\chemabove[6pt]{}{\cat}(-[:210]Br)-(-H)(-[:45]\rrr{H})-[:-45]H}
        \arrow{->[\bbb{\ch{Br-}}]}
        \chemname[-15pt]{\chemfig{R-[:-30](-[:-90]\bbb{Br})(-[:210]Br)-(-H)(-[:45]H)-[:-45]H}}{geminal halide}
      \schemestop{}
      \medskip
      
      \item Geminal halide: a carbon that contains two halides.
      \item Related practice problem that generates chiral carbon due to use of different reagents for each equivalence (needs to be verified still, I might be wrong):
      
      \medskip
      \schemestart{}
        \chemfig{-[:-30]-~-H}
        \arrow{->[HBr][HI]}[0,0.8]
        \chemfig{--[:-30](-[:-130]\bbb{Br})=(-[:30]H)-[:-30]H}
      \schemestop{}
      \medskip

      \medskip
      \schemestart{}
        \dots
        \arrow(--[braces]){->[H\bbb{I}]}[0,.8]
        \chemname[-25pt]{\chemfig{->:[:30](-[:90]\bbb{I})(<[:-90]Br)-(-H)(-[:45]H)-[:-45]H}}{\ttt{rectus}}
        \+
        \chemname[-25pt]{\chemfig{(<:[:90]-[:150])(-[:-90]\bbb{I})(<[:150]Br)-(-H)(-[:45]H)-[:-45]H}}{\fff{sinister}}
      \schemestop{}
      \medskip

      \medskip
      \schemestart{}
        \chemfig{}
      \schemestop{}
      \bigskip
      
    \end{itemize}
    
    \item Example with an internal alkyne; either side can be chosen if the R groups are not specified:
    
    \hspace{-43pt}
    \medskip
    \schemestart{}
      \chemfig{R-~-R'} 
      \arrow{->[HCl]}
      \chemfig{R-\chemabove[6pt]{}{\cat}=(-[:30]H)(-[:-30]R')}
      \arrow(--[braces]){->[\bbb{\ch{Cl-}}]}
      \chemname[-25pt]{\chemfig{R-[:-30](-[:210]\bbb{Cl})=(-[:30]H)(-[:-30]R')}}{\ttt{zusammen}}
      \+{,,-10pt}
      \chemname[-25pt]{\chemfig{\bbb{Cl}-[:-30](-[:210]R)=(-[:30]H)(-[:-30]R')}}{\fff{entegen}}
    \schemestop{}
    \medskip
    \begin{itemize}
      \item Cl is higher priority than any carbon containing gruop (R), and any R' is higher than H, so both top and bottom attacks are possibile.
      \item Related practice problem:
      
      \medskip
      \schemestart{}
        {\scriptsize\chemfig{[:-30]*6(-=-(-~-)=-=)}}
        \arrow{->[HCl]}
        {\scriptsize\chemname[-40pt]{\chemfig{[:-30]*6(-=-(-\chemabove[6pt]{}{\cat}=(-[:-30]H)(-[:30]))=-=)}}{preferred}}
        \quad{}or\quad
        {\scriptsize\chemfig{[:-30]*6(-=-(-[:-30](-[:240]H)=\chemabove[6pt]{}{\cat}-)=-=)}}
      \schemestop{}
      \medskip

      \item The former carbocation intermediate is preferred due to resonance of the benzene ring, which does a better job stabilizing the carbocation.
      \item Continuing the reaction:
      
      \medskip
      \schemestart{}
        {\scriptsize{\chemfig{[:-30]*6(-=-(-\chemabove[6pt]{}{\cat}=(-[:-30]H)(-[:30]))=-=)}}}
        \arrow(--[braces]){->[\bbb{\ch{Cl-}}]}
        \chemname[-25pt]{{\scriptsize{\chemfig{[:-30]*6(-=-(-[:-30](-[:240]Cl)-(-[:-30]H)(-[:30]))=-=)}}}}{\fff{entegen}}
        \+
        \chemname[-25pt]{{\scriptsize{\chemfig{[:-30]*6(-=-(-[:-30](-[:240]Cl)-(-[:30]H)(-[:-30]))=-=)}}}}{\ttt{zusammen}}
      \schemestop{}
      \bigskip
    \end{itemize}
  \end{itemize}

  \subsection{Hydration of Alkynes}\label{Hydration of Alkynes}
  \begin{itemize}
      \item Introductory example:
      
      \schemestart{}
        \chemfig{R-~-H}
        \arrow{->[\ch{H2O}][d. \ch{H2SO4}]}[0,1.1]
        \chemfig{R-\chemabove[6pt]{}{\cat}=(-[:30]\rrr{H})-[:-30]H}
        \arrow{->[\ch{H2O}]}
        \chemname[-25pt]{\chemfig{R-[:-30](-[:210]\bbb{OH})=(-[:30]H)-[:-30]H}}{enol}
      \schemestop{}
      
      \item Follows \hyperref[Markovnikov's Rule]{\ulink{Markonikov's Rule}}
      \item However, enols readily interconvert with more stable form:
      
      \medskip
      \schemestart{}
      \chemname[-25pt]{\chemfig{R-[:-30](-[:210]HO)=(-[:30]H)-[:-30]H}}{enol}
      \arrow{<->>}
      \chemname[-25pt]{\chemfig{R-[:-30](=[:210]O)-(-H)(-[:45]H)-[:-45]H}}{keto}
      \schemestop{}
      \medskip 

      \item \ddd{Tautomers}: a class of structural isomers (constitutional isomers) that readily interconvert, commonly due to the relocation of a proton (protonation-deprotonation) across a \(\pi \) bond.
        \begin{itemize}
          \item The chemical reaction is called tautomerization; the concept is referred to tautomerism, which is sometimes desmotropism.
        \end{itemize}
      \item Keto-enol tautomerism major product is the keto form, which does not allow for a second addition reaction.
      \item \ch{HgSO4} is frequently used instead of d. \ch{H2SO4}, due to the mercurinium intermediate vs.\ the carbocation intermediate, which is both \emph{faster} and a means to \emph{reduce the probability} of byproducts.
      
      \item Example of an alkyne in a \hyperref[Hydroboration-Oxidation]{\ulink{hydroboration-oxidation reaction}}:
      
      \medskip
      \hspace{25pt}\chemfig{\cadn{H}-[,1.2]\cadp{BH_2}}\\
      \medskip
      \schemestart{}
        \chemfig{R-\cadp{C}~\cadn{C}-H}
        \arrow{->[\ch{B2H6}][{\small\ch{H2O2}, \ch{NaOH, THF}}]}[0,1.8]
        \chemfig{(-[:150]\bbb{H})(-[:-150]R)=(-[:30]\rrr{BH_2})(-[:-30]H)}
        \arrow{->[{\small\ch{H2O2}, \ch{NaOH, THF}}]}[0,1.8]
        \dots
      \schemestop{}
      \medskip

      \medskip
      \schemestart{}
        \dots
        \quad
        \chemname[-30pt]{\chemfig{(-[:150]H)(-[:-150]R)=(-[:30]OH)(-[:-30]H)}}{enol}
        \arrow{<->>}
        \chemname[-30pt]{\chemfig{(-[:135]H)(-[:-135]R)(-[180])-(=[:30]O)(-[:-30]H)}}{aldehyde (a ketone)}
      \schemestop{}
      \bigskip
      
      \item \ddd{Aldehyde}: generally created by removing a hydrogen from an alcohol; in our case it is generated by the \emph{anti-Markovnikov} reaction that results in a terminal enol, which then undergoes tautomerization and produces the aldehyde as the major product.
  \end{itemize}
  
  \subsection{Hydrogenation of Alkynes}\label{Hydrogenation of Alkynes}
  \begin{itemize}
      \item Complete hydrogenation of an alkyne:
      
      \medskip
      \schemestart{}
        \chemfig{R-~-R'}
        \arrow{->[\ch{H2}][Pd-c]}
        \chemfig{R-(-[:90]H)(-[:-90]H)-(-[:90]H)(-[:-90]H)-R'}
      \schemestop{}
      \bigskip
  
      \item Alkyne \to\ \emph{cis}-alkene; use of lindlar catalyst (Pd-c poisoned with lead) limits further reduction by controlling hydrogens available:
      
      \medskip
      \schemestart{}
      \chemfig{R-~-R'}
      \arrow{->[\ch{H2}][lindlar]}
      \chemfig{(-[:150]H)(-[:210]R)=(-[:30]H)(-[:-30]R')}
      \schemestop{}
      \bigskip
      
      \item Alkyne \to\ \emph{trans}-alkene; using generation of free radicals (\bbb{\bullet}, single electron) that pair up with another electron generated by the dissociation of Na \to\ \rrr{\ch{Na+}}\plus\ \bbb{\ch{e-}} to create a free pair of electrons that then receive a hydrogen from \ch{NH3}:

      \medskip
      \schemestart{}
        \chemfig{R-~-R'}
        \arrow{->[Na][liq. \ch{NH3}]}
        \chemfig{(-[:120,.3,,,draw=none]\bbb{\bullet})(-[:-150]R)=(-[:30]R')(-[:-60,.3,,,draw=none]\bbb{\bullet})}
        \arrow{->}
        \chemfig{(-[:120,.3,,,draw=none]\bbb{\ani})(-[:-150]R)=(-[:30]R')(-[:-60,.3,,,draw=none]\bbb{\ani})}
        \dots
      \schemestop{}
      \bigskip

      \medskip
      \schemestart{}
        \dots
        \arrow{->[\ch{H-NH2}]}[0,1.2]
        \chemfig{(-[:150]H)(-[:210]R)=(-[:30]R')(-[:-30]H)}
      \schemestop{}
      \bigskip

        \begin{itemize}
          \item Note, in lecture the added hydrogens were drawn separately; I am unsure, but I assume they happen near-simultaneously.
        \end{itemize}
      
  \end{itemize}

  \subsection{Halogenation of Alkynes}\label{Halogenation of Alkynes}
  \begin{itemize}
      \item Introductory example:

      \medskip
      \schemestart{}
        \chemfig{R-~-R'}
        \arrow{->[\ch{I_2}][(1 eq.)]}
        \chemfig{R-[:30]@{c1}([:30]*3(=-@{i2}\chemabove[3pt]{\rrr{I}}{\cat}-[@{sb}]))--[:-30]R'}
        \arrow{->[\chemfig{@{i1}\bbb{I^\ani}}]}
        \chemfig{R-[:-30](-[:210]\bbb{I})=(-[:-30]R')(-[:30]\rrr{I})}
      \schemestop{}
      \chemmove[dash pattern= on 1pt off 1pt]{
        \draw(i1)..controls +(-80:1.8cm) and +(south:1cm).. (c1);
        \draw(sb)..controls +(110:0.5cm) and +(160:0.5cm).. (i2);
        }
      \bigskip
        
      \begin{itemize}
        \item The 3-membered cationic ring with a double bond is not very stable, so it does not form readily with a rate that is \(10^3\text{--}10^7\) times slower than that on an alkene.
      \end{itemize}

      \item Same example, but with 2 equivalent or excess:
      
      \medskip
      \schemestart{}
        \chemfig{R-~-R'}
        \arrow{->[\ch{I_2}][(2+ eq.)]}
        \chemfig{R-[:-30](-[:210]I)=(-[:-30]R')(-[:30]I)}
        \arrow{->[\ch{I_2}]}
        \chemfig{R-[:30]([:30]*3(--\chemabove[3pt]{\rrr{I}}{\cat}-))(-[:-90]I)-(-[:-90]I)-[:-30]R'}
        \arrow{}
        \dots
      \schemestop{}
      \bigskip
      
      \medskip
      \schemestart{}
        \dots
        \arrow{}
        \chemfig{(-[:60]R)(-[:120]I)(-[:-90]I)-[,1.3](-[:-60]R')(-[:-120]I)(-[:90]I)}
      \schemestop{}
      \bigskip

      \begin{itemize}
        \item Note: if you had controlled the second addition and added a different halide, then stereochemistry would have been important to consider since the products would have had chiral carbons.
      \end{itemize}
      
  \end{itemize}
  
\end{itemize}