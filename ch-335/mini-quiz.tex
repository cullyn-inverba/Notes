\documentclass[quiz]{inVerba-notes}
% chktex-file 36

\definecolor{title-color}{HTML}{5f89f5}
\newcommand{\theTitle}{\href{https://github.com/cullyn-inverba/notes/tree/master/ch-335}{Mini Quizzes}}

\begin{document}
\hypertarget{ToC}{\tableofcontents}

\clearpage
\section{Week 3 --- Chapter 16}\label{Week 3}
\begin{enumerate}
  \item What is the major product for the following reaction.

  \schemestart{}
    \chemfig{-[:-30](-[:-90]OH)-[:30]-[:-30]-[:30]}
    \arrow{->[\ch{CrO3}][\ch{H2SO4}]}[0,1.2]
  \schemestop{}
  \bigskip
  \begin{multicols}{2}
    \begin{itemize}
      \item \chemfig{-[:-30](=[:-90]O)-[:30]O-[:-30]-[:30]-[:-30]}
      \item \chemfig{-[:30]O-[:-30](=[:-90]O)-[:30]-[:-30]-[:30]}
      \item \chemfig{-[:-30](=[:-90]O)-[:30]-[:-30]-[:30]}
      \item \chemfig{-[:-30](-[:-90]OH)(-[:90]OH)-[:30]-[:-30]-[:30]}
    \end{itemize}
  \end{multicols}
  \basec{\begin{itemize}
    \item \ch{CrO3} is an oxidizing agent, which means electron density will be pulled away from the carbon to form a double bonded oxygen.
    \item Oxygen does not insert itself into the chain.
  \end{itemize}}
  \item Give the major product for the following oxidation.

  \medskip
  \schemestart{}
    \chemfig{(-[:90])(-[:-150])-[:-30]-[:30]-[:-30]OH}
    \arrow{->[PCC]}
  \schemestop{}
  \bigskip
  
  \begin{multicols}{2}
    \begin{itemize}
      \item \chemfig{(-[:90])(-[:-150])-[:-30]-[:30](=[:90]O)-[:-30]OH}
      \item \chemfig{(-[:90])(-[:-150])-[:-30]-[:30](=[:90]O)}
      \item \chemfig{-[:-30]-[:30]-[:-30]-[:30](=[:90]O)-[:-30]OH}
      \item \chemfig{-[:-30]-[:30]-[:-30]-[:30](=[:90]O)}
    \end{itemize}
  \end{multicols}

  \basec{\begin{itemize}
    \item PCC is a mild oxidizing agent that is commonly used for selective oxidation of alcohols to aldehydes or ketones. 
    \item In this case we started with a terminal enol, which would produce an aldehyde.
    \item The carbonyl group is not affected, it should not change.
  \end{itemize}}
  \newpage
  \item What is the major product from the following reaction?

  \medskip
  \schemestart{}
    \chemfig{-[:30]-[:-30](-[:90]Br)(-[:-90]Br)-[:30]-[:-30]}
    \arrow{->[2 \ch{NH2^{\fminus}}][]}[0,1.3]
  \schemestop{}
  \bigskip
  \begin{multicols}{2}
    \begin{itemize}
      \item \chemfig{-[:30]-[:-30](-[:-90,.8]Br)=[:30]-[:-30]}
      \item \chemfig{-[:-30]-~-}
      \item \chemfig{-[:30]-[:-30]-[:30]=[:-30]}
      \item[]
      \item \chemfig{-[:30]-[:-30]-~}
    \end{itemize}
  \end{multicols}

  \basec{\begin{itemize}
    \item This looks like dihaloalkane elimination. I took this quiz a week early, so this explanation might not be the best, but looks like \ch{NH2-} is acting as a reducing agent(?); causing the elimination of bromine, leaving the carbon to form a triple bond. 
  \end{itemize}}
  \item The major product of a hydroboration oxidation reaction on a terminal alkyne is
  \begin{multicols}{2}
    \begin{itemize}
      \item a carboxylic acid
      \item ketone
      \item alkane
      \item aldehyde
    \end{itemize}
  \end{multicols}

  \basec{\begin{itemize}
    \item Example of an alkyne undergiong a hydroboration-oxidation reaction:
      
      \medskip
      \hspace{25pt}\chemfig{\cadn{H}-[,1.2]\cadp{BH_2}}\\
      \medskip
      \schemestart{}
        \chemfig{R-\cadp{C}~\cadn{C}-H}
        \arrow{->[\ch{B2H6}][{\small\ch{H2O2}, \ch{NaOH, THF}}]}[0,1.8]
        \chemfig{(-[:150]\bbb{H})(-[:-150]R)=(-[:30]\rrr{BH_2})(-[:-30]H)}
        \arrow{->[{\small\ch{H2O2}, \ch{NaOH, THF}}]}[0,1.8]
        \dots
      \schemestop{}
      \medskip

      \medskip
      \schemestart{}
        \dots
        \quad
        \chemname{\chemfig{(-[:150]H)(-[:-150]R)=(-[:30]OH)(-[:-30]H)}}{enol}
        \arrow{<->>}
        \chemname{\chemfig{(-[:135]H)(-[:-135]R)(-[180])-(=[:30]O)(-[:-30]H)}}{aldehyde (a ketone)}
      \schemestop{}
      \bigskip
  \end{itemize}}
  \end{enumerate}
\clearpage
\section{Week 2 --- Chapter 15}\label{Week 2}
\begin{enumerate}
  \item The reagent needed to convert 2-butyne to cis-2-butene is
  \begin{itemize}
      \item \ch{H2/Pd-C}
      \item \ch{Li/NH3}
      \item \ch{Na/NH3}
      \item \ch{H2/Lindlar Catalyst}
    \end{itemize}
    \basec{\begin{itemize}
    \item Complete hydrogenation of an alkyne:
      
    \medskip
    \schemestart{}
      \chemfig{R-~-R'}
      \arrow{->[\ch{H2}][Pd-c]}
      \chemfig{R-(-[:90]H)(-[:-90]H)-(-[:90]H)(-[:-90]H)-R'}
    \schemestop{}
    \bigskip

    \item Alkyne \to\ \emph{cis}-alkene; use of lindlar catalyst (Pd-c poisoned with lead) limits further reduction by controlling hydrogens available:
    
    \medskip
    \schemestart{}
    \chemfig{R-~-R'}
    \arrow{->[\ch{H2}][lindlar]}
    \chemfig{(-[:150]H)(-[:210]R)=(-[:30]H)(-[:-30]R')}
    \schemestop{}
    \bigskip
    
    \item Alkyne \to\ \emph{trans}-alkene; using generation of free radicals (\bbb{\bullet}, single electron) that pair up with another electron generated by the dissociation of Na \to\ \rrr{\ch{Na+}}\plus\ \bbb{\ch{e-}} to create a free pair of electrons that then receive a hydrogen from \ch{NH3}:

    \medskip
    \schemestart{}
      \chemfig{R-~-R'}
      \arrow{->[Na][liq. \ch{NH3}]}
      \chemfig{(-[:120,.3,,,draw=none]\bbb{\bullet})(-[:-150]R)=(-[:30]R')(-[:-60,.3,,,draw=none]\bbb{\bullet})}
      \arrow{->}
      \chemfig{(-[:120,.3,,,draw=none]\bbb{\ani})(-[:-150]R)=(-[:30]R')(-[:-60,.3,,,draw=none]\bbb{\ani})}
      \dots
    \schemestop{}
    \bigskip

    \medskip
    \schemestart{}
      \dots
      \arrow{->[\ch{H-NH2}]}[0,1.2]
      \chemfig{(-[:150]H)(-[:210]R)=(-[:30]R')(-[:-30]H)}
    \schemestop{}
    \bigskip
  \end{itemize}}

  \item A mixture of 1-heptyne, 2-heptyne, and 3-heptyne was hydrogenated in the presence of a palladium catalyst until hydrogen uptake stopped. If one assumes that the hydrogenation went to completion for all the reactants present in the mixture, how many distinct seven-carbon isomers were produced?
  \begin{itemize}
    \item 0nly 1
    \item 2
    \item 4
    \item 6
  \end{itemize}

  \basec{
    \begin{itemize}
      \item \ch{H2/Pd-c} (palladium catalyst) generates completely saturated alkenes, thus the location of the double bond in a heptyne will make no difference overall.
    \end{itemize}}

  \item Give the best reagents for the reaction
  
  \medskip
  \schemestart{}
    \chemfig{\ch{(CH3)2CHCH2C}-[,1.7,,,draw=none]~CH}
    \arrow{}
    \chemname[-25pt]{\chemfig{\ch{(CH3)2CHCH2CH2CH}-[:-7,1.75,,,draw=none](=[:-90]OH)}}{}
  \schemestop{}
  \bigskip
  \begin{itemize}
    \item \ch{H2O}, \ch{H2OSO4}, \ch{HgSO4}
    \item \ch{BH3}, \ch{H2O2}, \ch{NaOH}
    \item \ch{K2Cr2O7}
    \item \ch{H2}, Lindlar Catalyst
  \end{itemize}
 
  \basec{\begin{itemize}
    \item First, this is a hydration reaction, so that limits just the first two options. 
    \item Hydration using \ch{H2O} and \ch{H2OSO4} or \ch{HgSO4} does have difference, but both follow Markovnikov's rule and end produce internal enols and thus internal ketones.
    \item Hydroboration-oxidation reaction follows anti-Markovnikov rule and produces a terminal enol and thus an aldehyde, which is the desired product.
  \end{itemize}}

  \item Which of the alkyne addition reactions below involves an enol intermediate?
  \begin{itemize}
    \item Hydroboration/oxidation
    \item dil. \ch{H2SO4} in \ch{HgSO4}
    \item Hydrogenation
    \item Both hydroboration/oxidation and dil. \ch{H2SO4} in \ch{HgSO4}
  \end{itemize}
  \basec{
    \begin{itemize}
      \item See question three, both hydroboration/oxidation and dil. \ch{H2SO4} in \ch{HgSO4} are used in hydration, which have enol intermediates.
      \item Hydrogenation only has to do with adding hydrogens to saturate the alkyne through elimination reactions, which question one covers. 
    \end{itemize}}
\end{enumerate}

\clearpage
\section{Week 1 --- Chapter 14}\label{Week 1}
\begin{enumerate}
    \item Name the structure: 
    
    \medskip
    \schemestart{}
        \chemfig{*7((-Cl)--=----)}
    \schemestop{}
    \bigskip
    
    \begin{itemize}
      \item 1-chloro-3-cycloheptene
      \item 4-chloro-1-cycloheptene
      \item 4-chloro-1-cyclohexene
      \item 6-chloro-1-cycloheptene
      \basec{\begin{itemize}
        \item When numbering the parent chain, the double bond should receive the lowest number possible; \emph{k=1}
          \begin{itemize}
            \item Note: define the location \(k\) of the double bond as being the number of its first carbon, not at the end.
          \end{itemize}
        \item The locant (\(k\)) of the double bond should be placed right before the suffix of ``ene,'' though, it was previously recommended before the parent (both are acceptable), e.g., 2-pentene = pent-2-ene; \emph{1-cycloheptene}
        \item Name and the side groups (other than hydrogen) according to the appropriate rules; \emph{chloro}
        \item Define the position of each side group as the number of the chain carbon it is attached to; \emph{4-}
      \end{itemize}}
    \end{itemize}

    \item Name the structure: 
    
    \medskip
    \schemestart{}
      \chemfig{ClCH_2CH_2-[:-30](-[:210]H)=(-[:-30]H)
      (-[:30]C(-[:150]H)=(-[:-30]H)-[:30]CH_3)
      }
    \schemestop{}
    \bigskip

    \begin{itemize}
      \item (2E,4E)-7-chloro-2,4-heptadiene
      \item (2Z,4Z)-7-chloro-2,4-heptadiene
      \item (2Z,4E)-7-chloro-2,4-heptadiene
      \item (2E,4Z)-7-chloro-2,4-heptadiene
      \basec{
        \begin{itemize}
          \item \ddd{E-Z notation}: recommended instead of \textit{cis} and \textit{trans} in order to account for cases that has more than two different groups attached to the double bond by first determining the priority using the \link{https://en.wikipedia.org/wiki/Cahn\%E2\%80\%93Ingold\%E2\%80\%93Prelog_priority_rules}{Cahn-Ingold-Prelog System}.
          \begin{itemize}
            \item \fff{E, entgegen, ``opposite''}.
            \item \ttt{Z, zusammen, ``together''}; ``on ze zame zide.''
        \end{itemize}
        \item When numbering the parent chain, the double bond should receive the lowest number possible; \emph{k=2}
          \begin{itemize}
            \item The two highest priority groups are on \fff{opposite} sides; \emph{2E}
          \end{itemize}
        \item There is more than one double bond; \emph{\(k_2=4\)}
          \begin{itemize}
            \item The two highest priority groups are on \ttt{zame} side; \emph{4Z}
          \end{itemize}
      \end{itemize}
      }
    \end{itemize}
    
    \item How many stereoisomeric product(s) do you get in the reaction below.
    
    \medskip
    \schemestart{}
      \chemfig{=[:30]-[:-30](-[:30]-[:-30])-[:-90]-[:-30]}
      \arrow{->[\ch{Hg(OAc)2}, \ch{H2O}, THF][\ch{NaBH4}]}[0,2.2]
    \schemestop{}
    \bigskip
    
    \basec{\begin{itemize}
      \item Oxymercuration-demercuration reactions follow Markovnikov's rule, i.e., \rrr{\(H^+\)} is added to the carbon with the \rrr{greatest} number of hydrogen atoms while the \bbb{\(X^-\) component} is added to the carbon with the \bbb{fewest} hydrogen atoms.
      \item Drawing the intermediate is not necessary, and no chiral centers are found in the products:
      
      \medskip
      \schemestart{}
        \dots 
        \arrow(--[braces]){->[\rrr{\ch{Hg(OAc)2}}, \bbb{\ch{H2O}}, THF][\ch{NaBH4}]}[0,2.2]
        \chemfig{\rrr{H}-[:30](<[:90]\bbb{OH})-[:-30](-[:30]-[:-30])-[:-90]-[:-30]}
        \+
        \chemfig{\rrr{H}-[:30](<:[:90]\bbb{OH})-[:-30](-[:30]-[:-30])-[:-90]-[:-30]}
      \schemestop{}
      \bigskip
      
    \end{itemize}}
    
    \item Which reaction intermediate is formed when Br2/CCl4 reacts with cyclohexene?
    
    \begin{multicols}{2}
    \begin{enumerate}[label=\roman*]
    
    \item 

    \medskip
    \schemestart{}
      \chemfig{*6(--(-[,.3,,,,draw=none]\cat)-(-Br)---)}
    \schemestop{}
    \bigskip

    \item 

    \medskip
    \schemestart{}
      \chemfig{*6(--(-Br)-(-[,.3,,,,draw=none]\cat)---)}
    \schemestop{}
    \bigskip

    \item 

    \medskip
    \schemestart{}
      \chemfig{*6(--(-[,.3,,,,draw=none]\cdot)-(-Br)---)}
    \schemestop{}
    \bigskip

    \item 

    \medskip
    \schemestart{}
      \chemfig{*6(--(-[:30]Br\,\cat)-(-[:-30,.55])---)}
    \schemestop{}
    \bigskip

    \item 

    \medskip
    \schemestart{}
      \chemfig{*6(--(-Br)-(-Br)---)}
    \schemestop{}
    \bigskip

    \end{enumerate}
    \end{multicols}

    \basec{\begin{itemize}
      \item \ddd{Halogenation}: a reaction that involves the addition of one or more halogens to a compound or material.
    \begin{itemize}
      \item The addition of halogens to alkenes proceeds via \emph{intermediate halonium ions}.
      \item \ddd{Halonium ion}: any onium ion containing a halogen atom carrying a positive charge. This cation has the general structure: \chemfig{R-[,.7]\rrr{+X}-[,.7]R'}
      \item \ddd{Onium ion}: a \rrr{cation} formally obtained by the protonation of mononuclear parent hydride of a pnictogen (group 15 of the periodic table), chalcogen (group 16), or halogen (group 17); \rrr{\(Br^\cat\)} in our case.
    \end{itemize}
    \end{itemize}}
\end{enumerate}
\end{document}