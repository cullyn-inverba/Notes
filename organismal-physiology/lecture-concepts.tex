\documentclass[12pt,a4paper]{article}
\usepackage{inverba}

\newcommand{\userName}{Cullyn Newman} 
\newcommand{\class}{BI 320} 
\newcommand{\institution}{Portland State Univeristy} 
\newcommand{\theTitle}{\color{g-Leaf} Organismal Physiology Lecture Concepts}

\begin{document}
%%%%%%%%%%%%%%%%%%%%%%%%%%%%%%%%%%%%%%%%%%%%%%%%%%%%%%%%%%%%%%%%%%%%%
\tableofcontents
\cleardoublepage
\fancyhead{}
\fancyhead[R]{\hyperlink{home}{\nouppercase\leftmark}}
%%%%%%%%%%%%%%%%%%%%%%%%%%%%%%%%%%%%%%%%%%%%%%%%%%%%%%%%%%%%%%%%%%%%%

%\begingroup
%%%%%%%%%%%%%%%%%%%%%%%%%%%%% Chapter 1 %%%%%%%%%%%%%%%%%%%%%%%%%%%%%
%\begingroup
\clearpage
\section*{Midterm I}\phantomsection
\addcontentsline{toc}{section}{\textbf{Midterm I}}
\fancyhead[R]{\hyperlink{home}{Week 1}}
\fancyhead[L]{\hyperlink{home}{Midterm I}}
\subsection{Week 1}
\begin{itemize}
    \item What are the two central questions of physiology? 
    \item Describe the different sub-disciplines of physiology.
    \item What is the Krogh principle? Explain what it means to physiology.
    \item Define the terms conformity and regulation and understand how they relate to physiological processes. Discuss examples.
    \item Define the term homeostasis. Understand the process of negative and positive feedback regulation.
    \item Describe how physiology changes with time in response to the external environment. Define the terms acclimation, adaptation, natural selection, and evolution.
    \item Describe the structure and chemical components of lipid membranes.
    \item What are the effects of low and high temperatures on membranes? How are membrane properties altered to offset these effects?
    \item Describe the five functional types of membrane proteins and their basic functions.
    \item What are the two primary roles of enzymes?
    \item Define the terms V\(_{\text{max}}\) and \(K_m\). Explain the factors that affect these reaction properties.
    \item Define activation energy. Understand the effect of enzyme catalysis on a reaction’s energy of activation.
    \item What are the effects of substrate concentration on the rate of an enzymatic reaction? How does enzyme-substrate affinity affect the reaction rate?
    \item Why are enzymatic rates unresponsive to increases in substrate concentration above a physiologically relevant range?
    \item Understand why conformational change is a critical part of enzyme function.
    \item Define the term isozyme and understand how they can contribute to natural selection.
    \item Define and understand the process of allosteric modulation.
\end{itemize}
%\endgroup
%%%%%%%%%%%%%%%%%%%%%%%%%%%%% Chapter 1 %%%%%%%%%%%%%%%%%%%%%%%%%%%%%

%%%%%%%%%%%%%%%%%%%%%%%%%%%%% Chapter 2 %%%%%%%%%%%%%%%%%%%%%%%%%%%%%
%\begingroup
\clearpage
\fancyhead[R]{\hyperlink{home}{Week 2}}
\subsection{Week 2}
\begin{itemize}
    \item Define the terms transcription, translation, and post-translational processing. Understand the differences between nRNA and mRNA and introns and exons. 
    \item Understand how to interpret information about the origin of physiological traits from a phylogenetic tree.
    \item Define the terms genome and genomics. Describe the methods, challenges, and major goals of genomics research.
    \item Describe an example for each major mechanism of gene modification, e.g. mutation accumulation, deletions, gene duplication.
    \item What does the phrase “from genotype to phenotype” mean? What are the limitations associated with this phrase?
    \item Define the terms transcriptome and transcriptomics. Describe the methods and challenges of transcriptomics research. How can the function of a gene’s expression be tested?
    \item Define the terms proteome and proteomics. Why is proteomics treated as a separate discipline rather than being lumped together with genomics and transcriptomics?
    \item What is two-dimensional gel electrophoresis? What kinds of data does it generate? How is it used in proteomics research?
    \item Define the term metabolomics. How does it differ from the other “omics” disciplines?
    \item Define the term epigenetics. Are epigenetic changes heritable from cell  to cell? From parents to offspring? Explain.
    \item Identify the two major mechanisms of epigenetic change and their consequences on gene transcription.
\end{itemize}
%\endgroup
%%%%%%%%%%%%%%%%%%%%%%%%%%%%% Chapter 2 %%%%%%%%%%%%%%%%%%%%%%%%%%%%%

%%%%%%%%%%%%%%%%%%%%%%%%%%%%% Chapter  %%%%%%%%%%%%%%%%%%%%%%%%%%%%%
%\begingroup
\clearpage
\fancyhead[R]{\hyperlink{home}{Week 3}}
\subsection{Week 3}
\begin{itemize}
    \item Define and describe the different types of passive and active solute transport. Understand how and why they differ from each other.
    \item Define: equilibrium, concentration gradient, electrical gradient, electrochemical gradient, electrogenic, and electroneutral. 
    \item Understand the forces imparted on ion movement as a result of reinforcing and opposing electrochemical gradients.
    \item Define the Fick diffusion equation and be able to use it to calculate a rate of diffusion. 
    \item What are the major limitations of diffusion as a transport mechanism? 
    \item What is a boundary layer and how does it impact diffusion?
    \item Describe the basic characteristics of the electrochemical environment for a typical animal cell. \ch{Na+} and \ch{K+} ion channels and the \ch{Na+}--\ch{K+} ATPase pump affect/regulate these characteristics?
    \item Describe the different types of gated ion channels and their function in ion transport.
    \item Understand the concept of potential energy stored in an electrochemical gradient. 
    \item What are cotransporters and countertransporters?
    \item What are the major mechanisms that generate diversity in transporters and/or modulate their function? Provide examples for each.
    \item What are colligative properties? List and define the major colligative properties important to physiology.
    \item Define osmosis and understand how osmotic pressure is generated and measured.
    \item Define the terms hyperosmotic, isosmotic, and hyposmotic.
    \item Understand the different effects electrolytes and organic molecules have on osmotic pressure.
    \item Describe the mechanisms of water transport.
    \item How is cell volume affected by its environment?
\end{itemize}
%\endgroup
%%%%%%%%%%%%%%%%%%%%%%%%%%%%% Chapter  %%%%%%%%%%%%%%%%%%%%%%%%%%%%%
%\endgroup
\end{document}