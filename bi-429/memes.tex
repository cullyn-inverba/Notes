\documentclass[basic]{inVerba-notes}

\newcommand{\userName}{Cullyn Newman}
\newcommand{\class}{BI:\@ 429}
\newcommand{\theTitle}{\color{link}{Biodiveristy Mini-Symposia: memes}}
\newcommand{\institution}{Portland State University}

\begin{document}
    \textbf{Intro}: hi, today my presentation is on memes---yes memes---I promise it is deeply related to biodiversity, but I don't have a lot of time, so I have to jump right into it.

    \textbf{Multidisciplinary Subject}: first, I'd like to point out that this statement was taken directly from our syllabus. The key point here is that conservation biology is a multidisciplinary subject that requires an integrative approach to problem-solving.

    But what is an integrative approach? Well as you can see, there are numerous fields of study within conservation biology.\minimal{~\cite{bennett2017conservation},~\cite{cooke2013conservation},~\cite{groom2006principles},~\cite{timberlake1989behavior}}

    Essentially, there are a lot of ideas floating around, all of which are part of a large knowledge ecosystem that is attempting to solve some of our most dire problems. These ideas are\dots\ well, \textbf{memes}. To be more specific\dots

    \textbf{Richard Dawkins}: in 1976 Richard Dawkins wrote an extremely influential book titled \minor{``The Selfish Gene''}. In the book he expands on Gorgeo C. Williams' theory of adaptation and natural selection by explaining how genes replicate and are inherently selfish, but also explains how such selfish behaviors actually generate the cooperation required for life as we see it.\minimal{~\cite{dawkins2016selfish}}

    What's extremely relevant, is that in this book he coined the word meme, which he defined as a unit of human cultural evolution. Dawkins essentially implies that the same evolutionary forces that act on genes, also act on the ideas, or memes.\minimal{~\cite{merriam-webster}} Right? So to summarize, \textbf{memes are to cultural as genes are to organisms.}

    \textbf{Genetic Diversity}: so this is where I relate memes to biodiversity. If genetic diversity is the ultimate source of biodiversity at all levels, by being the substrate that evolution acts upon, and if human culture is the most dangerous threat to (or potentially saviors of) biodiversity, then diversity in shared ideas, or ``memes,'' is essential for changing the culture that is currently destroying the planet, to one that will instead work towards restoring it. 

    \textbf{The Evolutionary Play}: now I made a pretty bold statement there, and originally had a lot more backing it up, but I've painfully cut out over 70\% of my presentation just to it under four minutes and 45 seconds. So for now let's back up bit and at least explain the guiding memes of conservation biology.

    The first one is that evolution is the basic axiom that unites the field of biology. This is super important! Understanding that memes themselves have phenotypic effects---akin to genes---allows us to apply evolutionary mechanisms to understanding how memes spread. This means in the other guiding principles should be applicable.
    
    The second guiding meme is called The Ecological Theater, which essentially states that the world is dynamic and in nonequilibrium; that is, there are increasingly complex, interacting systems on hierarchical scales that create an incredible theater, or play of life.\minimal{~\cite{groom2006principles}}
    
    This of course raises the third principle and really ties memes into the whole ordeal. This third meme states that we too are a part of such theater---we can't just wall off nature. Any attempts will ultimately fail~\minimal{\cite{groom2006principles}}. The only difference is that memes are not limited by physical interactions, instead the timescale of their evolution is in large part a function of the bandwidth information transfer in the environment. This gives evolution an unprecedented tool for enacting change on the environment. Right?---The potential for change is enormous. Understanding memes is thus an essential part of both biodiversity and implementing effective conservation biology practices on the societal level.

    \textbf{Goal of Conservation Biology}: another vital meme I want to emphasize is the following. This super important because it highlights a common misconception around the word ``conservation.'' Right?---we are really conserving the right to change, not the status quo. Now relating this to memes\dots what this means is that being open to changing your mind is one way to increase diversity in thought; and what I am saying is that such increase may be the most influential tool that conservation biology has to use in such an \textbf{integrative} discipline. Right? There is so much that goes into making these vital decisions.

    \textbf{Systems Thinking}: to conclude, I'd like to stress the importance of a certain meme---that is the necessity of adopting system like thinking. Broadly defined: it is the opposite of linear thinking; it's a holistic (integrative) vs.\ analytic (dissective). Ironically, this statement is very linear; right, it actually separates thinking into a dichotomy itself. But this irony does not subtract from the validity of the statement. The bigger issue is that our minds our not well-equipped to see the system as a whole, so sometimes we do have to draw lines. Again we are a part of an ecological theater, one that increasingly complex and never ends, so the lines are actually always being redrawn.

    \textbf{Donella Meadows}: I really wanted to talk more about this\dots\ but I didn't have time. However, I do strongly recommend reading this book by Donella Meadows. If you take one thing away from this presentation, then it's this. (READ SLIDE)
    
    This meme is life changing. I think most conversations are actually just clumsy arguments fueled by misunderstandings around purpose of the discussion. If you want to have productive conversations that lead to societal change, then pull back, use integrative---systems based thinking---approaches, clearly define issues at hand, and ultimately, never stop asking questions---the story never ends.
    

\nocite*{}
\bibliographystyle{apacite}
\bibliography{memes.bib}
\end{document}