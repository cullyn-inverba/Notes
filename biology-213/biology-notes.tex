\documentclass[12pt,a4paper]{article}
\usepackage{inverba}
\newcommand{\userName}{Cullyn Newman} 
\newcommand{\class}{BI 216} 
\newcommand{\institution}{Portland State University} 
\newcommand{\thetitle}{\hypertarget{home}{Biology Notes}}
\rfoot{\hyperlink{home}{\thepage}}

\begin{document}
%%%%%%%%%%%%%%%%%%%%%%%%%%%%%%%%%%%%%%%%%%%%%%%%%%%%%%%%%%%%%%%%%%%%%
\tableofcontents
\cleardoublepage
\fancyhead{}
\fancyhead[R]{\hyperlink{home}{\nouppercase\leftmark}}
%%%%%%%%%%%%%%%%%%%%%%%%%%%%%%%%%%%%%%%%%%%%%%%%%%%%%%%%%%%%%%%%%%%%%

%%%%%%%%%%%%%%%%%%%%%%%%%%%%% Chapter 44 %%%%%%%%%%%%%%%%%%%%%%%%%%%%
%\begingroup
\clearpage
\setcounter{section}{43}
\section{Animal Sensory Systems}
\subsection{How Do Sensory Organs Convey Information to the Brain?}
\begin{itemize}
    \item \textbf{Transduction}: conversion of an external stimulus to an internal signal in the form of action potentials along sensory neurons.
    \begin{itemize}
        \item Requires a sensory receptor:
        \begin{enumerate}
            \item \textit{Mechanoreceptor}: respond to changes in pressure.
            \item \textit{Photoreceptors}: respond to particular wavelengths of light.
            \item \textit{Chemoreceptors}: detect specific molecules.
            \item \textit{Thermoreceptors}: respond to changes in temperature.
            \item \textit{Nociceptors}: sense harmful stimuli such as tissue injury.
            \item \textit{Electroreceptors}: detect electric fields.
            \item \textit{Magnetoreceptors}: detect magnetic fields.
        \end{enumerate}
        \item Frequency of action potential firing rate can indicate the intensity of the stimulus.
    \end{itemize}
    \item \textbf{Transmission}: the process of sending the signal to the central nervous system.
\end{itemize}
\subsection{Mechanoreception: Sensing Pressure Changes}
\begin{itemize}
     \item \textbf{Statocyst}: an fluid filled organ that grabs use to help sense pressure created by gravity.
    \item Direct physical pressure causes ion channels to open and close, creating voltage gated action potentials.
    \item \textbf{The Mammalian Ear}:
    \begin{itemize}
        \item \textit{Outer ear}: collects incoming pressure waves and funnels them into tube known as the ear canal, which leads to the \textbf{tympanic membrane}, or eardrum.
        \item \textit{Middle ear}: contains three tiny bones that the eardrum passes vibrations to in order to amplify sound. One of the bones, called \textbf{stapes}, vibrates against a membrane called the \textbf{oval window}, which separates the middle ear from the inner ear.
        \item \textit{Inner ear}: the oval window oscillates in response to vibrations which generates waves in a chamber known as the \textbf{cochlea}. These waves are pressure inputs the hair cells respond to.
    \end{itemize}
    \item Hair cells, forming rows that sit in the middle chamber, are embedded in a tissue that sits atop the \textbf{basilar membrane}. In addition, the hair cells' stereocilia touch another smaller surface called the \textbf{tectorial membrane.}
    \item This sandwiching of hair cells produce very specific responses to various frequencies, allowing us to distinguish between them. 
    \item \textbf{Lateral line system}: a mechanoreceptor organ that most fish and larval amphibians use. Consists of embedded gel-like domed structures called cupulae that lay inside canals along the length of the body. 
\end{itemize}

\subsection{Photoreception: Sensing Light}
\begin{itemize}
    \item \textbf{The Insect Eye}:
        \begin{itemize}
            \item \textbf{Compound eyes}: eyes composed of hundreds of thousands of light-sensing columns called \textbf{ommatidia}.
            \item Each ommatidium has lens that focuses light into a smaller number of receptor cells--usually four.
        \end{itemize}
    \item \textbf{The Vertebrate Eye}:
        \begin{itemize}
            \item \textbf{Simple eye}: a structure with a single lens that focuses incoming light onto a layer of many receptor cells.
            \item \textbf{Structure of the Vertebrate Eye}:
                \begin{itemize}
                    \item \textit{white of the eye}: outermost layer of the eye that consists of tough rind of white tissue called the \textit{sclera}.
                    \item \textit{Cornea}: a transparent sheet of connective tissue on the front of the sclera.
                    \item \textit{Iris}: pigmented, round muscle just inside the cornea that can contract or expand to control the amount of light entering the eye.
                    \item \textit{Pupil}: hole in the center of the iris.
                    \item \textit{Lens}: works with the cornea to focus incoming light.
                    \item \textit{Retina}: a layer of photoreceptors and several layers of neurons.
                \end{itemize}
            \item The photoreceptors are held in place by the pigmented epithelium.
            \item photoreceptors synapse with an intermediate layer of connecting neurons called \textbf{bipolar cells}.
            \item Bipolar cells synapse with neurons called \textbf{ganglion cells}, which form the innermost layer of the retina.
            \item The axons of the ganglion cells project to the brian via optic nerve.
            \item Photoreceptors come in two distinct types:
                \begin{itemize}
                    \item \textbf{Rods}: sensitive to dim light but not to color.
                    \item \textbf{Cones}: sensitive to different wavelengths of light, but not dim light.
                \end{itemize}
            \item \textbf{Opsin}: a transparent membrane protein that associates with a molecule of pigment in the \textbf{retinal}.
            \item \textbf{Rhodopsin}: a two molecule complex similar ot opsin in rod cells. 
        \end{itemize}
\end{itemize}

\subsection{Chemoreception: Sensing Chemicals}
\begin{itemize}
    \item Chemoreception occurs when chemicals bind to chemoreceptors, initiating action potentials in sensory neurons.
    \item \textbf{Gustation}: the sense of taste.
    \item \textbf{Olfaction}: sense of smell.
    \item \textbf{Taste buds}: clustered structures containing about 100 spindle shaped taste cells that make synapses with sensory neurons.
    \item \textit{Salty}: due to sodium ions dissolved in food.
    \item \textit{Sourness}: due to presence of protons.
    \item \textit{Umami}: due to monosodium glutamate. 
    \item \textbf{Odorants}: airborne molecules.
    \item \textbf{Olfactory bulb}: part of the brain where olfactory signals are processed and interpreted.
    \item \textbf{Pheromone}: a chemical that is secreted into the environment that affects the behavior or physiology of animals.
\end{itemize}

\subsection{Other Sensory Systems}
\begin{itemize}
    \item Thermoreceptors are located in the central nervous system and also commonly found skin and other outer surfaces of animals. 
    \item \textit{nociceptor}: senses extreme temperatures as well as other painful stimuli produced by chemicals, excessive pressure, and tissue damage.
    \item \textbf{Ampullae of Lorenzini}: tiny pores scattered across a shark's head contain structures that are responsible for their electroreception. Sharks can detect electrical potentials as small as a nanovolt.
    \item \textit{Electrogenic fishes} have specialized organs near their tails that generate electric fields stronger than those of regular nerves or muscles. 
    \item \textit{Magnetoreception}: seen in many organisms, including fungi, invertebrates, and all other vertebrate classes.
\end{itemize}

%\endgroup
%%%%%%%%%%%%%%%%%%%%%%%%%%%%% Chapter 44 %%%%%%%%%%%%%%%%%%%%%%%%%%%%

%%%%%%%%%%%%%%%%%%%%%%%%%%%%% Chapter 45 %%%%%%%%%%%%%%%%%%%%%%%%%%%%
%\begingroup
\clearpage
\section{Animal Movement}
\subsection{How Do Muscles Contract?}
\begin{itemize}
    \item \textbf{Muscle fiber}: a muscle fiber is a long, thin muscle cell. 
    \item Within each muscle cell are many threadlike, contractile structures called \textbf{myofibrils}. 
    \item Myofibrils often look striped or striated due to alternating light and dark units called \textbf{sarcomeres}, which repeat along the length of a myofibril.
    \item The question of how muscles contract simplifies to how sarcomeres shorten.
    \item \textbf{Thin filaments} are composed of actin.
    \item \textbf{Thick filaments} are composed of myosin.
    \item \textbf{Sliding-filament model}: filaments slide past one another during contraction, with the sarcomere shorting with no change between the lengths of thick and thin filaments themselves. 
    \item ATP is required for myosin to release from actin once the two molecules have bound to each other.
    \item \textbf{Tropomyosin} and \textbf{Tropnin} work together to block the myosin binding sites on actin. When the sites are blocked, the myosin-actin interaction cannot occur, relaxing the muscle. 
\end{itemize}

\subsection{Muscle Tissues}
\begin{itemize}
    \item \textbf{Smooth Muscle:}
        \begin{itemize}
            \item Unbranched.
            \item Tappered at each end.
            \item Often organized into thin sheets.
            \item Lack sarcomeres, which gave them their smooth appearance.
            \item Relatively small and have a single nucleus.
            \item Essential for the lungs blood vessels, digestive system, urinary bladder, and reproductive system.
            \item Involuntary
        \end{itemize}
    \item \textbf{Cardiac Muscle:}
        \begin{itemize}
            \item Makes up walls of the heart.
            \item Contain sarcomeres and is satiated. 
            \item Unique branch structure.
            \item Directly connected end to end via specialized regions called nitercanated discs which are critical to the flow of electrical signals and coordination of the heartbeat.
            \item Involuntary
        \end{itemize}        
        \item \textbf{Skeletal Muscle:}
            \begin{itemize}
                \item Exceptionally long, unbranched muscle fibers.
                \item Muitnucleate
                \item Contains sarcomeres
                \item Voluntary
                \item Force depends on:
                    \begin{itemize}
                        \item relative proportion of different fiber types.
                        \item organization of fibers within the muscle.
                        \item how the muscle is used.
                    \end{itemize}
            \end{itemize}
        \item \textbf{Skeletal Muscle Fiber Types:}
            \begin{itemize}
                \item \textbf{Slow muscle fibers}:
                    \begin{itemize}
                        \item oxidative.
                        \item appear red, due to high concentration of myoglobin.
                        \item myosin hydrolyzes ATP at a slow rate, causing slow contraction.
                        \item slow to fatigue due to many mitochondria and aerobic respiration.
                    \end{itemize} 
                \item \textbf{Fast muscle fibers}:
                    \begin{itemize}
                        \item glycolytic.
                        \item appear white, due to low myoglobin concentration.
                        \item myosin hydrolyzes ATP at a rapid rate.
                        \item fast to fatigue due to primarily relying on glycolysis.
                    \end{itemize}
                \item \textbf{Intermediate muscle fibers}:
                    \begin{itemize}
                        \item combination of both fast and slow.
                    \end{itemize}
            \end{itemize}
        
\end{itemize}

%\endgroup
%%%%%%%%%%%%%%%%%%%%%%%%%%%%% Chapter 45 %%%%%%%%%%%%%%%%%%%%%%%%%%%%

%%%%%%%%%%%%%%%%%%%%%%%%%%%%% Chapter 46 %%%%%%%%%%%%%%%%%%%%%%%%%%%%
%\begingroup
\clearpage
\section{Chemical Signals in Animals}
\subsection{Cell-to-Cell Signaling}
\begin{itemize}
    \item \textbf{Autocrine Signals}: signals that affect the same cell that releases them.
    \item \textbf{Paracrine Signals}: diffuse locally and act on target cells near the source cell.
    \item \textbf{Endocrine Signals}: hormones carried to distinct cells by blood or other body fluids and released by discrete organs called glands.
    \item \textbf{Neural Signals}, or neurotransmitters, act on other neurons.
    \item \textbf{Neuroincocrine Signals}: hormones released by neurons that act on distant cells.
    \item Many hormones can act as multiple signal types.
    \item All three types of signaling pathway—endocrine, neuron-daocrine, and neuroendocrine-to-endocrine—are regulated by
    negative feedback, or feedback inhibition.
\end{itemize}
\subsection{How do Hormones Act on Target Cells?}

\begin{itemize}
    \item There are three chemical classes of hormones:
        \begin{itemize}
            \item Peptides ane polypeptides.
            \item Amino acid derivatives.
            \item Steroids.
        \end{itemize}
    \item Steroids and thyroid hormones cross plasma membranes much more readily than other types of hormones.
    \item \textbf{Cyclic AMP (cAMP)}: the key ingredient in the activation of phosphorylase, which catalyzes a reaction that cleaves glucose molecules off glycogen.
    \item \textbf{Signal transduction cascade} can allow the binding of just a single molecule of epinephrine may trigger the release of millions or even billions of glucose molecules.
    \item Most steroid and thyroid hormones act by inducing a change in
    gene expression.    
\end{itemize}
\subsection{What do Hormones Do?}
\begin{itemize}   
    \item Hormones coordinate the activities
    of cells in three arenas: 
        \begin{itemize}
            \item development, growth, and reproduction.
            \item response to environmental challenges.
            \item maintenance of homeostasis.
        \end{itemize} 
    \item  A single hormone may affect a wide array of cells and tissues and induce a variety of responses. 
    \item \textbf{Xenoestrogens}: foreign chemicals that bind to estrogen receptors and induce estrogen like effects.
    \item Cortisol triggers the long-term response to stressors by inducing changes that conserve glucose for use by the brain.    
    \item Glucocorticoids make amino acids available for glucose synthesis by promoting the degradation of contractile proteins in muscle. The resulting loss of muscle mass may cause severe weakness.
 \end{itemize}
    
\subsection{How is the Production of Hormones Regulated?}

\begin{itemize}
    \item In many cases, hormone production is directly or indirectly controlled by the nervous system, where the release of a hormone is regulated by hormones from the anterior pituitary.  
    \item Hormone-secreting cells in the anterior pituitary are regulated by hormones released by the hypothalamus.  
    \item Glucocorticoids accomplish feedback inhibition—they suppress their own production.    
    \item \textit{Cushing’s disease}: the result when feedback inhibition fails.
\end{itemize}
%\endgroup
%%%%%%%%%%%%%%%%%%%%%%%%%%%%% Chapter 46 %%%%%%%%%%%%%%%%%%%%%%%%%%%%

%%%%%%%%%%%%%%%%%%%%%%%%%%%%% Chapter 49 %%%%%%%%%%%%%%%%%%%%%%%%%%%%
%\begingroup
\setcounter{section}{48}
\clearpage
\section{An Introduction to Ecology}
\subsection{The Levels of ecological study}
    \begin{itemize}
        \item Organisms: study of the morphological, physiological, and behavioral adaptations that allow individuals to live in a particular areal
        \item Populations: focus on the number and distribution of individuals in a population and their change over time.
        \item Communities: focus on predation, parasitism, and competition, or explore how communities respond to fires, floods, and other disturbances.
        \item Ecosystems: study of all the organisms in a particular region along with nonliving components.
        \item Biosphere: the sum of all terrestrial and aquatic ecosystems.
    \end{itemize}
\subsection{What Determines the Distribution and Abundance of Organisms?}
    \begin{itemize}
        \item For each species, a unique combination of abiotic and biotic factors determines where individuals live and the size of populations
        \item Understanding historical events, such as the movement of entire continents, is important to interpreting current patterns of species distributions
        \item Abiotic and biotic factors often interact to produce a different effect on species distributions than either type of factor would have on its own.        
    \end{itemize}
\subsection{Abiotic Factors}
    \begin{itemize}
        \item Air circulation cells such as Hadley cells create bands of wet and dry habitats, and the tilt of Earth’s axis causes seasonality in the amount of sunlight that non-equatorial regions receive.
        \item The presence of mountains can create local areas of wet or dry habitats, and proximity to an ocean and the direction of ocean currents moderates temperatures in certain terrestrial habitats.
    \end{itemize}
\subsection{Global Climate Patterns}
    \begin{itemize}
        \item The major terrestrial biomes include tropical wet forest, subtropical desert, temperate grassland, temperate forest, boreal forest, and arctic tundra.
    \end{itemize}
\subsection{Terrestrial Biomes}

\subsection{Aquatic Biomes}
\begin{itemize}
    \item \textbf{Intertidal zone}: consists of a rocky shoreline, sandy beach, or mud flat that is exposed to the air at low tide but submerged at high tide. 
    \item \textbf{Neritic zone}: extends from the intertidal zone to depths of about 200 m. Its outermost edge is defined by the end of the continental shelf 
    \item \textbf{Oceanic zone}: the deepwater region beyond the continental shelf. 
    \item \textbf{Benthic zone}: bottom of the ocean at all depths. \item \textbf{Photic zone}: intertidal and sunlit regions of the neritic, oceanic, and benthic zones
    \item \textbf{Aphotic zone}: Areas that do not receive sunlight.
\end{itemize}

%\endgroup
%%%%%%%%%%%%%%%%%%%%%%%%%%%%% Chapter 49 %%%%%%%%%%%%%%%%%%%%%%%%%%%%

%%%%%%%%%%%%%%%%%%%%%%%%%%%%% Chapter 53 %%%%%%%%%%%%%%%%%%%%%%%%%%%%
%\begingroup
\setcounter{section}{52}
\clearpage
\section{Ecosystems and Global Ecology}
\subsection{How Does Energy Flow through Ecosystems}
\begin{itemize}
    \item \textbf{Primary producer}: a autotroph that allows energy to enter into ecosystems.
    \item \textbf{autotroph}: an organism that can synthesize it's own food from inorganic sources.
    \item \textbf{Gross Primary Productivity (GPP)}: total amount of chemical energy produced in a given area and time period.
    \item \textbf{Net primary productivity (NPP)}: energy that is invested by primary producers in building new tissue of offspring.
    \item NPP = GGP - R. R = energy used in cellular respiration or lost.
    \item It's estimated that 45\% of GPP to NPP and the other 55\% was used for either cellular respiration or lost to the environment.
\end{itemize}
    \subsubsection{What Happens to the Biomass of Autorophs?}
    \begin{itemize}
        \item \textbf{Primary consumers} eat living organisms, secondary eat primary, tertiary eat secondary, and so on.
        \item \textbf{Decomposers}: or detritivores, obtain energy from the remains of other organisms or waste products. 
        \item \textbf{trophic}: the level at which organisms obtain energy from the same source.
    \end{itemize}
    \subsubsection{Energy Transfer between Trophic Levels}
    \begin{itemize}
        \item All ecosystems share a pattern: the total biomass produced each year declines from lower trophic levels up th the higher levels.
        \item Productivity is measured in \SI{}{g/m^2/year} 
        \item Efficiency is a fraction of biomass transferred from one level to the next.
        \item Biomass production at each trophic level varies, but generally efficiency is only about 10\%.
        \item Large mammals are more efficient at producing biomass than small mammals due to less heat loss.
        \item Biomass production is more efficient in ectotherms than endotherms, since they primarily rely on heat gained rather than oxidizing sugars.
        \item \textbf{Biomagnification}: molecules that increase in concentration at higher trophic levels.
    \end{itemize}
    \subsubsection{Global Patterns in Productivity}
    \begin{itemize}
        \item Productivity is limited by anything that limits the rate of photosynthesis.
        \item NPP on land is much higher than it is in oceans.
        \item Wet tropics have the highest net productivity.
        \item Excluding deserts, NPP declines as you move away from the equator due to availability of sunlight and decreasing temperature.
        \item Marine productivity is highest along coastlines, due to increased nutrients from rivers and process called upwelling.
        \item Tropical wet and dry forests cover less than 5\% but account for over 30\% of NPP.
        \item In aquatic ecosystems, the most productive are algal beds, coral reefs, wetlands, and estuaries.
        \item It's estimated that humans are preventing of appropriating 24\% of potential NPP.
    \end{itemize}
\section{How Do Nutrients Cycle through Ecosystems?}
\subsection{Nutrient Cycling within Ecosystems}
\begin{itemize}
    \item \textbf{Biogeochemical cycle}: The path an element takes as it moves from abiotic systems, through biotic, and back. 
    \item \textbf{Soil organic matter}: the carbon-containing compounds that microscopic decomposers release.
    \item \textbf{Humus}: result of when detritus becomes completely decayed.
    \item the decomposition of detritus is the limiting factor in the overall rate of the biogeochemical cycle.
    \item Decomposition is affected by: abiotic conditions such as oxygen availability, temperature and precipitation; quality of nutrient source fot the fungi, bacteria, and archaea; abundance and diversity of detritivores present.
    \item Nutrients can be lost from one ecosystem and exported to another.
    \item Exported nutrients must be replaced by imports if ecosystems are to function.
    \item Mechanisms of nutrient replacement: ions from rocks are released due to weathering; wind or water transport; carbon via photosynthesis; nitrogen via nitrogen-fixing.
    \item Vegetation lowers the rate of nutrient export.
\end{itemize}

\subsubsection{Global Biogeochemical Cycles}
\begin{itemize}
    \item Human effects on the water cycle: asphalt and concrete limiting water absorption to deep soil; grasslands and forest converted to agricultural fields resulting in lost water in roots, more water off, and less percolates into groundwater; human consumption out pacing recharge rate.
    \item Human effects on the nitrogen cycle: cultivation of nitrogen fixing crops; industrially produced fertilizers, buringing of fossil fuels.
    \item Consequences of increased nitrogen: algal blooms in aquatic ecosystems, removing oxygen from the area; short term production increase at cost of diversity and long term productivity; acid rain, climate change, depletion of ozone layer.
    \item Humans have more than doubled the rate of nitrogen fixation.
    \item Humans have quadrupled the amount of phosphorus entering the global biogeochemical system. 
\end{itemize}
%\endgroup
%%%%%%%%%%%%%%%%%%%%%%%%%%%%% Chapter 53 %%%%%%%%%%%%%%%%%%%%%%%%%%%%
\end{document}