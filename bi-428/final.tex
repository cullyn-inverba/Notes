\documentclass[plain,basic]{inVerba-notes}

\newcommand{\userName}{Cullyn Newman}
\newcommand{\class}{BI:\@ 428}
\newcommand{\theTitle}{Human Genetics Final}
\newcommand{\institution}{Portland State University}

\begin{document}
\section{Week 6: Metabolic Pathways}
\minimal{\begin{enumerate}
  \item \textbf{Diagram and discuss the metabolic pathway for Tay-Sachs and Sandhoff Disease.}
  
  \begin{enumerate}
    \item Describe the phenotype, diagnosis and prognosis and look at the frequency of occurrence in various populations.
    
    \basec{
      \textbf{Phenotype:} Tay-Sachs is the classical phenotype of disease continuum based on the amount of residual \(\beta \)-hexosaminidase A (HEX A) enzyme activity~\cite{toro2020hexa}. It is characterized by progressive weakness, loss of motor skills beginning between ages three and six months, decreased visual attentiveness, and increased or exaggerated startle response with a cherry-red spot observable on the retina followed by developmental plateau and loss of skills after eight to ten months. Seizures are common by 12 months with further deterioration in the second year of life and death occurring between ages two and three years with some survival to five to seven years~\cite{toro2020hexa}.

      HEXB gene is referred to as Sandhoff disease, which is pretty close to TSD, but distinguished by Hepatosplenomegaly, skeletal abnormalities, deficiency of both HEX A \& HEX B enzyme activity~\cite{toro2020hexa}. 

      \textbf{Diagnosis:} Since the disease continuum is based on the amount of residual HEX A enzyme activity, then the molecular characteristics and impact are influenced by pathogenic variants. The disorder is typically divided into acute infantile, subacute juvenile, and late-onset disorders, with unique phenotypes common to each subset~\cite{toro2020hexa}.

      HEX A enzymatic activity, or specifically lack of activity, testing is done in the serum, white blood cells, or other tissues in the presence of normal or elevated activity of the \(\beta \)-hexosaminidase B (HEX B) enzyme to establish the diagnosis.

      Individuals with acute infantile TSD have none or extremely low HEX A enzymatic activity, while individuals with subacute juvenile or late-onset TSD have some minimal residual HEX A enzymatic activity~\cite{toro2020hexa}.

      \textbf{Prognosis:} as previously mentioned, the likely cause is due to disruption of the activity of an enzyme due to disruptions in the HEXA gene, disrupting a subunit of the enzyme responsible for the buildup of the molecule GM2 ganglioside within cells, leading to the destruction of nerve cells and spinal cord. Death often occurs in early childhood~\cite{fernandes2004tay}.

      \textbf{Prevalence:} TSD is very rare in the general population~\cite{fernandes2004tay}. Early incidence of TSD was estimated at 1:3,600~\cite{toro2020hexa}. However,  carrier rate for TSD is approximately 1:30 among Jewish Americans of Ashkenazi extraction~\cite{toro2020hexa}. Due to genetic counseling the incidence of TSD in the Ashkenazi Jewish population of North America has been reduced by greater than 90\%~\cite{toro2020hexa}.

      Other populations, such as French-Canadian communities of Quebec, the Old Order Amish community in Pennsylvania, and the Cajun population of Louisiana are at higher due to genetic isolation as well~\cite{fernandes2004tay}.
    }

    Points: [\qquad / \qquad ]

    Feedback: 

    \vspace*{50pt}
    \item Indicate the biochemical basis for the disease (i.e, the pathway).
    
    \basec{
      Tay-Sachs is a neurodegenerative disease resulting from mutations of the HEXA gene encoding the alpha subunit of beta-hexosaminidase, producing a destructive ganglioside accumulation in lysosomes, principally in neurons with the severity of the disease generally correlating with the level of residual Hex A activity~\cite{gravel1991biochemistry}.
    }

    Points: [\qquad / \qquad ]

    Feedback: 

    \vspace*{50pt}
    \item What is known about the gene/s and proteins involved – chromosomal location, structure of the gene and size, mutations known, protein structure and function.
    
    \basec{
      Lysosomal hexosaminidase occurs in two principal  forms,  Hex A and Hex B. Hex A is made up of one \(\alpha \) and one \(\beta \) sub-unit, while Hex B is made up of two \(\beta \) subunits~\cite{gravel1991biochemistry}. The \(\alpha \)-subunit is encoded by the HEXA gene on chromosome 15 and the \(\beta \)subunit by the HEXB gene on chromosome 5~\cite{gravel1991biochemistry}.

      HEXA and HEXB be have similar organizational structure; they are split into 14 exons spanning about 35 kb and 50 kb, respectively, and all but the first splice junction are located at identical positions in the aligned sequence~\cite{gravel1991biochemistry}.

      A variety of different HEXA mutations can ultimately cause the disease. TSD was one of the first genetic disorders with widespread screening and ended up being one of the first disorders in which prevalence of compound heterozygosity (condition of having two or more heterogeneous recessive alleles at a particular locus) was demonstrated.

      Most Tay-Sachs mutations probably do not directly affect protein functional elements (e.g., the active site). Instead, they cause incorrect folding (disrupting function) or disable intracellular transport~\cite{chelbi2000generalized}
  }

    Points: [\qquad / \qquad ]

    Feedback: 

    \vspace*{50pt}
    \item Any gene therapies?
    
    \basec{
      Research is ongoing to develop enzyme replacement therapy for Tay-Sachs disease, but has not proven successful in people with Tay-Sachs disease~\cite{nord}. The inability to find a way for the replacement enzyme to cross the blood-brain barrier, a protective networks of blood vessels and cells that allow some materials to enter the brain, while keeping other materials out is among the major problems with enzyme replacement therapy~\cite{nord}.

      There are some gene therapies also being studied. In these studies  the defective gene present in a patient is replaced with a normal gene to enable the production of active enzyme, likely leading to a cure if successful. However, there are still technical difficulties to resolve before gene therapy can succeed~\cite{nord}.

    }

    Points: [\qquad / \qquad ]

    Feedback: 

    \vspace*{50pt}
  \end{enumerate}

  \item \textbf{Diagram and discuss the metabolic pathway for Lesch-Nyhan Syndrome.}
  
  \begin{enumerate}
    \item Describe the phenotype, diagnosis and prognosis and look at the frequency of occurrence in various populations.
    
    \basec{
      \textbf{Phenotype:} Lesch-Nyhan syndrome (LNS) is at the most severe end of in the spectrum of \textit{HPRT1} disorders, with motor dysfunction resembling severe cerebral palsy, intellectual disability, and self-injurious behavior~\cite{jinnah2020hprt1}. It is caused by a deficiency of the enzyme hypoxanthine-guanine phosphoribosyltransferase (HGPRT). The HGPRT deficiency causes a build-up of uric acid in all body fluids. The combination of increased synthesis and decreased utilization of purines leads to high levels of uric acid production leading to varying degrees of neurologic and/or behavioral problems~\cite{jinnah2020hprt1}.

      \textbf{Diagnosis:} the diagnosis is established in a male families genetic history with suggestive clinical and laboratory findings and a hemizygous pathogenic variant in \textit{HPRT1} identified by molecular genetic testing and/or low HGprt enzyme activity identified on biochemical testing~\cite{jinnah2020hprt1}.

      \textbf{Prognosis:} The prognosis for individuals with LNS is poor. Death is usually due to renal failure in the first or second decade of life~\cite{page1995treatment}. Less sever forms of HGPRT deficiency have better prognoses. There is no standard treatment for LNS, but some symptoms can be managed temporarily.  

      \textbf{Prevalence:} the prevalence of LNS is approximately 1:380,000. The prevalence of the milder phenotypes (HND and HRH) is not well studied, but they appear to be less common than LNS~\cite{jinnah2020hprt1}. HPRT1 disorders occur in all populations that have been studied, and with relatively equal frequency~\cite{jinnah2020hprt1}.
    }

    Points: [\qquad / \qquad ]

    Feedback: 

    \vspace*{50pt}
    \item Indicate the biochemical basis for the disease (i.e, the pathway).
    
    \basec{
      LNS  disrupts the metabolism of the raw material of genes, specifically purines, which are essential part of DNA and RNA\@. The body can either make purines (de novo synthesis) or recycle them (the resalvage pathway)~\cite{nyhan1973lesch}.
      
      The product of the normal gene is the enzyme hypoxanthine-guanine phosphoribosyltransferase, which speeds up the recycling of purines from broken down DNA and RNA\@. Different types of mutations affect this gene, and the result is a very low level of the enzyme~\cite{nyhan1973lesch}.

      Low level of the enzyme results in failure to salvage the purines leading to accumulation of uric acid that normally would have been recycled into purines. The excess uric acid forms painful deposits in the skin (gout) and in the kidney and bladder (urate stones). Other physical manifestations occur due to neurological breakdown, such as unrestrained biting of fingers and tongues, mental retardation and severe muscle weakness~\cite{nyhan1973lesch}.

      Because a lack of HGPRT causes the body to poorly utilize vitamin B12, some males may develop megaloblatic anemia.~\cite{nyhan1973lesch}.
    }

    Points: [\qquad / \qquad ]

    Feedback: 

    \vspace*{50pt}
    \item What is known about the gene/s and proteins involved – chromosomal location, structure of the gene and size, mutations known, protein structure and function.
    
    \basec{
      The mutation is inherited in an X-linked fashion (location: Xq26.2-q26.3 according to OMIM). Females who inherit one copy of the mutation are not affected because they have two copies of the X chromosome (XX). Males are severely affected because they only have one X chromosome (XY), and therefore their only copy of the HPRT1 gene is mutated~\cite{nyhan1973lesch}.

      HPRT1 genes contain 9 exons, with a wide variety of mutations possible as previously mentioned. Deletions, insertions, single-base substitutions, and frame-shift mutations all can lead to decreased activity~\cite{nyhan1973lesch}. Essentially many types mutations in the HPRT1 gene can disrupt the function of the enzyme.
    }

    Points: [\qquad / \qquad ]

    Feedback: 

    \vspace*{50pt}
    \item Any gene therapies?
    
    \basec{
      There are no current gene therapies for Lesh-Nyhan syndromes, though many specific drug therapies can be used on a patient specific basis~\cite{nord2}. Treatment may even require the coordinated efforts of a team of specialists.

      Some individuals with Lesch-Nyhan syndrome have been reported to benefit from behavior modification techniques designed to reduce self-mutilating behaviors, but real success is unusual~\cite{nord2}.

      The drug allopurinol is used to control the excessive amounts of uric acid associated with Lesch-Nyhan syndrome and control symptoms associated with excessive amounts of uric acid. However, this treatment has no effect on the neurological or behavioral symptoms associated with this disorder~\cite{nord2}.

      No sustained treatment or drug therapy has proven uniformly effective for the treatment of the neurological problems associated with Lesch-Nyhan syndrome. Baclofen or benzodiazepines have been used to treat spasticity. Diazepam may be useful~\cite{nord2}.
    }

    Points: [\qquad / \qquad ]

    Feedback: 

    \vspace*{50pt}
  \end{enumerate}

\end{enumerate}
  
\newpage
\section{Week 8: Mapping and Genome Organization}
\begin{enumerate}
  \item \textbf{A synthesis of the Huntington’s Disease story}:
  
  \begin{enumerate}
    \item  Describe in detail the phenotype of this disorder.
    
    \basec{A progressive disorder of motor, cognitive, and psychiatric disturbances. Key signs of effected individuals is progressive motor disability featuring chorea, which is the involuntary, irregular, and unpredictable muscle movement. Changes in personality, cognitive decline, and depression are also very common~\cite{omim}. Symptoms are a result of progressive, selective neural cell loss and atrophy in the caudate and putamen~\cite{walker2007huntington}.}

    Points: [\qquad / \qquad ]

    Feedback: 

    \vspace*{50pt}
    \item Describe in detail the chromosomal location of Huntington’s Disease (HD) and how the location of this gene was worked. Give a detailed description of the process of positional cloning.
    
    \basec{
      Huntingtin (HTT;\@ 613004) on chromosome 4p16. Caused by an expanded trinucleotide repeat (CAG, encoding glutamine)~\cite{omim}.

      Pure ``positional cloning'' assumes no functional information and must locate the responsible gene solely on the basis of map position~\cite{collins1995positional}. Collins then outlines a different method called ``positional candidate'' approach, wherein the strategy relies on a combination of mapping to the correct chromosomal subregion, generally by use of linkage analysis, followed by a survey of the interval to see if potential candidates reside there~\cite{collins1995positional}.

      Typically, position involves the isolation of partially overlapping DNA segments from genomic libraries to progress along the chromosome toward a specific gene. However, HD did not have such genomic libraries and information.

      This lack of information helped birth the US-Venezuela Huntington's Disease Collaborative Project, where over a 10-year period from 1979, hundreds of scientists worked to locate the genetic cause of HD\@. 

      The causal gene was approximately located in 1983~\cite{gusella306watkins} then precisely located in 1993 on chromosome 4 (4p16.3) by use of haplotype analysis of linkage disequilibrium to spotlight the small segment responsible.~\cite{macdonald1993novel}. Here, they discovered the impact of the trinucleotide repeat (CAG repeat) in a coding region that was widely expressed leading to a dominant phenotype.


    }

    Points: [\qquad / \qquad ]

    Feedback: 

    \vspace*{50pt}
    \item Describe in detail what is known about the structure and function of the HD gene.
    
    \basec{
    \textbf{Penetrance/prevalence} Normal: 26 or fewer CAG repeats---Not at risk. Intermediate: 27 to 35---Not at risk, but has risk of having a child with a higher range due to mutations during germline reproduction. Pathogenic HD-causing alleles: 36 or more repeats~\cite{caron_2020}.
      \begin{itemize}
        \item \textbf{Reduced-penetrance}: 36--39 repeats. At risk, but may not develop symptoms. Elderly asymptomatic individuals are common in this range~\cite{caron_2020}.
        \item \textbf{Full-penetrance}: 40 or more. Chances of developing HD increases with both age and number of repeats. Mean age of onset is approximately 45 years~\cite{caron_2020}.
      \end{itemize}
  
    The fact that the phenotype of HD is completely dominant suggested that the disorder results from a gain-of-function mutation in which either the mRNA product or the protein product of the disease allele has some new property or is expressed inappropriately.~\shortcite{myers1982factors}
    
    The mutant huntingtin protein in HD results from an expanded CAG repeat leading to an expanded polyglutamine strand at the N terminus and a putative toxic gain of function. Neuropathologic studies show neuronal inclusions containing aggregates of polyglutamines.~\cite{walker2007huntington}
    
    In addition to Huntington disease, there are at least 8 other diseases of the central nervous system, each of which is known to be associated with a different protein containing an expanded polyglutamine sequence~\cite{caron_2020}.
    }

    Points: [\qquad / \qquad ]

    Feedback: 

    \vspace*{50pt}
    \item Develop the status of treatments and gene therapies. Be detailed here as well.
    
    \basec{
      Huntington's disease currently remains incurable and ultimately leads to fatal neurodegenerative disorders due to the CAG trinucleotide repeat from the huntingtin gene as described previously. However, there are been numerous gene therapies attempted, commonly using CRIPSR-Cas9, and I will briefly describe three of them here.

      First, a 2016 study demonstrates the power of a then novel strategy of inactivating the mutant huntingtin gene. Their goal was to focus on improving allele specificity inactivation for a given diplotype. They used two CRISPR-Cas9 guide RNAs that depend on Protospacer Adjacent Motif altering SNPs to target sites on the mutant alleles. They excised \approx44 kb of DNA spanning the promoter region, transcription start site, and CAG expansion mutation in the mutant HTT mRNA protein. This lead to a proof of concept strategy that unequivocally indicated permanent allele-specific inactivation of the HD mutant allele~\cite{shin2016permanent}. This was a big deal as perfect selectivity with broad applicability in diverse disease haplotypes support precision medicine through inactivation of other gain-of-function mutations.

      Next, a 2019 study used a small Cas9 ortholog packaged alongside a single guide RNA into a single adeno-associated virus vector in an attempt to disrupt the expression of mutant HTT gene in a mouse model. The authors found that the CRIPR-Cas9-mediated disruption resulted in \approx50\% decrease in neuronal inclusions and significantly improved lifespan and certain motor deficits~\cite{ekman2019crispr}. This study was an early demonstration of the potential for CRIPR-Cas9 to treat HD and other autosomal dominant neurodegenerative disorders caused by trinucleotide repeats using the emerging CRISPR-Cas9 applications. 

      Finally, a 2020 study reports an \textit{in vivo} cell conversion technology to reprogram striatal astrocytes into GABAergic neurons, as they are the neurons being lost in the huntingtin gene mutation and causing motor deficits. Using a mouse model, the authors report an astrocyte-to-neuron success rate of 80\% in the straitum and >50\% DARPP32\(^+\) medium spiny neurons~\cite{wu2020gene}. These converted neurons displayed action potential and synaptic events, and projected their axons to the targeted regions in a time-dependent manner; these results show that the not only were the neurons converted, but they were functional as well. Behavioral analyses also showed significant extension of life span and improvement of motor functions~\cite{wu2020gene}. This study was important as showed a successful application of \textit{in vivo} AtN conversion, implicating it another potential gene therapy to treat Huntington's disease and other neurodegenerative disorders.
    }

    Points: [\qquad / \qquad ]

    Feedback: 

    \vspace*{50pt}
  \end{enumerate}

 \end{enumerate}

\newpage
\section{Week 10: Ubiquitin}
\begin{enumerate}
  \item Describe the cellular components and their functions of ubiquitin.
  
  \basec{
      Ubiquitin, named for its ubiquitous finding in eukaryotic organisms, is a small regulatory protein that, when applied to a substrate protein called ubiquitlyation, affects proteins in a variety of ways; proteins can be marked for degradation, have their cellular location altered, change in activity, and prevent interactions with other proteins.

      The protein is small and performs its functions through conjugation to a large range of target proteins. Ubiquitin consists of only 76 amino acids and has a small molecular mass of about 8.6 kDa. Key features is the C-terminal tail and the 7 lysine residues. It's coded by 4 different genes in mammals, UBA52 and RPS27A for a single copy fused to ribosomal proteins and UBB and UBC for polyubiquitin precursor proteins. 

      Ubiquitination requires three types of enzyme: ubiquitin-activating enzymes, ubiquitin-conjugating enzymes, and ubiquitin ligases, known as E1s, E2s, and E3s, respectively. 

      1. \textbf{Activation} via two-step reaction by an E1 ubiquitin-activating enzyme, which is dependent on ATP, and involves the production of an ubiquitin-adenlyate intermediate. 

      2. \textbf{Conjugation} via E2 ubiquitin-conjugating enzymes catalyze the transfer of ubiquitin from E1 to the active site cysteine of the E2 via a trans(thio)esterification reaction. 

      3. \textbf{Ligation} via E3 ubiquitin ligases that catalyze the final step of the ubiquitination cascade by creating an isopeptide bond between a lysine of the target protein and the C-terminal glycine of ubiquitin. 
      
      There are also E4 enzymes, or ubiquitin-chain elongation factors, that are capable of adding pre-formed polyubiquitin chains to substrate proteins.

      The various levels of the cascade allow for tight regulation of the ubiquitlyation processes, with some variation existing. 

      Different types of substrate ubiquitlyation lead to various regulation of cellular processes, ranging from monoubiquitination (resulting in cellular processes such as membrane trafficking, endocytosis and viral budding) to polyubiquitination which can help regulate a large variety of functions. Overall, the system functions in a wide variety of processes, including: 
      
      \begin{multicols}{2}
      \begin{itemize}
        \item Antigen processing
        \item Apoptosis
        \item Biogenesis of organelles
        \item Cell cycle and division
        \item DNA transcription and repair
        \item Differentiation and development
        \item Immune response and inflammation
        \item Neural and muscular degeneration
        \item Maintenance of pluripotency
        \item Morphogenesis of neural networks
        \item Modulation of cell surface receptors, ion channels and the secretory pathway
        \item Response to stress and extracellular modulators
        \item Ribosome biogenesis
        \item Viral infection
      \end{itemize}
    \end{multicols}
    }

    Points: [\qquad / \qquad ]

    Feedback: 

    \vspace*{50pt}

  \item Find and describe a cancer related to malfunction of ubiquitin. Be thorough in your description---include things like the cause of the cancer, the consequences of that cancer and whether there are any therapies that are being considered.
  
  \basec{
    Many reviews have covered the role of ubiquitination in the development of cancer, including both tumor-suppressing and  tumor-promoting path-ways are regulated by ubiquitination depending on the nature of the substrate, a single ubiquitin ligase can act as both an oncogene and a tumor suppressor~\cite{popovic2014ubiquitination}.
    
    Ubiquitin ligases can affect oncogenesis in several ways, as they are able to change both fate and function of their substrates. Since proteins that are marked by ubiquitination are often trafficked to the proteasome or lysosome for degradation, then control over the cell cycle can be regulated, like in the ligase MDM2, which ubiquitinates the tumor suppressor p53 or the SCF and APC/C ligase complexes, which control cell cycle progression and mitotic exit10 by controlling the stability of cyclin–cyclin-dependent kinase (CDK) complexes as well as of the cell cycle inhibitor p27~\cite{popovic2014ubiquitination}. Another example cited is of mutations or aberrant expression of the E3 ligase c-Cbl have  been linked to primary colorectal cancer~\cite{popovic2014ubiquitination}.

    The review previously cited covered a wide range of examples, and provides w excellent table that indicates the targeted protein or pathway, the Ubiquitin chain type affected, and the resulting disease. There is almost too many examples to choose from, but I'll focus on the MDM2 E3 ligase, which as I previously described affects the p53 protein and thus affects the function of tumor suppression. The chain type affected is Lys48, which I did not cover in the previous chain type, but is indicated to be involved in protein degradation~\cite{jacobson2009lysine}. Of course, failure to suppress tumors and failure of protein degradation would result in cancer of various types depending on location. 
  
    The review previously cited covered a wide range of examples, and provides w excellent table that indicates the targeted protein or pathway, the Ubiquitin chain type affected, and the resulting disease. There is almost too many examples to choose from, but I'll focus on the MDM2 E3 ligase, which as I previously described affects the p53 protein and thus affects the function of tumor suppression. The chain type affected is Lys48, which I did not cover in the previous chain type, but is indicated to be involved in protein degradation~\cite{jacobson2009lysine}. Of course, failure to suppress tumors and failure of protein degradation would result in cancer of various types depending on location. 
  
    Therapy for this example is possible, since Ubiquitin ligases are crucial players that ensure substrate specificity as well as the spatiotemporal regulation of ubiquitination events, then the MDM2-mediated ubiquitination can selectively inhibit the small molecules HL198 or JNJ-26854165, resulting in p53 stabilization during developmental stage 1~\cite{popovic2014ubiquitination}. This results in the induction of tumor cell death and inhibition of tumor growth, as shown in preclinical studies; however, further studies are required to fully assess the pharmacokinetic activity of HL198 in cancer~\cite{popovic2014ubiquitination}.

    }

    Points: [\qquad / \qquad ]

    Feedback: 

\end{enumerate}

}

\nocite*{}
\bibliographystyle{apacite}
\bibliography{final.bib}
\end{document}