\chapter{18: Nuclear Magnetic Resonance Spectroscopy}\label{18: Nuclear Magnetic Resonance Spectroscopy}
\section{NMR Basics}\label{NMR Basics}
\begin{itemize}
    \item \ddd{NMR spectroscopy}: a technique to observe local magnetic fields around atomic nuclei using excitation of nuclei with radio waves.
      \begin{itemize}
        \item Radio waves triggers nuclear magnetic resonance, which changes resonance frequency, giving access to detail of the electronic structure of a molecule and its individual functional groups.
      \end{itemize}
    \item NMR usually involves three sequential steps:
      \begin{itemize}
        \item Alignment (polarization) of magnetic nuclear spins in an applied, constant magnetic field \(B_0\).
        \item Perturbations of the nuclear spins by weak oscillating magnetic fields, usually referred to as a radio-frequency (RF) pulse.
        \item Detection and analysis of electromagnetic waves emitted by the nuclei of the sample as a result of the perturbation.
      \end{itemize}
    \item Organic chemist are usually limited to proton (\(^1_1H\)) and carbon-13 (\(^{13}_6C\)), but can be used on more.
\end{itemize}

\section{Proton NMR Spectroscopy}\label{Proton NMR Spectroscopy}
\begin{itemize}
    \item \ddd{Proton NMR}: specific application of NMR spectroscopy with respect to hydrogen-1 nuclei within the molecule of a substance in order to determine structure.
      \begin{itemize}
        \item There are four main characteristics analyzed in order to determine the structure, signal quantity, location (shift), area (integration), and shape (spin-spin).
      \end{itemize}
    
    \subsection{Signal Quantity}\label{Signal Quantity}
    \begin{itemize}
        \item The number of signals identifies the number of different types of protons in the sample.
        \item \ddd{Chemically equivalent proton}: protons that share the same chemical environment.
          \begin{itemize}
            \item \ch{CH3}, \ch{CH2}, and \ch{CH} will always be non-equivalent to each other.
            \item Protons are chemically equivalent if they are equivalent via symmetry considerations.
          \end{itemize}
        \item Differences in chemical environment will show up as the same peak.
          \begin{itemize}
            \item E.g., ethane has 6 hydrogens that all have the same chemical environment, while butane has two different chemically equivalent groups \(2\times \)(\ch{CH3,~CH2})
          \end{itemize}
    \end{itemize}

    \subsection{Signal Location}\label{Chemical Shifts}
    \begin{itemize}
        \item \ddd{Chemical shift} \(\delta \): the resonant frequency of a nucleus relative to a standard in a magnetic field, i.e., \emph{where} the signal shows up on the \emph{x-axis} of the NMR spectrum.
          \begin{itemize}
            \item \(\delta = \dfrac{\nu_{sample}-\nu_{ref}}{\nu_{ref}}\)
            \item \(\nu_{sample}\): absolute resonance frequency of the sample.
            \item \(\nu_{ref} \): absolute resonance frequency of a standard reference compound in the same applied magnetic field \(B_0\).
            \item \(\delta \) expressed in parts per million.
          \end{itemize}
        \item Chemical shift can have deviations; exact value depends on molecular structure of solvent, temperature, magnetic field, and neighboring functional groups.
          \begin{itemize}
            \item Tells us about the chemical or electrical environment in regard to proton NMR\@.
          \end{itemize}
        \item NMR plots have an x-axis of 14--0 ppm.
          \begin{itemize}
            \item Closer to 0 is the ``upfield.''
            \item Closer to 14 is the  ``downfield.''
            \item Comparisons are relative. 
          \end{itemize}
        \item \ddd{Shielding}: an opposed magnetic field (\(B_i\)) to an applied field (\(B_0\)) induced by circulating electrons from adjacent bonds and atoms. 
          \begin{itemize}
            \item \(B = B_0 - B_i\)
            \item \ddd{Shielded}: when nucleus of interest is surrounded by high electron density, making it more \emph{upfield}.
            \item \ddd{Deshielded}: the inverse of shielded; when electron density is pulled away from the nucleus, making if more \emph{downfield}. 
          \end{itemize}
        \item Increased electronegativity of nearby atoms decreases shielding (increasing deshielding), thus making the chemical shift more downfield.
        \item Inductive effect decreases with distance, decreasing degree of chemical shift.
        \item Electron donating groups: OH, \ch{OCH3}, R, \ch{NR2}.
        \item Electron withdrawing groups: carbonyl groups (must be attached to ring), nitro groups.
    \end{itemize}
    
    \subsection{Signal Integration}\label{Signal Integration}
    \begin{itemize}
        \item NMR plots give the number of hydrogens responsible for a given peak by using an integration curve (area under the curve). 
    \end{itemize}

    \subsection{Signal Shape}\label{Signal Shape}
    \begin{itemize}
        \item The splitting pattern of signal peaks reveal how many hydrogens neighbors exist for a particular group of equivalent hydrogens.
        \item In general, splitting will result in \(N + 1\) peaks, where \(N\) is the number of hydrogens on the adjacent atom(s).
        \item \textbf{Singlet}: no hydrogens on adjacent atom.
        \item \textbf{Doublet}: one hydrogen on adjacent atom, resulting in two peaks of equal size.
        \item \textbf{Triplet}: two hydrogens on adjacent atom, with three peaks with an area ratio of 1:2:1.
        \item \textbf{Quartet}: three hydrogens on adjacent atoms, with four peaks with an area ratio of 1:3:3:1
        \item A peak is split by \(n\) identical protons into components whose sizes are the ratio of the \(n\)th row of Pascal's triangle.
        
    \end{itemize}  
\end{itemize}

\begin{center}\large
          
  \bigskip
  \begin{tikzpicture}
    \foreach \n in {0,...,7} {
      \foreach \k in {0,...,\n} {
        \node at (\k-\n/2,-\n) {$\binomialCoefficient{\n}{\k}$};
        }
        }
  \end{tikzpicture}

\end{center}