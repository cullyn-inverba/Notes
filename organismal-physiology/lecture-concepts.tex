\documentclass[12pt,a4paper]{article}
\usepackage{inverba}

\newcommand{\userName}{Cullyn Newman} 
\newcommand{\class}{BI 320} 
\newcommand{\institution}{Portland State Univeristy} 
\newcommand{\theTitle}{\color{g-Leaf} Organismal Physiology Lecture Concepts}

\begin{document}
%%%%%%%%%%%%%%%%%%%%%%%%%%%%%%%%%%%%%%%%%%%%%%%%%%%%%%%%%%%%%%%%%%%%%
\tableofcontents
\cleardoublepage
\fancyhead{}
\fancyhead[R]{\hyperlink{home}{\nouppercase\leftmark}}
%%%%%%%%%%%%%%%%%%%%%%%%%%%%%%%%%%%%%%%%%%%%%%%%%%%%%%%%%%%%%%%%%%%%%

%\begingroup
%%%%%%%%%%%%%%%%%%%%%%%%%%%%% Chapter 1 %%%%%%%%%%%%%%%%%%%%%%%%%%%%%
%\begingroup
\clearpage
\section*{Midterm I}\phantomsection
\addcontentsline{toc}{section}{\textbf{Midterm I}}
\fancyhead[R]{\hyperlink{home}{Week 1}}
\fancyhead[L]{\hyperlink{home}{Midterm I}}
\subsection{Week 1}
\begin{itemize}
    {\color{G-Moon}\item What are the two central questions of physiology? }
        \begin{itemize}
            \item The mechanism, which refers to how an organism functions.
            \item The origin, which asks what the adaptation significance of a mechanisms is.
        \end{itemize}
    {\color{G-Moon}\item Describe the different sub-disciplines of physiology.}
        \begin{itemize}
            \item Mechanistic(functions)
            \item Comparative(diversity)
            \item Evolutionary(origins)
            \item Environmental(ecology/physiology)
            \item Integrative(whole systems).
        \end{itemize}
    {\color{G-Moon}\item What is the Krogh principle? Explain what it means to physiology.}
        \begin{itemize}
            \item Limited number of animals of convenient study, but enough of the problems can be studied using comparative methods.
            \item There is unity in diversity; the differences allow for specialized studies of complex systems due to niche environments, while similarities allow for studies of functions on animals we otherwise couldn't study.
        \end{itemize}
    {\color{G-Moon}\item Define the terms conformity and regulation and understand how they relate to physiological processes. Discuss examples.}
        \begin{itemize}
            \item Conformity correlates passively with external changes.
            \item Regulation holds constant, using cellular mechanisms.
        \end{itemize}
    {\color{G-Moon}\item Define the term homeostasis. Understand the process of negative and positive feedback regulation.}
        \begin{itemize}
            \item Homeostasis: a coordinated process that maintains a relatively constant state though use of sensors and effectors.
            \item Positive feedback: less common, self reinforcing, difficult to regulate, powerful.
            \item Negative feedback: more common, self correcting. 
        \end{itemize}
    {\color{G-Moon}\item Describe how physiology changes with time in response to the external environment. Define the terms acclimation, adaptation, natural selection, and evolution.}
        \begin{itemize}
            \item Acute: short-term, reversible, and quick to response to the environment, typically minutes--hours.
            \item Chronic: long term due to prolonged/repetitive exposure, sometimes reversible, can lead to acclimation.
            \item Evolution acts on changing allele frequencies of a population overtime.
            \item Natural selection acts to remove the least fit individuals, which influences allele frequencies.
            \item Adaption is trait that allows for greater fitness compared to other individuals in competition.
        \end{itemize}
    {\color{G-Moon}\item Describe the structure and chemical components of lipid membranes.}
        \begin{itemize}
            \item Glycoproteins and glycolipids: carbohydrate chains attached to protein or lipids, respectivley.
            \item Glycocalyx: combination of both.
            \item Integral proteins: embedded in phospholipid bilayer.
        \end{itemize}
    {\color{G-Moon}\item What are the effects of low and high temperatures on membranes? How are membrane properties altered to offset these effects?}
        \begin{itemize}
            \item Higher temperature produces more brownian motion which increases membrane fluidity. 
            \item Membranes can have a different proportion of unsaturated/saturated hydrocarbon tails; the more double bounds (unsaturated) creates more fluidity at colder temperatures; more saturation increases stability at higher temperatures.
            \item Cholesterol acts to both stabilize membrane at higher temperatures and helps increases fluidity at lower temperatures.
        \end{itemize}
    {\color{G-Moon}\item Describe the five functional types of membrane proteins and their basic functions.}
        \begin{itemize}
            \item Channels: generall for water and used in osmosis. \item Tranporters: used in active transport and facilitated diffusion.
            \item Enzymes: decreases activation energy of reactions.
            \item Receptors: mediate responses to chemical messages.
            \item Structural proteins: used to attached to other molecules, effect cellular flexibility, and other structural relations.
        \end{itemize}
    {\color{G-Moon}\item What are the two primary roles of enzymes?}
        \begin{itemize}
            \item Accelerating and regulation chemical reactions.
        \end{itemize}
    {\color{G-Moon}\item Define the terms V\(_{\text{max}}\) and \(K_m\). Explain the factors that affect these reaction properties.}
        \begin{itemize}
            \item V\(_{\text{max}}\): maximum velocity of reaction, determined by relative quantity of enzymes to substrate and catalytic effectiveness of each molecule.
            \item K\(_{m}\): half-saturation constant, inversely related with enzyme-substrate affinity. 
        \end{itemize}
    {\color{G-Moon}\item Define activation energy. Understand the effect of enzyme catalysis on a reaction’s energy of activation.}
        \begin{itemize}
            \item The energy required for the substrate to enter the transition  (intermediate state between substrate and product).
            \item Enzymes lower the activation energy required.
        \end{itemize}
    {\color{G-Moon}\item What are the effects of substrate concentration on the rate of an enzymatic reaction? How does enzyme-substrate affinity affect the reaction rate?}
        \begin{itemize}
            \item Substrate concentration can increase reaction speed only relative to enzyme concentraion; recactions become saturated when all enzymes are occupied by a substrate, further concentraion increases would have no effect.
            \item Affinity affects the shape of the reaction velocity; an rate correlated with affinity.
        \end{itemize}
    {\color{G-Moon}\item Why are enzymatic rates unresponsive to increases in substrate concentration above a physiologically relevant range?}
        \begin{itemize}
            \item Physiologically relevant range: physiologically possbile (outer limits) or normal bodily concentrations.
            \item Enzymes are only functional in relevant ranges mainly due to selection pressure not acting on unusual ranges.
            \item Increases in substrates thus would have not any effect if enzymes are not functional.
        \end{itemize}
    {\color{G-Moon}\item Understand why conformational change is a critical part of enzyme function.}
        \begin{itemize}
            \item The shape of proteins deterins function, so changing the shape is critical for enzymes to influence reactions.
        \end{itemize}
    {\color{G-Moon}\item Define the term isozyme and understand how they can contribute to natural selection.}
        \begin{itemize}
            \item Isozyme: enzymes that catalyze the same reaction, but differ in amino acid sequence.
            \item The change in sequence can lead to slight differences in function, which selection pressure can act on, increasing the differences.
        \end{itemize}
    {\color{G-Moon}\item Define and understand the process of allosteric modulation.}
        \begin{itemize}
            \item Alosteric modulation: the modulation of catalytic properties.
            \item Allosteric activation and inhibition can increase or decreases affinity, and thus catalytic activity.
        \end{itemize}
\end{itemize}
%\endgroup
%%%%%%%%%%%%%%%%%%%%%%%%%%%%% Chapter 1 %%%%%%%%%%%%%%%%%%%%%%%%%%%%%

%%%%%%%%%%%%%%%%%%%%%%%%%%%%% Chapter 2 %%%%%%%%%%%%%%%%%%%%%%%%%%%%%
%\begingroup
\clearpage
\fancyhead[R]{\hyperlink{home}{Week 2}}
\subsection{Week 2}
\begin{itemize}
    {\color{G-Moon}\item Define the terms transcription, translation, and post-translational processing. Understand the differences between nRNA and mRNA and introns and exons.}
        \begin{itemize}
            \item Transcription: the process of turning DNA into mRNA.
            \item Translation: the process of turning mRNA into sequences of amino acids for protein synthesis.
            \item nRNA: first product of transcription, before introns have been spliced out, leaving just the exons for the mRNA.
            \item Post-translational processing: covalent and enzymatic modification (folding, cutting) of proteins after translation.
        \end{itemize}
    {\color{G-Moon}\item Understand how to interpret information about the origin of physiological traits from a phylogenetic tree.}
    {\color{G-Moon}\item Define the terms genome and genomics. Describe the methods, challenges, and major goals of genomics research.}
        \begin{itemize}
            \item The study of genomes---the full set of genetic material---of organisms.
            \item Mainly uses computational biology that uses various high-throughput methods (analysis of large amounts of data).
            \item Major goal: elucidate the evolution of current functioning of genes and genomes.
        \end{itemize}
    {\color{G-Moon}\item Describe an example for each major mechanism of gene modification, e.g. mutation accumulation, deletions, gene duplication.}
        \begin{itemize}
            \item Gene deleteion: disturbing the function of animal's genes to identify function by revealing deficient phenotypes.
            \item Duplication: forced overexpression, inverse of gene deletetion.
            \item RNAi: specific mRNA silencing, then compared. 
            \item CRISPR/Cas: insertion/editing of mRNA that are then tranlasted. 
        \end{itemize}
    {\color{G-Moon}\item What does the phrase “from genotype to phenotype” mean? What are the limitations associated with this phrase?}
        \begin{itemize}
            \item Refers to mapping genes with their functions. 
            \item Very difficult as there are complex post-translational effects and epigenetics.
        \end{itemize}
    {\color{G-Moon}\item Define the terms transcriptome and transcriptomics. Describe the methods and challenges of transcriptomics research. How can the function of a gene’s expression be tested?}
        \begin{itemize}
            \item Transcriptomics: study of which genes are transcribed to make mRNA and the rates at which they are transcribed.
            \item Transcriptome: full set of mRNA molecules that represent the full complement of genes being transcribed; useful for comparative methods.
        \end{itemize}
    {\color{G-Moon}\item Define the terms proteome and proteomics. Why is proteomics treated as a separate discipline rather than being lumped together with genomics and transcriptomics?}
        \begin{itemize}
            \item Study of proteins being synthesized by cells and tissues.
            \item Protein folding is increadily complex; incredibly large number proteins to simultaneously study.
        \end{itemize}
    {\color{G-Moon}\item What is two-dimensional gel electrophoresis? What kinds of data does it generate? How is it used in proteomics research?}
        \begin{itemize}
            \item Two-dimensional gel: primary method of proteomics that separates complex mixtures of proteins, based on isoelectric points and molecular weights.
        \end{itemize}
    {\color{G-Moon}\item Define the term metabolomics. How does it differ from the other “omics” disciplines?}
        \begin{itemize}
            \item Study or organic compounds in cell other than the macromolecules coded by the genome. 
        \end{itemize}
    {\color{G-Moon}\item Define the term epigenetics. Are epigenetic changes heritable from cell to cell? From parents to offspring? Explain.}
        \begin{itemize}
            \item Epigenetics: modification of gene expression with no change in DNA sequence that are transmitted when genes replicate.
            \item Heritable from cell to cell (mitotic) and (meiotic) 
        \end{itemize}
    {\color{G-Moon}\item Identify the two major mechanisms of epigenetic change and their consequences on gene transcription.}
        \begin{itemize}
            \item DNA methylation: addition of methyl groups that generally silences particular genes.
            \item Histone modification: modifies histones that make DNA more/less accessbile for transcription. 
                \begin{itemize}
                    \item Methylation, acetylation, phosphorylation, or other covalent bonding of chemical groups all play specific roles.
                    \item Also small RNA molecules helps perpetuate changes.
                \end{itemize}
        \end{itemize}
\end{itemize}
%\endgroup
%%%%%%%%%%%%%%%%%%%%%%%%%%%%% Chapter 2 %%%%%%%%%%%%%%%%%%%%%%%%%%%%%

%%%%%%%%%%%%%%%%%%%%%%%%%%%%% Chapter  %%%%%%%%%%%%%%%%%%%%%%%%%%%%%
%\begingroup
\clearpage
\fancyhead[R]{\hyperlink{home}{Week 3}}
\subsection{Week 3}
\begin{itemize}
    {\color{G-Moon}\item Define and describe the different types of passive and active solute transport. Understand how and why they differ from each other.}
        \begin{itemize}
            \item Passive transpot: movement of solutes towards equilibrium.
                \begin{itemize}
                    \item Simple diffusion: high solute concentration $\rightarrow$ low solute concentration.
                    \item Faciliated diffusion: solute that bind reversibly to tranporter proteins that allow for faster transport that simple diffusion; occurs in direction of electrochemical equilibrium.
                    \item Temperatue has more of an effect on facilitated diffusion.
                \end{itemize}
            \item Active transport: move molecules/solutes against a concentraion gradient using cellular energy.
                \begin{itemize}
                    \item Primary: uses protein pumps to transport mostly metal ions and normally uses ATP.
                    \item Secondary: uses potential energy through use of transporters to effect the electrochemical gradient.
                \end{itemize}
        \end{itemize}
    {\color{G-Moon}\item Define: equilibrium, concentration gradient, electrical gradient, electrochemical gradient, electrogenic, and electroneutral. }
        \begin{itemize}
            \item Equilibrium: state of minium capacity to do work under locally prevailing conditions.
            \item Cencentration gradient: difference in concentration between two solutions separated by a semipermeable membrane.
            \item Electrical gradient: differences in charge.
            \item Electrochemical gradient: combination of electrical and chemical gradient.
            \item Electrogenic: produces a change in the electrical potential of a cell.
            \item Electroneutral: no net change on electric charge.
        \end{itemize}
    {\color{G-Moon}\item Understand the forces imparted on ion movement as a result of reinforcing and opposing electrochemical gradients.}
        \begin{itemize}
            \item Chemical gradients influence electrial gradients and vice versa; interaction forms an electrochemical gradient. 
            \item Ions can be moved around with active transport to reinforce or oppose gradient in order to influence passive transport.
        \end{itemize}
    {\color{G-Moon}\item Define the Fick diffusion equation and be able to use it to calculate a rate of diffusion. }
        \begin{itemize}
            \item  \(J=D\dfrac{C_1 - C_2}{X}\)
            \item \(J\) is the net number of solute molecules passing into the low-concentration region from the high-concentration of solute particles, making it a colligative property.
            \item \textbf{Diffusion coefficient (\textit{D})}: proportionality factor determined by the permeability of the membrane or epithelium as well as the temperature.
        \end{itemize}
    {\color{G-Moon}\item What are the major limitations of diffusion as a transport mechanism? }
        \begin{itemize}
            \item Ion channels: integral membrane protein that permit passive transport; can be gated, allowing for responsive conformational changes.
            \item Permeability: ease at which solute can move through membrane by diffusion.
            \item Always moves towards equilibrium, so energy must eventually by used to alter gradients to allow diffusion to continue to do work.
        \end{itemize}
    {\color{G-Moon}\item What is a boundary layer and how does it impact diffusion?}
        \begin{itemize}
            \item The semipermeable layer between gradients.
            \item Ion channels and temperature can change permeability of boundary layer.
            \item Thickeness of decreases ease of diffusion.
            \item Charge and size of ions and molecules impact rates of diffusion.
        \end{itemize}
    {\color{G-Moon}\item Describe the basic characteristics of the electrochemical environment for a typical animal cell. \ch{Na+} and \ch{K+} ion channels and the \ch{Na+}--\ch{K+} ATPase pump affect/regulate these characteristics?}
        \begin{itemize}
            \item Common enzymes use the sodium-potassium pump that moves three {\color{pos}\ch{Na^+}} out for every two {\color{pos}\ch{K^+}} moved into the cell. 
            \item This can drastically change the electrochemical environment, often drastically impacting rates of diffusion.
        \end{itemize}
    {\color{G-Moon}\item Describe the different types of gated ion channels and their function in ion transport.}
        \begin{itemize}
            \item Voltage-gated: responds to changes in charge.
            \item Stretch-gated: responds to changes in tension.
            \item Phosphorylation-gated: responds to changes in protein phosphorylation.
            \item Ligan-gated: responds to various extracelluar signaling.
        \end{itemize}
    {\color{G-Moon}\item Understand the concept of potential energy stored in an electrochemical gradient. }
        \begin{itemize}
            \item Cations usually move down the electrochemical gradient using ATP to generate potential energy for passive transport.
        \end{itemize}
    {\color{G-Moon}\item What are cotransporters and countertransporters?}
        \begin{itemize}
            \item Cotransporters (symporter): two substrates are transported together in the same directions.
            \item Countertransporters (antiporter): one substrate is transported across while the other in tranporter in the opposite direction.
        \end{itemize}
    {\color{G-Moon}\item What are the major mechanisms that generate diversity in transporters and/or modulate their function? Provide examples for each.}
        \begin{itemize}
            \item The molecular form can vary considerably which allow for modulation of function and efficiency.
            \item Channels can be modulated through a lifetime via gene expression responses to environmental circumstances.
            \item Ligand (typically noncovalent) and phosphorylation (covalent) allow for rapid regulation.
            \item Insertion-and-retrieval: the location of proteins, and the inerting of removing from the membrane, can allow for relatively quick modulation.
        \end{itemize}
    {\color{G-Moon}\item What are colligative properties? List and define the major colligative properties important to physiology.}
        \begin{itemize}
            \item Colligative properties: properties of solutions that depend on the ratio between solute particles and solevent molecules.
            \item Vapour pressure: lowered when non-volatile solute is dissolved.
            \item boiling and freezing points: additions of solutes increases and decreases points respectivley; both are proportional to lowering of vapour pressure.
            \item Osmotic pressure: external pressure required to be applied so that there is no net movement of a solvent across the membrane. "Water wants to go where solutes are"
        \end{itemize}
    {\color{G-Moon}\item Define osmosis and understand how osmotic pressure is generated and measured.}
        \begin{itemize}
            \item Osmosis: primary means by which water in transported into and out of cells using transport of solvent through semipermable membrane towards region of higher solute concentration.
            \item Formula for osmotic pressure: \(K\dfrac{\Pi_1 - \Pi_2}{X}\)
            \item i.e., the rate at which water crosses the membrane by osmosis.
            \item Similar to the Fick equation for concentration gradient, except $\Pi_{1\&2}$ are the osmotic pressures of the solutions on each side of the membrane, and \(K\) is the osmotic permeability of the membrane + temperature.
        \end{itemize}
    {\color{G-Moon}\item Define the terms hyperosmotic, isosmotic, and hyposmotic.}
        \begin{itemize}
            \item Isosmotic: when two solutions have the same osmotic pressure.
            \item Hyperosmotic: when solution A has less solute than B
            \item Hyposmotic: when B has more solutes than A.
            \item The direction of net water movement by osmosis is from hyposmotic solution into the hyperosmotic one, i.e., \(A \rightarrow B\) 
        \end{itemize}
    {\color{G-Moon}\item Understand the different effects electrolytes and organic molecules have on osmotic pressure.}
        \begin{itemize}
            \item Nonpermating solutes create persistent osmotic-gradient components across semipermable membranes.
            \item Some solute can be dragged along with water, which alters electrochemical gradients, playing a continuous role in rates of passive transport.
            \item Active transport provdes means of indirect control of the strictly passive water tranport of osmosis through changes in osmotic pressure.
        \end{itemize}
    {\color{G-Moon}\item Describe the mechanisms of water transport.}
        \begin{itemize}
            \item Aquaporins: main mechanism that uses water channel proteins that greatly increase water transport.
        \end{itemize}
    {\color{G-Moon}\item How is cell volume affected by its environment?}
        \begin{itemize}
            \item Controlled by osmoregulation, which controls osmolarity, volume, and ionic regulation, all of which have effects on each other.
            \item Environment change change external pressure and electrochemical gradients, both of which have multiple effects and ultimately cell volume.
        \end{itemize}
\end{itemize}
%\endgroup
%%%%%%%%%%%%%%%%%%%%%%%%%%%%% Chapter  %%%%%%%%%%%%%%%%%%%%%%%%%%%%%
%\endgroup
\end{document}