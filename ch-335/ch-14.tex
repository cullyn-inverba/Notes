% chktex-file 36

\chapter{14: Alkenes}\label{14: Alkenes}
\section{Alkenes Basics/Review}\label{Alkenes Basics/Review}
\begin{itemize}
  \item Alkanes vs alkenes: 
  \medskip
  \begin{center}
    \schemestart{}
      \chemname{\chemfig{H-C(-[::+90]H)(-[::-90]H)-C(-[::+90]H)(-[::-90]H)-H}}{Saturated\\alk\emph{anes} eth\emph{ane}}
      \qquad
      \chemname{\chemfig{C(-[::150]H)(-[::-150]H)=C(-[::+30]H)(-[::-30]H)}}{Unsaturateed\\alk\emph{enes} eth\emph{ene}}
      \qquad
      \chemname{\chemfig{-[:30]-[:-30]}}{prop\emph{ane}}
      \qquad
      \chemname{\chemfig{-[:30]=[:-30]}}{prop\emph{ene}}
    \schemestop{}
  \end{center}
  \bigskip

  \item We will fully investigate the \hyperref[Nomenclature of Alkenes]{\dlink{nomenclature of alkenes}} later.
  \item Some relevant facts to start:
    \begin{itemize}
      \item \ddd{Alkene}: a hydrocarbon that contains a carbon–carbon double bond.
        \begin{itemize}
          \item IUPAC recommends using alkene for only acyclic hydrocarbons with just one double bond; an \ddd{olefin} is a hydrocarbon with one or more double bonds.
        \end{itemize}
      \item Generally prepared through beta elimination, which results in the formation of alkenes from alkanes.
    \end{itemize}
      
  \subsection{Types of Alkenes}\label{Types of Alkenes}
  \begin{itemize}
    \item Basic types of alkenes:
    
    \schemestart{}
      \chemname{\chemfig{=[:30,,,,emph]-[:-30]-[:30]}}{Terminal Alkene}
      \qquad
      \chemname{\chemfig{-[:30]=[:-30,,,,emph]-[:30]}}{Internal Alkene}
      \qquad
      \chemname{{\footnotesize\chemfig{*6(---=[,,,,emph]--)}}}{Cyloalkene}
    \schemestop{}
    \bigskip

    \item Types of terminal alkenes:  
    
    \schemestart{}
      \chemname{\chemfig{R=(-[:45]H)(-[:-45]H)}}{Methylene}
      \qquad
      \chemname{\chemfig{R-=[:-45]}}{Vinyl}
      \qquad
      \chemname{\chemfig{R--[:-45]=}}{Allyl}
    \schemestop{}
    \bigskip

    \begin{itemize}
      \item ``R'' always tells you it's a carbon containing functional group, or hydrogen. 
      \item ``A'' can be used to represent any functional group. 
    \end{itemize}
  \end{itemize}

  \subsection{Relevant Review}\label{Relevant Review}
  \begin{itemize}
    \item \ddd{Electronegativity}: negative charges on atoms with lower hybridization result in greater stability due to proximity (overlap) to positive nucleus. More s character results in greater stability.
      \begin{itemize}
        \item I.e., \(\bbb{sp~(50\%~s)} > sp^2~(33\%~s) > \rrr{sp^3~(25\%~s)}\)
        \item E.g., ethene has two carbons that are both \(sp^2\) due to one unhybridized p-orbital. This gives ethene a trigonal planar geometry.
      \end{itemize}

      \bigskip
      \begin{center}
      \hspace{-20pt}
      \schemestart{}
        \chemfig{C(-[::150]H)(-[::-150]H)=C(-[::+30]H)(-[::-30]H)}
        \arrow{<->}
        \quad
        \chemfig{
          \orbital{s}
            -[:-30,.85]
              {\orbital[scale=1.8]{p}}
              {\orbital[angle=0,scale=1.6,half,color=dark]{p}}
              {\orbital[angle=150,half]{p}}
              {\orbital[angle=-150,half]{p}}
              (-[:-150,.85]\orbital{s})
            =[,.75]
            =[,.75]
            {\orbital[angle=180,scale=1.6, half, color=dark]{p}}
            {\orbital[scale=1.8]{p}}
            {\orbital[angle=30,half]{p}}
            {\orbital[angle=-30,half]{p}}
            (-[::30,.85]{\orbital{s}})(-[::-30,.85]{\orbital{s}})
        }
        \arrow{0}
        \hspace{-20pt}
        \orbital[scale=1.6]{p}
        \quad (unhybridized p-orbital)
      \schemestop{}
      \end{center}
      \bigskip
      \bigskip

    \item \ddd{Hydrogen deficiency index (HDI)}: the measure of degrees of unsaturation. 
      \begin{itemize}
        \item E.g., two degrees of unsaturation results in a HDI of 2.
        \item Degrees of freedom help represent possible structures, indicating possible double bounds, triple bounds, rings, or various combinations of each.
        \item Only helpful when molecular formula is known for certainty.
        \item Formula: \emph{HDI = \(\frac{1}{2}(2C + 2 + N - H - X)\)}
        \begin{itemize}
            \item \(X\): halogen atoms.
        \end{itemize}
      \end{itemize}
    \item What is the HDI for the following molecules?
    
    \begin{multicols}{2}  
    \begin{itemize}
        \item[i] \chemfig{-[:30]=[:-30]-[:30]=[:-30]-[:30]=[:-30]}
        \item[ii]\quad{\footnotesize\chemfig{*6(-=-=-=)}}
        \item[i] \(\frac{1}{2}(2(7) + 2 + (0) - (3+(5(1))+2) - 0) = \emph{3}\)
        \smallskip 
        \item[ii] \(\frac{1}{2}(2(6) + 2 + 0 - (6(1)) - 0) = \emph{4}\)
      \end{itemize}
    \end{multicols}

    \item \ddd{Degree of substitution}: not a substitution reaction, but the \emph{number of groups} connected to the double bond.
    \begin{itemize}
      \item 
      
        \chemname{\chemfig{R-[:45]=}}{Monosubstituted}
        \qquad
        \chemname{\chemfig{R-[:45]=-[:45]R}}{Disubstituted}
        \qquad
        \chemname{\chemfig{R-[:45]=(-[:-45]R)-[:45]R}}{Trisubstituted}
        \qquad
        \chemname{\chemfig{R-[:45](-[:135]R)=(-[:-45]R)-[:45]R}}{Tetrasubstituted}
        \bigskip
        
    \end{itemize}
  \end{itemize}
  \subsubsection{Common Patterns Between Formal Charge and Lone Pairs}
    \begin{itemize}
        \item \ddd{Associated Patterns for Oxygen}
            \begin{itemize}
                \item A \bbb{negative (\(\fminus \))} charge corresponds with \bbb{1 bond} and \bbb{3 lone pairs}.
                \item The \minor{absence} of charge corresponds with \minor{2 bonds} and \minor{2 lone pairs}.
                \item A \rrr{positive (\(\fplus \))} charge corresponds with \rrr{3 bonds} and \rrr{1 lone pair}.
            \end{itemize}
        \item \ddd{Associated Patterns for Nitrogen}
            \begin{itemize}
                \item A \bbb{negative} charge corresponds with \bbb{2 bonds} and \bbb{2 lone pairs}.
                \item The \minor{absence} of charge corresponds with \minor{3 bonds} and \minor{1 lone pair}.
                \item A \rrr{positive} charge corresponds with \rrr{4 bonds} and \rrr{0 lone pairs}.
            \end{itemize}
    \end{itemize}

  \subsubsection{Chirality}\label{Chirality}
  \begin{itemize}
    \item \ddd{Achiral (nonsuperimposable)}: when an object's mirrored version is identical to the actual object.
    \item \ddd{Chiral}: objects that are not superimposable.
      \begin{itemize}
        \item The most common source of molecular chirality is the presence of a \emph{carbon bearing four different groups}.
      \end{itemize}
    \item All three-dimensional objects can be classified as either chiral or achiral.
    \item \ddd{Enantiomer}: the nonsuperimposable mirror image of a chiral compound.
      \begin{itemize}
          \item Can be used in speech the same way the word \emph{twin} is used
          \item The easiest way to draw enantiomers is to simply change wedges and dashes, but there are multiples ways to mirror a molecule, so it can be more complex.
      \end{itemize}
    \item \ddd{Diastereomers}: non-identical stereoisomers (nonsuperimposable) that are \emph{not mirror images} of one another. 
      \begin{itemize}
          \item  Enantiomers have the same physical properties, while diastereomers have \emph{different physical properties}.
          \item Differences between enantiomers and diastereomers are especially relevant when comparing compounds with \emph{more than one chiral center}.
          \item \emph{Maximum} (could be less) number of stereoisomers: \emph{\(2^n\)}
        \begin{itemize}
          \item \(n\): number of chiral centers
          \item \(\dfrac{2^n}{2}\): max pairs of enantiomers.
        \end{itemize}
      \end{itemize}
    \end{itemize}

    \medskip

    \subsubsection{Cahn-Ingold-Prelog System}\label{Cahn-Ingold-Prelog System}
    \begin{itemize}
    \item \ddd{Chan-Ingold-Prelog system}: a system of nomenclature for Identifying each enantiomer individually.
    \begin{enumerate}
      \item Assign priorities to each of the four groups based on atomic number; the highest atomic number has the highest priority.
      \item Rotate the molecule so that the fourth priority group is on a dash (behind)
      \item Determine the configuration, i.e., sequence of 1--2--3 groups;
        \begin{itemize}
          \item \ttt{clockwise (R, \textit{rectus}, right)} or \fff{counterclockwise (S, \textit{sinister}, left)}.
        \end{itemize}
    \end{enumerate}
    \item If there is a tie between the atoms connected, then continue outward until a difference is found.
      \begin{itemize}
        \item Do not add the sum all atomic numbers attached to each atom, just the first in which the atoms differ.
        \item Any multiple bonded atom, (2 or 3) is treated as if connected to multiple atoms equal to number of bonds.
      \end{itemize}
    \item Switching any two groups on a chiral center will invert the configuration, e.g.,
    
      \medskip
      \schemestart{}
        \fff{\chemfig{2-[:30](-[:-30]3)(<[:130]1)(<:[:50]4)}}
        \arrow{->}
        \ttt{\chemfig{2-[:30](-[:-30]3)(<[:130]4)(<:[:50]1)}}
      \schemestop{}
      \bigskip 

    \item Switching twice results in a change without changing configuration, e.g.,
            
      \medskip
      \schemestart{}
        \ttt{\chemfig{
            2-[:30](<[:130]1)(<:[:50]3)-[:-30]4
            }}
        \arrow{->}
        \fff{\chemfig{
            2-[:30](<[:130]1)(<:[:50]4)-[:-30]3
            }}
        \arrow{->}
        \ttt{\chemfig{
            1-[:30](<[:130]2)(<:[:50]4)-[:-30]3
            }}
      \schemestop{}
      \bigskip
    
    \bigskip

    \item \textbf{Configuration in IUPAC nomenclature}:
    \begin{itemize}
        \item The configuration of the chiral center is indicated at the beginning of the name, italicized, and surrounded by parentheses.
        \item When multiple centers are present, then each must be preceded by a locant.
    \end{itemize}
  \end{itemize}

  \subsubsection{Rearrangements}\label{Rearrangements}
    \begin{itemize}
        \item There are several kinds of rearrangements, but only those relating to carbocation rearrangements are focused here.
        \item \ddd{Hyperconjugation}: carbocations that can be stabilized by neighboring groups due to molecular orbitals that slightly overlap with empty \textit{p} orbitals, placing some of its electron density there.
            \begin{itemize}
                \item \emph{Primary (\ang{1}), secondary (\ang{2}), and tertiary (\ang{3})}: refers to the number of groups directly attached to the carbocation.
                \item Tertiary are the most stable (more slight overlap) and primary are the least (less overlap)
            \end{itemize}
        \item \ddd{Hydride shift}: involves the migration of a \bbb{\ch{H^-}}.
            \begin{itemize}
                \item Involves the rearrangement of the carbocation to a more stable variant due to change migration of the \bbb{\ch{H^-}}.
            \end{itemize}
        \item \ddd{Methyl shift}: similar to a hydride, except a whole methyl group is migrated instead.
            \begin{itemize}
                \item The methyl group must be attached to the carbon atom that is adjacent to the carbocation for this to occur.
            \end{itemize}
    \end{itemize}
\end{itemize}

\clearpage
\section{Nomenclature of Alkenes}\label{Nomenclature of Alkenes}
\begin{itemize}
  \item Alkenes are named using the same four steps in the previously used nomenclature, though the suffix of \minimal{``ane''} is replaced with \emph{``ene.''}
  \item When choosing the parent chain, choose the parent chain that \emph{includes} the double bond.
  \item When numbering the parent chain, the double bond should receive the \emph{lowest} number possible.
    \begin{itemize}
      \item Define the location \(k\) of the double bond as being the number of its first carbon.
      \item The locant (\(k\)) of the double bond should be placed right before the suffix of ``ene,'' though, it was previously recommended before the parent (both are acceptable), e.g., 2-pentene = pent-2-ene
    \end{itemize}
  \item Name and the side groups (other than hydrogen) according to the appropriate rules.
  \item Define the position of each side group as the number of the chain carbon it is attached to.
  \item \ddd{E-Z notation}: recommended instead of \textit{cis} and \textit{trans} in order to account for cases that has more than two different groups attached to the double bond by first determining the \hyperref[Cahn-Ingold-Prelog System]{\ulink{CIP priority}}. 
    \begin{itemize}
      \item \fff{E, entgegen, ``opposite''}.
      \item \ttt{Z, zusammen, ``together''}; ``on ze zame zide.''
    \end{itemize}
  \item Commonly recognized alternative names:

    \medskip
    \begin{center}
    \schemestart{}
      \chemname{\chemfig{=[:30]}}{Ethylene}
      \qquad
      \chemname{\chemfig{=[:30]-[:-30]}}{Propylene}
      \qquad
      \chemname{{\tiny\chemfig{*6(-=-(-=[:-30])=-=)}}}{Styrene}
    \schemestop{}
    \end{center}
    \bigskip

    \item Groups containing \ch{C=C} have common names as well, which can be found under \hyperref[Types of Alkenes]{\ulink{types of alkenes}}.
    
    \item If there is \emph{more than 1} functional group, then the \emph{alcohol} has the \emph{higher priority} over alkenes.
    \begin{itemize}
        \item There are more rules depending on functional groups, but for now the distinction between alcohol and alkenes are all that is needed.
        \item I.e., find the longest chain and number in a way that the carbon \emph{containing the OH} gets the \emph{lowest} possible number while still including both the OH and alkene.
        \item An example with alcohol:
    
        \medskip
        \schemestart{}
        \chemname{\chemfig{-[:30]-[:-30]3(=[:-90,,,,emph]4-[:-30,,,,ttt]5-[:-90]6)-[:30,,,,ttt]2-[:-30]1-[:30]\emph{OH}}}{(\ttt{3Z})-3-propyl-hex-3-\emph{en}-1-\emph{ol}\\or\\\textit{(\ttt{3Z})-3-propyl-3-hexen-1-ol}}
        \schemestop{}
        \bigskip

        \item An example with OH still being lowest (name still ends with -ol), but alkene happens to be lower:
        
        \medskip
        \schemestart{}
            \chemname{\chemfig{[:30]10-[:-30]9-[:30]8-[:-30]7(<:[:-90,.9,,,ttt])-[:30]6-[:-30]5(<[:-90,,,,fff]\ch{OH})-[:30,,,,ttt]4=[:-30]3-[:30,,,,fff]2-[:-30]1}}{(\fff{3E},\fff{5S},\ttt{7R})-4-methyl-3-en-5-ol}
        \schemestop{}
        \bigskip

        \item Note: if planar geometry was not given, \ttt{R} and \fff{S} could not be determined. Check out the \hyperref[Cahn-Ingold-Prelog System]{\ulink{Cahn-Ingold-Prelog System}} for review.
    \end{itemize}
\end{itemize}

\clearpage
\section{Addition Reactions of Alkenes}\label{Addition Reactions of Alkenes}
\begin{itemize}
  \item[]
  \subsection{Syn Addition}\label{Syn Addition}
  \begin{itemize}
    \item Syn addition means that both components (A \& B) came in from the same side.

    \medskip
    \schemestart{}
      \chemfig{C(-[::150])(-[::-150])=C(-[::+30])(-[::-30])}
      \+
      \chemfig{A-B}
      \arrow{}
      \chemfig{-C(-[:90])(-[:-90]\ttt{A})-C(-[:90])(-[:-90]\ttt{B})-}
    \schemestop{}
    \bigskip
        
    \item When there is mixture of products, then the stereochemistry is important---A \& B could come from different directions.
    \item Switching to a 3D perspective, with groups pointing towards and away from the viewer, rather than a flat view:
        
    \medskip
    \schemestart{}
      \chemfig{C(-[::150])(-[::-150])=C(-[::+30])(-[::-30])}
      \arrow{<->}
      \chemfig{C(<:[::150])(<[::-150])=C(<:[::+30])(<[::-30])}
      \arrow{->[A---B][\ttt{Syn}]}
      \chemfig{C(<:[:190])(<[:-110])(-[:90]\ttt{A})-C(-[:90]\ttt{B})(<[:-70])(<:[:-10])}
    \schemestop{}
    \bigskip
        
    \item The final product above results in no chiral compounds (no carbons contain four different groups). If a product does have chiral compounds, you must show the other possible products.

    \medskip
    \schemestart{}
      \chemfig{C(<:[::150])(<[::-150]-[:180,.6])=C(<:[::30]-[,.6])(<[::-30])}
      \arrow{->[A---B][\ttt{Syn}]}
      \chemfig{C(<:[:190])(<[:-110]-[:180,.6])(-[:90]\ttt{A})-C(-[:90]\ttt{B})(<[:-70])(<:[:-10]-[,.6])}
    \schemestop{}
    \bigskip
        
    \item The product in this case has  \hyperref[Chirality]{\ulink{chiral centers}}, so enantiomers must be considered.
        
    \medskip
    \schemestart{}
      \dots
      \arrow(--[braces]){->[A---B][\ttt{Syn}]}
      \chemfig{C(<:[:190])(<[:-110]-[:180,.6])(-[:90]\ttt{A})-C(-[:90]\ttt{B})(<[:-70])(<:[:-10]-[,.6])}
      \+
      \chemfig{C(<:[:-190])(<[:110]-[:180,.6])(-[:-90]\ttt{A})-C(-[:-90]\ttt{B})(<:[:70]-[,.6])(<[:10])}
    \schemestop{}
    \bigskip

  \end{itemize}

  \subsection{Anti Addition}\label{Anti Addition}
  \begin{itemize}
    \item Anti addition means that the added components come in from different sides.
    
    \medskip
    \hspace{-28pt}
    \schemestart{}
      \chemfig{C_1(<:[::150])(<[::-150]-[:180,.6])=C_2(<:[::30]-[,.6])(<[::-30])}
      \arrow(--[braces]){->[A---B][\fff{Anti}]}
      \chemfig{C_1(<:[:190])(<[:-110]-[:180,.6])(-[:90]\fff{A})-C_2(-[:-90]\fff{B})(<:[:70]-[,.6])(<[:10])}
      \+
      \chemfig{C_1(<:[:90])(<[:140]-[:180,.6])(-[:-90]\fff{A})-C_2(-[:90]\fff{B})(<[:-90])(<:[:-40]-[,.6])}
    \schemestop{}
    \bigskip
  \end{itemize}

  \subsection{Hydrogenation}
  \begin{itemize}
    \item \ddd{Hydrogenation}: a chemical reaction between molecular hydrogen \ch{H2} and another compound or element, usually in the presence of a catalyst such as nickel, palladium (typically pd, due to cost) or platinum.
      \begin{itemize}
        \item Commonly employed to \emph{reduce} (\(\pi \) bonds) or \emph{saturate} (add hydrogens) organic compounds.
        \item Usually constitutes the addition of pairs of hydrogen atoms to a molecule, often an alkene.
        \item Catalysts are required for the reaction to be usable; non-catalytic hydrogenation takes place only at very high temperatures.
        \item Always occurs with \ttt{syn addition}, with hydrogen entering from the least hindered side.
      \end{itemize}
    
    \item Hydrogenation of 2-methyl-3-ethyl-2-pentene:
      \begin{itemize}
        \item First convert to 3D view:
        
        \medskip
        \schemestart{}
          \chemfig{(-[::150])(-[::-150])=(-[::30]-[,.6])(-[::-30]-[,.6])}
          \arrow{<->}
          \chemfig{(<:[::150])(<[::-150])=(<:[::30]-[,.6])(<[::-30]-[,.6])}
        \schemestop{}
        \bigskip

        \item Then apply \ch{H2} with catalyst to trigger syn addition:
        
        \medskip
        \schemestart{}
          \chemfig{(<:[::150])(<[::-150])=(<:[::30]-[,.6])(<[::-30]-[,.6])}
          \arrow{->[\ch{H2}][Pd-c]}
          \chemfig{(<:[:200])(<[:240])(-[:90]\ttt{H})-(-[:90]\ttt{H})(<:[:-20]-[,.6])(<[:-60]-[,.6])}
        \schemestop{}
        \bigskip

        \item Product is achiral; no enantiomers. Bottom would attack just flips the orientation; not a different product. 
      \end{itemize}
  \end{itemize}

  \subsubsection{Hydrogenation Practice Problems}\label{Hydrogenation Practice Problems}
  \begin{enumerate}\footnotesize
      \item

      \medskip
      \schemestart{}
        \chemfig{(<:[:150])(<[:-150]-[180,.6])=(<:[:30](-[:60,.6])(-[:-10,.6]))(<[:-30])}
        \arrow{->[\ch{H2}][Pd-c]}
        \chemfig{(<:[:200])(<[:-100]-[:180,.6])(-[:90])-(-[:90])(<:[::-20](-[::60,.6])(-[::-10,.6]))(<[:-80])}
      \schemestop{}
      \bigskip

      \item

      \medskip
      \schemestart{}
        \chemfig{*6(---(-)=(-)--)}
        \arrow{->[\ch{H2}][Pd-c]}
        \chemfig{*6(---(<)-(<)--)}
      \schemestop{}
      \bigskip

      \item

      \medskip
      \schemestart{}
        \chemfig{*6(---(--[:-30])=(-)--)}
        \arrow(--[braces]){->[\ch{H2}][Pd-c]}
        \chemfig{*6(---(<-[:-30])=(<)--)}
        \+
        \chemfig{*6(---(<:-[:-30])=(<:)--)}
      \schemestop{}
      \bigskip
  \end{enumerate}

  \subsection{Hydrohalogenation}
  \begin{itemize}
    \item \ddd{Hydrohalogenation}: the electrophilic addition (\(\pi \to 2\sigma \) bonds) of hydrohalic acids (\ch{HX}, e.g., \ch{HCl} or \ch{HBr}) to alkenes to yield the corresponding haloalkanes.
    \item \ddd{Markovnikov's rule}: an addition of a protic acid (HX) or other polar reagent to an asymmetric alkene results in the \rrr{electropositive part (usually H)} gets attached to the carbon with \rrr{more hydrogen substituents}, and the \bbb{electronegative part (usually the halide)} attaches to the carbon with \bbb{more alkyl substituents}. 
      \begin{itemize}
        \item Alternatively: \rrr{\(H^+\)} is added to the carbon with the \rrr{greatest} number of hydrogen atoms while the \bbb{\(X^-\) component} is added to the carbon with the \bbb{fewest} hydrogen atoms.
      \end{itemize}

    \medskip
    \schemestart{}
      \chemfig{C_1(<:[:150])(<[:-150](-[:190,.6])(-[:250,.6]))=[@{db}]C_2(<:[:30]H)(<[:-30])}
      \arrow{->[\chemfig{@{hx}\rrr{H}@{xx}\bbb{X}}]}
      \chemname[-20pt]{\chemfig{
        \chemabove[3pt]{C_1}{\cat}
        (<:[:150])(<[:-150](-[:190,.6])(-[:250,.6]))
        -C_2
        (-[:90]@{h}\rrr{H})(<:[:250]H)(<[:-50,.9])}
        }{most stable carbocation intermediate}
      \arrow{->}
      \chemname[-20pt]{\chemfig{
        C_1
        (-[:90]@{x}\bbb{X})(<:[:150])(<[:-150](-[:190,.6])(-[:250,.6]))
        -C_2
        (-[:90]\rrr{H})(<:[:250]H)(<[:-50,.9])}
      }{final product} 
    \schemestop{}
    \chemmove[shorten <=2pt,dash pattern= on 1pt off 1pt]{
      \draw(db)..controls +(90:1cm)and+(north:1.5cm)..(hx);
      \draw(hx)..controls +(90:1cm)and+(north:1.5cm)..(h);
      \draw(xx)..controls +(90:1.5cm)and+(north:1.8cm)..(x);
    }
    \bigskip

    \item However, we have a chiral carbon, so we have to take enantiomers into account:
    
    \medskip
    \schemestart{}
      \chemfig{(<:[:150])(<[:-150](-[:190,.6])(-[:250,.6]))=[@{db}](<:[:30]H)(<[:-30])}
      \arrow(--[braces]){->[\rrr{H}\bbb{X}]}
      \chemname[-20pt]{\chemfig{
        (-[:90]\bbb{X})(<:[:150])(<[:-150](-[:190,.6])(-[:250,.6]))-
        (-[:90]\rrr{H})(<:[:250]H)(<[:-50,.9])}}
        {\ttt{Syn}}
      \+
      \chemname[-20pt]{\chemfig{
        (-[:-90]\bbb{X})(<:[:80])(<[:120](-[:80,.6])(-[:160,.6]))-
        (-[:90]\rrr{H})(<:[:250]H)(<[:-50,.9])}}
        {\fff{Anti}}
    \schemestop{}
    \bigskip
    
    \item The previous example is an example of a regioselective reaction, i.e., it's a chemical reaction where one reaction site is preferred over another.
      \begin{itemize}
        \item Markovnikov additions are common examples of regioselective reactions since there is a specific region in where the hydrogen is supposed to go.
      \end{itemize}
    
    \item Another example that contains multiple products:
    
    \medskip
    \schemestart{}
      \chemfig{*6(---(-)=(-)--)}
      \arrow(--[braces]){->[\rrr{H}-\bbb{I}]}
      \chemfig{*6(---(<)(<:[:-30]\rrr{H})-(-[:30,0.3,,,draw=none]\cat)(-)--)}
      \+
      \chemfig{*6(---(<\rrr{H})(<:[:-30])-(-[:30,0.3,,,draw=none]\cat)(-)--)}
    \schemestop{}
    \bigskip

    \item Again, this first step shows the most stable carbocation intermediate, which helps determine where the \rrr{\ch{H+}} will attach to. In this case either location works, so four products are formed, each with syn and anti enantiomers.
   
    \medskip
    \hspace{-20pt}
    \schemestart{}
      \dots
      \arrow(--[braces]){->[\rrr{H}-\bbb{I}]}
      {\footnotesize
      \chemname[-40pt]{\chemfig{*6(---(<)(<:[:-30]\rrr{H})-(<:[:30])(<\bbb{I})--)}}{\fff{Anti}}
      \+
      \chemname[-40pt]{\chemfig{*6(---(<)(<:[:-30]\rrr{H})-(<:[:30]\bbb{I})(<)--)}}{\ttt{Syn}}
      \+ 
      \chemname[-40pt]{\chemfig{*6(---(<\rrr{H})(<:[:-30])-(<:[:30])(<\bbb{I})--)}}{\ttt{Syn}}
      \+
      \chemname[-40pt]{\chemfig{*6(---(<\rrr{H})(<:[:-30])-(<:[:30]\bbb{I})(<)--)}}
      {\fff{Anti}}
      }
    \schemestop{}
    \bigskip

    \item The above products are all \emph{stereoisomers} of each other. Note: not all chemical reactions produce all possible stereoisomers.
  \end{itemize}

  \subsubsection{Hydrohalogenation Practice Examples}

    {\footnotesize
    \medskip
    \schemestart{}
      \chemfig{*6(---=(-)--)}
      \arrow{->[\rrr{H}\bbb{Br}]}
      \chemfig{*6(---(-\rrr{H})-(-)(-[:30,0.3,,,draw=none]\cat)--)}
      \arrow(--[braces]){}[,,shorten >=6pt]
      \chemfig{*6(---(-\rrr{H})-(<:[:120]\bbb{Br})(<[:50])--)}
      \+
      \chemfig{*6(---(-\rrr{H})-(<[:120]\bbb{Br})(<:[:50])--)}
    \schemestop{}
    \bigskip
    }

  \subsection{Addition of Water or Alcohol}
  \ddd{Addition of Water or Alcohol (ROH)}: analogous to that of hydrohalogenation (HX); both of which involve a carbocation intermediate, syn/anti stereochemistry, and Markovnikov regioselectivity.
    \begin{itemize}
      \item Note: the hydronium ion (\rrr{\ch{H3O+}}) forms via the reaction of \rrr{\ch{H2SO4}} with \bbb{\ch{H2O}}, and is the source of the proton that reacts with the starting alkene.
      \item \ch{H-OH} and \ch{RO-H} helps to visualize where the proton is coming from. What ever works with water also works with alcohol and vice versa. 
      \item An example that is very similar to the example from \hyperref[Hydrohalogenation]{\ulink{hydrohalogenation}}:
        
        \hspace{-30pt}
        \bigskip
        \schemestart{}
         \chemfig{(<:[::150])(<[::-150]-[:180,.6])=[,.8](<:[::30]H)(<[::-30]-[,.6])}
         \arrow{->[\bbb{\ch{H2O}}][dil \rrr{\ch{H2SO4}}]}
         \chemfig{
           \chemabove[3pt]{C_1}{\cat}
           (<:[:150])(<[:-150](-[:190,.6]))-
           (-[:90]\rrr{H})(<:[:250]H)(<[:-50,.9](-[,.6]))}
         \arrow{->[\bbb{\ch{H2O}}]}[,,shorten <=8pt]
         \chemfig{
           (-[:90]\bbb{OH})(<:[:200])(<[:-110](-[:190,.6]))-
           (-[:90]\rrr{H})(<:[:250]H)(<[:-50,.9](-[,.6]))}
        \schemestop{}
        
        \begin{itemize}
          \item Note: \rrr{\ch{H2SO4}} is a \rrr{strong acid} and can be generalized to \rrr{\ch{H+}} or \rrr{HA}. 
          \item Note: I occasionally use \rrr{red} for \rrr{cations}, \rrr{acids}, and \elec{} (electrophile). I also use \bbb{blue} for \bbb{anions}, \bbb{bases}, and \nuc{} (nucleophile) to help easily keep track of things. This is an arbitrary color choice, but it holds weight in these notes. However, not all bases/acids are nucleophile/electrophiles, so keep that in mind. 
        \end{itemize}

      \item Again, this reaction is stereoselective, so there is actually more than one product:
        
        \medskip
        \schemestart{}
          \dots
          \arrow(--[braces]){->[\ch{H2O}]}[,,shorten <=8pt]
          \chemname[-30pt]{\chemfig{
           (-[:-90]\bbb{OH})(<[:-200](-[:190,.6]))(<:[:110])-
           (-[:90]\rrr{H})(<:[:260]H)(<[:-40,.9](-[,.6]))
           }}{\fff{Anti}}
          \+
          \chemname[-30pt]{\chemfig{
           (-[:90]\bbb{OH})(<:[:200])(<[:-110](-[:190,.6]))-
           (-[:90]\rrr{H})(<:[:250]H)(<[:-50,.9](-[,.6]))
           }}{\ttt{Syn}}
        \schemestop{}
        \bigskip
        
      \item Example 2:
      
      \medskip
      \schemestart{}
        \chemfig{=[:-30]-[:30](<[:90,.8])-[:-30]-[:30]}
        \arrow{->[\bbb{\ch{H2O}}/\rrr{\ch{H+}}]}
        \chemfig{\rrr{H}-[:-90,.8]-[:-30]\chemabove[6pt]{}{\cat}-[:30](<[:90,.8])-[:-30]-[:30]}
      \schemestop{}
      \bigskip
        
      \item \hyperref[Rearrangements]{\ulink{Rearrangement}} is always a possibility to be considered when cation are generated. In this case, a \ang{2} carbocation was generated, but we can do better:
        \begin{itemize}
          \item Note: showing H is not necessary, but useful to visualize the hydride shift and avoid the urge to move the methyl. \minor{``Why move a sofa when you can move a chair?''}
        \end{itemize}
      
      \hspace{-20pt}
      \medskip
      \schemestart{}
        \chemfig{\rrr{H}-[:-90,.8]-[:-30]@{c}\chemabove[6pt]{}{\cat}-[:30](<:[:30]@{h}H)(<[:90,.8])-[:-30]-[:30]}
        \arrow{->}[,,shorten <=12pt]
        \chemname[-20pt]{\chemfig{\rrr{H}-[:-90,.8]-[:-30]-[:30](-[:160,0.3,,,draw=none]\cat)(-[:90,.8])-[:-30]-[:30]}}{\ang{3} carbocation intermediate}
        \arrow{->[\bbb{\ch{H2O}}]}
        \chemfig{\rrr{H}-[:-90,.8]-[:-30]-[:30](-[:130,.8])(-[:50,.8]\bbb{OH})-[:-30]-[:30]}
      \schemestop{}
      \chemmove[dash pattern= on 1pt off 1pt]{
        \draw(h)..controls +(south:1cm) and +(south:1cm).. (c);}
      \bigskip

      \item Note: the carbocation intermediate changes planar geometry due to change in hybridization (\(sp^3 \to sp^2\)). 
      \item Top and bottom attacks do not need to be considered in this case, as there are chiral centers; no enantiomers.
      \item Reminder, \ch{H2O} and alcohols (ROH) behave very similarly; the reduction of the \(\pi \) bond generates \(2\sigma \) bonds, one of which that takes a \rrr{hydrogen}, and the other (with possible carbocation rearrangement first) which first takes up the \rrr{cation} generated in the medium, then interacts with \bbb{base}, removing an \rrr{H} and leaving \bbb{OH}.
        \begin{itemize}
          \item Note: the OH is not actually negatively charged on the end, I just keep the blue there to help show where it's coming from; it helps determine syn/anti if needed.
        \end{itemize}
        
        \medskip
        \schemestart{}
          \bbb{\ch{H2O}}/\rrr{\ch{H+}} 
          \arrow{->}[,,shorten <=8pt]
          \rrr{\chemfig{H-\chemabove{OH_2}{\cat}}}
          \+
          \pi~bond
          \arrow{->}[,,shorten <=8pt]
          \chemfig{-[:90]\rrr{H}}
          \+
          \chemfig{-[:90]\rrr{\ch{OH2+}}}
          \arrow{->[\bbb{\ch{H2O}}]}[,,shorten <=8pt]
          \chemfig{-[:90]OH}
        \schemestop{}
        \bigskip

        \schemestart{}
          \bbb{\ch{ROH}}/\rrr{\ch{H+}} 
          \arrow{->}[,,shorten <=8pt]
          \rrr{\chemfig{H-\chemabove{ROH_2}{\cat}}}
          \+
          \pi~bond
          \arrow{->}[,,shorten <=8pt]
          \chemfig{-[:90]\rrr{H}}
          \+
          \chemfig{-[:90]\rrr{\ch{OHR+}}}
          \arrow{->[\bbb{\ch{ROH}}]}[,,shorten <=8pt]
          \chemfig{-[:90]OH}
        \schemestop{}
        
      \item \textbf{Problems of going through carbocation intermediate}:
        \begin{itemize}
          \item Carbocations are \(sp^2\), which is makes them trigonal planar, so \nuc{} can attack from top or bottom.
          \item \(C^\cat{}\) can cause rearrangements leading to \emph{multiple products}.
        \end{itemize}
    \end{itemize}
  
  \subsection{Oxymercuration-Demercuration}
  \begin{itemize}
      \item \ddd{Oxymercuration-Demercuration} is another electrophilic addition organic reaction that transforms an alkene into a neutral alcohol.
        \begin{itemize}
          \item Reacts with mercuric acetate (\ch{AcO–Hg–OAc}) in aqueous solution to yield the addition of an acetoxymercury (\ch{HgOAc}) group and a hydroxy (OH) group across the double bond.
  
          \medskip
          \begin{center}\footnotesize
          \hspace{-20pt}
          \schemestart{}
            \chemname[-35pt]{\chemfig{CH_3-C(=[:-90]O)-OH}}{acitic acid, OAc}
            \arrow{->}[,,shorten <=8pt]
            \chemname[-35pt]{\chemfig{CH_3-C(=[:-90]O)-\bbb{O^-}}}{acitate, \ch{AcO-}}
          \schemestop{}
          \end{center}
          \medskip

          \item Carbocations are not formed in this process and thus rearrangements are not observed. 
          \item The reaction follows \hyperref[Markovnikov's rule]{\ulink{Markovnikov's rule}}.
          \item The reaction is stereospecific---it is always an \fff{anti addition}.
        \end{itemize}
      \item Example using propylene:
      \begin{itemize}
        \medskip
        \item 
          \schemestart{}
            \chemfig{-[:30]=[:-30]}
            \arrow{->[\ch{Hg(OAC)2},\bbb{\ch{H2O}},THF][\rrr{\ch{NaBH4}}, \ch{NaOH}, THF]}[0,2.2]
          \schemestop{}
          \bigskip
          
          \item \ch{H2O} is \nuc{}; \ch{ROH} can be used instead.
          \item \ch{THF} has no function directly; it is the solvent.
          \item \rrr{\ch{NaBH4}} is the reducing agent. 
          
          \medskip
          \schemestart{}
            \chemfig{\ch{AcO}-[:30,1.5]\chlewis{:,180}{Hg}-[:-30,1.6]\ch{OAc}}
            \arrow{->}
            \chemfig{\rrr{\fplus{} \chlewis{:,90}{Hg}}-[,1.5]\ch{OAc}}
          \schemestop{}
          \bigskip

          \medskip
          \hspace{-30pt}
          \schemestart{}
            \chemfig{-[:30]=[@{db}:-30]}
            \arrow{->}[0,.6]
            \chemfig{@{cat}\rrr{\fplus{} \chlewis{:,90}{Hg}}-[,1.5]\ch{OAc}}
            \arrow{->}[0,.6]
            \chemfig{-[:30](-[@{cat2}:-90,.3,,,draw=none]\cat)-[:-30]-[:30]@{hg}\rrr{\chlewis{:,90}{Hg}}-[,1.5]\ch{OAc}}
            \arrow{->}[0,.6]
            \chemfig{-[:30]*3(--{\cat{}Hg}(-\ch{OAc})-)}
          \schemestop{}
          \chemmove[dash pattern= on 1pt off 1pt]{
            \draw(db)..controls +(north:1.5cm) and +(west:0.5cm).. (cat);
            \draw(hg)..controls +(north:0.5cm) and +(north:1cm).. (cat2);}
          \bigskip
          
          \item Note: the above is a concerted (one step) process, but drawn out for illustration.
          \item Also, we are not done, we have a chiral center that forms, plus it is just an intermediate step, a nucleophilic attack will occur on the strained (highly reactive) epoxide:
          
          \medskip
          \hspace{-40pt}
          \schemestart{}
            \dots
            \arrow(--[braces]){->}[0,.9]
            \chemname[-35pt]{\chemfig{-[:30]@{c1}*3(-<{\cat{}Hg}(-\ch{OAc})>)}
            \+
            \chemfig{-[:30]*3(-<:{\cat{}Hg}(-\ch{OAc})>:)}}{Mecurinium intermediates}
            \arrow(--[braces]){->[\chemfig{@{wa}\bbb{\ch{H2O}}}]}[0,.9]
            \chemfig{-[:-30](<:[:-90])-[:30]-[:90]Hg-[:90]OAc}
            \+
            \chemfig{-[:-30](<[:-90])-[:30]-[:90]Hg-[:90]OAc}
          \schemestop{}
          \chemmove[dash pattern= on 1pt off 1pt]{
            \draw(wa)..controls +(south:1.5cm) and +(south:1cm).. (c1);}
          \bigskip

          \item Notice the nucleophile attack the carbon with the grater flow of electrons (more substituted), since there was a positive charge on mercury. Epoxides with no charge are attacked on the side with less steric hindrance.
          \item Also, the nucleophilic attack occurs on the opposite side of the plane since there is less hindrance on that side.
          
        \end{itemize}
      \end{itemize}
    \subsubsection{Related Practice Problems}
    \begin{enumerate}\footnotesize
      \item 
      \schemestart{}
        \chemfig{-[:-30]-[:30](<[:90])-[:-30]=[:30]}
        \arrow{->[\ch{CH3\bbb{OH}}/\rrr{\ch{H+}}]}
        \chemfig{-[:-30]-[:30](<[:90])(<:[:45]\rrr{H})-[:-30]\chemabove[6pt]{}{\cat}-[:30]}
        \arrow{->}[0,.8]
        \chemfig{-[:-30]-[:30](-[:30,.3,,,draw=none]\cat)(-[:90])-[:-30]-[:30]}
      \schemestop{}
      \bigskip
    
      \medskip
      \schemestart{}
        \dots
        \arrow{->}
        \chemfig{-[:-30]-[:30](-[:45]\bbb{O}CH_3)(-[:135])-[:-30]-[:30]}
      \schemestop{}
      \bigskip
      
    
      \item 
      \schemestart{}
        \chemfig{(-[::150])(<[::-150]-[:180,.6])=[,.8](-[::30]H)(<[::-30]-[,.6])} 
        \arrow{->[ROH][d. \rrr{\ch{H2SO4}}]}
        \chemfig{\chemabove[6pt]{}{\cat}(-[::150])(<[::-150]-[:180,.6])-(<:[:-100]H)(-[:90]\rrr{H})(<[::-30]-[,.6])} 
      \schemestop{}
      \bigskip
    
      \medskip
      \schemestart{}
        \dots
        \arrow(--[braces]){->[R\bbb{OH}]}
        \chemfig{(<:[:220])(<[:-100]-[:180,.6])(-[:90]\bbb{OR})-(<:[:-100]H)(-[:90]\rrr{H})(<[:-40]-[,.6])} 
        \+
        \chemfig{(<:[:-220])(<[:100]-[:180,.6])(-[:-90]\bbb{OR})-(<:[:-90]H)(-[:90]\rrr{H})(<[:-40]-[,.6])} 
      \schemestop{}
      \bigskip
      
      \item + Alternate ways of drawing rings (was done in lecture):
    
      \medskip
      \schemestart{}
        \chemfig{-[:-30]-[:30]=[:-30]}
        \arrow(--[braces]){->[\ch{Hg(OAC)2},\ch{H2O},THF][\ch{NaBH4}, \ch{NaOH}, THF]}[0,2.2]
        \chemname[-45pt]{\chemfig{-[:-30]-[:30]*3(-<{\cat{}Hg}(-\ch{OAc})>)}
            \+
            \chemfig{-[:-30]-[:30]*3(-<:{\cat{}Hg}(-\ch{OAc})>:)}}{Mecurinium intermediates}
      \schemestop{}
      \bigskip
      
      \hspace{-40pt}
      {\small
      \schemestart{}
        \dots
        \arrow(--[braces]){->}[0,.6]
        \chemname[-45pt]{\chemfig{-[:-30]-[:30]*3(-<{\cat{}Hg}(-\ch{OAc})>)}
          \+
          \chemfig{-[:-30]-[:30]*3(-<:{\cat{}Hg}(-\ch{OAc})>:)}}{Mecurinium intermediates}
        \arrow(--[braces]){<->[or]}[0,.8]
        \chemname[-45pt]{\chemfig{-[:-30]<[:30]*3(--{\cat{}Hg}(-\ch{OAc})-)}
          \+
          \chemfig{-[:30]<[:-30]*3(-{\cat{}Hg}(-[:-90]\ch{OAc})--)}}{Mecurinium intermediates}
      \schemestop{}
      \bigskip
      }
    
      \textbf{Oxymercuration}:
      
      \hspace{-40pt}
      {\footnotesize
      \schemestart{}
        \dots
        \arrow(--[braces]){->}[0,.6]
        \chemfig{-[:-30]-[:30]*3(-<{\cat{}Hg}(-\ch{OAc})>)}
        \+
        \chemfig{-[:-30]-[:30]*3(-<:{\cat{}Hg}(-\ch{OAc})>:)}
        \arrow(--[braces]){->[\chemfig{\bbb{\ch{H2O}}}]}[0,1.2]
        \chemfig{-[:30]-[:-30](<:[:-90])-[:30]-[:90]Hg-[:90]OAc}
        \+
        \chemfig{-[:30]-[:-30](<[:-90])-[:30]-[:90]Hg-[:90]OAc}
      \schemestop{}
      }
      \bigskip
    
      \textbf{Demercuration}:
      
      \medskip
      \hspace{-50pt}
      {\footnotesize
      \schemestart{}
        \dots
        \arrow(--[braces]){->}[0,.6]
        \chemfig{-[:30]-[:-30](<:[:-90])-[:30]-[:90]Hg-[:90]OAc}
        \+
        \chemfig{-[:30]-[:-30](<[:-90])-[:30]-[:90]Hg-[:90]OAc}
        \arrow(--[braces]){->[\ch{NaBaH4}][NaOH,THF]}[0,1.5]
        \chemfig{-[:30]-[:-30](<:[:-90]OH)-[:30]}
        \+
        \chemfig{-[:30]-[:-30](<[:-90]OH)-[:30]}
      \schemestop{}
      }
      \bigskip
    
      \item 

      \hspace{-9pt}
      \schemestart{}
      \chemfig{=[:-30]-[:30](<[:90])-[:-30]-[:30]}
      \arrow(--[braces]){->[\rrr{\ch{Hg(OAC)2}},\bbb{\ch{CH3OH}},THF][\ch{NaBH4}, \ch{NaOH}, THF]}[0,2.5]
      \chemfig{\rrr{H}-[:-30](<[:-90]\bbb{\ch{OCH3}})-[:30](<[:90])-[:-30]-[:30]}
      \+
      \chemfig{\rrr{H}-[:-30](<:[:-90]\bbb{\ch{OCH3}})-[:30](<[:90])-[:-30]-[:30]}
      \schemestop{}
      \bigskip
    
      \item 
      
      \schemestart{}
        \chemfig{*6(---=(-)--)}
        \arrow{->[\ch{Hg(OAC)2},\ch{H2O},THF][\ch{NaBH4}, \ch{NaOH}, THF]}[0,2.2]
        \chemfig{*6(---(-[,.3,,,,draw=none]\rrr{H})=(-\bbb{OH})--)}
        \arrow{->}[0,.8]
        \chemfig{*6(----(-[:120]\bbb{OH})(-[:60])---)}
      \schemestop{}
      \bigskip
      
      \item (was done in lecture)
      
      \medskip
      \schemestart{}
          \chemfig{*6(--(-)-=(-)--)}
          \arrow{->[\ch{Hg(OAC)2},\ch{H2O},THF][\ch{NaBH4}, \ch{NaOH}, THF]}[0,2.2]
          \chemfig{*6(--(-)-(-[,.3,,,draw=none]\rrr{H})=(-\bbb{OH})--)}
      \schemestop{}
      \bigskip
    
      \item[\small\color{minor}\textbullet] We know these are the locations of the H and OH, so we can start by labeling them.
      
      \medskip
      \schemestart{}
        \chemfig{*6(--(-)-(-[,.3,,,draw=none]\rrr{H})=(-\bbb{OH})--)}
        \arrow(--[braces]){->}
        \chemfig{*6(--(-)-(<\rrr{H})-(-\bbb{OH})--)}
        \+
        \chemfig{*6(--(-)-(<:\rrr{H})-(-\bbb{OH})--)}
      \schemestop{}
      \bigskip
    
      \item[\small\color{minor}\textbullet] The methyl will be the same no matter what, so we can ignore that.
      \item[\small\color{minor}\textbullet] \rrr{H} has two options, which gives us the above.
      \item[\small\color{minor}\textbullet] Since oxymercuration-demercuration is always an \fff{anti addition}, then we know \bbb{OH} must be \fff{anti} to the hydrogen in both of the products:
      
      \medskip
      \schemestart{}
        \chemfig{*6(--(-)-(-[,.3,,,draw=none]\rrr{H})=(-\bbb{OH})--)}
        \arrow(--[braces]){->}
        \chemfig{*6(--(-)-(<\rrr{H})-(<:\bbb{OH})--)}
        \+
        \chemfig{*6(--(-)-(<:\rrr{H})-(<\bbb{OH})--)}
      \schemestop{}
      \bigskip
    \end{enumerate}
  
  \subsection{Hydroboration-Oxidation}
  \begin{itemize}
    \item \ddd{Hydroboration-Oxidation}: a two-step hydration reaction that converts an alkene into an alcohol.
      \begin{itemize}
        \item Results in the \ttt{syn addition} of a hydrogen and a hydroxyl group where the double bond had been.
        \item An anti-Markovnikov reaction---the hydroxyl group attaches to the less-substituted carbon.
        \item Provides a more stereospecific and complementary regiochemical alternative to other hydration reactions such as acid-catalyzed addition (stereoselective) and oxymercuration-demercuration (stereospecific for anti).
        \item Usually \ch{BH3}, \ch{B2H6}, \ch{R2BH} are used. The presence of Boron is key. 
      \end{itemize}
    \item Like \ch{H2O} and \ch{ROH}, it's useful to think of them in the following way: 
    
    \bigskip
    \schemestart{}
      \chemfig{BH3}
      \arrow{<->}
      \chemfig{\bbb{\chemabove{H}{\text{\pcn}}}-\rrr{\chemabove{BH_2}{\text{\pcp}}}}
      \qquad
      \qquad
      \chemfig{R2BH}
      \arrow{<->}
      \chemfig{\bbb{\chemabove{H}{\text{\pcn}}}-\rrr{\chemabove{BR_2}{\text{\pcp}}}}
    \schemestop{}
    \bigskip
    
    \item Unlike water/alcohol, the hydrogen is now the partial negative component and the boron is the positive.

    \medskip
    \hspace{-35pt}
    \schemestart{}
      \chemfig{-[:-30]-[:30]=[:-30]}
      \arrow{->[\ch{BH3}][\ch{H2O2}, \ch{NaOH}]}[0,1.5]
      \chemfig{-[:-30]-[:30]!\pcpab{}(-[:60,,,,,draw=none]\bbb{H}-[:-30]\rrr{BH_2})=[:-30]\pcnab{}}
      \arrow{->[\ch{BH3}]}
      \chemfig{-[:-30]-[:30](-[:90]\bbb{H})-[:-30]-[:30]\rrr{BH_2}}
    \schemestop{}
    \bigskip
    
    \item There are no chiral centers, but there is still one more step---the oxidation:

    \medskip
    \schemestart{}
      \chemfig{-[:-30]-[:30](-[:90]\bbb{H})-[:-30]-[:30]\rrr{BH_2}}
      \arrow{->[\ch{BH3}][\ch{H2O2}, \ch{NaOH}]}[0,1.5]
      \chemfig{-[:-30]-[:30](-[:90]\bbb{H})-[:-30]-[:30]\rrr{OH}}
    \schemestop{}
    \bigskip
    
    \item If we had used \ch{Hg(OAc)2}, then the above reaction would be different; partial charges \\
    \bigskip
    change the possible final products: \chemfig{\bbb{\chemabove{H}{\text{\pcn}}}-\rrr{\chemabove{BH_2}{\text{\pcp}}}} vs.\ \chemfig{\rrr{\chemabove{H}{\text{\pcp}}}-\bbb{\chemabove{OH_2}{\text{\pcn}}}}:

    \hspace{-40pt}
    {\small
    \medskip
    \schemestart{}
      \chemfig{-[:-30]-[:30](-[:90]\bbb{OH})-[:-30]-[:30]\rrr{H}}
      \arrow(---[braces]){->[\ch{Hg(OAC)2},\ch{H2O},THF][\ch{NaBH4}, \ch{NaOH}, THF]}[0,2.2]
      \chemfig{-[:-30]-[:30](<[:90]\bbb{OH})-[:-30]-[:30]\rrr{H}}
      \+
      \chemfig{-[:-30]-[:30](<:[:90]\bbb{OH})-[:-30]-[:30]\rrr{H}}
    \schemestop{}
    }
    \bigskip

    \item Revisiting a practice example, using different reagents:
      
    \medskip
    \schemestart{}
        \chemfig{*6(--(-)-=(-)--)}
        \arrow(--[braces]){->[\ch{BH3}][\ch{H2O2}, \ch{NaOH}, THF]}[0,2.2]
        \chemname[-40pt]{\chemfig{*6(--(-)-(<\rrr{OH})-(<\bbb{H})(<:[:30])--)}}{\ttt{Syn}}
        \+
        \chemname[-40pt]{\chemfig{*6(--(-)-(<:\rrr{OH})-(<:\bbb{H})(<[:30])--)}}{\ttt{Syn}}
    \schemestop{}
    \bigskip

    \item If the untouched methyl happened to be above or below the plane initially, then you would keep that the same, e.g:
    
    \medskip
    \schemestart{}
        \chemfig{*6(--(-)-=(-)--)}
        \arrow(--[braces]){->[\ch{BH3}][\ch{H2O2}, \ch{NaOH}, THF]}[0,2.2]
        \chemname[-40pt]{\chemfig{*6(--(<)-(<\rrr{OH})-(<\bbb{H})(<:[:30])--)}}{\ttt{Syn}}
        \+
        \chemname[-40pt]{\chemfig{*6(--(<)-(<:\rrr{OH})-(<:\bbb{H})(<[:30])--)}}{\ttt{Syn}}
    \schemestop{}
  \end{itemize}

  \subsection{Halogenation}
  \begin{itemize}
    \item \ddd{Halogenation}: a reaction that involves the addition of one or more halogens to a compound or material.
    \begin{itemize}
      \item The addition of halogens to alkenes proceeds via intermediate halonium ions.
      \item \ddd{Halonium ion}: any onium ion containing a halogen atom carrying a positive charge. This cation has the general structure: \chemfig{R-[,.7]\rrr{+X}-[,.7]R'}
      \item \ddd{Onium ion}: a cation formally obtained by the protonation of mononuclear parent hydride of a pnictogen (group 15 of the periodic table), chalcogen (group 16), or halogen (group 17).
    \end{itemize}
    
    \hspace{-20pt}
    {\small
    \medskip
    \schemestart[][west]
      \chemfig{-[:30]@{db1}=[:-30]@{db2}}
      \arrow(--[braces]){->[\chemfig{@{br1}\bbb{Br}-[@{sb},,]@{br2}\bbb{Br}}]}[0,1.2]
      \chemfig{-[:30]*3(-<\chemabove[3pt]{\rrr{Br}}{\cat}>)}
      \+{,,20pt}
      \chemfig{-[:30]*3(-<:\chemabove[3pt]{\rrr{Br}}{\cat}>:)}
      \arrow(--[braces]){<->[or]}
      \chemfig{>[:30]*3(--\chemabove[3pt]{\rrr{Br}}{\cat}-)}
      \+
      \chemfig{>[:-30]*3(-\chembelow[6pt]{\rrr{Br}}{\cat}--)}
    \schemestop{}
    }
    \chemmove[dash pattern= on 1pt off 1pt]{
      \draw(br1)..controls +(north:1cm) and +(north:0.5cm).. (db1);
      \draw(db2)..controls +(north:1cm) and +(west:0.5cm).. (br1);
      \draw(sb)..controls +(north:1cm) and +(north:0.5cm).. (br2);
      }
    \bigskip
    
    \item In the above example we see a cyclic bromonium ion intermediate being formed. Next, one bromine is now left to act as the \nuc:
    
    \medskip
    \schemestart{}
    \dots
    \arrow(--[braces]){->[\chemfig{@{br}\chemabove[3pt]{\bbb{Br}}{\ani}}]}
    \chemfig{-[:30]@{at}*3(-<@{cat1}\chemabove[3pt]{\rrr{Br}}{\cat}>)}
    \+{,,20pt}
    \chemfig{-[:30]@{att}*3(-<:@{cat2}\chemabove[3pt]{\rrr{Br}}{\cat}>:)}
    \arrow(--[braces]){->}
    \chemname[-40pt]{\chemfig{(<[:90]\bbb{Br})(-[:-130])-<:[:-90]\rrr{Br}}}{\fff{Anti}}
    \+
    \chemname[-40pt]{\chemfig{(<:[:90]\bbb{Br})(-[:-130])-<[:-90]\rrr{Br}}}{\fff{Anti}}
    \schemestop{}
    \chemmove[dash pattern= on 1pt off 1pt]{
      \draw(br)..controls +(south:2cm) and +(south:1cm).. (at);
      \draw(br)..controls +(south:3cm) and +(south:1cm).. (att);
      \draw(at)..controls +(west:1cm) and +(west:1cm).. (cat1);
      \draw(att)..controls +(west:1cm) and +(west:1cm).. (cat2);
      \draw(br2)..controls +(220:2cm) and +(165:6cm).. (br);
      }
    \bigskip
    
    \item Alternatively, if you did the reaction in \ch{H2O}, then water would become the dominant \nuc, leaving OH\@.
    
    \medskip
    \schemestart{}
      \chemfig{-[:30]=[:-30]}
      \arrow(--[braces]){->[\ch{Br2}][\bbb{\ch{H2O}}]}
      \chemname[-40pt]{\chemfig{
        (<[:90]\bbb{OH})(-[:-130])-<:[:-90]\rrr{Br}
        }}{\fff{Anti}}
      \+
      \chemname[-40pt]{\chemfig{
        (<:[:90]\bbb{OH})(-[:-130])-<[:-90]\rrr{Br}
        }}{\fff{Anti}}
    \schemestop{}
  \end{itemize}
\end{itemize}