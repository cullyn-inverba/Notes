\documentclass{inVerba-notes}

\definecolor{title-color}{HTML}{78b07a}
\newcommand{\theTitle}{\href{https://github.com/cullyn-inverba/notes/tree/master/bi-429} {Conservation Biology}}

\begin{document}
\tableofcontents

%%%%%%%%%%% What is Conservation Biology? %%%%%%%%%%%
%\begingroup
\chapter{What is Conservation Biology?}\label{What is Conservation Biology}
\section{Conservation Biology Basics}\label{Conservation Biology Basics}
\begin{itemize}
    \item \ddd{Biodiversity}: the variety and variability of life; generally it is a measure of variation at the genetic, species, and ecosystem level.
        \begin{itemize}
            \item Terrestrial biodiversity is usually greater near the equator, which is the result of the warm climate and high primary productivity.
        \end{itemize}
    \item \ddd{Primary productivity}: the synthesis of organic compounds from atmospheric or aqueous carbon dioxide.
        \begin{itemize}
            \item Principally occurs through the process of photosynthesis, but it also occurs through chemosynthesis.
            \item Almost all life on Earth relies directly or indirectly on primary production. 
            \item \ddd{Autotrophs (primary producers)}: organisms responsible for primary production and form the base of the food chain.
            \item \ddd{Gross primary production (GPP)}: the amount of chemical energy, typically expressed as carbon biomass, that primary producers create in a given length of time. 
            \item \ddd{Net Primary production (NPP)}: the remaining fixed energy left after cellular/growth respiration used by primary producers.
                \begin{itemize}
                    \item NPP = GPP \minus\ respiration [by plants]
                    \item In terrestrial ecosystems, it is expressed as mass of carbon per unit area per unit time interval (\si{gCm^{-2}yr^{-1}})
                \end{itemize}
        \end{itemize}
    \item \ddd{Stewardship}: an ethic that embodies the responsible planning and management of resources.
        \begin{itemize}
            \item The stewardship of natural biodiversity requires that a
            strong link be forged between conservation biology and environmentally sustainable development.
        \end{itemize}

    \subsection{Field of Conservation Biology}\label{Field of Conservation Biology}
    \begin{itemize}
        \item \ddd{Conservation Biology}: study of the conservation of nature and of Earth's biodiversity with the aim of protecting species, their habitats, and ecosystems from excessive rates of extinction and the erosion of biotic interactions.
        \item A recent development, circa 1980, that \emph{aims to unite} the modern era of \emph{theoretically oriented} conservation science and \emph{applied policy}.
        \item The \emph{inherent multidisciplinary} basis for conservation biology has led to new subdisciplines including conservation social science, conservation behavior, and conservation physiology.
        \item Views all of \emph{nature's diversity} as important and having \emph{inherent value}, vs.\ the historical approach of a human-centric utilitarian philosophy aiming to manage the environment as a means to maximize our benefits.
        \item \ddd{Tragedy of the commons}: the main social dilemma that faces shared natural resources management; one that describes depletion of a shared resource through uncoordinated action of independent and self-interested actions that themselves alone are not bad, but done collectively then results in destruction.
            \begin{itemize}
                \item In a modern economic context, ``commons'' is taken to mean any open-access and unregulated resource such as the atmosphere, oceans, rivers, ocean fish stocks.
                \item In legal context, it is a type of property that is neither private nor public, but rather held jointly by the members of a community, who govern access and use through social structures, traditions, or formal rules.
            \end{itemize}
    \end{itemize}

    \subsection{Conservation in the United States}\label{Conservation in the United States}
    \begin{itemize}
        \item American conservation efforts can be traced to three
        philosophical movements, two of the nineteenth century
        and one of the twentieth.
        \item \ddd{The Romantic-Transcendental Conservation Ethic}: spoke of nature in a \emph{quasi-religious sense}, argued for the preservation of nature for nature’s sake is a pristine state.
        \begin{itemize}
            \item Derived from Ralph Waldo Emerson and Henry David Thoreau (eastern United States) and John Muir (western United States)
        \end{itemize}
        \item \ddd{Resource Conservation Ethic}: an \emph{anthropocentric} valuing of natural resources; viewed as a means to feed the economic machine and contribute to the material quality of human life for today’s and future generations.
        \begin{itemize}
            \item Popularized by Gifford Pinchot who based his approach on John Stuart Mill's utilitarian philosophy.
            \item Stressed \emph{equity}---a fair distribution of resources among consumers for both the present and future while also emphasizing \emph{efficiency}.
            \item Lead to the \ddd{multiple-use concept}, which is remains the mandate of the U.S. Forest Service and Bureau of Land Management.
        \end{itemize}
        \item \ddd{Evolutionary-Ecological Land Ethic}: demonstrated that nature was not a simple collection of independent parts but a complicated in integrated system of interdependent processes and components that was \emph{more than the sum of its parts}.
        \begin{itemize}
            \item Popularized by Aldo Leopold after his death in 1949 who was educated in Pinchot's tradition, but saw it as inadequate.
            \item Marked the beginning of the development of ecology and evolution as scholarly disciplines.
        \end{itemize}
    \end{itemize}
    
    \subsection{Guiding Principles For Conservation Biology}\label{Guiding Principles For Conservation Biology}
    \begin{itemize}
        \item \ddd{The Evolutionary Play}: evolution is the basic axiom that unites the field of biology.
            \begin{itemize}
                \item Population geneticist Theodosius Dobzhansky once said, “Nothing in biology makes sense except in the light of evolution.”
                \item Evolution is the only reasonable mechanism able to explain the patterns of biodiversity we see in the world today; it establishes the ``ground rules''.
                \item The goal is \emph{not to stop evolutionary} change, i.e., not to try and conserve the \textit{status quo}, but rather to ensure that populations may \emph{continue to  respond to environmental change in adaptive manner}.
            \end{itemize}
        \item \ddd{The Ecological Theater}: the ecological world is dynamic and largely \emph{nonequilibrial}.
            \begin{itemize}
                \item Pushes back on classic paradigm that there was an equilibrium, or preferable a stable point. Instead, the new view focuses on \emph{dynamic processes} and relative physical contexts. 
                \item The goal is to understand how the \emph{interplay between nonequilibrial} processes and the hierarchy of \emph{species interactions} determines \emph{community structure and biodiversity.}
            \end{itemize}
        \item \ddd{Human Presence}: human presence must be included in conservation planning.
            \begin{itemize}
                \item  Conservation efforts that attempt to \emph{wall
                off nature} and safeguard it from humans will \emph{ ultimately fail}.
                \item Many conservationists feel that the only realistic path to conservation in the long term is to ensure a \emph{reasonable standard of living for all people}.
                    \begin{itemize}
                        \item This involves attention to a number of
                        other issues, including birth control, revised concepts of land ownership and use, education, health care, empowerment of women, and inequality.
                    \end{itemize}
            \end{itemize}
    \end{itemize}
\end{itemize}

\section{Pervasive Aspects of Conservation Biology}\label{Pervasive Aspects of Conservation Biology}
\begin{itemize}
    \item Action must often be taken without complete knowledge; requiring flexibility, creativity, building of intuition, and dealing with uncertainty.
    \item Decisions are usually politically and economically charged and cannot wait for detailed studies that take months or even years.
    \item Conservation biologists must walk a fine line between strict scientific credibility resulting in \emph{potentially destructive \stress{inaction}}, and naive advice based on incomplete knowledge that may result in \emph{potentially misguided \stress{action}}.
    
    \subsection{A Multidisciplinary Science}\label{A Multidisciplinary Science}
    \begin{itemize}
        \item Much like the system it studies, conservation biology is part of a complex system on interacting sciences.
        \item A non-exhaustive list that has connects to both the natural sciences and humanities:
        \begin{multicols}{2}
            \begin{itemize}
                \item Endangered species management
                \item Reserve design
                \item Ecological economics
                \item Restoration ecology
                \item Ecosystem conservation
                \item Environmental ethics
                \item Environmental law
                \item Environmental business
                \item Conservation journalism
                \item Conservation marketing
                \item Eco-arts
            \end{itemize}
        \end{multicols}
    \end{itemize}
    
    \subsection{An Inexact and Value-laden Science}\label{An Inexact and Value-laden Science}
    \begin{itemize}
        \item  Ecological systems are complex, their dynamics are expressed in probabilities, stochastic influences may be strong, and many significant processes are nonlinear.
            \begin{itemize}
                \item \ddd{Stochastic}: refers to the property of being well described by a random probability distribution.
            \end{itemize}
        \item Uncertainty is inherently part of ecology and conservation, and \emph{probabilistic answers}, rather than prescriptive answers, to problems are the norm.
        \item \ddd{Precautionary principle}: avoid practices that could lead to irrevocable harm or serious environmental degradation in the absence of scientific certainty about whether such harm will occur. 
            \begin{itemize}
                \item If an ongoing practice is suspect, then it should be suspended until and unless it is shown not to be harmful.
                \item Essentially, it is the ultimate safety margin that prevents us from taking potentially damaging actions unless we are reasonably sure they will cause no serious harm. 
            \end{itemize}
        \item An emerging, consensus answer seems to be that scientists have a clear responsibility to society to lend their knowledge and expertise toward the value-laden goal of biodiversity preservation.
            \begin{itemize}
                \item Objectivity in how science is conducted cannot be compromised to reach predetermined goals, for then all scientific credibility is lost. 
                \item Some argue that the ``mission-oriented'' goal of conservation biology may interfere with objectivity, though others argue the goal emerges from objective science.  
            \end{itemize}
    \end{itemize}
    
    
\end{itemize}


%\endgroup
%%%%%%%%%%% What is Conservation Biology? %%%%%%%%%%%

%%%%%%%%%%% Global Biodiversity %%%%%%%%%%%
%\begingroup
\chapter{Global Biodiversity}\label{Global Biodiversity}
\section{Components of Biodiversity}\label{Components of Biodiversity}
\begin{itemize}
    \item \hyperref[Biodiversity]{\ulink{Biodiversity}} was previously described, but in reality it is \emph{increasingly complex} can be considered on a variety of levels.
        \begin{itemize}
            \item Biodiversity is a part dynamic and hierarchical system, where there are \emph{constantly changing} borders in which definitions can be applied; where we draw the borders depends on the purpose of the discussion. 
        \end{itemize}
    \subsection{Genetic Diversity}\label{Genetic Diversity}
    \begin{itemize}
        \item Genetic variability is the ultimate source of biodiversity at all levels and the material on which evolution acts upon.
        \item The number of genes varies by more than three orders of magnitude, which is only compounded by alleles, or alternate forms, that make potential alterations enormous.
        \item Preservation of genetic diversity helps accomplish the \hyperref[The Evolutionary Play]{\ulink{evolutionary play principle}} of conservation biology.
        \item Genetic diversity among domesticated species is just as important, allowing for innovation that leads to greater efficiencies and advancements in human welfare globally.
    \end{itemize}
    
    \subsection{Population Diversity}\label{Population-Level Diversity}
    \begin{itemize}
        \item Genetic changes lead to differences among individuals in a population, which shape the history and future potential for adaption.
        \item Local ecological conditions that act on genetic differences is the main driving force that leads to variation among populations and thus potential for future adaption.
        \item Loss of local populations reduces the ability of ecosystems to provide goods other species (including people) and damage ability to recover to disasters. 
        \item Massive declines in abundances and range of many species are a great concern to conservation biology.
    \subsubsection{Human Culture Diversity}\label{Human Culture Diversity}
        \begin{itemize}
            \item Variety in human cultures is an important component of population diversity, as it embodies the reservoir of human knowledge.
            \item Much like genes influence population diversity, ``memes'' coined by Richard Dawkins, represent ideas, knowledge, skills, values, traditions, etc., that act as the ultimate driving force that evolution can act on for human culture.
            \item Differences in ideas allow for the adaption to changing social environments, and increases in homogeneity in the idea space are also of grave concern.
        \end{itemize}
    \end{itemize}
    
    \subsection{Species Diversity}\label{Species Diversity}
    \begin{itemize}
        \item Species are frequently the most common unit of measurement in terms of biodiversity and are primary targets of conservation legislation.
        \item Defining a species is difficult, and generally depends on the purpose of the discussion, much like any systems' science.
        \item Many of the concepts are quite similar or overlap, so they are not easy to count: the biologist R. L. Mayden recorded about 24 concept while philosopher of science John Wilkins counted 26.
        \item Many concepts overlap, so Wilkins grouped concepts into seven basic kinds:
        \begin{itemize}
            \item \emph{agamospecies} for asexual organisms;
            \item \emph{biospecies} for reproductively isolated sexual organisms;
            \item \emph{ecospecies} based on ecological niches;
            \item \emph{evolutionary} species based on lineage;
            \item \emph{genetic} species based on gene pool;
            \item \emph{morphospecies} based on form or phenotype;
            \item and \emph{taxonomic} species based on determined by a taxonomist.
        \end{itemize}
        \item The total number of species is estimated to be between 8 and 8.7 million. However, the vast majority of them are not studied or well documented; it may take over 1000 years to fully catalog them all (assuming it's feasible).
        \item \minimal{Note: the book discusses species extensively, but I will skip most of it here as it already seems a bit dated. I have covered species concepts quite extensively myself in evolutionary biology, and we spent significant lecture time on it, so I am not too concerned about reviewing it.}
    \end{itemize}
    
    \subsection{Biological Community Diversity}\label{Biological Community Diversity}
    \begin{itemize}
        \item Biological communities are defined by the species within them and their interactions.
        \item Commonly used measures fall into three major categories that emphasize numbers, evenness, and differences.
            \begin{itemize}
                \item \ddd{Species richness}: the number of species in a given area.
                \item \ddd{Species diversity}: the measure of evenness, or diversity, based on some index of importance such as abundance, productivity, or size; the highest value of evenness is obtained if all are equally abundant/represented.
                \item \ddd{Species differences}: the measure of similarity, generally genetic similarity across lineages, or diversity of habitat across ecosystems/landscapes.
            \end{itemize}
        \item An unweighted measure of species richness is usually used as a means to capture rare species that characterize most biotas.
            \begin{itemize}
                \item Species richness fails to differentiate between native and nonnative species, as well as other measures that assess variability such as dynamic responses to disturbances. 
            \end{itemize}
        \item Obtaining a better understanding how biological communities behave, interact, and evolve on these larger scales is another key area of interest for conservation biology.
    \end{itemize}
    
    \subsection{Ecosystem and Biome Diversity}\label{Ecosystem and Biome Diversity}
    \begin{itemize}
        \item \ddd{Ecosystem}: generally defined as a community of living organisms in conjunction with the nonliving components of their environment, interacting as a system.
            \begin{itemize}
                \item Climate tends to define borders in an ecosystem, giving rise to various biomes.
                \item Ecosystems are often a target of conservation because they support a wide variety of critical functions that maintenance the global ecosystem, which includes the human population.
            \end{itemize}
        \item \ddd{Biome}: collection of plants and animals that have common characteristics for the environment they exist in. They can be found over a range of continents. 
            \begin{itemize}
                \item Biome is more broad than habitat; many habitats can exist in a single biome.
                \item Understanding biomes are important to understand how various changes propagate into global interactions. 
                \item Conservation of biomes is critical to keep ecosystems functioning.
                \item \ddd{Biota}: the total collection of organisms of a geographic region or a time period, from local geographic scales and instantaneous temporal scales all the way up to whole-planet and whole-timescale spatio-temporal scale; the biotas of the Earth make up the biosphere.
            \end{itemize}
        \item \ddd{Habitat}: the array of resources, both physical and biotic in an area, that support the survival and reproduction of a particular species. 
        \item Much like any system's science, the distinction between these levels are usually arbitrary and often are comprised of multiple interacting factors.
    \end{itemize}
\end{itemize}
\clearpage
\section{Distribution}\label{Distributions}
\begin{itemize}
    \item Biodiversity is not evenly distributed, rather it varies greatly across the globe as well as within regions.
    \item The biota depends on many factors, such as temperature, precipitation, altitude, soils, geography and the presence of other species.
    \item Terrestrial biodiversity is thought to be up to 25 times greater than ocean biodiversity.
    \item Forests, especially wet forests, tend to harbour most of Earth's terrestrial biodiversity---this makes our interaction a key focus for conservation biology.
    
    \subsection{Latitudinal Gradients}\label{Latitudinal Gradients}
    \begin{itemize}
        \item \ddd{Latitudinal gradient}: the general trend of biodiversity \emph{increasing from the poles to the tropics}, which also tends to make localities at \emph{lower latitudes have more species} than localities at higher latitudes.
        \item Many factors contribute, but the ultimate factor behind many of them is the \emph{greater mean temperature} at the equator compared to that of the poles.
        \item Some studies claim that this characteristic is unverified in aquatic ecosystems, especially in marine ecosystems. 
        \item Parasites do not appear to follow the same rule.
        \item An alternative hypothesis, ``the fractal biodiversity'', was proposed to explain the biodiversity in latitudinal gradients.
            \begin{itemize}
                \item Considerations include \emph{temperature, moisture, and NPP} as the main variables of an ecosystem niche and as the axis of the ecological \emph{hypervolume}, whose fractal dimension rises to three moving towards the equator.
            \end{itemize}
    \end{itemize}
    
    \subsection{Hotspots}\label{Hotspots}
    \begin{itemize}
        \item \ddd{Endemism}: the state of a species \emph{being native to a single defined geographic location}, such as an island, state, nation, country or other defined zone.
            \begin{itemize}
                \item Organisms that are indigenous to a place are not endemic to it if they are also found elsewhere.
            \end{itemize}
        \item \ddd{Biodiversity hotspot}: a region with a high level of endemic species that have experienced great habitat loss.
            \begin{itemize}
                \item The majority are in \emph{forest areas} and most are located in the \emph{tropics}.
            \end{itemize}
        \item Many regions of high biodiversity and/or endemism arise from specialized habitats which require unusual adaptations.
        \item Accurately measuring differences in biodiversity can be difficult; selection bias among researchers may contribute to biased empirical research for modern estimates of biodiversity. 
    \end{itemize}
\end{itemize}

\clearpage
\section{Ecosystem Services}
\begin{itemize}
    \item \ddd{Ecosystem services}: the suite of benefits that ecosystems provide to humanity.
    \item There is much debate and claims of the impact biodiversity has on various ecosystems and services, but in general they fall into three categories: 
        \begin{itemize}
            \item \ddd{Provisioning services}: services that aid in the production of renewable resources.
            \item \ddd{Regulating services}: services that lessen volatility of environmental change.
            \item \ddd{Cultural services}: services that represent human value and enjoyment.
        \end{itemize}
    \item \link{https://www.nature.com/articles/nature11148}{A 2012 study} performed an exhaustive survey of peer-reviewed literature. More detailed explanations surrounding claims can be found on the \link{https://en.wikipedia.org/wiki/Biodiversity\#Ecosystem_services}{Wikipedia article}. 
    \item \minimal{Note: I will revisit ecosystem services when we cover it in lecture. The book seems dated, so I fear I will be wasting time with it. Also, the wiki includes plenty of concise explanations that I think will suffice for now.}
\end{itemize}
%\endgroup
%%%%%%%%%%% Global Biodiversity %%%%%%%%%%%
\end{document}