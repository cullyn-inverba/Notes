\chapter{19: Free Radical Reactions}\label{19: Free Radical Reactions}
\section{Free Radical Basics/Review}\label{Free Radical Basics/Review}
\begin{itemize}
    \item \ddd{Free radical}: a highly reactive intermediate atom, molecule, or ion that has an unpaired valence electron formed upon homolytic cleavage, or homolysis. 
      \begin{itemize}
        \item Homolytic cleavage occurs when breaking a covalent bond, leaving each fragment with \emph{one electron}.
        \item Breaking bonds require energy, so \emph{heat (\(\Delta \)) or light (\(h\nu \))} is needed for homolysis to take place.
        \item Radical forming reactions proceed in \emph{nonpolar solvents}, since radicals do not have a charge.
        \item No rearrangements are possibile with radicals.
      \end{itemize}
    \item \ddd{Bond-dissociation energy (BDE)}: one measure of strength of a chemical bond. 
      \begin{itemize}
        \item BDE is based on energy required for homolytic bond cleavage:
          \begin{itemize}
            \item Higher BDE = less stable radical formed.
            \item Lower BDE = more stable radical.
          \end{itemize}
      \end{itemize}
    \item Carbon radical geometry is not perfectly trigonal planar, instead it is a hybrid between trigonal pyramidal and trigonal planar.
    \item Compounds bearing \ch{C-H} bonds react with radicals in the following order, reflecting \ch{C-H} BDE\@:
      \begin{itemize}
        \item Least stable \(\leftarrow \) \emph{\ch{CH3^.} < \ang{1}C < \ang{2}C < \ang{3}C < benzyl < allyl} \(\rightarrow \) most stable
      \end{itemize}
    \item \ddd{Captodative effect}: the stabilization of radicals by synergistic effect of electron-withdrawing group (EWG\@; ``captor'' group) and an electron-donating group (EDG\@; ``dative'' group)
      \begin{itemize}
        \item Free radical can be \emph{stabilized by resonance }in a manner directly analogous to that of carbanions and carbocations.
      \end{itemize} 
\end{itemize}

\section{Radical Halogenation of Alkanes}\label{Radical Halogenation of Alkanes}
\begin{itemize}
    \item \ddd{Free-radical halogenation}: a type of halogenation typical of alkanes and alkyl-substituted aromatics under application of UV light that proceeds by a free-radical chain mechanism.
    \item General mechanism (using chlorination of methane as example):
      \begin{itemize}
        \item \ddd{Initiation:} homolysis of halogen by light or heat forming free radicals:
        
        \medskip
        \schemestart{}
          \chemfig{Cl-Cl}
          \arrow{->[\(h\nu \)]}
          \ch{Cl.} + \ch{Cl.}
        \schemestop{}
        \bigskip
        
        \item \ddd{Chain propagation}: a hydrogen is pulled off the carbon leaving a methyl radical, then the methyl radical pulls a \ch{Cl.} from \ch{Cl2}:
        
        \medskip
        \schemestart{}
          \ch{CH4} + \ch{Cl.} 
          \arrow{}[0,.8]
          \ch{CH3.} + \ch{H-Cl} 
          \arrow{}[0,.8] 
          \ch{CH3.} + \ch{Cl-Cl}
          \arrow{}[0,.8]
          \ch{CH3-Cl} + \ch{Cl.}
        \schemestop{}
        \bigskip
        
        \begin{itemize}
          \item If there is sufficient chlorine, then other products could be formed, for example a continuation to \ch{CH2Cl2}.
        \end{itemize}
        \item \ddd{Chain termination}: recombination of two free radicals, needed to stop the reaction. Could result in two methyl radicals forming an impurity in final mixture. Ideally, however, the following net reaction occurs:
        
        \medskip
        \schemestart{}
          \ch{CH4}
          \arrow{->[\ch{Cl2}][{\(h\nu \)}]}
          \ch{CH3-Cl} + \ch{H-Cl}
        \schemestop{}
        \bigskip
        
      \end{itemize}
    \item Radical halogenation allows for the addition of a halogen on an alkene or other substituent, thus allowing for further reactions to take place on such normally unreactive substituents (e.g., haloalkanes) which are often needed in other reactions.
    \item \ddd{Statistical product distribution}: when all hydrogens in a substituent are equivalent and have equal chance of being replaced.
      \begin{itemize}
        \item Hydrogens that form \ch{CH2} or \ch{CH} will be preferred on propane and above for alkanes.
      \end{itemize}
    \item The reactivity of different halogens varies considerably.
      \begin{itemize}
        \item F > Cl > Br > I.
        \item Fluorine is difficult to control (too exothermic), with chlorine moderate to fast, bromine slow (with high UV levels) and iodine practically nonexistent (since it's endothermic). 
        \item Use of bromine will increase BDE preference due to lesser thermodynamic activation energy available.
      \end{itemize}
\end{itemize}