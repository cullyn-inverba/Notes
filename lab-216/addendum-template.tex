\documentclass[12pt,a4paper]{article}
\usepackage{inverba}
\newcommand{\userName}{Cullyn Newman} 
\newcommand{\class}{BI 216} 
\newcommand{\institution}{Portland State University} 
\newcommand{\thetitle}{\hypertarget{home}{Lab x Addendum: title}}
\rfoot{\hyperlink{home}{\thepage}}

\begin{document}
\raggedright
\section*{Part A: }
\begin{enumerate}[font=\bfseries, wide]
    {\color{gray}\item Hello}\par
    
    Ion channels form aqueous pores across the lipid bilayer and allow inorganic ions of appropriate size and charge to cross the membrane down their electrochemical gradients at rates about 1000 times greater than those achieved by any known transporter. The channels are “gated” and usually open transiently in response to a specific perturbation in the membrane, such as a change in membrane potential (voltage-gated channels), or the binding of a neurotransmitter to the channel (transmitter-gated channels).\par
    \ch{K+}-selective leak channels have an important role in determining the resting membrane potential across the plasma membrane in most animal cells.  Voltage-gated cation channels are responsible for the amplification and propagation of action potentials in electrically excitable cells, such as neurons and skeletal muscle cells. Transmitter-gated ion channels convert chemical signals to electrical signals at chemical synapses. Excitatory neurotransmitters, such as acetylcholine and glutamate, open transmitter-gated cation channels and thereby depolarize the postsynaptic membrane toward the threshold level for firing an action potential. Inhibitory neurotransmitters, such as GABA and glycine, open transmitter-gated \ch{Cl^-} or \ch{K^+} channels and thereby suppress firing by keeping the postsynaptic membrane polarized. A subclass of glutamate-gated ion channels, called NMDA-receptor channels, is highly permeable to \ch{Ca^2+}, which can trigger the long-term changes in synapse efficacy (synaptic plasticity) such as LTP and LTD that are thought to be involved in some forms of learning and memory.\par 
    Ion channels work together in complex ways to control the behavior of electrically excitable cells. A typical neuron, for example, receives thousands of excitatory and inhibitory inputs, which combine by spatial and temporal summation to produce a combined postsynaptic potential (PSP) at the initial segment of its axon. The magnitude of the PSP is translated into the rate of firing of action potentials by a mixture of cation channels in the initial segment membrane.
\end{enumerate}

\section*{Part B: Measuring Salinity in Estuaries}
\subsection*{Level 3: Measuring Salinity in Estuaries}
\begin{enumerate}[font=\bfseries, wide, resume]
    \item Question \#5 from the activity: Which statement represents a valid conclusion based on the graph?  Enter the correct letter and the statement (0.5pts)\par

    C. A rainstorm on Oct 25 may have caused the decrease in salinity on Oct 27

    \item What may have caused Delta Smelt to be found outside of their normal range? (0.5pts)\par 

    There was a signicant amount of rainfall during the times the salinity was higher than 2, and salinity also spiked when ever rainfail increased. 
\end{enumerate}

\subsection*{Level 4 - Research Question: Predicting the Return of the Atlantic Sturgeon}
\begin{enumerate}[font=\bfseries, wide, resume]
    \item To get started, use the online Fact Sheet to select an estuary where Atlantic Sturgeon are found. Record the estuary name and location here:  (0.5pts) \par 

    The locaiton we chose was Chesapeake Bay, MD. 

    \item Write your research question in the space below. (1pt)\par 

    How does dissolved oxygen and temperature change over 

    \item Complete the table (1pt)\par
    \begin{table}[ht]
        \centering
        \caption[]{Caption}
        \begin{tabular}{m{3.5cm}m{3cm}m{4cm}m{5cm}}
            \toprule
            Location & Water Quality \par Parameter & Range of dates & Notes \\
            \midrule
            Otter Point Creek & Temp, Salinity & & \\
             & & & \\
             & & & \\
             & & & \\
             & & & \\
             & & & \\
            \bottomrule
            \end{tabular}
    \end{table}
    \item Can you identify a time period when the water temperature is within the range for the sturgeon to return? (0.5pts)
    \item What is the range of the other water quality parameters during that time period? (0.5pts)
    \item Can you identify a time period when all the conditions look right for the sturgeon to return to spawn? (0.5pts)
    \item Do the same conditions occur around the same time, year after year? (0.5pts)
\end{enumerate}

\subsection*{Level 5: Work as a team to develop your own investigation}
\begin{enumerate}[font=\bfseries, wide, resume]
    \item Read through Level 5 on your own, and then work with your team to develop your research question. State your research question here: (1 pt)
    \item State your hypothesis: (1 pt)
    \item \textit{Make a Plan:} Make a lis tbelow of the specific data you will need to answer the question (1 pt) \par
    \begin{table}[ht]
        \centering
        \caption[]{Caption}
        \begin{tabular}{m{3.5cm}m{3cm}m{4cm}m{5cm}}
            \toprule
            Location & Water Quality \par Parameter & Range of dates & Notes \\
            \midrule
             & & & \\
             & & & \\
             & & & \\
             & & & \\
            \bottomrule
            \end{tabular}
    \end{table}
    \item Other than the data listed above, what other information (if any) will you need to answer your question? (1 pt)
    \item Insert figure here (1 pt)
    \item \textit{Interpret the data:} What does your data show? Be specific and descriptive. Does the data suppport your hypothesis? (1 pt)
    \item \textit{Draw a Conclusion:} What is the answer to your question? Use evidence and data to support your conclusion. (1 pt)
    \item Give a specific example of why it would be biologically relevant to measure \textbf{temperature} in an aqautic envrionment. (1 point)
    \item Give a specific example of why it would be biologically relevant to measure \textbf{dissolved oxygen} in an aqautic envrionment. (1 point)
    \item Give a specific example of why it would be biologically relevant to measure \textbf{carbon dioxide} in an aqautic envrionment. (1 point)
\end{enumerate}
\end{document}