\chapter{21: Benzene and Aromaticity}\label{21: Benzene and Aromaticity}

\section{Determining Aromaticity}\label{Determining Aromaticity}
\begin{itemize}
  \item \ddd{Aromaticity}: a property of cyclic, planar structures with \(\pi \) bonds in resonance, giving increased stability relative to other arrangements with same set of atoms.
  \begin{itemize}
    \item Most aromatic compounds are derivatives of benzene; the word aromatic often informally refers to benzene derivatives despite the existence of non-benzene aromatic compounds.
  \end{itemize}
  \item \ddd{Characteristics of aromatic systems}:
    \begin{itemize}
      \item A \emph{delocalized conjugated} \(\pi \) system.
      \item \emph{Planar}, with all contributing atoms in the same plane.
      \item \emph{Cyclic}, arranged in one or more rings.
      \item Follows \emph{H\"uckel's rule}; the number of \(\pi \) electrons is \emph{even}, but \emph{not a multiple of four}.
        \begin{itemize}
          \item \ddd{H\"uckel's rule}: if a molecule has \(4n + 2\pi \)-electrons, it is aromatic; \(n= 0\) or any integer.
          \item \ddd{Antiaromatic}: when an aromatic system has \(4n~\pi \)-electrons (a multiple of four), making it highly unstable and reactive; \(n\neq 0\)
        \end{itemize}
    \end{itemize}
  \item \rrr{Least stable} \(\leftarrow \) \emph{antiaromatic < non-aromatic < aromatic} \(\rightarrow \) \bbb{most stable}
    \begin{itemize}
      \item Reactions will only move compounds away from the relatively more stable state if it is done in the presence of a catalyst.
      \item Antiaromatic will typically become non-planar, breaking the \(\pi \) interactions, and typically become non-aromatic.
    \end{itemize}
  \item Introducing a charge to a non-aromatic cyclic compound can induce a delocalized conjugated \(\pi \) system.
    \begin{itemize}
      \item \bbb{Negative charges} can share in resonance, increasing number of \(\pi \)-electrons in the system.
      \item \rrr{Positive charges} do not change the number of \(\pi \) electrons in the system.
      \item Either one can lead to an antiaromatic or aromatic compound, depending on H\"uckel's rule.
    \end{itemize}
  \subsection{Aromatic Compounds}\label{Aromatic Compounds}
  \begin{itemize}
      \item \ddd{Polycyclic}: aromatic molecules containing two or more simple aromatic rings fused together by sharing two neighboring carbon atoms.
        \begin{itemize}
          \item Not all fused carbons are necessarily equivalent, some electrons are not delocalized over the entire molecule. 
        \end{itemize}
      \item \ddd{Heterocyclic}: when one or more of the atoms in the ring is of an element other than carbon.
        \begin{itemize}
          \item Tends to lessen the ring's aromaticity, increasing reactivity.
          \item The addition of the lone pair electrons from unusual atoms often contributes one of the electron's p-orbital to the aromatic \(\pi \) system if conjugation is need to be aromatic.
          \item However, if the system is already aromatic, then the lone pair will act as a base, not increasing number of \(\pi \)-electrons.
        \end{itemize}
  \end{itemize}
\end{itemize}

\section{Benzene Derivative Nomenclature}\label{Benzene Derivative Nomenclature}
\begin{itemize}
  \item \emph{Monosubstituted} benzenes are named by the IUPAC system using benzene as the suffix.  
    \begin{itemize}
      \item Common monosubstituted benzene names:
      
      \begin{center}
        \medskip
        \hspace{-30pt}
        \schemestart{}
        \chemname[-40pt]{{\footnotesize\chemfig{*6(=-=-(-)=-)}}}{Toluene}
        \qquad
        \chemname[-40pt]{{\footnotesize\chemfig{*6(=-=-(-\ch{OH})=-)}}}{Phenol}
        \qquad
        \chemname[-40pt]{{\footnotesize\chemfig{*6(=-=-(-\ch{NH2})=-)}}}{Aniline}
        \qquad
        \chemname[-40pt]{{\footnotesize\chemfig{*6(=-=-(-\ch{OCH3})=-)}}}{Amisole}
        \qquad
        \schemestop{}
        \bigskip
      \end{center}
      
    \end{itemize}
  \item \emph{Disubstituted} benzenes are named using numbers such as 1,2(ortho; o)-  1,3(meta; m)-  and 1,4(para; p)-.
    \begin{itemize}
      \item Differing substituents on the same ring are listed in alphabetical order.
      \item The parent suffixes toluene, phenol and aniline are also used in conjunction with a second substituent.
    \end{itemize}
  \item \emph{Polysubstituted} benzenes substituents are alphabeticalized with the numbers around the ring to giving the lowest numerical total.
    \begin{itemize}
      \item Use of common names described above are used when possible and designate the first position.
    \end{itemize}
\end{itemize}

