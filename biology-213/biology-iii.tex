\documentclass[12pt,letterpaper]{article}
\usepackage{inverba}
\newcommand{\userName}{Cullyn Newman} 
\newcommand{\class}{BI 213} 
\newcommand{\institution}{Portland State University} 
\newcommand{\thetitle}{\hypertarget{home}{Animals and Ecology}}
\rfoot{\hyperlink{home}{\thepage}}

\begin{document}

%%%%%%%%%%%%%%%%%%%%%%%%%%%%%%%%%%%%%%%%%%%%%%%%%%%%%%%%%%%%%%%%%%%%%%%%%%%%%%%%%%%%%%%%%
%                               %   %   %   %   %   %   %                               %
%                           %   %   %   %   %   %   %   %   %                           %
%                       %   %                               %   %                       %
%   %   %   %   %   %   %   %   O   U   T   L   I   N   E   %   %   %   %   %   %   %   %
%                       %   %                               %   %                       %
%                           %   %   %   %   %   %   %   %   %                           %
%                               %   %   %   %   %   %   %                               %
%%%%%%%%%%%%%%%%%%%%%%%%%%%%%%%%%%%%%%%%%%%%%%%%%%%%%%%%%%%%%%%%%%%%%%%%%%%%%%%%%%%%%%%%%
%\begingroup

\begin{chapbox}{Animals and Ecology}{ 
\begin{enumerate}[font=\bfseries, wide]
    \setcounter{enumi}{38}
\item \hyperlink{39}{\textbf{Animal Form and Function}}
    \begin{itemize}
        \item \hyperlink{39.1}{Form, Function, and Adaptation}
        \item \hyperlink{39.2}{Tissues, Organs, and Systems}
        \item \hyperlink{39.3}{How Does Body Size Affect Animal Physiology?}
        \item \hyperlink{39.4}{Homeostasis}
        \item \hyperlink{39.5}{Thermoregulation: A Closer Look}
        \item \hyperlink{39.r}{Review}
    \end{itemize}
\item \hyperlink{40}{\textbf{Water and Electrolye Balance}}
    \begin{itemize}
        \item \hyperlink{40.1}{Osmoregulation and Exertion}
        \item \hyperlink{40.2}{Water and Electrolyte in Marine Fishes}
        \item \hyperlink{40.3}{Water and Electrolyte in Freshwater Fishes}
        \item \hyperlink{40.5}{Water and Electrolyte in Vertebrates}
        \item \hyperlink{40.r}{Review}
    \end{itemize}
\item \hyperlink{41}{\textbf{Nutrition}}
    \begin{itemize}
        \item \hyperlink{41.1}{Nutritional Requirements}
        \item \hyperlink{41.3}{How Are Nutrients Digested and Absorbed?}
        \item \hyperlink{41.4}{Nutritional Homeostasis}
        \item \hyperlink{41.r}{Review}
    \end{itemize}
\item \hyperlink{42}{\textbf{Gas Exchange}}
    \begin{itemize}
        \item \hyperlink{42.1}{The Resipitory and Circulatory System}
        \item \hyperlink{42.2}{Air and Water as Resipitory Media}
        \item \hyperlink{42.3}{Organs of Gas Exchange}
        \item \hyperlink{42.4}{How are Oxygen and Carbon Dioxide Transported in Blood?}
        \item \hyperlink{42.5}{Circulation}
        \item \hyperlink{42.r}{Review}
    \end{itemize}
\item \hyperlink{43}{\textbf{Nervous System}}
    \begin{itemize}
        \item \hyperlink{43.1}{Principles of Electric Signaling}
        \item \hyperlink{43.2}{Dissecting Action Potential}
        \item \hyperlink{43.3}{The Synapse}
        \item \hyperlink{43.4}{The Vertebrate Nervous System}
        \item [--] \hyperlink{43.r}{Review}
    \end{itemize}
\item \hyperlink{44}{\textbf{Sensory Systems}}
    \begin{itemize}
        \item 
    \end{itemize}
\item \hyperlink{45}{\textbf{Animal Movement}}
    \begin{itemize}
        \item 
    \end{itemize}
\item \hyperlink{46}{\textbf{Chemical Signals}}
    \begin{itemize}
        \item 
    \end{itemize}
\end{enumerate}
}\end{chapbox}

%\endgroup
%%%%%%%%%%%%%%%%%%%%%%%%%%%%%%%%%%%%%%%%%%%%%%%%%%%%%%%%%%%%%%%%%%%%%%%%%%%%%%%%%%%%%%%%%
%                               %   %   %   %   %   %   %                               %
%                           %   %   %   %   %   %   %   %   %                           %
%                       %   %                               %   %                       %
%   %   %   %   %   %   %   %       N   O   T   E   S       %   %   %   %   %   %   %   %
%                       %   %                               %   %                       %
%                           %   %   %   %   %   %   %   %   %                           %
%                               %   %   %   %   %   %   %                               %
%%%%%%%%%%%%%%%%%%%%%%%%%%%%%%%%%%%%%%%%%%%%%%%%%%%%%%%%%%%%%%%%%%%%%%%%%%%%%%%%%%%%%%%%%



%%%%%%%%%%%%%%%%%%%%%%%%%%%%%%%%%%%%%%%%%%%%%%%%%%%%%%%%%%%%%%%%%%%%%%%%%%%%%%%%%%%%%%%%%
%  vvvvvvvvvvvvvvvvvvvvvvvvvvvvvvvvvv   Chapter 39   vvvvvvvvvvvvvvvvvvvvvvvvvvvvvvvvv  %
%\begingroup
\clearpage

\renewcommand{\thetitle}{\hypertarget{39}{Form, Function, and Adaptation}}
\rfoot{\hyperlink{40}{39 --- \thepage}}
\hypertarget{39}{} 

\begin{chapbox}{\hyperlink{home}{Chapter 39: Form, Function, and Adaptation}}
    \begin{enumerate}
        \item \hyperlink{39.1}{Form, Function, and Adaptation}
        \item \hyperlink{39.2}{Tissues, Organs, and Systems}
        \item \hyperlink{39.3}{How Does Body Size Affect Animal Physiology?}
        \item \hyperlink{39.4}{Homeostasis}
        \item \hyperlink{39.5}{Thermoregulation: A Closer Look}
        \item [--] \hyperlink{39.r}{Review}
    \end{enumerate}
\end{chapbox}

\hypertarget{39.1}{}
\begin{secbox}{\hyperlink{39}{Form, Function, and Adaptation}}{
    \begin{itemize}
        \item\textbf{Anatomy}
        \item\textbf{Physiology}
        \item\textbf{Adaptations}
        \item\textbf{Trade-off}
        \item\textbf{Spermatophore}
        \item\textbf{Acclimatization}
    \end{itemize}    
}\end{secbox}

\hypertarget{39.2}{}
\begin{secbox}{\hyperlink{39}{Tissues, Organs, and Systems}}{
    \begin{itemize}
        \item \textbf{Tissue}
        \item \textbf{Connective tissue}
        \begin{itemize}
            \item \textbf{Loose connective tissue }
            \item \textbf{Dense connective tissue}
            \item \textbf{Supporting connective tissue}
            \item \textbf{Fluid connective tissue }
        \end{itemize}
        \item \textbf{Nervous tissue}
        \begin{itemize}
            \item \textbf{Neurons}
            \item \textbf{Dendrites}
            \item \textbf{Axon}
        \end{itemize}
        \item \textbf{Muscle tissue}
        \begin{itemize}
            \item \textbf{Skeletal muscle}
            \item \textbf{Cardiac muscle}
            \item \textbf{Smooth muscle }
        \end{itemize}
        \item \textbf{Epithelial tissues }
        \begin{itemize}
            \item \textbf{Organ}
            \item \textbf{Gland}
            \item \textbf{Apical }
            \item \textbf{Basolateral}
            \item \textbf{Basal lamina}
        \end{itemize}
        \item \textbf{Organ system }
    \end{itemize}
}\end{secbox}

\hypertarget{39.3}{}
\begin{secbox}{\hyperlink{39}{How Does Body Size Affect Animal Physiology?}}{
    \begin{itemize}
        \item \textbf{Surface area vs. volume theory}
        \item \textbf{Metabolic rate}
        \item \textbf{Basal metabolic rate (BMR)}
        \item \textbf{Gills}
        \item \textbf{Adaptations for increased surface area}
        \begin{itemize}
            \item \textbf{Flattening}
            \item \textbf{Folding}
            \item \textbf{Branching}
        \end{itemize}
    \end{itemize}
}\end{secbox}

\hypertarget{39.4}{}
\begin{secbox}{\hyperlink{39}{Homeostasis}}{
    \begin{itemize}
        \item \textbf{Homeostasis}
        \item \textbf{Regulate}
        \item \textbf{Conform}
        \item \textbf{Set point}
        \item \textbf{Homeostatic system:}
        \begin{itemize}
            \item \textbf{Sensor }
            \item \textbf{Integrator}
            \item \textbf{Effector}
        \end{itemize}
        \item \textbf{Hypothalamus}
        \item \textbf{Negative feedback }
    \end{itemize}
}\end{secbox}

\hypertarget{39.5}{}
\begin{secbox}{\hyperlink{39}{Thermoregulation: A Closer Look}}{
    \begin{itemize}
        \item \textbf{Thermoregulate}
        \item \textbf{Thermoregulatory strategies:}
        \begin{itemize}
            \item \textbf{Endotherm }
            \item \textbf{Ectotherm}
            \item \textbf{Homeotherms}
            \item \textbf{Poikilotherms}
            \item \textbf{Torpor}
            \item \textbf{Hibernation}
        \end{itemize}
        \item \textbf{Countercurrent exchanger}
    \end{itemize}
}\end{secbox}   

\hypertarget{39.r}{}
\begin{probbox}{\hyperlink{39}{Chpater 39: Review}}{
    \hyperlink{39.1}{Form, Function, and Adaptation}
    \begin{itemize}
            \item Animal structures and their functions represent adaptations, which are heritable traits that improve survival and reproduction in a certain environment.
            \item Adaptations involve trade-offs, or inescapable compromises between traits.
            \item Acclimatization is a reversible response to the environment that improves physiological function in that environment.
        \end{itemize}
    \hyperlink{39.2}{Tissues, Organs, and Systems}
    \begin{itemize}
            \item Animal cells with a common function are grouped together into
            four general types of tissue: connective tissue, nervous tissue, muscle tissue, and epithelial tissue.
            \item Organs are structures that are composed of two or more tissues that together perform specific tasks.
            \item Organ systems comprise organs that work together in an integrated fashion to perform one or more functions.
    \end{itemize}
    \hyperlink{39.3}{How Does Body Size Affect Animal Physiology?}
    \begin{itemize}
            \item Large animals have smaller surface area/volume ratios than small animals. As animals grow, their volume increases more rapidly than their surface area. 
            \item Large animals have low mass-specific metabolic rates, in keeping with their relatively small surface area for exchanging the oxygen and nutrients required to support metabolism and the wastes and heat produced by metabolism.
            \item The relatively high surface area of small animals means that they lose heat extremely rapidly.
    \end{itemize}
    \hyperlink{39.4}{Homeostasis}
    \begin{itemize}
            \item Homeostasis refers to relatively constant physical and chemical conditions inside the body. 
            \item Homeostasis in a fluctuating environment is usually achieved by regulation. 
            \item Animals have set points, or target values, for various body parameters. When a parameter is not at its set point, negative feedback occurs. Responses to negative feedback return the parameter to the set point and result in homeostasis. 
            \item Most animals have a set point for body temperature. If an individual overheats, it may pant, sweat, or seek a cool environment; if an individual is cold, it may shiver, bask in sunlight, or fluff its fur or feathers.
        \end{itemize}
    \hyperlink{39.5}{Thermoregulation: A Closer Look}
    \begin{itemize}
            \item Animals vary from endothermic to ectothermic and from homeothermic to poikilothermic. 
            \item Endotherms can be active in cold environments but must obtain a lot of energy to fuel their metabolism. Ectotherms do not require as much energy, but their activity depends on environmental temperature. 
            \item Countercurrent heat exchangers have vessels in close contact that carry warm and cool fluids in opposite directions.
    \end{itemize}        
}\end{probbox}
%\endgroup
%  ^^^^^^^^^^^^^^^^^^^^^^^^^^^^^^^^^^   Chapter 39   ^^^^^^^^^^^^^^^^^^^^^^^^^^^^^^^^^^ %   
%%%%%%%%%%%%%%%%%%%%%%%%%%%%%%%%%%%%%%%%%%%%%%%%%%%%%%%%%%%%%%%%%%%%%%%%%%%%%%%%%%%%%%%%%

%%%%%%%%%%%%%%%%%%%%%%%%%%%%%%%%%%%%%%%%%%%%%%%%%%%%%%%%%%%%%%%%%%%%%%%%%%%%%%%%%%%%%%%%%
%  vvvvvvvvvvvvvvvvvvvvvvvvvvvvvvvvvv   Chapter 40   vvvvvvvvvvvvvvvvvvvvvvvvvvvvvvvvvv %
%\begingroup

\clearpage

\renewcommand{\thetitle}{\hypertarget{40}{Water and Electrolye Balance}}
\rfoot{\hyperlink{40}{40 --- \thepage}}
\hypertarget{40}{}
\setcounter{section}{40}

\begin{chapbox}{\hyperlink{home}{Chapter 40: Water and Electrolyte}}
    \begin{enumerate}
        \item \hyperlink{40.1}{Osmoregulation and Exertion}
        \item \hyperlink{40.2}{Water and Electrolyte in Marine Fishes}
        \item \hyperlink{40.3}{Water and Electrolyte in Freshwater Fishes}
        \item [5.] \hyperlink{40.5}{Water and Electrolyte in Vertebrates}
        \item [--] \hyperlink{40.r}{Review}
    \end{enumerate}
\end{chapbox}

\hypertarget{40.1}{}
\begin{secbox}{\hyperlink{40}{Osmoregulation and Exertion}}{
    \begin{itemize}
        \item \textbf{Electrolyte}
        \item \textbf{Diffusion}
        \item \textbf{Osmosis}
        \item \textbf{Osmolarity}
        \item \textbf{Osmoregulation }
        \item \textbf{Osmoconformers}
        \item \textbf{Isosmotic}
        \item \textbf{Osmoregulators}
        \item \textbf{Hyperosmotic }
        \item \textbf{Hyposmotic}
        \item \textbf{Aquaporins}
        \item \textbf{Ammonia}
        \item \textbf{Urea}
        \item \textbf{Uric acid}
    \end{itemize}
}\end{secbox}

\hypertarget{40.2}{}
\begin{secbox}{\hyperlink{40}{Water and Electrolyte in Marine Fish}}{
    \begin{itemize}
        \item \textbf{Rectal gland}
        \item \textbf{Ouabain }
        \item \textbf{Interstitial fluid}
    \end{itemize}
}\end{secbox}

\hypertarget{40.3}{}
\begin{secbox}{\hyperlink{40}{Water and Electrolyte in Freshwater Fish}}{
    \begin{itemize}
        \item \textbf{Osmoregulatory cells may be in different locations}
        \item \textbf{Different forms of Na\(^+\)/K\(^+\)-ATPase may be activated}
        \item \textbf{The orientation of key transport proteins “flips.”}
    \end{itemize}
}\end{secbox}

\hypertarget{40.5}{}
\begin{secbox}{\hyperlink{40}{Water and Electrolyte in Vertebrates}}{
    \begin{itemize}
        \item \textbf{Kidney}
        \item \textbf{Ureter}
        \item \textbf{Bladder}
        \item \textbf{Urethra}
        \item \textbf{Nephron}
        \item \textbf{Cortex}
        \item \textbf{Medulla}
        \item \textbf{The Mammalian Kidney:}
        \item \textbf{Filtration}
        \item \textbf{Proximal tubule}
        \item \textbf{Microvilli}
        \item \textbf{Ion and Water Movement:}
        \item \textbf{Loop of Henle}
        \item \textbf{A Comprehensive View of the Loop of Henle:}
        \item \textbf{Vasa recta}
        \item \textbf{Distal tubule}
        \item \textbf{Collecting duct}
        \item \textbf{Hormones Involved:}
        \item \textbf{Cloaca}
    \end{itemize}
}\end{secbox}

\hypertarget{40.r}{}
    \begin{probbox}{\hyperlink{40}{Chaper 40: Review}}{
        \hyperlink{40.1}{Osmoregulation and Exertion}
        \begin{itemize}
            \item Solutes move across membranes via passive transport, facilitated diffusion, or active transport. Water moves across membranes by osmosis. 
            \item In most animals, epithelial cells that selectively transport water and electrolytes are responsible for homeostasis. 
            \item The mechanisms involved in regulating water and electrolyte balance vary widely among animal groups because different habitats present different types of osmotic stress.
            \item The type of nitrogenous waste excreted by an animal is affected by its phylogeny and its habitat type. Most fishes excrete ammonia; mammals and most adult amphibians excrete urea; and insects and reptiles excrete uric acid.
        \end{itemize}
        \hyperlink{40.2}{Water and Electrolyte in Marine Fishes}
        \begin{itemize}
            \item Seawater is strongly hyperosmotic to the tissues of marine bony fishes, so they tend to lose water by osmosis and gain electrolytes by diffusion. 
            \item Marine bony fishes are osmoregulators, whereas cartilaginous fishes including sharks are osmoconformers. 
            \item Epithelial cells in the shark rectal gland and in the gills of marine bony fishes excrete excess salt using Na\(^+\)/K\(^+\)­ATPase and Na\(^+\)/Cl\(^+\)/K\(^+\) cotransporters located in the basolateral membrane. 
            \item Similar salt­excreting cells also exist in the salt glands of marine birds and other reptiles and in the kidneys of mammals.
        \end{itemize}
        \hyperlink{40.3}{Water and Electrolyte in Freshwater Fishes}
        \begin{itemize}
            \item Freshwater is strongly hyposmotic to the blood of freshwater fishes, so they tend to gain water by osmosis and lose electrolytes by diffusion. 
            \item Epithelial cells in the gills of freshwater fishes import ions using Na\(^+\)/K\(^+\)­ATPase located in the basolateral membrane and Na\(^+\)/Cl\(^-\)/K\(^+\) cotransporters located in the apical membrane.
        \end{itemize}
        \hyperlink{40.5}{Water and Electrolyte in Vertebrates} 
        \begin{itemize}
            \item Nephrons in the vertebrate kidney form a filtrate in the renal corpuscle and then reabsorb valuable nutrients, electrolytes, and water in the proximal tubule. 
            \item A solution containing urea and electrolytes flows through the loop of Henle of mammalian kidneys, where changes in the permeability of epithelial cells to water and salt—along with active transport of salt—create a steep osmotic gradient. 
            \item Antidiuretic hormone increases the water permeability of the collecting duct, causing water to be reabsorbed along the osmotic gradient and hyperosmotic urine to be produced. 
            \item The nephrons of fishes, amphibians, and non­avian reptiles do not have loops of Henle and therefore cannot produce urine that is hyperosmotic to the body fluids. However, some of these vertebrates can produce hyperosmotic urine by reabsorbing water from the cloaca or bladder.
        \end{itemize}    
    }\end{probbox}
%\endgroup
%  ^^^^^^^^^^^^^^^^^^^^^^^^^^^^^^^^^^   Chapter 40   ^^^^^^^^^^^^^^^^^^^^^^^^^^^^^^^^^^ %  
%%%%%%%%%%%%%%%%%%%%%%%%%%%%%%%%%%%%%%%%%%%%%%%%%%%%%%%%%%%%%%%%%%%%%%%%%%%%%%%%%%%%%%%%%

%%%%%%%%%%%%%%%%%%%%%%%%%%%%%%%%%%%%%%%%%%%%%%%%%%%%%%%%%%%%%%%%%%%%%%%%%%%%%%%%%%%%%%%%%
%  vvvvvvvvvvvvvvvvvvvvvvvvvvvvvvvvvv   Chapter 41   vvvvvvvvvvvvvvvvvvvvvvvvvvvvvvvvvv %
%\begingroup

\clearpage

\renewcommand{\thetitle}{\hypertarget{41}{Nutrition}}
\rfoot{\hyperlink{41}{41 --- \thepage}}
\hypertarget{41}{}
\setcounter{section}{41}

\begin{chapbox}{\hyperlink{home}{Chapter 41: Nutrition}}
    \begin{enumerate}
        \item \hyperlink{41.1}{Nutritional Requirements}
        \item [3.]\hyperlink{41.3}{How are Nutrients Digested and Absorbed?}
        \item [4.]\hyperlink{41.4}{Nutritional Homeostasis}
        \item [i.]\hypertarget{41.r}{Review}
    \end{enumerate}
\end{chapbox}

\hypertarget{41.1}{}
\begin{secbox}{\hyperlink{41}{Nutritional Requirements}}{
    \begin{itemize}
        \item \textbf{Ingestion}
        \item \textbf{Digestive tract}
        \item \textbf{Digestion}
        \item \textbf{Absorption}
        \item \textbf{Essential nutrient}
        \item \textbf{Essential amino acids }
        \item \textbf{Essential fatty acids }
        \item \textbf{Vitamins}
        \item \textbf{Minerals}
    \end{itemize}
}\end{secbox}

\hypertarget{41.3}{}
\begin{secbox}{\hyperlink{41}{How are Nutrients Digested and Absorbed?}}{
    \begin{itemize}
        \item \textbf{Incomplete digestive tracts }
        \item \textbf{Complete digestive tracts}
        \item \textbf{Feces}
        \item \textbf{Salivary amylase}
        \item \textbf{Lingual lipase}
        \item \textbf{Salivary glands}
        \item \textbf{Esophagus}
        \item \textbf{Peristalsis}
        \item \textbf{Crop}
        \item \textbf{Stomach}
        \item \textbf{Sphincters}
        \item \textbf{Pepsin}
        \item \textbf{Parietal cells}
        \item \textbf{Mucous cell}
        \item \textbf{Carbonic anhydrase}
        \item \textbf{Ulcer}
        \item \textbf{Ruminants}
        \item \textbf{Symbiosis}
        \item \textbf{Cellulase}
        \item \textbf{Small intestine}
        \item \textbf{Villi}
        \item \textbf{Microvilli}
        \item \textbf{Lacteal}
        \item \textbf{Proteases}
        \item \textbf{Pancreas}
        \item \textbf{Trypsin}
        \item \textbf{Secretin}
        \item \textbf{Cholecystokinin}
        \item \textbf{Gastrin}
        \item \textbf{Pancreatic amylase}
        \item \textbf{Pancreatic lipase}
        \item \textbf{Emulsification}
        \item \textbf{Liver}
        \item \textbf{Bile}
        \item \textbf{Gallbladder}
        \item \textbf{Large intestine}
        \item \textbf{Colon}
        \item \textbf{Rectum}
        \item \textbf{Cecum}
        \item \textbf{Appendix}
        \item \textbf{Coprophagy}
        \item \textbf{Cloaca}
    \end{itemize}
}\end{secbox}

\hypertarget{41.4}{}
\begin{secbox}{\hyperlink{41}{Nutritional Homeostasis}}{
    \begin{itemize}
        \item \textbf{Diabetes mellitus}
        \item \textbf{Insulin}
        \item \textbf{Glucagon}
        \item \textbf{Gluconeogenesis}
        \item \textbf{Diabetes mellitus}
            \begin{itemize}
                \item [(1)]
                \item [(2)]
            \end{itemize}
        \item \textbf{Body mass index (BMI)}
    \end{itemize}
}\end{secbox}

\hypertarget{41.r}{}
\begin{probbox}{\hyperlink{41}{Chapter 41: Review}}{
    \hyperlink{41.1}{Nutritional Requirements}
    \begin{itemize}
        \item The diets of animals include fats, carbohydrates, and proteins that provide energy; vitamins that serve as coenzymes and perform other functions; minerals that are used as components of enzyme cofactors or structural materials; and ions of electrolytes required for osmotic balance and normal membrane function. 
        \item Fats contain more energy (about 9 kcal/g) than carbohydrates or proteins (about 4 kcal/g), making fats an efficient way to store energy in the body.
    \end{itemize}
    \hyperlink{41.3}{How Are Nutrients Digested and Absorbed?}
    \begin{itemize}
        \item Most animals have a digestive tract that begins at the mouth and ends at the anus. 
        \item In many animals, chemical digestion of food begins in the mouth. In mammals, salivary amylase hydrolyzes bonds in starch and glycogen, and lingual lipase hydrolyzes bonds in fats. 
        \item Once food is swallowed, it is propelled down the esophagus by peristalsis. 
        \item Digestion continues in the stomach. In the human stomach, a highly acidic environment denatures proteins, and the enzyme pepsin begins the cleavage of peptide bonds that link amino acids. 
        \item Food passes from the stomach into the small intestine, where it is mixed with secretions from the pancreas and liver. 
        \item In the small intestine, carbohydrate digestion is continued by pancreatic amylase; fats are emulsified by bile salts and digested by pancreatic lipase; and protein digestion is completed by a suite of pancreatic proteases. 
        \item Cells that line the small intestine absorb the nutrients released by digestion. In many cases, uptake is driven by an electrochemical gradient established by Na\(^+\)/K\(^+\)-ATPase that favors the diffusion of Na\(^+\) into the cells. 
        \item As solutes leave the lumen of the small intestine and enter cells, water follows by osmosis. 
        \item Water reabsorption is completed in the large intestine, where feces form. 
        \item The structure of organs in the digestive tract varies widely among species, in ways that support processing of the food each species ingests.
    \end{itemize}
    \hyperlink{41.4}{Nutritional Homeostasis}
    \begin{itemize}
        \item Diabetes mellitus is a condition in which the level of glucose in the blood is abnormally high.
        \item Type 1 diabetes mellitus is caused by a defect in the production of insulin—a hormone secreted by the pancreas that promotes the uptake of glucose from the blood.
        \item Type 2 diabetes mellitus is characterized by a failure of cells to respond to insulin.
        \item The development of type 2 diabetes is correlated with obesity. The incidence of this disease has reached epidemic proportions in many populations.
    \end{itemize}
}\end{probbox}

%\endgroup
%  ^^^^^^^^^^^^^^^^^^^^^^^^^^^^^^^^^^   Chapter 41   ^^^^^^^^^^^^^^^^^^^^^^^^^^^^^^^^^^ %  
%%%%%%%%%%%%%%%%%%%%%%%%%%%%%%%%%%%%%%%%%%%%%%%%%%%%%%%%%%%%%%%%%%%%%%%%%%%%%%%%%%%%%%%%%

%%%%%%%%%%%%%%%%%%%%%%%%%%%%%%%%%%%%%%%%%%%%%%%%%%%%%%%%%%%%%%%%%%%%%%%%%%%%%%%%%%%%%%%%%
%  vvvvvvvvvvvvvvvvvvvvvvvvvvvvvvvvvv   Chapter 42   vvvvvvvvvvvvvvvvvvvvvvvvvvvvvvvvvv %
%\begingroup

\clearpage

\renewcommand{\thetitle}{\hypertarget{42}{Gas Exchange}}
\rfoot{\hyperlink{42}{42 --- \thepage}}
\hypertarget{42}{}
\setcounter{section}{42}

\begin{chapbox}{\hyperlink{home}{Chapter 42: Gas Exchange}}
    \begin{enumerate}
        \item \hyperlink{42.1}{The Resipitory and Circulatory System}
        \item \hyperlink{42.2}{Air and Water as Resipitory Media}
        \item \hyperlink{42.3}{Organs of Gas Exchange}
        \item \hyperlink{42.4}{How are Oxygen and Carbon Dioxide Transported in Blood?}
        \item \hyperlink{42.5}{Circulation}
        \item [--] \hyperlink{42.r}{Review}
    \end{enumerate}
\end{chapbox}

\hypertarget{42.1}{}
\begin{secbox}{\hyperlink{42}{The Resipitory and Circulatory System}}{
    \begin{itemize}
        \item \textbf{Ventilation}
        \item \textbf{Diffusion at the respiratory surface}
        \item \textbf{Circulation}
        \item \textbf{Diffusion at the tissues}
        \item \textbf{Cellular respiration}
        \item \textbf{Respiratory system}
        \item \textbf{Circulatory system}
    \end{itemize}
}\end{secbox}

\hypertarget{42.2}{}
\begin{secbox}{\hyperlink{42}{Air and Water as Resipitory Media}}{
    \begin{itemize}
        \item \textbf{Partial pressure}
        \item \textbf{Solubility of the gas in water}
        \item \textbf{Temperature of the water}
        \item \textbf{Presence of other solutes}
        \item \textbf{Partial pressure of the gas in contact with the water}
    \end{itemize}
}\end{secbox}

\hypertarget{42.3}{}
\begin{secbox}{\hyperlink{42}{Organs of Gas Exchange}}{
    \begin{itemize}
        \item \textbf{Gills}
        \item \textbf{Operculum}
        \item \textbf{Gill filament}
        \item \textbf{Gill lamellae}
        \item \textbf{Countercurrent system}
        \item \textbf{Tracheae (insect)}
        \item \textbf{Spiracles}
        \item \textbf{Trachea (human)}
        \item \textbf{Bronchi }
        \item \textbf{Bronchioles}
        \item \textbf{Lungs}
        \item \textbf{Alveoli}
        \item \textbf{Positive pressure ventilation}
        \item \textbf{Negative pressure ventilation}
        \item \textbf{Diaphragm}
        \item \textbf{Dead space}
    \end{itemize}
}\end{secbox}

\hypertarget{42.4}{}
\begin{secbox}{\hyperlink{42}{How are Oxygen and Carbon Dioxide Transported in Blood?}}{
    \begin{itemize}
        \item \textbf{Plasma}
        \item \textbf{Platelets }
        \item \textbf{White blood cells }
        \item \textbf{Red blood cells}
        \item \textbf{Hemoglobin}
        \item \textbf{Heme}
        \item \textbf{Oxygen– hemoglobin equilibrium curve}
        \item \textbf{Cooperative binding}
        \item \textbf{Bohr shift}
        \item \textbf{Carbonic anhydrase}
    \end{itemize}
}\end{secbox}

\hypertarget{42.5}{}
\begin{secbox}{\hyperlink{42}{Circulation}}{
    \begin{itemize}
        \item \textbf{Open circulatory system}
        \item \textbf{Heart}
        \item \textbf{Closed circulatory system}
        \item \textbf{Arteries}
        \item \textbf{Capillaries}
        \item \textbf{Veins}
        \item \textbf{Aorta}
        \item \textbf{Valves}
        \item \textbf{Lymphatic system}
        \item \textbf{Lymph}
        \item \textbf{Atrium}
        \item \textbf{Ventricle}
        \item \textbf{Pulmonary artery}
        \item \textbf{Pulmonary veins}
        \item \textbf{Pulmonary circuit}
        \item \textbf{Systemic circuit}
        \item \textbf{Venae cavae}
        \item \textbf{Heart murmur}
        \item \textbf{Pacemaker cells}
        \item \textbf{Sinoatrial (SA) node}
        \item \textbf{Intercalated discs}
        \item \textbf{Electrocardiogram}
        \item \textbf{Atrioventricular (AV) node}
        \item \textbf{Systole}
        \item \textbf{Diastole}
        \item \textbf{Cardiac cycle}
        \item \textbf{Systolic blood pressure}
        \item \textbf{Diastolic blood pressure}
        \item \textbf{Hypertension}
        \item \textbf{Baroreceptors}
        \item \textbf{Cardiovascular disease}
        \item \textbf{Arteriosclerosis}
        \item \textbf{Myocardial infarction}
    \end{itemize}
}\end{secbox}

\hypertarget{42.r}{}
\begin{probbox}{\hyperlink{42}{Chapter 42: Review}}{
    \hyperlink{42.1}{The Resipitory and Circulatory System}
    \begin{itemize}
        \item Animal gas exchange involves ventilation, exchange of gases between the environment and the blood, and exchange of gases between blood and tissues. 
        \item Animal circulation involves transportation of gases, nutrients, wastes, and other substances throughout the body.
    \end{itemize}
    \hyperlink{42.2}{Air and Water as Resipitory Media}
    \begin{itemize}
        \item As media for exchanging oxygen and carbon dioxide, air and water are dramatically different. 
        \item Compared with water, air contains much more oxygen and is much less dense and viscous. As a result, terrestrial animals have to process a much smaller volume of air to extract the same amount of O2, and the amount of work required to do so is less than in aquatic animals. 
        \item Both terrestrial and aquatic animals pay a price for exchanging gases: Land-dwellers lose water to evaporation; freshwater animals lose ions and gain excess water; marine animals gain ions and lose water.      
    \end{itemize}
    \hyperlink{42.3}{Organs of Gas Exchange}
    \begin{itemize}
        \item The structure of gills, tracheae, lungs, and other gas exchange organs minimizes the cost of ventilation while maximizing the diffusion rates of O2 and CO2. 
        \item Consistent with predictions made by Fick’s law of diffusion, respiratory epithelia tend to be extremely thin and to be folded to increase surface area. 
        \item In fish gills, countercurrent exchange ensures that the differences in O2 and CO2 partial pressures between water and blood are favorable for gas exchange over the entire length of the ventilatory surface. 
        \item Insect tracheae carry air directly to and from tissues. 
        \item In bird lungs, structural adaptations lead to a high ratio of useful ventilatory space to dead space. 
        \item Breathing rate is regulated to keep the carbon dioxide content of the blood stable during rest and exercise.
    \end{itemize}
    \hyperlink{42.4}{How are Oxygen and Carbon Dioxide Transported in Blood?}
    \begin{itemize}
        \item The tendency of hemoglobin to give up oxygen varies as a function of the PO2 in surrounding tissue in a sigmoidal fashion. As a result, a relatively small change in tissue PO2 causes a large change in the amount of oxygen released from hemoglobin. 
        \item Oxygen binds less tightly to hemoglobin when pH is low. Because CO2 reacts with water to form carbonic acid, the existence of high CO2 partial pressures in exercising muscle tissues lowers their pH and makes oxygen less likely to stay bound to hemoglobin and more likely to be unloaded into tissues. 
        \item The CO2 that diffuses into red blood cells from tissues is rapidly converted to carbonic acid by the enzyme carbonic anhydrase. The protons that are released as carbonic acid dissociates bind to deoxygenated hemoglobin. In this way, hemoglobin acts as a buffer that takes protons out of solution and prevents large fluctuations in blood pH.
    \end{itemize}
    \hyperlink{42.5}{Circulation}
    \begin{itemize}
        \item In many animals, blood or hemolymph moves through the body via a circulatory system consisting of a pump (heart) and vessels. 
        \item In open circulatory systems, overall pressure is low and tissues are bathed directly in hemolymph. 
        \item In closed circulatory systems, blood is contained in vessels that form a continuous circuit. Containment of blood allows higher pressures and flow rates, as well as the ability to direct blood flow accurately to tissues that need it the most. 
        \item In organisms with a closed circulatory system, a lymphatic system collects excess fluid that leaks from the capillaries and returns it to the circulation. 
        \item In amphibians and some reptiles, blood from the pulmonary and systemic circuits may be mixed in the single ventricle. 
        \item In mammals and birds, a four-chambered heart pumps blood into two circuits, which separately serve the lungs and the rest of the body. Crocodilians have a similar heart with a bypass vessel that can shunt blood from the pulmonary to the systemic circuit. 
        \item In vertebrates, the cardiac cycle is controlled by electrical signals that originate in the heart itself. 
        \item Heart rate, cardiac output, and constriction of both arterioles and veins are regulated by chemical signals and by electrical signals from the brain. 
        \item Cardiovascular disease is the leading cause of death in humans.
    \end{itemize}
}\end{probbox}
%\endgroup
%  ^^^^^^^^^^^^^^^^^^^^^^^^^^^^^^^^^^   Chapter 42   ^^^^^^^^^^^^^^^^^^^^^^^^^^^^^^^^^^ %  
%%%%%%%%%%%%%%%%%%%%%%%%%%%%%%%%%%%%%%%%%%%%%%%%%%%%%%%%%%%%%%%%%%%%%%%%%%%%%%%%%%%%%%%%%

%%%%%%%%%%%%%%%%%%%%%%%%%%%%%%%%%%%%%%%%%%%%%%%%%%%%%%%%%%%%%%%%%%%%%%%%%%%%%%%%%%%%%%%%%
%  vvvvvvvvvvvvvvvvvvvvvvvvvvvvvvvvvv   Chapter 43   vvvvvvvvvvvvvvvvvvvvvvvvvvvvvvvvvv %
%\begingroup

\clearpage

\renewcommand{\thetitle}{\hypertarget{43}{}}
\rfoot{\hyperlink{43}{43 --- \thepage}}
\hypertarget{43}{}
\setcounter{section}{43}

\begin{chapbox}{\hyperlink{home}{Chapter 43:}}
    \begin{enumerate}
        \item \hyperlink{43.1}{Principles of Electric Signaling}
        \item \hyperlink{43.2}{Dissecting Action Potential}
        \item \hyperlink{43.3}{The Synapse}
        \item \hyperlink{43.4}{The Vertebrate Nervous System}
        \item [--] \hyperlink{43.r}{Review}
    \end{enumerate}
\end{chapbox}

\hypertarget{43.1}{}
\begin{secbox}{\hyperlink{43}{Principles of Electric Signaling}}{
    \begin{itemize}
        \item \textbf{Nerve net}
        \item \textbf{Central nervous system (CNS)}
        \item \textbf{Sensory neurons}
        \item \textbf{Interneurons}
        \item \textbf{Motor neurons}
        \item \textbf{Nerves}
        \item \textbf{Peripheral nervous system}
        \item \textbf{Reflex}
        \item \textbf{Cell body}
        \item \textbf{Dendrites}
        \item \textbf{Axons}
        \item \textbf{Electrical potential}
        \item \textbf{Membrane Potential}
        \item \textbf{Resting potential}
        \item \textbf{Ion channel}
        \item \textbf{Leak channels}
        \item \textbf{Equilibrium potential}
        \item \textbf{Action potential }
        \item \textbf{Three-Phase Signal}
            \begin{itemize}
                \item \textbf{Depolarization}
                \item \textbf{Repolarization}
                \item \textbf{Hyperpolarization}
            \end{itemize}
        \item \textbf{Excitable membranes}
    \end{itemize}
}\end{secbox}

\hypertarget{43.2}{}
\begin{secbox}{\hyperlink{43}{Dissecting Action Potential}}{
    \begin{itemize}
        \item \textbf{Voltage-gated channels}
        \item \textbf{Voltage clamping}
        \item \textbf{Patch clamping}
        \item \textbf{Neurotoxins}
        \item \textbf{Action potential propagation}
        \item \textbf{Refractory state}
        \item \textbf{Oligodendrocytes}
        \item \textbf{Schwann cells }
        \item \textbf{Glia}
        \item \textbf{Schwann cells }
        \item \textbf{Glia}
        \item \textbf{Myelin sheath}
        \item \textbf{Node of Ranvier}
        \item \textbf{Multiple sclerosis (MS)}
    \end{itemize}
}\end{secbox}

\hypertarget{43.3}{}
\begin{secbox}{\hyperlink{43}{The Synapse}}{
    \begin{itemize}
        \item \textbf{Synapses}
        \item \textbf{Neurotransmitters}
        \item \textbf{Synaptic cleft}
        \item \textbf{Synaptic vesicles}
        \item \textbf{Presynaptic neuron}
        \item \textbf{Postsynaptic cell}
        \item \textbf{Ligand-gated channels}
        \item \textbf{Second messengers}
        \item \textbf{Excitatory postsynaptic potentials (EPSPs)}
        \item \textbf{Inhibitory postsynaptic potentials (IPSPs)}
        \item \textbf{Summation}
        \item \textbf{Axon hillock}
    \end{itemize}
}\end{secbox}

\hypertarget{43.4}{}
\begin{secbox}{\hyperlink{43}{The Vertebrate Nervous System}}{
    \begin{itemize}
        \item \textbf{Afferent division }
        \item \textbf{Efferent division}
        \item \textbf{Somatic nervous system}
        \item \textbf{Autonomic nervous system}
        \item \textbf{Parasympathetic nervous system}
        \item \textbf{Sympathetic nervous system}
        \item \textbf{Enteric nervous system}
        \item \textbf{General Anatomy of the Human Brain }
            \begin{itemize}
                \item \textbf{Cerebrum}
                \item \textbf{Cerebellum}
                \item \textbf{Diencephalon}
                \item \textbf{Brain stem}
                \item \textbf{Frontal lobe}
                \item \textbf{Parietal lobe}
                \item \textbf{Occipital lobe}
                \item \textbf{Temporal lobe}
                \item \textbf{Corpus callosum}
                \item \textbf{Hippocampus}
            \end{itemize}
        \item \textbf{Optogenetics}
        \item \textbf{Synaptic plasticity}
        \item \textbf{Neurogenesis}
    \end{itemize}
}\end{secbox}

\hypertarget{43.r}{}
\begin{probbox}{\hyperlink{43}{Chapter 43: Review}}{
    \hyperlink{43.1}{Principles of Electric Signaling}
    \begin{itemize}
        \item Most neurons have a cell body, multiple short dendrites that receive signals from other cells, and a single axon that transmits electrical signals to other neurons or to effector cells in glands or muscles. 
        \item Studies of the squid giant axon established that neurons have a resting potential maintained by the sodium–potassium pump and potassium leak channels. When Na+/K+-ATPase hydrolyzes ATP, it transports 3 Na+ out of the cell and 2 K+ in.
    \end{itemize}
    \hyperlink{43.2}{Dissecting Action Potential}
    \begin{itemize}
        \item Most neurons have a cell body, multiple short dendrites that receive signals from other cells, and a single axon that transmits electrical signals to other neurons or to effector cells in glands or muscles. 
        \item Studies of the squid giant axon established that neurons have a resting potential maintained by the sodium–potassium pump and potassium leak channels. When Na+/K+-ATPase hydrolyzes ATP, it transports 3 Na+ out of the cell and 2 K+ in.     
    \end{itemize}
    \hyperlink{43.3}{The Synapse}
    \begin{itemize}
        \item Most neurons have a cell body, multiple short dendrites that receive signals from other cells, and a single axon that transmits electrical signals to other neurons or to effector cells in glands or muscles. 
        \item Studies of the squid giant axon established that neurons have a resting potential maintained by the sodium–potassium pump and potassium leak channels. When Na+/K+-ATPase hydrolyzes ATP, it transports 3 Na+ out of the cell and 2 K+ in.
    \end{itemize}
    \hyperlink{43.4}{The Vertebrate Nervous System}
    \begin{itemize}
        \item Most neurons have a cell body, multiple short dendrites that receive signals from other cells, and a single axon that transmits electrical signals to other neurons or to effector cells in glands or muscles. 
        \item Studies of the squid giant axon established that neurons have a resting potential maintained by the sodium–potassium pump and potassium leak channels. When Na+/K+-ATPase hydrolyzes ATP, it transports 3 Na+ out of the cell and 2 K+ in.
    \end{itemize}
}\end{probbox}
%\endgroup
%  ^^^^^^^^^^^^^^^^^^^^^^^^^^^^^^^^^^   Chapter 43   ^^^^^^^^^^^^^^^^^^^^^^^^^^^^^^^^^^ %  
%%%%%%%%%%%%%%%%%%%%%%%%%%%%%%%%%%%%%%%%%%%%%%%%%%%%%%%%%%%%%%%%%%%%%%%%%%%%%%%%%%%%%%%%%

\end{document}