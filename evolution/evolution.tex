\documentclass[12pt,a4paper]{article}
\usepackage{inverba}

\newcommand{\userName}{Cullyn Newman} 
\newcommand{\class}{BI 358} 
\newcommand{\institution}{Portland State Univsrsity} 
\newcommand{\theTitle}{\color{G-Leaf}Evolution}

\begin{document}
%%%%%%%%%%%%%%%%%%%%%%%%%%%%%%%%%%%%%%%%%%%%%%%%%%%%%%%%%%%%%%%%%%%%%
\tableofcontents
\cleardoublepage
\fancyhead{}
\fancyhead[R]{\hyperlink{home}{\nouppercase\leftmark}}
%%%%%%%%%%%%%%%%%%%%%%%%%%%%%%%%%%%%%%%%%%%%%%%%%%%%%%%%%%%%%%%%%%%%%

\clearpage
\fancyhead[L]{Quiz 1}
%\begingroup
%%%%%%%%%%%%%%%%%%%%%%%%%%%%% Chapter 1 %%%%%%%%%%%%%%%%%%%%%%%%%%%%%
%\begingroup
\clearpage
\section{Evolutionary Thinking}\phantomsection
\subsection{Introduction (Lecture)}
\begin{itemize}
    \item Essential questions of evolutionary biology: 
        \begin{itemize}
            \item Why do organisms look so different?
            \item Why develop elaborate sexual traits?
            \item Why do organisms senesce?
        \end{itemize}
    \item Evolution is mainly an historical science and thus must relay on other methods of reconstructing the past or making inferences about evolutionary forces. 
    \item \textbf{Proximate}: a question about a mechanistic cause; provides an immediate explanation about {\color{o-Sun}how} a mechanistic cause functions.
    \item \textbf{Ultimate}: {\color{o-Sun}why}, or the reason, a trait or organism is the way it is; an evolutionary explanation.
    \item Example of proximate vs ultimate in Galapagos finches: 
        \begin{itemize}
            \item Proxmate: developmental growth factor is increased/decreased in some birds.
            \item Ultimate: different habits are selected on breaks that maxmize food gathering ability.
        \end{itemize}
    \item Evolutionary biology's approach to answering questions:
        \begin{itemize}
            \item \textbf{Empirical data}: {\color{o-Sun}observation} studies, experiments; the \textit{comparative method}.
            \item \textbf{Theory}: {\color{o-Sun}predictions} that use models and mathematical reasoning which can be be {\color{o-Sun}tested} with empirical data.
        \end{itemize}
    \item Overview of the components of evolution by natural selection:
        \begin{itemize}
            \item Genetic variation exists, via mutations.
            \item Mutations are heritable.
            \item The is an advantage to survival and/or reproduction from the mutation.
            \item Individuals with the advantage in survival/reproduction are selected for.
        \end{itemize}
\end{itemize}
%\endgroup
%%%%%%%%%%%%%%%%%%%%%%%%%%%%% Chapter 1 %%%%%%%%%%%%%%%%%%%%%%%%%%%%%

%%%%%%%%%%%%%%%%%%%%%%%%%%%%% Chapter 10 %%%%%%%%%%%%%%%%%%%%%%%%%%%%%
%\begingroup
\clearpage
\setcounter{section}{9}
\section{Studying Adaptation}\phantomsection
\subsection{Hypothesis Testing: Oxpeckers Reconsidered}
\begin{itemize}
    \item \textbf{Adaption}: a trait, or a suite of traits, that increases the fitness of its possessor.
    \item No hypothesis for the adaptive value of a trait should be accepted simply because of its plausibility.
    \item Oxpeckers and impalas traditionally were thought to have a mutally beneficial existence; oxpeckers ate ticks and impalas provided a safe environment.
    \item Experiments on cattle were done to test whether this observation was true:
        \begin{itemize}
            \item Results show red-billed oxpeckers have no effect tick loads of cattle.
            \item Red-billeld oxpeckers maintained open wounds, even enlarging existing wounds to feed on the cattle's blood.
            \item Red-billed oxpeckers removed hosts' earwax; whether this is good of bad is unclear.
            \item Even these results must remain in question, as cattle are not the native host for the birds.
        \end{itemize}
    \item Other important points to remember: 
        \begin{itemize}
            \item Differences among populations or species are not always adaptive.
            \item Not every trait is adaptive.
            \item Not every adaptation is perfect, often the adaption just happened to work well enough or by chance better than other adaptations.
        \end{itemize} 
\end{itemize}

\subsection{Experimental Design}
\begin{itemize}
    \item Defining and testing effective control groups is critical.
    \item Treatments of controls and experimental measures must be handled as close to exactly alike as possible.
    \item Randomization is a key technique for equalizing miscellaneous effects and a tool to avoid bias.
    \item Reproduction is essential in order to help remove potential outlier effects.
        \begin{itemize}
            \item Allows for greater understanding of precision, accuracy, and variation by providing more data for statistical tests.
        \end{itemize}
\end{itemize}
%\endgroup
%%%%%%%%%%%%%%%%%%%%%%%%%%%%% Chapter 10 %%%%%%%%%%%%%%%%%%%%%%%%%%%%%

%%%%%%%%%%%%%%%%%%%%%%%%%%%%% Chapter 4 %%%%%%%%%%%%%%%%%%%%%%%%%%%%%
%\begingroup
\clearpage
\setcounter{section}{3}
\section{Evolutionary Trees}\phantomsection
\subsection{How to Read an Evolutionary Tree}
\begin{itemize}
    \item \textbf{Phylogeny}: aka evolutionary tree or phylogenetic tree, is a diaggram showing the history of divergence and evolutionary change. Essentially, it's the {\color{o-Sun}genealogical relationships} of organisms based on descent with modification.
        \begin{itemize}
            \item \textbf{Taxa}: the units you are analyzing, e.g. certain species or DNA sequences.
                \begin{itemize}
                    \item \textbf{Character}: a feature or trait present among the taxa of interest, e.g. teeth of mammals or nucleotides of DNA sequences.
                        \begin{itemize}
                            \item \textit{Character {\color{o-Sun}state(s)}}: an {\color{o-Sun}alternative condition} of a character, which are able to evolve one to another, e.g. pointed/flat teeth of mammals.
                        \end{itemize}
                    \item \textit{{\color{true}Ancestral} character}: a trait that was {\color{true}possessed by the common ancestor}.
                    \item \textit{{\color{false}Derived} character}: a trait the was {\color{false}not possessed by the common ancestor} and instead {\color{o-Sun}evolved} in at least one of the descendants.
                        \begin{itemize}
                            \item \textbf{Synapomorphy}: derived character state shared by {\color{o-Sun} two or more} taxa and used to define a clade of taxa.
                            \item \textbf{Autapomorphy}: derived character state in only {\color{o-Sun}one} taxon.
                        \end{itemize}
                    \item \textbf{Outgroup}: a taxon or taxa that are used to root the phylogeny or determine ancestral character states.
                    \item \textbf{Ingroup}: the set of taxa that are the focus of the phylogeny.
                \end{itemize}
            \item \textbf{Nodes}: points at which the tree splits; represents mutations, speciation events, or {\color{o-Sun}character changes}.
        \item {\color{false}\textbf{Anagensis}}: descent with modification, but {\color{false}no speciation}.
        \item {\color{true}\textbf{Cladogenesis}}: {\color{true}speciation}, origin of clades.
        \begin{itemize}
            \item \textbf{Clade}: also known as a {\color{o-Sun}monophyletic group}, an ancestor and {\color{o-Sun}all} of its descendants.
            \item \textbf{Paraphyletic group}: a group of organisms consisting of an ancestor and {\color{o-Sun}some} of its descendants.
        \end{itemize}
        \item \textbf{Sister}: a taxa or clade that are most closely related to each other; they {\color{o-Sun}share the most recent} common ancestor. 
    \end{itemize}
    \item \textbf{Homology}: similarity due to common descent; {\color{true}continuity} of a trait, character, or character state through time.
        \begin{itemize}
            \item \textit{Homologous trait}: found in a taxa that inherited the trait from a common ancestor.
        \end{itemize}
    \item \textbf{Homoplasy}: or analogous, similarity in the characters or traits in different taxa due to convergent evolution, parallelism, or reversal, but {\color{false}not due to common descent}.
        \begin{itemize}
            \item \textbf{Convergent evolution}: similar traits due to selective forces and {\color{false}not shared ancestry}.
            \item \textbf{Parallelism}: convergent evolution in {\color{o-Sun}recently diverged} taxa.
            \item \textbf{Reversal}: derived traits or character states that revert to the ancestral form.
        \end{itemize}
\end{itemize}

\subsection{Inferring Phylogenetic Trees}
\begin{itemize}    
    \item \textbf{Parsimony}: the hypothesis of relationshipsthat requires the {\color{o-Sun}smallest number} of {\color{o-Sun} character changes} is most likely to be correct.
        \begin{itemize}
            \item Based on {\color{false}derived} traits(synapomorphies).
            \item Reconstruction using parsimony:
                \begin{itemize}
                    \item[1.] Code characters.
                    \item[2.] Make up a taxon x character matrix.
                    \item[3.] Search for synapomorphies, and theshortest tree.
                    \item Outgroups can help polarize the characters. 
                \end{itemize}
            \item \textbf{Treelength}: a measure of evolutionary change using parsimony. 
                \begin{itemize}
                    \item Shortest tree length produces most parsimonious tree.
                    \item Length determined by number of synapomorphies.
                    \item Homoplasious characters increase tree length.
                \end{itemize}
        \end{itemize}
    \item \textbf{Distance Methods}: converts a sequence alignment to genetic distances between pairs of sequences.
        \begin{itemize}
            \item Branch length is proportional to genetic differences.
        \end{itemize}
    \item \textbf{Bayesian}
    \item \textbf{Bootstraping}
\end{itemize}
%\endgroup
%%%%%%%%%%%%%%%%%%%%%%%%%%%%% Chapter 4 %%%%%%%%%%%%%%%%%%%%%%%%%%%%%

%%%%%%%%%%%%%%%%%%%%%%%%%%%%% Chapter 3 %%%%%%%%%%%%%%%%%%%%%%%%%%%%%
%\begingroup
\clearpage
\setcounter{section}{2}
\section{Natural Selection}\phantomsection
\subsection{Lecture Notes}
\begin{itemize}
    \item 
\end{itemize}

%\endgroup
%%%%%%%%%%%%%%%%%%%%%%%%%%%%% Chapter 3 %%%%%%%%%%%%%%%%%%%%%%%%%%%%%
%\endgroup

%%%%%%%%%%%%%%%%%%%%%%%%%%%%% Chapter 6 %%%%%%%%%%%%%%%%%%%%%%%%%%%%%
%\begingroup
\clearpage
\setcounter{section}{5}
\section{Mendelian Genetics}\phantomsection
\subsection{}
\begin{itemize}
    \item 
\end{itemize}
%\endgroup
%%%%%%%%%%%%%%%%%%%%%%%%%%%%% Chapter 6 %%%%%%%%%%%%%%%%%%%%%%%%%%%%%

\clearpage
\fancyhead[L]{Quiz 2}
%\begingroup
%%%%%%%%%%%%%%%%%%%%%%%%%%%%% Chapter 7 %%%%%%%%%%%%%%%%%%%%%%%%%%%%%
%\begingroup
\clearpage
\section{}\phantomsection
\subsection{}
\begin{itemize}
    \item 
\end{itemize}
%\endgroup
%%%%%%%%%%%%%%%%%%%%%%%%%%%%% Chapter 7 %%%%%%%%%%%%%%%%%%%%%%%%%%%%%
%\endgroup

\clearpage
\fancyhead[L]{Quiz 3}
%\begingroup


%%%%%%%%%%%%%%%%%%%%%%%%%%%%% Chapter 8 %%%%%%%%%%%%%%%%%%%%%%%%%%%%%
%\begingroup
\clearpage
\section{}\phantomsection
\subsection{}
\begin{itemize}
    \item 
\end{itemize}
%\endgroup
%%%%%%%%%%%%%%%%%%%%%%%%%%%%% Chapter 8 %%%%%%%%%%%%%%%%%%%%%%%%%%%%%

%%%%%%%%%%%%%%%%%%%%%%%%%%%%% Chapter 9 %%%%%%%%%%%%%%%%%%%%%%%%%%%%%
%\begingroup
\clearpage
\section{}\phantomsection
\subsection{}
\begin{itemize}
    \item 
\end{itemize}
%\endgroup
%%%%%%%%%%%%%%%%%%%%%%%%%%%%% Chapter 9 %%%%%%%%%%%%%%%%%%%%%%%%%%%%%
%\endgroup

\clearpage
\fancyhead[L]{Quiz 4}
%\begingroup
%%%%%%%%%%%%%%%%%%%%%%%%%%%%% Chapter 11 %%%%%%%%%%%%%%%%%%%%%%%%%%%%%
%\begingroup
\setcounter{section}{10}
\clearpage
\section{}\phantomsection
\subsection{}
\begin{itemize}
    \item 
\end{itemize}
%\endgroup
%%%%%%%%%%%%%%%%%%%%%%%%%%%%% Chapter 11 %%%%%%%%%%%%%%%%%%%%%%%%%%%%%

%%%%%%%%%%%%%%%%%%%%%%%%%%%%% Chapter 12 %%%%%%%%%%%%%%%%%%%%%%%%%%%%%
%\begingroup
\clearpage
\section{}\phantomsection
\subsection{}
\begin{itemize}
    \item 
\end{itemize}
%\endgroup
%%%%%%%%%%%%%%%%%%%%%%%%%%%%% Chapter 12 %%%%%%%%%%%%%%%%%%%%%%%%%%%%%

%%%%%%%%%%%%%%%%%%%%%%%%%%%%% Chapter 13 %%%%%%%%%%%%%%%%%%%%%%%%%%%%%
%\begingroup
\clearpage
\section{}\phantomsection
\subsection{}
\begin{itemize}
    \item 
\end{itemize}
%\endgroup
%%%%%%%%%%%%%%%%%%%%%%%%%%%%% Chapter 13 %%%%%%%%%%%%%%%%%%%%%%%%%%%%%

%%%%%%%%%%%%%%%%%%%%%%%%%%%%% Chapter 14 %%%%%%%%%%%%%%%%%%%%%%%%%%%%%
%\begingroup
\clearpage
\section{}\phantomsection
\subsection{}
\begin{itemize}
    \item 
\end{itemize}
%\endgroup
%%%%%%%%%%%%%%%%%%%%%%%%%%%%% Chapter 14 %%%%%%%%%%%%%%%%%%%%%%%%%%%%%

%%%%%%%%%%%%%%%%%%%%%%%%%%%%% Chapter 16 %%%%%%%%%%%%%%%%%%%%%%%%%%%%%
%\begingroup
\clearpage
\setcounter{section}{15}
\section{}\phantomsection
\subsection{}
\begin{itemize}
    \item 
\end{itemize}
%\endgroup
%%%%%%%%%%%%%%%%%%%%%%%%%%%%% Chapter 16 %%%%%%%%%%%%%%%%%%%%%%%%%%%%%
%\endgroup

\clearpage
\fancyhead[L]{Quiz 5}
%\begingroup

%%%%%%%%%%%%%%%%%%%%%%%%%%%%% Chapter 17 %%%%%%%%%%%%%%%%%%%%%%%%%%%%%
%\begingroup
\clearpage
\section{}\phantomsection
\subsection{}
\begin{itemize}
    \item 
\end{itemize}
%\endgroup
%%%%%%%%%%%%%%%%%%%%%%%%%%%%% Chapter 17 %%%%%%%%%%%%%%%%%%%%%%%%%%%%%

%%%%%%%%%%%%%%%%%%%%%%%%%%%%% Chapter 18 %%%%%%%%%%%%%%%%%%%%%%%%%%%%%
%\begingroup
\clearpage
\section{}\phantomsection
\subsection{}
\begin{itemize}
    \item 
\end{itemize}
%\endgroup
%%%%%%%%%%%%%%%%%%%%%%%%%%%%% Chapter 18 %%%%%%%%%%%%%%%%%%%%%%%%%%%%%
%\endgroup

\clearpage
\fancyhead[L]{Quiz 6}
%\begingroup
%%%%%%%%%%%%%%%%%%%%%%%%%%%%% Chapter 20 %%%%%%%%%%%%%%%%%%%%%%%%%%%%%
%\begingroup
\clearpage
\setcounter{section}{19}
\section{}\phantomsection
\subsection{}
\begin{itemize}
    \item 
\end{itemize}
%\endgroup
%%%%%%%%%%%%%%%%%%%%%%%%%%%%% Chapter 20 %%%%%%%%%%%%%%%%%%%%%%%%%%%%%

%%%%%%%%%%%%%%%%%%%%%%%%%%%%% Chapter 15 %%%%%%%%%%%%%%%%%%%%%%%%%%%%%
%\begingroup
\clearpage
\setcounter{section}{14}
\section{}\phantomsection
\subsection{}
\begin{itemize}
    \item 
\end{itemize}
%\endgroup
%%%%%%%%%%%%%%%%%%%%%%%%%%%%% Chapter 15 %%%%%%%%%%%%%%%%%%%%%%%%%%%%%
%\endgroup
\end{document}