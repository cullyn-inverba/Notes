\documentclass[12pt,a4paper]{article}
\usepackage{inverba}

\newcommand{\userName}{Cullyn Newman} 
\newcommand{\class}{BI 358} 
\newcommand{\institution}{Portland State University} 
\newcommand{\theTitle}{\color{g-Leaf} Discussion Questions}

\begin{document}
%%%%%%%%%%%%%%%%%%%%%%%%%%%%%%%%%%%%%%%%%%%%%%%%%%%%%%%%%%%%%%%%%%%%%
\tableofcontents
\cleardoublepage
\fancyhead{}
%%%%%%%%%%%%%%%%%%%%%%%%%%%%%%%%%%%%%%%%%%%%%%%%%%%%%%%%%%%%%%%%%%%%%


%%%%%%%%%%%%%%%%%%%%%%%%%%%%% Week 1 %%%%%%%%%%%%%%%%%%%%%%%%%%%%%
%\begingroup
\clearpage
\section*{Week 2}\phantomsection
\addcontentsline{toc}{section}{\textbf{Week 2}}
\fancyhead[R]{\hyperlink{home}{Week 2}}
\fancyhead[L]{\hyperlink{home}{Dog Domesticaion}}
\subsection{Dog Domestication}
\begin{itemize}
    \item Background and Goals
        \begin{itemize}
            {\color{darklc} \item What hypotheses about dog domestication are examined in this paper?}
                \begin{itemize}
                    \item The origin of domestic dogs was the main focus; genetic data suggests East Asia, but other data suggests Europe and Siberia. 
                \end{itemize}
            {\color{darklc} \item  What predictions do these hypotheses make? What would the expected phylogenetic trees look like under each of these hypotheses (i.e. what are the expected topologies)?}
                \begin{itemize}
                    \item Dogs and wolves would fall under a monophyletic group from a particular area indicating origin.
                \end{itemize}
        \end{itemize}
    \item Methods
        \begin{itemize}
            {\color{darklc} \item  What taxa were sampled for this study?} 
                \begin{itemize}
                    \item Hypothesis was tested using DNA extracted to trace mitochondrial genomes from 18 prehistoric canids and 20 modern wolves from Eurasian and American origin. 
                \end{itemize}
            {\color{darklc} \item  What gene sequences were used for this study?}
                \begin{itemize}
                    \item Wolves, dogs, including divergent breeds, recently published chinese indigenous dogs, and coyotes. (148 mitochondrial genomes)
                \end{itemize}
            {\color{darklc} \item  What phylogenetic methods were used to reconstruct the phylogenetic tree in Figure 1?}
                \begin{itemize}
                    \item Maximum likelyhood, coalescence, and bayesian.
                \end{itemize}
        \end{itemize}
    \item Results and Discussion
        \begin{itemize}
            {\color{darklc} \item In Figure 1, which wolf sequence is most closely related to the extant dog clade A, and what geographic region is it from?}
                \begin{itemize}
                    \item Argentina (pre-Columbian within clade A)
                \end{itemize}
            {\color{darklc} \item In Figure 1, which clade of modern dog sequences has the oldest origin? Approximately how old is this clade?}
                \begin{itemize}
                    \item Dog clade C, Germany, ~14,700 years ago.
                \end{itemize}
            {\color{darklc} \item Based upon Figure 1, how many times does it appear that modern extant dogs could possibly have been domesticated?}
                \begin{itemize}
                    \item Figure 1 suggests 4, but the the authors state, {\color{G-Moon}"the inferred recent divergence of clade B from wolves now found in Sweden and Ukraine implies that is might represent a mitochondrial genome introgressed from wolves rather than one established by domestication."}
                \end{itemize}
            {\color{darklc} \item What was the topology of the phylogenetic tree that was recovered, and therefore, which hypothesis was supported?}
                \begin{itemize}
                    \item The findings support the conclusion that legacy of dogs derives from wolves of European origin.
                    \item Analysis of coalescence times support divergence time of >15,000 years ago.
                    \item Past mitochondrial and Y chromosome analysis suggestesd non-European, but had less supported data than these findings.
                \end{itemize}
        \end{itemize}
    \item Conclusions
        \begin{itemize}
            {\color{darklc} \item  What are the major conclusions of the paper?}
                \begin{itemize}
                    \item Three of four modern dog clades are more closely related to sequences from ancient European rather than extant wolves with divergence times >15,000 years ago.
                \end{itemize}
            {\color{darklc} \item  Was the taxonomic sampling sufficient to rule out alternative hypotheses? }
                \begin{itemize}
                    \item Ancient panel did not contain specimens from Middle East or China. No, no ancient dog remains older than 13,000 years are known from those regions.
                    \item The mtDNA sequence tree is well supported, but represents a single genetic locus. Independent loci could offer more power to resolve phylogenetic relations.
                \end{itemize}
        \end{itemize}
\end{itemize}
%\endgroup
%%%%%%%%%%%%%%%%%%%%%%%%%%%%% Week 1 %%%%%%%%%%%%%%%%%%%%%%%%%%%%%

%%%%%%%%%%%%%%%%%%%%%%%%%%%%% Chapter 2 %%%%%%%%%%%%%%%%%%%%%%%%%%%%%
%\begingroup
\clearpage
\section*{Week 3}\phantomsection
\addcontentsline{toc}{section}{\textbf{Week 3}}
\fancyhead[R]{\hyperlink{home}{Week 3}}
\fancyhead[L]{\hyperlink{home}{[Article]}}

%\endgroup
%%%%%%%%%%%%%%%%%%%%%%%%%%%%% Chapter 2 %%%%%%%%%%%%%%%%%%%%%%%%%%%%%
\end{document}