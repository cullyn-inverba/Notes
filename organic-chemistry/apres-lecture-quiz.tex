\documentclass[12pt,a4paper]{article}
\usepackage{inverba}

\newcommand{\userName}{Cullyn Newman} 
\newcommand{\class}{CH 334} 
\newcommand{\institution}{Portland State University} 
\newcommand{\theTitle}{\color{B-Cold} Apres Lecture Quizzes}

\begin{document}
%%%%%%%%%%%%%%%%%%%%%%%%%%%%%%%%%%%%%%%%%%%%%%%%%%%%%%%%%%%%%%%%%%%%%
\tableofcontents
\cleardoublepage
\fancyhead{}
\fancyhead[R]{\hyperlink{home}{\nouppercase\leftmark}}
%%%%%%%%%%%%%%%%%%%%%%%%%%%%%%%%%%%%%%%%%%%%%%%%%%%%%%%%%%%%%%%%%%%%%


%%%%%%%%%%%%%%%%%%%%%%%%%%%%% Week 1 %%%%%%%%%%%%%%%%%%%%%%%%%%%%%
%\begingroup
\clearpage
\section{Week 1}\phantomsection
\subsection{Friday, October 2}
\begin{itemize}
    \item Determing formal charge:
    \begin{itemize}
        \item Formula: {\color{o-Sun}\(FC = V - N - \dfrac{B}{2}\)}
        \item V = valance electrons of element
        \item N = lone pair electrons; B = bonded electrons
    \end{itemize}
    \item[1.] What is the formal charge on P in the following structure?  Each F and O has three lone pair of electrons.
        \begin{itemize}
            \item P = 5 - O - 8(0.5); P = {\color{pos}+1}
        \end{itemize}
    \item[2.] What is the formal charge on O in the structure above?
        \begin{itemize}
            \item O = 6 - 6 - 2(0.5); O = {\color{neg}-1}
        \end{itemize}
    \item[3.] What is the formal charge on P in the following structure?  Each F still has three lone pairs of electrons, and O had the tow pairs indicated.
        \begin{itemize}
            \item P = 5 - 0 - 10(0.5); P = \textbf{0}
        \end{itemize}
    \item[4.] Of the two structures shown for \ch{POF3}, which is the most stable, and will, therefore, be the most abundant form?
        \begin{itemize}
            \item \textbf{Structure II}
            \item \ch{O} has formal charge of \textbf{0} and is the 
            {\color{neg}most electronegative} element with difference in charge between the resonance structures.
            \item \ch{F} has greater electronegativity, but remains the same between both structures, so it's not relevant.
            \item Key difference: the double bond in structure II gives oxygen the {\color{o-Sun}lower magnitude} formal charge between the two.
        \end{itemize}
    \item[5.] The fundamental concept upon which VSEPR, and hence molecular shapes, is based is that:
        \begin{itemize}
            \item Electrons pairs repel each other;
                \begin{itemize}
                    \item negative charge repels other negative charges.
                \end{itemize}
            \item Electron repulsion is minimized by maximum angular separation; 
                \begin{itemize}
                    \item in other words, angular separation maximizes distance between electrons.
                \end{itemize}
            \item Bonding pair electrons and lone pair electrons both occupy regions around the central atom;
                \begin{itemize}
                    \item if they didn't occupy the same space than they wouldn't interact and thus wouldn't affect shape.
                \end{itemize}
            \item The electron dommain geometry and the molecular geometry is identical if there all of the electrons are bonding electrons;
                \begin{itemize}
                    \item the lone pairs are have a greater influence than bonded pairs, resulting in less space for bonded pairs.
                \end{itemize}
            \item \textbf{All of the above}
        \end{itemize}
    \item General method of determining structure:
    \begin{itemize}
        \item[1.] Count steric number---the total number of electron pairs in a molecule. Can be bonds or lone pairs.
        \item[2.] Determine predicted geomterical structure predicted (EDG) by VSEPR using steric number.
            \begin{itemize}
                \item Octahedral:6, Bipyramid:5, Tetrahedral:4, Trigonal:3, Linear:2
            \end{itemize}
        \item[3.] Determin impact (the MG) of lone pairs; more lone pairs results in less space between bonded pairs. Shape depends on EDG.
    \end{itemize}
    \item[6.] A resonance form of \ch{SOF2}, completely consistent with the octet rule,  is shown below.  What is the electron domain geometry (EDG), and molecular geometry (MG) of this molecule?
        \begin{itemize}
            \item \textbf{Tetrahedral EDG and trigonal pyramidal MG}
        \end{itemize}
    \item[7.\textbf{The lone pair electrons take up more space than bonding pair electrons.}] Draw a Lewis dot structure of formaldehyde (\ch{CH2O}): what is the molecular shape of this molecule?
        \begin{align*}
            \chemfig{C(=[2,]O)(-[5]H)(-[7]H)}
        \end{align*}
        \begin{itemize}
            \item Steric number = 3
                \begin{itemize}
                    \item Double bonds count as 1 for steric number.
                \end{itemize}
            \item No lone pairs on central atom, C, so it's shape planar. 
            \item \textbf{Trigonal planar}
        \end{itemize}
    \item[8.] The EDG for \ch{CH3-} (a carbanion) is tetrahedral, and the MG is trigonal pyramidal.  Why are the \ch{H-C-H} bond angles less than \ang{109.5} as in a perfect tetrahedron? 
        \begin{itemize}
            \item \textbf{The lone pair electrons take up more space than bonding pair electrons.}
        \end{itemize}
\end{itemize}
%\endgroup
%%%%%%%%%%%%%%%%%%%%%%%%%%%%% Week 1 %%%%%%%%%%%%%%%%%%%%%%%%%%%%%

%%%%%%%%%%%%%%%%%%%%%%%%%%%%% Chapter 2 %%%%%%%%%%%%%%%%%%%%%%%%%%%%%
%\begingroup
\clearpage
\section{Week 2}\phantomsection
\subsection{Monday, October 5}
\begin{itemize}
    \item 
\end{itemize}

\subsection{Wednesday, October 7}
\begin{itemize}
    \item 
\end{itemize}

\subsection{Friday, October 9}
\begin{itemize}
    \item 
\end{itemize}
%\endgroup
%%%%%%%%%%%%%%%%%%%%%%%%%%%%% Chapter 2 %%%%%%%%%%%%%%%%%%%%%%%%%%%%%
\end{document}