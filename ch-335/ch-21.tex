\chapter{21: Benzene and Aromaticity}\label{21: Benzene and Aromaticity}

\section{Determining Aromaticity}\label{Determining Aromaticity}
\begin{itemize}
  \item \ddd{Aromaticity}: a property of cyclic, planar structures with \(\pi \) bonds in resonance, giving increased stability relative to other arrangements with same set of atoms.
  \begin{itemize}
    \item Most aromatic compounds are derivatives of benzene; the word aromatic often informally refers to benzene derivatives despite the existence of non-benzene aromatic compounds.
  \end{itemize}
  \item Characteristics of aromatic systems:
    \begin{itemize}
      \item A \emph{delocalized conjugated} \(\pi \) system.
      \item \emph{Planar}, with all contributing atoms in the same plane.
      \item \emph{Cyclic}, arranged in one or more rings.
      \item Follows \emph{H\"uckel's rule}; the number of \(\pi \) electrons is \emph{even}, but \emph{not a multiple of four}.
        \begin{itemize}
          \item \ddd{H\"uckel's rule}: if a molecule has \(4n + 2\pi \)-electrons, it is aromatic; \(n= 0\) or any integer.
          \item \ddd{Antiaromatic}: when an aromatic system has \(4n~\pi \)-electrons (a multiple of four), making it highly unstable and reactive; \(n\neq 0\)
        \end{itemize}
    \end{itemize}
  \item Least stable \(\leftarrow \) antiaromatic < non-aromatic < aromatic \(\rightarrow \) most stable
  \item 
\end{itemize}


\section{Benzene Derivative Nomenclature}\label{Benzene Derivative Nomenclature}
\begin{itemize}
  \item 
\end{itemize}
