\chapter{17: Mass Spectrometry and Infrared Spectroscopy}\label{17: Mass Spectrometry and Infrared Spectroscopy}
\section{Infrared Spectrometry}\label{Infared Spectrometry}
\begin{itemize}
  \item \ddd{IR spectroscopy}: measurement of the interaction of infrared radiation with matter by absorption, emission, or reflection.
  \begin{itemize}
    \item Used to study and identify chemical substances or functional groups. 
    \item Uses an infrared spectrometer to produce an infrared spectrum.
  \end{itemize}
  \item The infrared spectrum is plotted on a graph with absorbance (or transmittance) vs.\ frequency.
  \item \ddd{Wavelength} \(\lambda \): the distance over which a wave shape repeats.
  \item \ddd{Frequency} \(\nu, f \): number of occurrences per unit of time (period; \(T\)), i.e., \(\nu = \dfrac{1}{T}\) 
  \item \(\lambda \) and \(\nu \) are inversely proportional:
  \begin{align*}
    c &= \lambda\nu &
    \frac{c}{\lambda} &= \nu &
    \lambda &\propto \frac{1}{\nu} &
    c\left(\frac{1}{\lambda}\right) &= \nu &
    c\tilde{\nu} &= \nu
  \end{align*}
  \item \ddd{Wavenumber} \(\tilde{\nu}=\si{\per\cm} \): wave number is the spatial frequency of a wave, rather than number of waves per distance, making \(\tilde{\nu} \propto \nu \)
  \begin{itemize}
    \item Energy: \(E = h\nu \); \(h\)= Planck's constant.
    \item Thus, \emph{\(E \propto \nu \propto \tilde{\nu}\)}
    \item Charts usually use frequency represented in wavenumbers.
  \end{itemize}
  \item Bonds can be thought of as springs that are constantly moving. IR graphs exploits the fact that molecules absorb frequencies that are characteristic of their structure.
  \begin{itemize}
    \item Peaks in the IR spectrum arise when a bond absorb energy that matches its vibrational energy, provides it has a dipole moment. 
    \item The \emph{stronger the bond}, the faster its vibrations and the \emph{greater the \(\tilde{\nu} \)}.
    \item The \emph{smaller the atoms} in a bond, the faster its vibrations and the \emph{greater the \(\tilde{\nu} \)}.
  \end{itemize}
  \item Calculating \(\tilde{\nu} \):
  \begin{equation*}
    \tilde{\nu} = \frac{1}{2\pi c}\sqrt{\frac{k}{\mu}}
  \end{equation*}
  Where \(k\) is the spring constant for the bond, \(c\) is the speed of light, and \(\mu \) is the relative mass.
  \item A spectrograph is often interpreted as having two regions:
    \begin{itemize}
      \item \ddd{Functional group region}: \(\geq 1500~\tilde{\nu} \).
      \begin{itemize}
        \item Most work done in this class will be in functional group region.
        \item Shapes of the troughs help determine compounds in functional group region:
          \begin{itemize}
            \item \ch{-OH} has a large wide ``tongue shape''.
            \item \ch{-COOH} has a wide and staggered ``beard shape''.
            \item \ch{R-NH2} has a double ``fang shape''.
            \item \ch{R2-NH} has a single ``fang shape''.
          \end{itemize}
      \end{itemize}
      \item \ddd{Fingerprint region}: \(< 1500~\tilde{\nu} \).
        \begin{itemize}
          \item Generally there are many troughs which form intricate patters, which can used to determine certain compounds.
        \end{itemize}
    \end{itemize}
  \item \ddd{Conjugation}: alternating \(\sigma \) and \(\pi \) bonds. Allows for electron delocalization which in general lowers overall energy of the molecule and increases stability.
    \begin{itemize}
      \item This leads to \emph{lower \(\tilde{\nu} \)} due to the electron delocalization.
    \end{itemize}
  \item Strong inductive effects towards electronegative atoms can retain double bond character if it is near one. This double bond ``sealing'' increases \(\tilde{\nu} \) relative to resonance found in anhydrides due to increased double bond character.
    \begin{itemize}
      \item Inversely, inductive effects towards double bonded oxygen decrease \(\tilde{\nu} \) due to decreased double bond character.
    \end{itemize}
  \item Carboxylic acid has even more increased resonance if it close to another one, which decreases overall double bond character and thus decreases \(\tilde{\nu} \). 
    \begin{itemize}
      \item Likewise, with \ch{NH2} rather than \ch{OH}, then there is even more of a tendency to enter resonance.
    \end{itemize}
\end{itemize}
  
\section{Mass Spectrometry}\label{Mass Spectrometry}
\begin{itemize}
  \item \ddd{Mass Spectrometry (MS)}: an analytical technique that is used to measure the mass-to-charge ratio of ions; typically presented as a mass spectrum.
    \begin{itemize}
      \item \ddd{Mass-to-charge ratio} \(\frac{m}{Q}\): a physical quantity that is used in electrodynamics of charged particles; two particles with the same mass-to-charge ratio move in the same path in a vacuum when subjected to same magnetic field.
      \item \ddd{Mass spectrum}: an intensity of \(\frac{m}{z}\) (a dimensionless unit of \(\frac{m}{Q}\)) representing chemical analysis; used to represent the distribution of ions by mass in a sample.
    \end{itemize}
  \item Basic technique involves taking a sample, vaporizing it to gas form, then bombarding the ions with high energy electrons, then analyzing the unstable radical cations in order to determine relative abundance of ions by mass.
    \begin{itemize}
      \item \ddd{Molecular ion peak} \(M^+\): corresponds to that of the molecule with all of its atoms intact, rather than a fragment, which themselves cause other minor peaks.
      \item \ddd{Base peak}: The tallest peak, often is the same as the molecular ion peak, but not always.
      \item \ddd{M + 1 peak}: peaks that are due to molecular ions containing heavier isotopes of their atoms.
        \begin{itemize}
          \item E.g., relative intensity of M+1 peak divided by 1.1 gives you the number of carbon atoms if the M peak has an intensity of 100\%.
        \end{itemize}
      \item \ddd{M + 2 peak}: used to determine if there are halogens in the molecule.
        \begin{itemize}
          \item Relative natural abundance of chlorine: 3:1
          \item Relative natural abundance of bromine: 1:1
          \item Thus, roughly equal M + 2 peaks signal for Br, while uneven peaks signal for Ch.
        \end{itemize}
      \item The atoms or molecules can then be identified by correlating known masses of the molecule or atoms to the identified masses through characteristic fragmentation patterns.
      \item Molecules containing only \ttt{C, H and/or O} atoms will have an \ttt{even molecular weight}.
      \item \ddd{Nitrogen rule}: molecules with an \fff{odd number of nitrogen} atoms will have an \fff{odd molecular weight}.
      \item In general, one can do the following to find the molecular formula:
        \begin{enumerate}
          \item Assign M, M + 1, and M + 2 peaks.
          \item Multiple peaks by what ever factor makes M's peak 100\%.
          \item Use M to determine if nitrogen is present, most of the time there will just be 1 if there is.
          \item Estimate number of carbons by dividing M + 1 relative abundance by 1.1.
          \item Determine if Cl or Br is present by comparison of M + 2 peaks. (use lower number isotope for calculation)
          \item Determine if oxygen is needed after accounting for halides and nitrogen. 
          \item Add remaining hydrogens needed to reach M's amu.
        \end{enumerate}
    \end{itemize}
  
\end{itemize}
