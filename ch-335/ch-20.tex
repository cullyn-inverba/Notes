\chapter{20: Conjugation and Resonance}\label{20: Conjugation and Resonance}
\section{Conjugation Basics/Review}\label{}
\begin{itemize}
  \item \ddd{Delocalization}: electrons not associated with a single atom or covalent bond.
    \begin{itemize}
      \item Refers to resonance in conjugated systems and aromatic compounds.
    \end{itemize}
  \item Conjugation: a system of connected p orbitals with delocalized electrons that generally lowers overall energy and increases stability.
    \begin{itemize}
      \item Represented as alternating \(\sigma \) and \(\pi \) bonds, i.e., the overlap of a p orbital with another across an adjacent \(\sigma \) bond.
      \item Most efficient when the p orbitals are coplanar.
    \end{itemize}
  \item General rules of contributing significance:
    \begin{enumerate}
      \item The greatest number of filled octets.
      \item The greatest number of covalent bonds.
      \item Minimize formally charged atoms.
      \item Minimization of unlike charged and maximization of like charges.
      \item Negative charges are on the most electronegative atoms, positive charges are on the most electropositive atoms.
      \item Do not deviate substantially from idealized bond lengths and angles.
      \item Maintain aromatic substructures locally while avoiding anti-aromatic molecules.
    \end{enumerate}


  \subsection{Conjugated Dienes}\label{Conjugated Dienes}
  \begin{itemize}
    \item Additions reaction of with HX and dienes result in both 1,2-addition and 1,4-addition pathways:
    
    \medskip
    \schemestart{}
    \chemfig{=[:30]-[:-30]=[:30]}
    \arrow{->[\ch{H-Br}]}
    \chemfig{4=[:30]3-[:-30]\pbe{2}-[:30]1(-[:90]H)}
    \+
    \chemfig{\pbe{4}-[:30]3=[:-30]2-[:30]1(-[:90]H)}
    \arrow{->[\bbb{\ch{Br^{-}}}]}
    \dots
    \schemestop{}
    \bigskip
    
    \medskip
    \schemestart{}
    \dots
    \qquad
    \chemfig{4=[:30]3-[:-30]2(-[:-90]Br)-[:30]1(-[:90]H)}
    \+
    \chemfig{4(-[:-90]Br)-[:30]3=[:-30]2-[:30]1(-[:90]H)}
    \schemestop{}
    \bigskip
    
    \item Temperature can affect which diene addition pathway is taken.
    \begin{itemize}
      \item \SI{0}{\celsius} resulted in a preference for the 1,2-pathway.
      \item \SI{40}{\celsius} resulted in a preference for the 1,4-pathway.
    \end{itemize}
    \item \ddd{Kinetic product}: the 1,2-product, which is formed more rapidly due to low activation energy, but it is less thermodynamically stable.
    \item \ddd{Thermodynamic product}: the 1,4-product, which has a higher activation energy, but is more thermodynamically stable.
  \end{itemize}
\end{itemize}


\clearpage
\section{Pericyclic Reactions}\label{Pericyclic Reactions}
\begin{itemize}
    \item \ddd{Pericyclic Reaction}: 
\end{itemize}


