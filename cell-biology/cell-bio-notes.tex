\documentclass[12pt,a4paper]{article}
\usepackage{inverba}
\newcommand{\userName}{Cullyn Newman} 
\newcommand{\class}{BI 336} 
\newcommand{\institution}{Portland State University} 
\newcommand{\thetitle}{\hypertarget{home}{Cell Biology Notes}}
\rfoot{\hyperlink{home}{\thepage}}

\begin{document}
%%%%%%%%%%%%%%%%%%%%%%%%%%%%%%%%%%%%%%%%%%%%%%%%%%%%%%%%%%%%%%%%%%%%%
\tableofcontents
\cleardoublepage
\fancyhead{}
\fancyhead[R]{\hyperlink{home}{\nouppercase\leftmark}}
%%%%%%%%%%%%%%%%%%%%%%%%%%%%%%%%%%%%%%%%%%%%%%%%%%%%%%%%%%%%%%%%%%%%%

%%%%%%%%%%%%%%%%%%%%%%%%%%%%% Chapter 1 %%%%%%%%%%%%%%%%%%%%%%%%%%%%%
%\begingroup
\clearpage
\setcounter{section}{14}
\section{Cell Signaling}
\subsection{Principles of Cell Signaling}
\begin{itemize}
    \item Extracellular signaling \(\rightarrow\) Receptor protein \(\rightarrow\) Intracellular signaling proteins \(\rightarrow\) Effector proteins.
    \item Effector proteins consists transcription regulators, ion channels, or metabolic enzymes.a
    \item More than 1500 genes encode for receptor proteins, and variation in the proteins are increased even more due to alternative RNA splicing and post-translational modifications.
\end{itemize}

\subsubsection{Extra Signals Can Act Over Short or Long Distances}
\begin{itemize}
    \item \textbf{Contact dependent signaling}: many extracellular signals are bound to the surface of cells that require direct contact from other membrane bound signaling cells in order for activation.
    \item \textbf{Local mediators}: secreted molecules that generally only act on cells in the local environment.
    \item \textbf{Paracrine signaling}: result of signaling using local mediators, and generally acts on cells of different types.
    \item \textit{Autocrine signaling}: results of local mediators acting on the cells that secret them.
    \item \textbf{Endocrine cells}: signaling over long distance using hormones that travel through the bloodstream. 
\end{itemize}

\subsubsection{Classes of Cell-Surface Receptor Proteins}
\begin{itemize}
    \item Most extracellular signals do not enter the nucleus, instead they act as signal transducers of extracellular ligand-binding events.
    \item \textbf{Ion-channel-coupled receptors}: or ionotropic receptors, are involved in rapid synaptic signaling between nerve cells of other electronically excitable cells.
    \item \textbf{G-protein-coupled receptors}: indirectly regulating separate membrane-bound target proteins, usually using enzymes or ion channels, resulting in small intracellular signaling molecules to alter the behavior of different signaling proteins in the cell.
    \item \textbf{Enzyme-coupled receptors}: either function as or directly associate with enzymes they activate. Usually are single pass transmembrane proteins that have ligand-binding site outside the cell and the catalytic site inside.
    \item Majority of enzyme are either protein kinases of associate with protein kinases.
\end{itemize}

\subsubsection{Cell-Surface Receptors Relay Signals Via Intracellular Signaling Molecules}
\begin{itemize}
    \item \textbf{Second messengers}: small chemicals involved in intracellular signaling that are generated in large amounts in response to receptor activation and diffuse to spread signals to other parts of the cell.
    \item Either water soluble or lipid soluble, resulting in an extension of the signal by binding/altering the behavior of selected signaling or effector proteins.
    \item Most intracellular molecules are proteins that act like a chain of molecular switches to transmit signals.
    \item \textbf{Protein kinase}: covalently adds one or more phosphate groups.
    \item \textbf{Protein phosphatase}: removes phosphate groups.
    \item 30-50\% of human proteins contain covalently attached phosphate.
    \item Many intracellular proteins are protein kinases themselves, resulting in \textbf{kinase cascades}.
    \item \textit{trimeric GTP-binding proteins}: help relay signals from G-proteins-coupled receptors that activate them.
    \item \textbf{monomeric GTPases} help relay signals from many classes of cell-surface receptors.
    \item \textbf{GTPase-activating proteins (GAPs)}: drive proteins into an "off" state by increasing rate of hydrolysis of bound GTP. 
    \item \textbf{Guanine nucleotide exchange factors (GETs)}: promote the release of bound GDP, allowing new GTP to bind, which activates the proteins.
\end{itemize}

\subsubsection{Modular Interaction Domans Mediate Interactions Between Intracellular Signaling Proteins}
\begin{itemize}
    \item \textit{Induced proximity}: the result of a signal triggering the assembly of a signaling complex, which can activate other proteins simply due to proximity.
    \item \textbf{Interaction domains}: highly conserved domains that are found in many intracellular signaling proteins and highly affect assembly of proximity complexes.
    \item The modular function helped evolution because the new pathways can be inserted with out disturbing the proteins folding of function.
    \item \textbf{Adaptors}: two or more interaction domains that link two proteins together in a signaling pathway.
    \item \textbf{Primary cilium}: an example of how to bring receptors and intracellular signaling together by concentrating them in them specific region of the cell (an antenna like object that projects from the surface of most vertebrate cells).
\end{itemize}

\subsubsection{The Relationship Between Signal and Response Varies in Different Signaling Pathways}
\begin{itemize}
    \item All signaling systems do not work in the same way and have a variety of response factors.    
    \item \textit{Response timing}: can range from milliseconds to hours or days.
    \item \textit{Sensitivity}: controlled by changes to numbers or affinity of the receptors on the target cell and often further influenced by amplification factors.
    \item \textit{Dynamic range}: related to sensitivity. More focused on the broader range sensitivity, responding to the changes in sensitivity itself, which is know as \textbf{adaptation} to the responsiveness according to prevailing amounts of signal.
    \item \textit{Persistance}: how long signaling is active, regulated by numerous mechanisms, including positive feedback.
    \item \textit{Signal processing}: converting simple signals into complex responses; regulated by switch like or oscillatory responses.
    \item \textit{Integration}: allows a response to be governed by multiple inputs; equivalent to AND gates.
    \item \textit{Coordination}: multiple responses due to a single extracellular signal. Depends on mechanisms for distributing a signal to multiple effectors, even sometimes modulating the strength of a response to other signals.
\end{itemize}
\subsection{Signaling Through G-Protein-Coupled Receptors}
\begin{itemize}
    \item \textbf{G-protein-coupled receptors (GPCRs)}: the largest family of cell-surface receptors and the mediators of most responses to cell signaling.
    \item The same signal molecule can activate many different GPCR family members.
    \item GPCRs consists of a single polypeptide chain that threads across the membrane layer seven times, forming a cylindrical structure, often with a ligand binding site in the center.
\end{itemize}

\subsubsection{Trimeric G Proteins Relay Signals From GPCRs}
\begin{itemize}
    \item \textbf{trimeric GTP-binding protein (G protein)}: activated when a receptor undergoes a conformational change and couples the receptor to enzymes or ion channels in the membrane.
    \item The G protein can be associated with the receptors before of after activation depending on the type.
    \item G proteins are composed of three subunits ---$\alpha$, $\beta$, and $\gamma$.
    \item Unstimulated state: $\alpha$ subunit has GDP bound and G protein is inactive.
    \item When GPCR is activated it acts like GEF and induces $\alpha$ subunit to release bound GDP.
    \item GTP binding induces conformational change in G$\alpha$ subunit, releasing G protein from receptor and dissociation of GTP-bound G$\alpha$ from G$\beta$$\gamma$ pair.
    \item \textbf{regulator of G protein signaling (RGS)}: proteins that act as a $\alpha$-subunit-specific GAPs and help shut off G-protein-mediated responses.
\end{itemize}
\subsubsection{Some G Proteins Regulate the Production of Cyclic AMP}
\begin{itemize}
    \item \textbf{Cyclic AMP (cAMP)}: acts as a second messenger in some pathways and helps with signal transduction.
    \item cAMP is synthesized from ATP by \textbf{adenylyl cyclase} and destroyed by \textbf{cyclic AMP phosphodiesterase}.
    \item Many extracellular signals work by increasing cAMP.
\end{itemize}

\subsubsection{Cyclic-AMP-Dependent Protein Kinase (PKA) Mediates Most of the Effects of Cyclic AMP}
\begin{itemize}
    \item \textbf{Cyclic-AMP-dependent protein kinase (PKA)}: activated by cAMP and used to regulates selected proteins through phosphorylating specific serines of threonines.
    \item PKA consists of a complex of two catalytic subunits and two regulatory subunits.
    \item Binding of cAMP causes the regulatory subunits to dissociate and are free to phosphorylate specific target proteins.
    \item \textbf{CRE-biding (CREB) protein}: a transcription regulator that recruits a transcriptional coactivator called \textit{CREB-binding protein (CBP)}. Thus, CREB can transform a short cAMP signal into a long-term change in the cell. (May play a role in some forms of learning and memory)
\end{itemize}

\subsubsection{Some G protins Signal Via Phospholipids}
\begin{itemize}
    \item \textbf{Phospholipase C-$\beta$ (PLC$\beta$)}: plasma membrane bound enzyme that many GPCRs exert their effects through.
    \item \textbf{Phosphatidyblinositol 4,5-bisphosphate [PI(4,5)P$_2$]}: a phosphorylated inositol phospholipid (phosphoinositide) that C-$\beta$ acts on and is present in small amounts in the inner half ot the plasma membrane.
    \item \textbf{G$_q$}: a G protein that activate the inositol phospholipid signaling pathway.
    \item \textbf{IP$_3$} and \textbf{diacylglycerol}: products of the cleaved PI(4,5)P$_2$ and the step at which signaling pathways split.
    \item IP$_3$ is a water soluble molecule that enters the ER and binds there, releasing the stored \ch{Ca^2+} into the cytosol.
    \item Diacylglycerol acts as as second messengers embedded in the plasma membrane, with a major function that activates a protein kinase called \textbf{PKC} (named due to \ch{Ca^2+} dependency).
    \item One activated, PKC various proteins depending on cell type.
    \item Diacylglycerol can also be cleaved further to release arachiconic acid, which acts a signal of used to make other small lipids caleld \textit{eicosanoinds}, which take part in pain and inflammatory responses.
\end{itemize}

\subsubsection{\texorpdfstring{\ch{Ca^2+}}- Functions as a Ubiquitous Intracellular Mediator}
\begin{itemize}
    \item \ch{Ca^2+} has numerous other functions in a variety of cell types.
    \item \textbf{Ryanodine receptor}: a second type of regulated \ch{Ca^2+} channel located in the ER membrane that opens in response to \ch{Ca^2+} and amplifies the \ch{Ca^2+} signal.
\end{itemize}

\subsubsection{Feedback Generates \texorpdfstring{\ch{Ca^2+}}- Waves and Oscillations}
\begin{itemize}
    \item Both IP$_3$ receptors and ryanodine receptors are stimulated by low to moderate cytoplasmic \ch{Ca^2+} concentrations.
    \item If the extracellular signals are sufficiently strong and persistant, then they may activate nearby IP$_3$ and ryanodine receptors, resulting in a wave of \ch{Ca^2+}, similar to actions potentials.
    \item High concentrations of \ch{Ca^2+} are inhibited and leads to a release of \ch{Ca^2+}. This results in an oscillations in \ch{Ca^2+} concentrations.
    \item Frequency dependent responses can also be oscillatory or non-oscillatory.
\end{itemize}

\subsubsection{\texorpdfstring{\ch{Ca^2+}}-/Calmodulin-Dependent Protein Kinases Mediate Many Responses to \texorpdfstring{\ch{Ca^2+}}- Signals}
\begin{itemize}
    \item \textbf{Calmodulin}: a multipurpose intracellular \ch{Ca^2+} receptor found in all eukaryotic cells and constitute as much as 1\% of the cell's total protein mass.
    \item Calmodulin displays a sigmoidal response to increasing concentrations of \ch{Ca^2+}.
    \item \textbf{CaM-kinase II}: found in most animal cells and is especially enriched in the nervous system--highly concentrated in synapses.
    \item Mutant mice lacking the enzyme have specific defects in memory, suggesting it plays a significant role in memory and learning.
    \item CaM-kinase II can use its intrinsic memory mechanism to decode the frequency of \ch{Ca^2+} oscillations.
\end{itemize}

\subsubsection{Some G Proteins Directly Regulate Ion Channels}
\begin{itemize}
    \item G proteins to not exclusively regulate membrane-bound enzymes. 
    \item G proteins can directly activate ion channels in the plasma membrane of target cells--thereby altering ion electrical excitability of the membrane.
\end{itemize}

\subsubsection{Smell and Vision Depend on GPCRs That Regulate Ion Channels}
\begin{itemize}
    \item \textbf{Olfactory receptors}: receptors with specific GPCRs that have modified cilia, which project from the surface of the neuron.
    \item Binding orderants activate the G\(_{t}\) which activates adenylyl cyclase, increasing cAMP that in turn opens the cyclic-AMP-gated cation channels.
    \item Vision has a similar proccess, but using a different nucleotide---\textbf{GMP}.
    \item Receptor activation by light causese a fall rather than a rise in the cyclic nucleotide.
    \item Rods are made of a stack of discs that are densley packed with rhodopsin.
    \item Unlike most synaptic signaling, cyclic GMP causes hyperpolarization, causing the decrease in GMP and closing of channels.
    \item Activated rhodopsin alters the conformation of G$_t$, causing causing transducin $\alpha$ subunit to activate \textbf{cyclic GMP phosphodiesterase}.
    \item Rods use several negative feedback loops that allow the cell to quickly revert to a resting state and also help rods adapt to contiously exposed light--allowing small changes in light to be possible.
    \item Families of G proteins are determined by amino acid sequence relatedness of the $\alpha$ subunits.
\end{itemize}

\subsubsection{Nitric Oxide is a Gaseous Signaling Mediator that Passes Between Cells}
\begin{itemize}
    \item \textbf{Nitrid oxide (NO)}: a signaling molecule in many tissues of plants and animals that can readily cross the plasma membrane.
    \item NO acts locally due to a short half-life, about 5-10 seconds in the extracellular space.
    \item NO is made by the deamination of arginine and catalyzed by \textbf{NO synthases (NOS)}.
    \item NO binds reversibly to iron in the active site of guanylyl cyclase, stimulating synthesis of cyclic GMP.
    \item NO can increase cyclic GMP in the cytosol within seconds.
\end{itemize}

\subsubsection{GPCR Desensitization Depends on Receptor Phosphorylation}
\begin{itemize}
    \item There are three general modes of adaptation:
        \begin{itemize}
            \item \textit{Receptor sequestration}: temporary movement into interior of the cell to remove ligand access.
            \item \textit{Receptor down-regulation}: destroyoed in lysosomes after internalization.
            \item \textit{Receptor inactivation}: altered in order to stop interaction with G protein.
        \end{itemize}
    \item All modes depend on the family of \textbf{GPCR kinases(GRKs)}
    \item GRKs phosphorylates multiple serines and threonines on a GPCR only after ligand binding due to the activated receptor being the one to allosterically activate GRK.
    \item Once a receptor has been phosphorylated by GRK, then it binds with high affinity t00o a member of the arrestin family of proteins.
    \item Bound arrestin helps in two ways: by preventing the activated receptor from interacting with G proteins and serves as an adaptor protein to help couple receptors in the clathrin-dependent endocytosis machinery.
\end{itemize}

\subsection{Signaling Through Enzyme-Coupled Receptors}
\begin{itemize}
    \item \textbf{Enzyme coupled receptors}: similar to GPCRs, with are transmembrane proteins with the ligand binding domain on the extracellular side, with the catalytic domain with on the cytosolic side.
    \item enzyme-coupled receptors have only a single transmembrane helix.
\end{itemize}

\subsubsection{Activated Receptor Tyrosine Kinases Phosphorylate Themselves}
\begin{itemize}
    \item \textbf{Receptor tyrosine kinases (RTKs)}: most common enzyme-coupled receptors for many growth factors, cytokines, and hormones.
    \item Most RTKs ligand binding causes receptors to dimerize, bringing cytoplasmic kinase domains closer together and promoting activity.
    \item Phosphorylated tyrosines on RTKs serve as docking sites for intracellular signaling proteins.
\end{itemize}

\subsubsection{Proteins with SH2 Domains Bind to Phosphorylated Tyrosines}
\begin{itemize}
    \item \textit{Interaction domains}: modular domain that mediates protein-protein interactions tht relay the signal of activated RTKs.
    \item \textbf{Phospholipase C-$\gamma$ (PLC$\gamma$)}: activates the inositol phospholipid signlaning pathway, which allows RTKs to increase cytosolic \ch{Ca^2+} levels that activate PKC.
    \item \textbf{SH2 domains}, and \textit{PTB domains}: highly conserved phophotyrosine-binding domains. These interaction domains enable proteins to bind to activated RTKs and other intracelluar signaling porteins that have been transiently phosphorylated.
    \item Not all proteins that bind to RTKs via SH2 domains relay the signal, some provide negative feedback.
\end{itemize}

\subsubsection{The GTPase Ras Mediates Signaling By Most RTKs}
\begin{itemize}
    \item \textbf{Ras superfamily}: a faminly of monomeric GTPases that relies on signals form cell-surface receptors and act can act as a signaling hub.
    \item In general, the Ras family is responsible for cell proliferation or differentiation. With Rho for cell morphology, Ran for nuclear transport, and Rab and Arf for vesicle transport.
    \item 30\% of human tumors express hyperactive mutant forms of Ras due to locked in of GTP-bound active state.
\end{itemize}

\subsubsection{Ras Activates a MAP Kinase Signaling Module}
\begin{itemize}
    \item \textbf{MAP kinase module}: coverts short lived signaling events into longer lasting ones that can relay it downstream to the nucleus in order to alter gene expression.
    \item MAP has three well conserved protein kinase components:
        \begin{itemize}
            \item Raf = MAPKKK
            \item Mek = MAPKK
            \item Erk = MAPK 
        \end{itemize}
    \item Scaffold proteins help prevent cross-talk between parallel MAP kinase modules.
\end{itemize}

\subsubsection{Rho Family GTPases Functionally Couple Cell-Surface Receptors to the Cytoskeleton}
\begin{itemize}
    \item Rho family regulate both the actin and microtubule cytoskeletons in order to control shape, polarity, motility, and adhesion.
    \item Rho also regulates cell-cycle progression, gene transcription, and membrane transport.
    \item Rho family are often bound to \textit{guanine nucleotide dissociation inhibitors (GDIs)} in the cytosol, preventing interaction with Rho-GEFs at the plasma membrane.
    \item \textbf{Ephrin}: a family of proteins that serve as the ligand for the eph receptors, which compose the largest known subfamily of RTKs.
\end{itemize}

\subsubsection{The PI 3-Kinase Produces Lipid Docking Sites in the Plasma Membrane}
\begin{itemize}
    \item \textbf{PI 3-kinase}: a kinase that principally phosphorylates inositol phospholipids rather than proteins and plays a central part in promoting cell survival and growth.
    \item 
    \item 
\end{itemize}
%\endgroup
%%%%%%%%%%%%%%%%%%%%%%%%%%%%% Chapter 1 %%%%%%%%%%%%%%%%%%%%%%%%%%%%%

%%%%%%%%%%%%%%%%%%%%%%%%%%%%% Chapter 16 %%%%%%%%%%%%%%%%%%%%%%%%%%%%
%\begingroup
\clearpage
\section{The Cytoskeleton}
\subsection{Function and Origin of the Cytoskeleton}
\begin{itemize}
    \item The cytoskeleton functions depend on three families: \textit{actin filaments, microtubules, and intermediate filaments}
    \item Actin determins shape, helps in whole-cell locomotion, and cell division.
    \item Microtubules determine position of membrane-enclosed organelles, direct intracellular transport, and form miotic spindles that segregate chromosomes.
    \item Intermediate filaments provide mechanical strength and interact with hundereds of accessory proteins that regulate and link the filaments.
    \item \textbf{Protofilaments}: linear strings of subunits joined end-to-end and associate with one another laterally to form hollow cylinders in order to provide strength and adaptability.
    \item Subunit assembly and disassembly constantly remodel all three types of cytoskeletal filaments.
\end{itemize}

\subsubsection{Actin anc Actin-Binding Proteins}
\begin{itemize}
    \item Actin is extraordinarily conserved among eukaryotes.
    \item $\alpha$-actin is expressed in only muscle cells.
    \item $\beta$ and $\gamma$ actin are found together in almost all non-muscle cells.
    \item \textbf{minus end}: "pointed" polar side of F-actin that grows slowly.
    \item \textbf{plus end}: "barbed" polar side that grows faster.
    \item Nucleation is the rate limiting step in the formation of acting filaments.
    \item \textbf{Critical concentration}: concentration of free subunits after equilibrium between subunit dissociation and addition.
    \item Cells catalyze filament nucleation at specific sites in order to determin the location at which new actin filaments are assembled.
    \item ATP hydrolysis within actin filaments leads to treadmilling at steady state.
    \item Polymerization follows as sigmoidal curve due to the lag phase form initial nucleation. 
    \item ATP caps go on actin filaments while GTP caps go on microtubules.
    \item \textbf{Dynamic instability}: microtubules depolymerize 100 times faster from an end with GDP-tubulin than GTP, allowing for slow growth and disassembly.
\end{itemize}

\subsubsection{Monomer Availability Controls Actin Filament Assembly}
\begin{itemize}
    \item \textbf{Thymosin}: a small protein that binds to actin monomers, preventing association with the plus of minus end and can neither hydrolyze nor exchange their bound nucleotide.
    \item \textbf{Profilin}: a protein that binds to face opposite to the ATP-binding cleft, blocking minus end association.
\end{itemize}
\subsubsection{A-Nucleating Factors Accelerate Polymerization and Generate Branched of Straight Filaments}
\begin{itemize}
    \item \textbf{Arp 2/3 complex}: nucleates actin filament growth from the minus end, allowing rapid elongation at the plus end and forms a tree like web.
    \item \textbf{Formins}: dimeric proteins that nucleate the growth of straight, unbranched filaments that can form parallel bundles.
    \item Profilin, arp 2/3, and formins primarily occurs at the plasma membrane. 
    \item \textbf{Cell cortex}: layer just beneath the plasma membrane, and the area where actin filaments determin the shape and movement of the cell surface.
\end{itemize}

\subsubsection{Actin-Filament-Binding Porteins Alter Filament Dynamics}
\begin{itemize}
    \item \textbf{Tropomyosin}: an elongated protein that binds to six or seven adjacent actin subunits along each of the two grooves of the helical actin filament. This stabilizes, stiffens, and prevents actin filament form interacting with other proteins.
    \item \textbf{CapZ}: stabilizes an actin filament at the plus end, reducing growth and depolymerization. 
    \item \textbf{Tropomodulin}: binds tightly to the minus end of actin filaments. Also can transiently cap pure actin filaments. There is a large family or tropomodulin proteins.
\end{itemize}

\subsubsection{Severing Proteins Regulate Actin Filaments Depolymerization}
\begin{itemize}
    \item \textbf{Gelsolin superfamily}: a class of actin-severing proteins that are activated by high levels or cytosolic \ch{Ca^2+}.
    \item After severing, gelsolin remains attached to the actin filament anc caps to the new plus end.
    \item \textbf{Cofilin}: binds along the length of the actin filament, forcing the filament to twist, making it more easily severed by thermal motions.
    \item Cofilin preferentially binds to ADP-containing filaments, which results in a tendency to dismantle older filaments in the cell.
    \item Tropomyosin binding can protect filaments from cofilin.
\end{itemize}

\subsubsection{Filament Arrays Influence Cellular Mechanical Properties and Signaling}
\begin{itemize}
    \item Actin filaments are organized into several types of arrays: dendritic networks, bundles, and weblike networks. 
    \item \textit{bundling porteins}: cross link actin filaments into a parallel array.
    \item \textit{Gel-forming proteins}: hold two actin filaments together at a large angle to each other, creating a looser meshwork.
    \item \textbf{Fimbrin}: excludes myosin which prevents them from contracting. 
    \item \textbf{$\alpha$-actinin}: cross links oppositely polarized filaments into loose bundles, allowing binding of myosin and formation of contractile actin bundles. 
    \item \textbf{Filamin}: promotes formation of loose and viscous gel by clamping together two actin filaments roughly at right angels.
    \item \textit{Lamellipodia}: sheetlike membrane projections that help cells crawl along solid surfaces.
    \item \textbf{Spectrin}: long flexible protein made out of four elongated polypeptide chains (two $\alpha$ and two $\beta$ subunits) that concentrate just beneath the plasma membrane, resulting in a network that creates a strong, yet flexible cell cortex, providing mechanical support for the overlying plasma membrane. 
\end{itemize}

\subsection{Myosin and Actin}
\begin{itemize}
    \item Actin-based members ot the myosin superfamily.
    \item \textit{Myosin II}: the first motor protein identified and is responsible for generating force for muscle contraction.
    \item Myosin generates force by coupling ATP hydrolysis to conformational changes.
    \item Each step of the movement along actin is generated by the swinging of $\alpha$ helix, or lever arm.
    \item Sliding of myosin II along actin filaments causes muscles to contract.
    \item \textbf{Mayofibril}: a cylindrical structure that is often as long as the muscle itself. It consits of \textit{sacromeres}, which give the striated appearance.
    \item Each sacromere consists of precisely ordered array or partly overlapping thin and thick filaments.
    \item \textit{Thin filaments}: are composed of actin and associated proteins with their plus ends attached to Z discs with the minus ends extending toward the middle and overlapping with thick filaments. 
    \item \textit{Thick filaments}: bipolar assemblies formed from specific muscle isoforms of myosin II. 
    \item \textbf{Nebulin}: a repeating 35-amino-acid actin-binding motif that determines the length of the small filaments.
    \item Sudden rise in cytosolic \ch{Ca^2+} concentration initiates muscle contraction.
    \item Actin and myosin perform a variety of functions in non-muscle cells.
\end{itemize}

\subsection{Microtubules}
\begin{itemize}
    \item \textbf{Tubulin}: a heterodimer formed from two closely related globular proteins; $\alpha$-tubulin and $\beta$-tubulin that make up microtubules.
    \item Each monomer has a binding site for one GTP.
    \item $\beta$-tubulin can have either GTP or GDP, while $\alpha$-tubulin GTP is static.
    \item Microtubules are hollow cylindrical built from 13 parallel portofilaments, each with $\alpha$$\beta$-tublin stacked hea to tail and then folded into a tube.
    \item Protofilaments line up with the same polar ends, making microtubules have a distinct structural polarity: $\alpha$=minus, $\beta$=plus.
    \item Like actin, plus grows or shrinks faster than the minus end.
    \item Two types of microtubule structures can exist, "D form" with GDP which tends to depolymerize, or "T form" with GTP which tends to polymerize.
    \item \textbf{Dynamic instability}: rapid interconversion between a growing and shrinking state, at uniform free subunit concentration. 
    \item Microtubules functions are inhibited by both polymer-stabilizing and polymer-destabilizing drugs.
\end{itemize}
\subsubsection{A Protein Complex Containing Tubulin Nucleates Microtubules}
\begin{itemize}
    \item $\gamma$-tubulin is present in small amounts and is involved in the nucleation of microtubules.
    \item \textbf{Microtubule-organizing center (MTOC)}: a specific intracellular location where $\gamma$-tubulin is most enriched.
    \item \textbf{$\gamma$-tubulin ring complex ($\gamma$-TuRC)}: two accessory proteins that bind to $\gamma$-tubulin, along several other proteins that create a spiral ring of $\gamma$-tubulin and serves as a template for a microtubule with 13 protofilaments.
\end{itemize}

\subsubsection{Microtubules Emanate from the Centrosome in Animal Cells}
\begin{itemize}
    \item \textbf{Centrosome}: located near the nucleus and from which microtubules are nucleated from their minus end.
    \item \textbf{centrioles}: a pair of cylindrical structures arranged at right angles. 
    \item \textit{pericentriolar material}: the location at which microtubule nucleation takes place using proteins and the centrioles.
    \item Centrosomes form the poles of mitotic spindles after duplication.
\end{itemize}

\subsubsection{Microtubule-Binding Proteins Modulate Filament Dynamics and organization}
\begin{itemize}
    \item \textbf{Microtubule-associated proteins (MAPs)}: proteins that bind to microtubules. 
    \item Some MAPs can stabilizes against disassembly, or mediate interactions with other cell components
    \item MAPs have at least on domain that binds to the microtubule and one that projects outward.
    \item MAPs are the target of several protein kinases.
\end{itemize}

\subsubsection{Plus-End Proteins Modulate Microtubule Dynamics and Attachments}
\begin{itemize}
    \item \textbf{XMAP215}: binds free tubulin subunits and delivers them to the plus end, promoting polymerization.
    \item Phosphorylation of XMAP215 during mitosis inhibits it's activity, resulting in tenfold increase of dynamic instability, crucial for the efficient construction of mitotic spindle.
    \item \textbf{Plus-end tracking proteins(+TIPs)}: accumulate at active plus ends, dissociating when microtubules begin to shrink.
    \item +TIPs act modulate growth and shrinkage of microtubule, control positioning, and interaction with other structures.
\end{itemize}

\subsubsection{Microtubule-Severing Proteins Destabilize Microtubules}
\begin{itemize}
    \item \textbf{Stathmin (Op18)}: binds to two tubulin heterodimers and prevents their addition to the ends of microtubules, preventing assembly.
    \item Phosphorylation inhibits stathmin binding, resulting in microtubule elongation and supression of dynamic instability.
    \item \textbf{Katanin}: a protein that is responsible for severing microtubules. 
    \item Katanin is thought to contribute to the rapid microtubule depolymerization observed at the poles of spindles during mitosis.
\end{itemize}

\subsubsection{Two Types of Motor Proteins Move Along Microtubules}
\begin{itemize}
    \item \textbf{Kinesins and dyneins}: two major classes of microtubule based mortors.
    \item Most of the Kinesins have the motor domain at the N-terminus of the heavy chain and walk toward the plus end. 
    \item Dyneins are family of minus end directed microtubules, consisting of one to three heavy chains, and a large variable number of associated intermediate to light chains.
    \item \textit{cytoplasmic dyneins}: one branch which are homodimers of two heavy chains and they help to traffic organelles and mRNA, positioning of the centrosome and nucleus, and construction of the microtubule spindle in mitosis and meiosis. 
    \item \textit{Axonemal dyneins}: comprise the second branch; they are highly specialized for rapid sliding movements of microtubules that drive the beating of cilia and flagella.
    \item Dyneins are the largest and fastest known molecular motors.
\end{itemize}

\subsubsection{Microtubules and Motors Move Organelles and Vesicles}
\begin{itemize}
    \item \textit{Anterograde axonal transport}: rapid movement of mitochondria, secretory vesicle precursors, and various synapse components down the microtubule highways by Kinesin to distant nerve terminals. (towards plus)
    \item \textit{Retrograde axonal transport}: movement in the opposite direction by cytoplasmic dynein. (towards minus)
    \item When intracellular cyclic AMP levels decrease, the stronger kinesin is inactivated, leaving dynein free to do work.
\end{itemize}


%\endgroup
%%%%%%%%%%%%%%%%%%%%%%%%%%%%% Chapter 16 %%%%%%%%%%%%%%%%%%%%%%%%%%%%
\end{document}