\documentclass[12pt,a4paper]{article}
\usepackage{inverba}
\newcommand{\userName}{Cullyn Newman} 
\newcommand{\class}{BI 341} 
\newcommand{\institution}{Portland State University} 
\newcommand{\thetitle}{\hypertarget{home}{Genetics Notes}}
\rfoot{\hyperlink{home}{\thepage}}

\begin{document}
%%%%%%%%%%%%%%%%%%%%%%%%%%%%%%%%%%%%%%%%%%%%%%%%%%%%%%%%%%%%%%%%%%%%%
\tableofcontents
\cleardoublepage
\fancyhead{}
\fancyhead[R]{\hyperlink{home}{\nouppercase\leftmark}}
%%%%%%%%%%%%%%%%%%%%%%%%%%%%%%%%%%%%%%%%%%%%%%%%%%%%%%%%%%%%%%%%%%%%%
\setcounter{section}{7}
%%%%%%%%%%%%%%%%%%%%%%%%%%%%% Chapter 8 %%%%%%%%%%%%%%%%%%%%%%%%%%%%%
%\begingroup
\clearpage
\section{Chapter}
\subsection{Section}
\begin{itemize}
    \item 
\end{itemize}
%\endgroup
%%%%%%%%%%%%%%%%%%%%%%%%%%%%% Chapter 8 %%%%%%%%%%%%%%%%%%%%%%%%%%%%%
\setcounter{section}{9}
%%%%%%%%%%%%%%%%%%%%%%%%%%%% Chapter 10 %%%%%%%%%%%%%%%%%%%%%%%%%%%%%
%\begingroup
\clearpage
\section{Genomics, Proteomics, and Genetic Engineering}
\subsection{Genome Sequencing}
\begin{itemize}
    \item In most methods, short, doubled stranded \textbf{adapter molecules}, which are complementary to oligonucleotide primers allowing for PCR amplification.
    \item \textbf{Sequencing by synthesis}: shear DNA, spread out over a flat surface, and subjected to PCR amplification. Result is many small clusters of PCR products, that are analyzed with reversible terinators, with 3' end blocked, forcing addition of one base as well as a floresent tags.
    \item \textbf{Ion torrent sequencing}: a method of DNA sequencing based on the detection of hydrogen ions that are released during the polymerization of DNA. This technology differs from other sequencing-by-synthesis technologies in that no modified nucleotides or optics are used.
    \item \textbf{Single-molecule sequencing}: A single DNA polymerase enzyme is affixed at the bottom of a ZMW with a single molecule of DNA as a template. The ZMW is a structure that creates an illuminated observation volume that is small enough to observe only a single nucleotide of DNA being incorporated by DNA polymerase.
    \item \textbf{Nanopore sequencing}: a single molecule of DNA or RNA can be sequenced without the need for PCR amplification or chemical labeling of the sample. Has higher rate of error, up to 5-10\%.
    \item \textbf{Comparative genomics}: a field of research where organisms are compared, mostly by alignment of genome sequences and checking extent of conservation.
    \item Comparative genomics can also help target regulatory motifs, which are hard to find due to their relatively short sequences and ability to change position. 
\end{itemize}

\subsection{Genomics and Proteomics}
\begin{itemize}
    \item \textbf{Functional genomics}: a dynamic focus on genome-wide paterns of gene expression and coordination instead of just DNA structures.
    \item \textbf{DNA mircoarray}: a collection of microscopic DNA spots attached to a solid surface that allow for the measurement of the expression levels of large numbers of genes simultaneously or to genotype multiple regions of a genome. 
\end{itemize}

\subsection{Recombinant DNA}
\begin{itemize}
    \item \textbf{Recombinant DNA}: isolated DNA is cut into fragments by one or more restircion enzyme, joined back together in a new combination, and then reintroduced into a cell or organism.
    \item Restriction enzymes have the same sticky ends regardless of organism as log as they were produced by same enzyme.
    \item Most useful vectors are easily introduced, contain a replication origin, and allow the growth of a cell on a solid selective medium.
    \item Some vectors can accept large DNA fragments, up to 100-200 kb, and are called \textit{artificial chromosomes}. 
    \item Most common are bacterial artificial chromosomes (BACs).
    \item The sticky ends are joined together using DNA ligase.
    \item Reverse transcriptase can synthesize a complementary strand of DNA called cDNA.
    \item The 3' end of cDNA can fold back on itself, creating a hairpin primer for second-strand synthesis.
    \item \textit{Polylinker}, or multiple cloning site (MCS), in a vectors makes directional cloning possible. 
\end{itemize}
%\endgroup
%%%%%%%%%%%%%%%%%%%%%%%%%%%% Chapter 10 %%%%%%%%%%%%%%%%%%%%%%%%%%%%%
\setcounter{section}{12}
%%%%%%%%%%%%%%%%%%%%%%%%%%%% Chapter 13 %%%%%%%%%%%%%%%%%%%%%%%%%%%%%
%\begingroup
\clearpage
\section{Chapter}
\subsection{Section}
\begin{itemize}
    \item 
\end{itemize}
%\endgroup
%%%%%%%%%%%%%%%%%%%%%%%%%%%% Chapter 13 %%%%%%%%%%%%%%%%%%%%%%%%%%%%%

%%%%%%%%%%%%%%%%%%%%%%%%%%%% Chapter 14 %%%%%%%%%%%%%%%%%%%%%%%%%%%%%
%\begingroup
\clearpage
\section{Chapter}
\subsection{Section}
\begin{itemize}
    \item 
\end{itemize}
%\endgroup
%%%%%%%%%%%%%%%%%%%%%%%%%%%% Chapter 14 %%%%%%%%%%%%%%%%%%%%%%%%%%%%%

%%%%%%%%%%%%%%%%%%%%%%%%%%%% Chapter 14 %%%%%%%%%%%%%%%%%%%%%%%%%%%%%
%\begingroup
\clearpage
\section{Chapter}
\subsection{Section}
\begin{itemize}
    \item 
\end{itemize}
%\endgroup
%%%%%%%%%%%%%%%%%%%%%%%%%%%% Chapter 14 %%%%%%%%%%%%%%%%%%%%%%%%%%%%%
\end{document}