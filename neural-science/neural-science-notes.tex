\documentclass[12pt,a4paper]{article}
\usepackage{inverba}
\newcommand{\userName}{Cullyn Newman} 
\newcommand{\class}{} 
\newcommand{\institution}{Portland State University} 
\newcommand{\thetitle}{\hypertarget{home}{Principles of Neural Science}}
\rfoot{\hyperlink{home}{\thepage}}

\begin{document}
%%%%%%%%%%%%%%%%%%%%%%%%%%%%%%%%%%%%%%%%%%%%%%%%%%%%%%%%%%%%%%%%%%%%%
\tableofcontents
\cleardoublepage
\fancyhead{}
\fancyhead[R]{\hyperlink{home}{\nouppercase\leftmark}}
%%%%%%%%%%%%%%%%%%%%%%%%%%%%%%%%%%%%%%%%%%%%%%%%%%%%%%%%%%%%%%%%%%%%%

\fancyhead[L]{Part I Overall Perspective}
%\begingroup
%%%%%%%%%%%%%%%%%%%%%%%%%%%%% Chapter 1 %%%%%%%%%%%%%%%%%%%%%%%%%%%%%
%\begingroup
\clearpage
\section{The Brain and Behavior}

\subsection{The Brain Has Distinct Functional Regions}

    \textbf{The Central Nervous System Has Seven Main Parts}
    \begin{itemize}
        \item \textbf{Spinal cord}: most caudal part of the central nervous system. It is subdivided into cervical, thoracic, lumbar, and sacral regions.
        \item \textbf{Brain stem}: consists of the medulla oblongata, pons, and midrain. Relays input from the spinal cord and back, and controls input to and from the head.
        \item \textbf{Medulla oblongata}: rostral to spinal cord and includes several centers responsible for vital autonomic functions. 
        \item \textbf{Pons}: rostral to medulla and conveys information about movement.
        \item \textbf{Cerebellum}: lies behind pons, modulates force and range of movement, and involved in learning motor skills.
        \item \textbf{Diencephalon}: lies rostral to midrain and contains two structures, thalamus (processes information reaching cerebral cortex) and hypthalamus (regulates autonomic, endocrine, and visceral functions).
        \item \textbf{Cerebrum}: comprises two cerebral hemispheres, each consisting of wrinkled outer layer (the cerebral cortex), and three deep lying structures (basal ganglia, the hippocampus, and the amygdaloid nuclei).
        \item \textbf{Cerebral cortex}: divided into four distinct lobes--- frontal, parietal, occipital, and temporal. The frontal lobe is largely concerned with short-term memory and planning, as well as movement; the parietal lobe with somatic sensation, forming a body image, and relating it to extrapersonal space; the occipital lobe with vision; and the temporal lobe with hearing---combined with deeper structures---with learing, memory, and emotion.
    \end{itemize}
%\endgroup
%%%%%%%%%%%%%%%%%%%%%%%%%%%%% Chapter 1 %%%%%%%%%%%%%%%%%%%%%%%%%%%%%

%%%%%%%%%%%%%%%%%%%%%%%%%%%%% Chapter 2 %%%%%%%%%%%%%%%%%%%%%%%%%%%%%
%\begingroup
\clearpage
\section{Nerve Cells, Neural Circuitry, and Behavior}
\subsection{The Nervous System Has Two Classes of Cells}
\begin{itemize}
    \item There are two main classes of cells in the nervous system: nerve cells, or neurons, and glial cells, or glia.
    \item A neuron has four defined regions:
        \begin{itemize}
            \item \textbf{Cell body}: or \textit{soma}, is the metabolic center of the cell, containing normal cell organelles.
            \item \textbf{Dendrites}: branch out in tree-like fashion and are main apparatue for receiving signals.
            \item \textbf{Axon}: extends some distance from a cell and carries signals to other neurons.
            \item \textbf{Presynaptic terminals}: specialized enlarged regions of it's axon's branches and is responsible for tranfer of signals.
        \end{itemize}
    \item \textbf{Principle of dynamic polarization}: electrical signals only forlow in one direction in neurons.
    \item \textbf{Connectional specificity}: nerve cells do not connect randomly with one another in these formation of networks.
    \item Neurons are classified into three groups:
        \begin{itemize}
            \item \textbf{Unipolar}: simpiliest due to single primary process, which gives rise to many branches. One branch as axon and others as receiving structures. These cells predominate invertebrates; they orccur in the autonomic nervous system in vertebrates. 
            \item \textbf{Bipolar}: oval soma that gives rise to two processes: a dendritic structure that receives signals and an axon that carries information towards the central nervous system. Many sensory cells are bipolar, and pain receptors are pseudo-unipolar. 
            \item \textbf{Multipolar}: predominate nervous system of vertebrates and vary greatly in shape; typically containing a single neuron and many dendritic points emerging from various points around the cell body.
        \end{itemize}
    \item Glial cells support nerve cells and greatly outnumber neurons. 
    \item Glial cells surround the cell bodies, axons, and neurons and can be divided into two major classes:
        \begin{itemize}
            \item \textbf{Microglia}: immune system cells that become phagocytes during injury, infection, or degenerative diseases. 
            \item There are three main types of \textbf{macroglia}: oligodendrocytes, Schwann cells, and astrocytes. About 80\% of all brain cells are macrogalia.
        \end{itemize}
\end{itemize}
%\endgroup
%%%%%%%%%%%%%%%%%%%%%%%%%%%%% Chapter 2 %%%%%%%%%%%%%%%%%%%%%%%%%%%%%

%%%%%%%%%%%%%%%%%%%%%%%%%%%% Chapter 3 %%%%%%%%%%%%%%%%%%%%%%%%%%%%%%
%\begingroup
\clearpage
\section{Genes and Behavior}

\begin{center}
    This chapter has been intentionally left blank, see genetic notes for more information.
\end{center}
%\endgroup
%%%%%%%%%%%%%%%%%%%%%%%%%%%%% Chapter 3 %%%%%%%%%%%%%%%%%%%%%%%%%%%%%
%\endgroup

\clearpage
\fancyhead[L]{Part II Molecular Biology of the Neuron}
%\begingroup
%%%%%%%%%%%%%%%%%%%%%%%%%%%%% Chapter 4 %%%%%%%%%%%%%%%%%%%%%%%%%%%%%
%\begingroup
\clearpage
\section{The Cells of the Nervous System}
\begin{center}
    This chapter was intentionally left blank, see cell biology notes for more information. 
\end{center}
%\endgroup
%%%%%%%%%%%%%%%%%%%%%%%%%%%%% Chapter 4 %%%%%%%%%%%%%%%%%%%%%%%%%%%%%

%%%%%%%%%%%%%%%%%%%%%%%%%%%%% Chapter 5 %%%%%%%%%%%%%%%%%%%%%%%%%%%%%
%\begingroup
\clearpage
\section{Ion Channels}
\subsection{Rapid Signaling in the Nervous System Depends on Ion Channels}
\begin{itemize}
    \item Up to 100 million ions can pass through a single channel each second, comaprable to the turnover rate of the fastest enzymes, catalase and carbonic anhydrase.
    \item Each channel allows only one or a few types on ions to pass.
    \item Many open and close, however, some remain open resulting in significant contribution to resting potential. 
    \item Ions pumps maintained gradients and are 100 to 100,000 times slower than channels.
    \item Questions for this chapter:
        \begin{itemize}
            \item Why do nerve cells have channels?
            \item How can channels conduct ions at such high rates adn still be selective?
            \item How are channels gated?
            \item How are properties of theses channels modified by various intrinsic and extrinsic conditions?
        \end{itemize}
\end{itemize}

\subsection{Ion Channels are Proteins That Span the Cell Membrane}
\begin{itemize}
    \item Cells have channels in order the transport ions across lipid bilayer easily and eliminate the need to be stripped of waters of hydration.
    \item The smaller the ion, the greater attraction to water, and the lower its mobility. This partially explains selection, but does how does the inverse selection, that selecting of lower mobility, occur?
    \item Some ions bind to proteins that can transport them, but this is far to slow for some cases.
    \item An extension of pore theory says that channels have narrow regions that act as molecular sieves, where the ion sheds most of it's water and only is let through by a binding to a specifically charged selectivity filter. 
\end{itemize}

\subsection{Ion Channels in ALl Cells Share Several Characteristics}
\begin{itemize}
    \item The opening and closing of a channel invole conformational changes.
    \item \textit{Gating}: the transition of a channel between theses stable functional states.
    \item Three major gating mechanisms:
        \begin{itemize}
            \item Ligand: binding of chemical ligands known as agonists at either cellular site; transmitters on the extracelluar; others that activate signaling cascades; and more. 
            \item Voltage-gated: changes in electrochemical changes as often as temperature sensors.
            \item Mechanical stretch or physical changes in the membrane.
        \end{itemize}
\end{itemize}
%\endgroup
%%%%%%%%%%%%%%%%%%%%%%%%%%%%% Chapter 5 %%%%%%%%%%%%%%%%%%%%%%%%%%%%%

%%%%%%%%%%%%%%%%%%%%%%%%%%%%% Chapter 6 %%%%%%%%%%%%%%%%%%%%%%%%%%%%%
%\begingroup
\clearpage
\section{Electrical Properties of the Neuron}
\begin{center}
    This chapter was intentionally left blank. No alternatvie notes, but may need to review chemistry and physics if this chapter is needed.
\end{center}
%\endgroup
%%%%%%%%%%%%%%%%%%%%%%%%%%%%% Chapter 6 %%%%%%%%%%%%%%%%%%%%%%%%%%%%%

%%%%%%%%%%%%%%%%%%%%%%%%%%%%% Chapter 7 %%%%%%%%%%%%%%%%%%%%%%%%%%%%%
%\begingroup
\clearpage
\section{Propagated Signaling}
\subsection{Personal Notes}
\begin{itemize}
    \item The variety of voltage-sensitive ion channels and the influence of cytoplasmic factors may be analogous to bias/weights or other hyperperamters in neural networks.
    \item Genetic changes thus change change these networks and may be how transfer learning takes place instead of starting from complete scratch for every organism. 
    \item Innate abilities could represent earlier layers in the network and heavily genetically promgrammed, while later layers are given more time to develop based on environment. 
    \item Can epigenetics have a relatively fast acting change on inherited intelligence? 
\end{itemize}

\begin{center}
    The rest of this chapter has been left blank, as it is more of an extension of molecular gentics, which I may need to return to later to answer questions like those listed above.
\end{center}
%\endgroup
%%%%%%%%%%%%%%%%%%%%%%%%%%%%% Chapter 7 %%%%%%%%%%%%%%%%%%%%%%%%%%%%%
%\endgroup

\clearpage
\fancyhead[L]{Part III Synaptic Transimission}
%\begingroup
%%%%%%%%%%%%%%%%%%%%%%%%%%%%% Chapter 8 %%%%%%%%%%%%%%%%%%%%%%%%%%%%%
%\begingroup
\clearpage
\section{Overview of Synaptic Transmission}
\subsection{Synapses Are Either Electrical or Chemical}
\begin{itemize}
    \item Average neuron forms and receive several thousand synaptic connections each, with the Purkinje cell of the cerebellum receiving up to 100,000.
    \item Both forms of transmission can be enhanced or diminished by cellular activity.
    \item Electrial synapses are used to send rapid stereotyped depolarizing signals.
    \item Chemical synapses are capable of more complex behaviors due to vairable signaling. 
    \item Most synapses are chemical.
\end{itemize}
\subsection{Electrical Synapses Provide Instantaneous Signal Transmission}
\begin{itemize}
    \item Presynaptic terminals must be big enough for its membrane to contain many ion channels to trigger initial depolarization.
    \item Postsynaptic terminals must be relatively to small in due to Ohm's law.
    \item Even weak subthreshold depolarizing currents can be carried to the postsynaptic neuron and depolarize it.
    \item Electrical synapses have a specialized region of contact called the gap junction, with seperation of only 4 nm, bridged by gap-junction channels specialized to conduct ionic current.
    \item Electrical transmission can be used to orchestrate actions or large groups of neurons.
    \item Groups of electrically coupled cells allows for explosive reactions.
    \item Gap junctions are formed between glial cells as well as neurons.
\end{itemize}
\subsection{Chemical Synapses Can Amplify Signals}
\begin{itemize}
    \item Chemical synapses are used to amplify or inhibit signals. 
    \item The synaptic cleft is 20-40 nm wide and depend on the diffusion of neurotransmitters to carry out signaling.
    \item Neurotransmitters are clustered at speclizied regions called \textit{active zones}, which allow for selective activate of nearby postsynaptic receptors, which lead to the opening or closing of ion channels.
    \item Chemical synapses can be as short as 0.3 ms but often last several ms.
    \item Weak activations of chemical synapses can activate larger electrial synapses. 
    \item The action of a transmitter depends on the properties of the postsynaptic receptor, not the chemical properties of itself.
    \item Neurotransmitters control the opening of ion channels in the postsynaptic cell either directly or indirectly.
    \item Indirect effects tend to last seconds to minutes and often modulate behavior due to alterations in the excitability of neurons and their synaptic connections.
\end{itemize}
%\endgroup
%%%%%%%%%%%%%%%%%%%%%%%%%%%%% Chapter 8 %%%%%%%%%%%%%%%%%%%%%%%%%%%%%

%%%%%%%%%%%%%%%%%%%%%%%%%%%%% Chapter 9 %%%%%%%%%%%%%%%%%%%%%%%%%%%%%
%\begingroup
\clearpage
\section{Directly Gated Transmission}   
\begin{center}
    This chapter has been intentionally left blank. Unlikely to return to this chatper.
\end{center}
%\endgroup
%%%%%%%%%%%%%%%%%%%%%%%%%%%%% Chapter 9 %%%%%%%%%%%%%%%%%%%%%%%%%%%%%

%%%%%%%%%%%%%%%%%%%%%%%%%%%%% Chapter 10 %%%%%%%%%%%%%%%%%%%%%%%%%%%%%
%\begingroup
\clearpage
\section{Synaptic Integration in CNS}
\subsection{Central Neurons Receive Excitatory and Inhibitory Inputs}
\begin{itemize}
    \item Generation of an action potential often requires the near-synchronous firing of a number of sensor neurons.
    \item Small inhibitory postsynaptic potential (ISPS), if strong enough, can counteract the sum of the excitatory actions and prevent membrane potential from reaching threshold potential.
    \item \textbf{Sculpting} function of synaptic inhibition that exerts control over action potentials in neurons that are spontaneously active due to intrinsic pacemaker channels, often completely shaping the firing patterns of cells.
    \item Most transmitters usually are inhibitory or excitatory despite being able to be either type. 
\end{itemize}

\subsection{Inhibitory Synaptic Action}
\begin{itemize}
    \item Inhibitory synapses play an essential role in the nervous system both by preventing too much excitation and by helping coordinate activity among networks of neurons. 
    \item Inhibitory inputs that hyperpolarize the cell perform subtraction on the excitatory inputs, where those that \textit{shunt} perform division. 
    \item Adding or removing nonshunting inhibitory inputs results in summation, while the combination or excitatory with the removal of inhibitory shunt produces a multiplication.
\end{itemize}

%\endgroup
%%%%%%%%%%%%%%%%%%%%%%%%%%%%% Chapter 10 %%%%%%%%%%%%%%%%%%%%%%%%%%%%%
%\endgroup


%%%%%%%%%%%%%%%%%%%%%%%%%%%%% Chapter 11 %%%%%%%%%%%%%%%%%%%%%%%%%%%%%
%\begingroup
\clearpage
\section{Second Messengers}
\begin{center}
    This chapter has been intentionally left blank. Molecular focused chapters may be reviewed later when a more narrow question needs to be ansewered. 
\end{center}
%\endgroup
%%%%%%%%%%%%%%%%%%%%%%%%%%%%% Chapter 11 %%%%%%%%%%%%%%%%%%%%%%%%%%%%%





























\end{document}