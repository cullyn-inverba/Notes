\documentclass[plain,basic]{inVerba-notes}

\newcommand{\userName}{Cullyn Newman}
\newcommand{\class}{BI:\@ 428}
\newcommand{\theTitle}{Journal Article Summary --- Week 6}
\newcommand{\institution}{Portland State University}

\begin{document}    

\begin{center}
    \textbf{\Large{The role of exome sequencing in newborn screening for inborn errors of metabolism}}
\end{center}

\section{Key Points}
\begin{itemize}
    \item \textbf{Newborn Screening (NBS)}: public health newborn screening that on a population scale aimed to identify rare, treatable conditions in infants.
    \item \textbf{Tandem Mass Spectrometry (MS/MS)}: a technique where two or more mass spectrometry tests are coupled together with an additional reaction step to increase their analytical ability, particularly rare inborn errors in this case.
    \item \textbf{Inborn Errors of Metabolism (IEMs)}: a class of genetic diseases that involve birth defects and disorders of metabolism; generally do errors in enzyme coding genes.
    \item \textbf{Whole Exome Sequencing (WES)}: a genomic technique for sequencing all the protein-coding regions of genes in the genome (i.e., the exome). 
    \item WES was used as an innovative methodology for NBS\@;
        \begin{itemize}
            \item This study used archived blood data from nearly all  infants with IEMs in the 4.35 million born from 2005 to 2013, including those with false positive MS/MS tests.
            \item WES had overall sensitivity of 88\% and specificity of 98.4\%.        
            \item MS/MS had 99.0\% and 99.8\% respectively.
            \item WES could be used as secondary test for false positive screening of MS/MS tests.
        \end{itemize}
    \item This study represents the largest sequencing effort of an entire population study of IEM-affected cases to date, allowing for unbiased assessment of WES as tool for population screening.
\end{itemize}


\nocite{adhikari2020role}
\bibliographystyle{apacite}
\bibliography{summaries.bib}
\end{document}

