\documentclass[12pt,a4paper]{article}
\usepackage{inverba}

\newcommand{\userName}{Cullyn Newman} 
\newcommand{\class}{BI 358} 
\newcommand{\institution}{Portland State Univsrsity} 
\newcommand{\theTitle}{\color{G-Leaf}Evolution}

\begin{document}
%%%%%%%%%%%%%%%%%%%%%%%%%%%%%%%%%%%%%%%%%%%%%%%%%%%%%%%%%%%%%%%%%%%%%
\tableofcontents
\cleardoublepage
\fancyhead{}
\fancyhead[R]{\hyperlink{home}{\nouppercase\leftmark}}
%%%%%%%%%%%%%%%%%%%%%%%%%%%%%%%%%%%%%%%%%%%%%%%%%%%%%%%%%%%%%%%%%%%%%

%%%%%%%%%%%%%%%%%%%%%%%%%%%%% Chapter 1 %%%%%%%%%%%%%%%%%%%%%%%%%%%%%
%\begingroup
\clearpage
\section{Evolutionary Thinking}\phantomsection
\subsection{Introduction (Lecture)}
\begin{itemize}
    \item Essential questions of evolutionary biology: 
        \begin{itemize}
            \item Why do organisms look so different?
            \item Why develop elaborate sexual traits?
            \item Why do organisms senesce?
        \end{itemize}
    \item Evolution is mainly an historical science and thus must relay on other methods of reconstructing the past or making inferences about evolutionary forces. 
    \item \textbf{Proximate}: a question about a mechanistic cause; provides an immediate explanation about {\color{o-Sun}how} a mechanistic cause functions.
    \item \textbf{Ultimate}: {\color{o-Sun}why}, or the reason, a trait or organism is the way it is; an evolutionary explanation.
    \item Example of proximate vs ultimate in Galapagos finches: 
        \begin{itemize}
            \item Proxmate: developmental growth factor is increased/decreased in some birds.
            \item Ultimate: different habits are selected on breaks that maxmize food gathering ability.
        \end{itemize}
    \item Evolutionary biology's approach to answering questions:
        \begin{itemize}
            \item \textbf{Empirical data}: {\color{o-Sun}observation} studies, experiments; the \textit{comparative method}.
            \item \textbf{Theory}: {\color{o-Sun}predictions} that use models and mathematical reasoning which can be be {\color{o-Sun}tested} with empirical data.
        \end{itemize}
    \item Overview of the components of evolution by natural selection:
        \begin{itemize}
            \item Genetic variation exists, via mutations.
            \item Mutations are heritable.
            \item The is an advantage to survival and/or reproduction from the mutation.
            \item Individuals with the advantage in survival/reproduction are selected for.
        \end{itemize}
\end{itemize}
%\endgroup
%%%%%%%%%%%%%%%%%%%%%%%%%%%%% Chapter 1 %%%%%%%%%%%%%%%%%%%%%%%%%%%%%

%%%%%%%%%%%%%%%%%%%%%%%%%%%%% Chapter 10 %%%%%%%%%%%%%%%%%%%%%%%%%%%%%
%\begingroup
\clearpage
\setcounter{section}{9}
\section{Studying Adaptation}\phantomsection
\subsection{Hypothesis Testing: Oxpeckers Reconsidered}
\begin{itemize}
    \item \textbf{Adaption}: a trait, or a suite of traits, that increases the fitness of its possessor.
    \item No hypothesis for the adaptive value of a trait should be accepted simply because of its plausibility.
    \item Oxpeckers and impalas traditionally were thought to have a mutually beneficial existence; oxpeckers ate ticks and impalas provided a safe environment.
    \item Experiments on cattle were done to test whether this observation was true:
        \begin{itemize}
            \item Results show red-billed oxpeckers have no effect tick loads of cattle.
            \item Red-billeld oxpeckers maintained open wounds, even enlarging existing wounds to feed on the cattle's blood.
            \item Red-billed oxpeckers removed hosts' earwax; whether this is good of bad is unclear.
            \item Even these results must remain in question, as cattle are not the native host for the birds.
        \end{itemize}
    \item Other important points to remember: 
        \begin{itemize}
            \item Differences among populations or species are not always adaptive.
            \item Not every trait is adaptive.
            \item Not every adaptation is perfect, often the adaption just happened to work well enough or by chance better than other adaptations.
        \end{itemize} 
\end{itemize}

\subsection{Experimental Design}
\begin{itemize}
    \item Defining and testing effective control groups is critical.
    \item Treatments of controls and experimental measures must be handled as close to exactly alike as possible.
    \item Randomization is a key technique for equalizing miscellaneous effects and a tool to avoid bias.
    \item Reproduction is essential in order to help remove potential outlier effects.
        \begin{itemize}
            \item Allows for greater understanding of precision, accuracy, and variation by providing more data for statistical tests.
        \end{itemize}
\end{itemize}
%\endgroup
%%%%%%%%%%%%%%%%%%%%%%%%%%%%% Chapter 10 %%%%%%%%%%%%%%%%%%%%%%%%%%%%%

%%%%%%%%%%%%%%%%%%%%%%%%%%%%% Chapter 4 %%%%%%%%%%%%%%%%%%%%%%%%%%%%%
%\begingroup
\clearpage
\setcounter{section}{3}
\section{Evolutionary Trees}\phantomsection
\subsection{How to Read an Evolutionary Tree}
\begin{itemize}
    \item \textbf{Phylogeny}: aka evolutionary tree or phylogenetic tree, is a diaggram showing the history of divergence and evolutionary change. Essentially, it's the {\color{o-Sun}genealogical relationships} of organisms based on descent with modification.
        \begin{itemize}
            \item \textbf{Taxa}: the units you are analyzing, e.g. certain species or DNA sequences.
                \begin{itemize}
                    \item \textbf{Character}: a feature or trait present among the taxa of interest, e.g. teeth of mammals or nucleotides of DNA sequences.
                        \begin{itemize}
                            \item \textit{Character {\color{o-Sun}state(s)}}: an {\color{o-Sun}alternative condition} of a character, which are able to evolve one to another, e.g. pointed/flat teeth of mammals.
                        \end{itemize}
                    \item \textit{{\color{true}Ancestral} character}: a trait that was {\color{true}possessed by the common ancestor}.
                    \item \textit{{\color{false}Derived} character}: a trait the was {\color{false}not possessed by the common ancestor} and instead {\color{o-Sun}evolved} in at least one of the descendants.
                        \begin{itemize}
                            \item \textbf{Synapomorphy}: derived character state shared by {\color{o-Sun} two or more} taxa and used to define a clade of taxa.
                            \item \textbf{Autapomorphy}: derived character state in only {\color{o-Sun}one} taxon.
                        \end{itemize}
                    \item \textbf{Outgroup}: a taxon or taxa that are used to root the phylogeny or determine ancestral character states.
                    \item \textbf{Ingroup}: the set of taxa that are the focus of the phylogeny.
                \end{itemize}
            \item \textbf{Nodes}: points at which the tree splits; represents mutations, speciation events, or {\color{o-Sun}character changes}.
        \item {\color{false}\textbf{Anagensis}}: descent with modification, but {\color{false}no speciation}.
        \item {\color{true}\textbf{Cladogenesis}}: {\color{true}speciation}, origin of clades.
        \begin{itemize}
            \item \textbf{Clade}: also known as a {\color{o-Sun}monophyletic group}, an ancestor and {\color{o-Sun}all} of its descendants.
            \item \textbf{Paraphyletic group}: a group of organisms consisting of an ancestor and {\color{o-Sun}some} of its descendants.
        \end{itemize}
        \item \textbf{Sister}: a taxa or clade that are most closely related to each other; they {\color{o-Sun}share the most recent} common ancestor. 
    \end{itemize}
    \item \textbf{Homology}: similarity due to common descent; {\color{true}continuity} of a trait, character, or character state through time.
        \begin{itemize}
            \item \textit{Homologous trait}: found in a taxa that inherited the trait from a common ancestor.
        \end{itemize}
    \item \textbf{Homoplasy}: or analogous, similarity in the characters or traits in different taxa due to convergent evolution, parallelism, or reversal, but {\color{false}not due to common descent}.
        \begin{itemize}
            \item \textbf{Convergent evolution}: similar traits due to selective forces and {\color{false}not shared ancestry}.
                \begin{itemize}
                    \item \textbf{Parallelism}: convergent evolution in {\color{o-Sun}recently diverged} taxa.
                \end{itemize}
            \item \textbf{Reversal}: derived traits or character states that revert to the ancestral form.
        \end{itemize}
\end{itemize}

\subsection{Inferring Phylogenetic Trees}
\begin{itemize}    
    \item \textbf{Parsimony}: relationships that require the {\color{o-Sun}smallest number of character changes} are most likely to be correct.
        \begin{itemize}
            \item Based on shared and {\color{false}derived} traits(synapomorphies).
            \item Reconstruction using parsimony:
                \begin{itemize}
                    \item[1.] Code characters.
                    \item[2.] Make up a taxon$\times$character matrix.
                    \item[3.] Search for synapomorphies, and the shortest tree.
                    \item Outgroups can help polarize (ancestral vs derived) the characters. 
                \end{itemize}
            \item \textbf{Treelength}: a measure of evolutionary change using parsimony. 
                \begin{itemize}
                    \item Shortest tree length produces most parsimonious tree.
                    \item Length determined by number of synapomorphies.
                    \item Homoplasious characters increase tree length.
                \end{itemize}
        \end{itemize}
    \item \textbf{Distance Methods}: converts a sequence alignment to genetic distances between pairs of sequences.
        \begin{itemize}
            \item Branch length is proportional to genetic differences.
        \end{itemize}
    \item \textbf{Maximum likelihood}: a method of estimating the parameters of a probability distribution by {\color{o-Sun}maximizing a \textit{likelihood function}}. 
        \begin{itemize}
            \item One of the more dominant means of statistical inference.
            \item \textbf{Likelihood}: measure of goodness of fit of a statistical model to a sample of data for given values of the unknown parameters.
            \item {\color{o-Sun}\(P(D|H)\)}; probability(P), Data(D), Hypothesis(H)
            \item \textbf{Bayesian}: uses the likelihood function to create a quantity called the \textit{posterior probability} of trees using a model of evolution based on prior probabilities in order to produce the most likely tree.
            \item \textbf{Bootstraping}: creating a value that indicates how many times out of 100 (normally) that the same branch was observed when repeating the phylogenetic reconstruction on re-sampled (pseudoreplicated) set of dat.
        \end{itemize}
\end{itemize}
%\endgroup
%%%%%%%%%%%%%%%%%%%%%%%%%%%%% Chapter 4 %%%%%%%%%%%%%%%%%%%%%%%%%%%%%

%%%%%%%%%%%%%%%%%%%%%%%%%%%%% Chapter 6 %%%%%%%%%%%%%%%%%%%%%%%%%%%%%
%\begingroup
\clearpage
\setcounter{section}{5}
\section{Mechanisms of Evolutionary Change}\phantomsection
\subsection{Hardy-Weinberg Equilibrium}
\begin{itemize}
    \item \textbf{Population}: a group of interbreeding individuals and their offsring.
    \item \textbf{Gene pool}: the set of all genes, or genetic information, in any population.
    \item \textbf{Genotypic frequency}: number of individuals with a given genotype divided by the total number of individuals in the population.
        \begin{itemize}
            \item The proportion (i.e., \(0<f<1\)) of genotypes in a population.
        \end{itemize}
    \item \textbf{Allele frequencies}: relative frequency of an allele at a particular locus in a population.
        \begin{itemize}
            \item \textbf{Locus}: a fixed position on a chromosome where a particular gene of genetic marker is.
            \item Monoploids: frequency of an allele is the result of the number of copies of the allele divided by sample size.
                \begin{itemize}
                    \item \(p=i/N\)
                    \item \(p\): frequency |  \(i\): copies of alleles |  \(N\): sample size
                \end{itemize}
            \item Diploids: frequency of alleles within three possbile genotypes at a locus with two alleles.
                \begin{itemize}
                    \item \({\color{o-Sun}p}=f({\color{o-Sun}AA})+\frac{1}{2}f(AB)\)~~frequency of {\color{o-Sun}A}-allele 
                    \item \({\color{o-Sun}q}=f({\color{o-Sun}BB})+\frac{1}{2}f(AB)\)~~frequency of {\color{o-Sun}B}-allele
                \end{itemize}
            \item Allele frequency can always be calculated from genotype frequency, whereas the reverse requires the \textit{Hardy-Weinberg principle} of random mating apply.
        \end{itemize}
    \item \textbf{Hardy-Weinberg principle}: allele and genotype {\color{o-Sun}frequencies} in a population will remain {\color{o-Sun}constant} in the {\color{o-Sun}absence of evolutionary influences}.
        \begin{itemize}
            \item Allele frequencies do not change from one generation to the next.
            \item Genotypic frequencies after one generation of random mating: \(p^2 + 2pq + q^2\)
            \item Evolutionary influences: genetic drift, mate choice, assortative mating, natural selection, sexual selection, mutation, gene flow, meiotic drive, genetic hitchhiking, population bottleneck, founder effect, and inbreeding.
                \begin{itemize}
                    \item \textit{Most of these influences will be discussed later}.
                \end{itemize}
        \end{itemize}
\end{itemize}

\subsection{Selection}
\begin{itemize}
    \item \textbf{Fitness}: success at which a organism produces fertile offspring.
    \item \textbf{Competition}: an interaction between organism in which the fitness of one is lowered by the presence of another.
    \item \textbf{Selection}: the act on a heritable phenotypic trait due to competition.
    \begin{itemize}
        \item Can be members of the same of different species.
        \item Not always directional and adaptive, instead selection pressure is applied and removes the less fit variants.
        \item Can be classified in different ways, such as effect on a trait, on genetic diversity, by life cycle, by unit of selection, or by the resource in competition.
        \item Most effective on large populations.
    \end{itemize}
    \subsubsection{By Effect on a Trait}
    \begin{itemize}
        \item \textbf{Stabilizing selection}: the simplies case in which selection acts to hold a trait at a stable optimum.
            \begin{itemize}
                \item Reduces the individuals in the trails of the trait's distribution, reducing variation.
            \end{itemize}
        \item \textbf{Directional selection}: favours extreme values of a trait.
            \begin{itemize}
                \item Directional selection on a continuous trait changes the average value of the trait in the population.
                \item Can reduce variation in the population, generally not by large amounts though.
            \end{itemize}
        \item \textbf{Disruptive selection (diversifying selection)}: acts during transition periods when current mode is sub-optimal, but alters trait in more than one direction.
            \begin{itemize}
                \item \textbf{Univariate}: when the trait is both quantitatively favoured in either direction and can lead to speciation.
                \item Generally {\color{o-Sun}increases the variance} on continuous traits. 
                \item May be more common than generally recognized.
            \end{itemize}
        \item All three increases the mean fitness of the population.
    \end{itemize}
    \subsubsection{By Effect on Genetic Diversity}
    \begin{itemize}
        \item \textbf{Purifying selection}: aka negative selection; acts to remove genetic variation from the population.
        \item \textbf{\textit{de novo} mutation}: introduces new variation and opposes negative selection.
        \item \textbf{Balancing selection}: acts to maintain genetic variation, even in absence of \textit{de novo} mutation by frequency-dependent selection.
            \begin{itemize}
                \item \textbf{Frequency-dependent selection}: fitness that depends of the phenotypic or genotypic {\color{o-Sun}composition} of a population.
                    \begin{itemize}
                        \item {\color{pos}Positive}: fitness {\color{pos}increases} as frequency of the trait {\color{pos}increases}.
                        \item {\color{neg}Negative}: fitness {\color{neg}decreases} as the frequency of the trait {\color{pos}increases}.
                    \end{itemize}
                \item \textbf{Overdominance}, \textit{aka heterozygote advantage}: when a combination of alleles confers a selective advantage over individuals with one allele.
                \item \textbf{Underdominance}, \textit{aka heterozygote disadvantage}: when the heterozygote has lower fitness than either homozygote. 
            \end{itemize}
    \end{itemize}
    \subsubsection{By Life Cycle Stage}
    \begin{itemize}
        \item \textbf{Viability selection}: \textit{aka survival selection}: increases probability of survival. 
            \begin{itemize}
                \item Can act to improve probability of survival before and after reproduction.
            \end{itemize}
        \item \textbf{Fecundity selection}: increases the rate of reproduction given survival.
            \begin{itemize}
                \item May be split into sub-components including sexual selection, gametic selection, gamete viability, compatability selection, and zygote formation.
            \end{itemize}
    \end{itemize}
\end{itemize}

\subsection{Mutation}
\begin{itemize}
    \item \textbf{Mutation}: alteration in the nucleotide sequence of the genome of an organism.
        \begin{itemize}
            \item May not produce discernible phenotypic changes.
            \item The ultimate source of genetic variation.
            \item Have several types of changes, from no effect, to small changes, or complete loss of function.
        \end{itemize}
    \subsubsection{Large-Scale Structural Mutations}
    \begin{itemize}
        \item \textbf{Gene duplications}, \textit{aka amplifications}: repetition of a chromosomal segment or attachment of extra piece of chromosome to another, leading to multiple copies of chromosomal regions.
        \item Deletions of large chromosmal regions.
        \item \textbf{Fusion genes}: mutations that join previously separated genes into one new distinct gene.
        \item \textbf{Chromosmal rearrangement}: large scale changes in structure of chromosomes, leading to speciation in isolated, inbred populations. Includes:
            \begin{itemize}
                \item \textbf{Chromosomal translocations}: interchange of genetic parts from nonhomologous chromosomes.
                \item \textbf{Chromosomal inversions}: reversing the orientation of a chromosomal segment.
                \item Non-homologous chromosomal crossover.
                \item \textbf{Interstitial deletions}: inverse of fusion genes; removes a segment of DNA joining distant genes.
            \end{itemize}
        \item \textbf{Loss of heterozygosity}: loss of one allele, by deletion or genetic recombination, in a organism that previously had two different alleles. 
    \end{itemize}
    \subsubsection{Small-Scale Mutations}
    \begin{itemize}
        \item \textbf{Point mutation}: a single nucleotide base change, that can result in a variety of effects.
        \item \textbf{Insertions}: add one or more extra nucleotides into the DNA.
            \begin{itemize}
                \item Usually caused by transposable elements, or errors during replication or repeating elements.
                \item Can cause\textit{reading frame shift}, possibly effecting how many codons are read, and thus altering the gene product.
            \end{itemize}
        \item \textbf{Deletions}: remove one or more nucleotides from the DNA.
            \begin{itemize}
                \item Also can cause a reading frame shift like insertions.
                \item Generally irreversible.
            \end{itemize}
        \item \textbf{Substitutions}: exchange of a single nucleotide for another.
            \begin{itemize}
                \item Often classified as transitions or transversions.
                \item Generally a purine (A-G) for a purine, or a pyrimidine (C-T) for a pyrimidine.
                \item Can be reversed by another point mutation.
            \end{itemize}
    \end{itemize}
    \subsubsection{Impact on Protein Sequence}
    \begin{itemize}
        \item Effect of mutation depends heavily on where it occurs, particularly in a coding or non-coding region.
        \item Regulator sequences, e.g. promoters, enhancers, silencers, can alter gene expression but are less likely to alter protein sequence.
        \item \textbf{Frameshift mutation}: caused by insertion or deletion of nucleotides that is not divisible by three, resulting in a different translation from the original.
        \item \textbf{Synonymous substitution}: a condon replacement with another that codes for same amino acid. 
            \begin{itemize}
                \item \textbf{Silent substitution}: no phenotypic difference after a synonymous substitution.
            \end{itemize}
        \item \textbf{Nonsynonymous substitution}: a codon replacement that codes for a different amino acid.
            \begin{itemize}
                \item \textbf{Missense mutation}: codon replacement that renders the resulting protein nonfunctional.
                \item \textbf{Nonsense mutation}: codon replacement that results in a premature stop codon that produces a truncated and often nunfunctional protein.
            \end{itemize}
    \end{itemize}
\end{itemize}
\subsection{Migration}
\begin{itemize}
    \item \textbf{Gene flow}: movement of alleles, or genetic variation, between populations.
        \begin{itemize}
            \item If the rate of gene flow is high enough, then two populations are considered to have equivalent allele frequencies and thus a single population.
            \item Constrains speciation by combining gene pools of the groups.
            \item May result in the addition of novel genetic variants in the gene pool.
        \end{itemize}
        \item Gene flow is expected to be lower in species that:
            \begin{itemize}
                \item have low mobility or dispersal.
                \item occur in fragmanted habits.
                \item have long distances 2between populations.
                \item have small population sizes.
            \end{itemize}
        \item \textbf{Allopatric speciation}: when gene flow is blocked by {\color{o-Sun}physical} barriers that inhibit gene flow.
        \item \textbf{Sympatric speciation}: result of gene flow that is blocked due to {\color{o-Sun}non-physical} barriers that inhibit gene flow.
\end{itemize}

\subsection{Genetic Drift}
\begin{itemize}
    \item \textbf{Genetic drift}: the change in the {\color{o-Sun} allele frequencies} in a population due to {\color{o-Sun}random sampling}.
        \begin{itemize}
            \item Not influenced by environmental factors.
        \end{itemize}
    \item May cause certain gene variants to become fixed or lost by chance.
    \item Generally drives populations towards genetic uniformity over time, {\color{o-Sun}decreasing heterozygosity}.
    \item Only mutation or gene flow can introduce new alleles, which acts against genetic drift.
    \item \textbf{Founder effect}: result of sampling error which has an increased likelyhood on populations with low numbers.
        \begin{itemize}
            \item By chance certain alleles can be dominant when they otherwise wouldn't be in a new founding population.
            \item Often acts to drastically increase rate of genetic drift.
        \end{itemize}
    \item \textbf{Genetic bottleneck}: a sharp reduction in the size of population due to environmental events. 
        \begin{itemize}
            \item Can essentially cause a founder effect, though it's not a new population.
        \end{itemize}
    \subsubsection{Coalescent Theory}
    \begin{itemize}
        \item \textbf{Coalescent theory}: how gene variants sampled from a population may have {\color{o-Sun}originated} from a common ancestor.
            \begin{itemize}
                \item Assumes no recombination, no natural selection, no gene flow in the simpiliest case.
            \end{itemize}
        \item Aims to look backward in time by merging allels into a single ancestral copy according to a random process in coalescence events.
        \item Many theoretical genealogies are made in order to compare to observed data in order to test assumptions about demographic history of a population.
            \begin{itemize}
                \item Used to make inference about population genetic parameters, such as migration, population size, and recombination.
            \end{itemize}
        \item \textbf{Coalescent time}: number of preceding generations where the coalescence took place, not calender time.
            \begin{itemize}
                \item Estimation of the time can be made multiplied by \(2N_e\) with the average time between generations.
                \item Time to coalescence for a pair of allels at a locus is {\color{o-Sun}dependent} on population size.
                \item Formula: \(P_c(t)=\left(1-\dfrac{1}{2N_e}\right)^{t-1}\left(\dfrac{1}{2N_e}\right)\)
            \end{itemize}
        \item Can also be used to model the amount of variation in DNA sequences expected from genetic drift and mutation.
    \end{itemize}
\end{itemize}

\subsection{Molecular Evolution}
\begin{itemize}
    \item \textbf{Molecular evolution}: the process of change in the sequence composision of cellular molecules across generations.
    \item \textbf{Polymorphism}: occurrence of two of more clearly different morphs, or alternative phenotypes, in the population of a species.
    \begin{itemize}
        \item \textbf{Substitution}: when allels become fixed or lost in a population and polymorphism is ended.
            \begin{itemize}
                \item Substitution rate (\(k\)): \(k = sN\mu\)
                \item \(s\) = probability of fixation.
                \item \(N\mu\)= mutation rate of population.
            \end{itemize}
    \end{itemize}
    \subsubsection{Recombination}
    \begin{itemize}
    \item \textbf{Recombination}: the process that results in genetic exchange between chromosomes or chromosomal regions.
        \begin{itemize}
            \item Can also cause mutations due to misalignment after recombination.
            \item \textbf{Gene repair}: a type of recombination that is the product of DNA repair that corrects damage using a homologous template. 
                \begin{itemize}
                    \item Often responsbile for homogenizing sequences of duplicate genes over long periods of time, which reduces nucleotide divergence.
                \end{itemize}
        \end{itemize}
    \item \textbf{Genetic hitchhiking}: change in allele frequency not because of natural selection, but due to proximity to a gene undering selective sweep.
        \begin{itemize}
            \item \textbf{Selective sweep}: a beneficial mutation that increases frequency and generally becomes fixed. 
        \end{itemize}
    \end{itemize}
    \subsubsection{Neutral Theory}
        \begin{itemize}
        \item \textbf{Neutral theory of molecular evolution}: most evolutionary changes occur at the molecular level.
        \item Most variation is due to random genetic drift of mutant alleles that are selectively neutral.
        \item Compatible with phenotypic evolution, as phenotypes are driven by molecular changes.
        \item Most mutations are neutral with respect to fitness.
        \item A minority of mutation are advantageous.
        \item Substitution rate predicted to be neutral, equal to per-individual mutation rate, {\color{o-Sun}independent} of population size. 
        \item \(K_A / K_s\) test used to determin direction selection based on evolutionary history.
            \begin{itemize}
                \item \(K_A\): number of nonsynonymous substitutions (replacement).
                \item \(K_S\): number of synonymous substitutions (silent).
                \item {\color{pos}\(K_A > K_s\)} signals for {\color{pos}diversifying selection}.
                \item {\color{neg}\(K_A < K_s\)} signals for {\color{neg}purifying selection}.
            \end{itemize}
        \end{itemize}
    \subsubsection{Molecular Clocks}
    \begin{itemize}
        \item \textbf{Molecular clocks}: the average rate at which species' genomes accumulates {\color{o-Sun}neutral mutations} over time.
        \begin{itemize}
            \item A linear rate is often easy to establish.
            \item Used to measure evolutionary divergence.
        \end{itemize}
    \end{itemize}
\end{itemize}

\subsection{Nonrandom Mating}
\begin{itemize}
    \item \textbf{Inbreeding}: production of offspring from closely genetically related individuals.
        \begin{itemize}
            \item Results in homozygosity, which can increase chances chances of offspring being affected by deleterious or recessive traits.
            \item \textbf{Inbreeding depression}: the reduced fitness in a given population due to inbreeding.
                \begin{itemize}
                    \item Usually caused by population bottlenecks or the founder effect.
                \end{itemize}
            \item Can also result in purging of deleterious allels through purifying selection.
            \item Can allow for the expression of advantageous phenotypes, which if outweighs the disadvantages, then could potentially lead to speciation.
    \item \textbf{Coefficient of inbreeding}: the probability that two alleles at any locus in an individual are identical by descent.
    \item Nonrandom mating does not alter allele frequencies and not a mechanism of evolution.
        \begin{itemize}
            \item Can alter the frequencies of genotypes, changing the distribution of phenotypes in a population, which can alter patters of natural selection.
        \end{itemize}
    \end{itemize}
    \item \textbf{Assortative mating}: mating based on phenotypic factors.
        \begin{itemize}
            \item Can play a role in sympatric speciation.
            \item A form of sexual selection.
            \item Can be either positive of negative, selecting for similar or different phenotypes respectively.
        \end{itemize}
\end{itemize}
%\endgroup
%%%%%%%%%%%%%%%%%%%%%%%%%%%%% Chapter 7 %%%%%%%%%%%%%%%%%%%%%%%%%%%%%
%\endgroup

%%%%%%%%%%%%%%%%%%%%%%%%%%%%% Chapter 8 %%%%%%%%%%%%%%%%%%%%%%%%%%%%%
%\begingroup
\clearpage
\setcounter{section}{7}
\section{Evolution at Multiple Loci}\phantomsection
\subsection{Linkage Equilibrium and Diesquilibrium}
\begin{itemize}
    \item \textbf{Linkage equilibrium}: when the genotype of a chromosome at one locus is {\color{o-Sun}independent} of its genotype at another locus.
    \item \textbf{Linkage disequilibrium}: the non-random association of alleles at different loci in a given population. 
        \begin{itemize}
            \item Occurs when frequency of the association between loci's different alleles is higher or lower than expected.
        \end{itemize}
    \item \textbf{Haplotype}: a group of alleles in an organism that are inherited together from a single parent.
        \begin{itemize}
            \item Used to mean the collection of specific alleles that represent a phenotype and likely to be conserved.
            \item Also can be used to mean a set of linked single-nucleotide polymorphism alleles that are associated statistically.
        \end{itemize}
    \item Factors that influences disequilibrium: selection, rate or genetic recombination, mutation rate, genetic drift, system of mating, population structure.
        \begin{itemize}
            \item Undertanding linkage desequilibrium in a genome can be a powerful signal of the population genetic processes that structure it.
            \item Selection, genetic drift, assortative mating, and population admixture act to {\color{pos}create disequilibrium}. 
            \item Recombination and outbreeding act to {\color{neg}reduce disequilibrium}.
        \end{itemize}
    \item Level of linkage disequilibrium can between \(A\) and \(B\) can be quantified by the {\color{o-Sun}coefficient of linkage disequilibrium, \(D_{AB}\)}.
        \begin{itemize}
            \item Formula: \(D_{AB}=P_{AB} - P_{A}P_{B}\)
            \item \(P_{AB}\): the frequency with which both occur together on same gamete, or the frequency of the \(AB\) haplotype.
            \item \(P_AP_B\): product of the probabilities give the probability they occur together.
            \item When there is a difference, the magnitude of the coefficient rises, indicating linkage disequilibrium.
            \item Strong recent selection can be indicated by linkage disequilibrium of allels located next to neutral allels.
        \end{itemize}
    \subsubsection{Recombination's Effect on  Linkage Diesquilibrium}
    \begin{itemize}
        \item Linkage diesquilibrium (\(D\)) will converge to zero depending of the magnitude of the recombination rate (\(c\)) between two loci the absence of natural selection, inbreeding, and genetic drift.
        \item The smaller the distance between the two loci, the smaller the rate of convergence of D to zero.
        \item Genetic recombination tends to randomize genotypes, thus it tends to reduce frequency of overrespresented chromosome haplotypes and increases underrepresented haplotypes. 
        \item In short, recombination reduces linkage disequilibrium.
    \end{itemize}
\end{itemize}

\subsection{Adaptive Significance of Sex}
\begin{itemize}
    \item \textbf{Parthenogenesis}: a natural form of {\color{o-Sun}asexual reproduction} in which growth and development of embryos occur without fertilization.
    \item \textbf{Hermaphrodite}: an organism that has complete or partial reproductive organs and can produce gametes normally associated with both sexes.
        \begin{itemize}
            \item Allows the either partner to act as male of female in sexual reproduction.
            \item \textbf{Sequential}: bon as one sex, but later changes into the opposite.
            \item \textbf{Simultaneous}: fully functioning male and female genitalia.
        \end{itemize}
    \subsubsection{Maynard Smith Assumptions}
    \begin{itemize}
        \item[1.] A female's reproductive mode does not affect how many offsping she makes.
        \item[2.] A female's reproductive mode does not affect the probability that her offspring will survive.
        \item Designed to question advantage of either mode of reproduction (asexual vs sexual).
        \item Predicts asexual females should be dominant reproduction form.
            \begin{itemize}
                \item Due to the {\color{o-Sun}two-fold cost of sex}: requires two individuals and some of the offspring are male.
                \item Asexual reproduction of females would have all of the individuals reproduce and pass on genes, leading to quick dominance.
            \end{itemize}
        \item The prediction is not seen, so it raises question of what the benifits of sexual reproductive are.
    \end{itemize}
    \subsubsection{Violations of Maynard Smith Assumptions}
    \begin{itemize}
        \item[1.] Male parental care can influence how many offspring survives or can be made, though this violatin is rare.
        \item[2.] Presence of self-compatible hermaphrodites and males in same organism means males must:
            \begin{itemize}
                \item collectively fertilize a proportion of eggs that exceed their own frequency in the population ($\alpha$);
                \item and/or produce offsring that have a higher relative fitness (\(w\)).
                \item Condition for male persistance: \(\alpha w>2\)
            \end{itemize}
        \item Raises the question how male fertilization increases relative fitness.
    \end{itemize}
    \subsubsection{Consequences of Sex}
    \begin{itemize}
        \item \textbf{Sexual reproduction}: meiosis with crossing over and mating between unrealted individuals.
            \begin{itemize}
                \item Meiosis results in genetic recombination.
                \item Mating between unrealted individuals (outcrossing) results in allelic segregation.
            \end{itemize}
        \item {\color{o-Sun}Segregation} tends to restore the population to the {\color{o-Sun}Hardy-Weinberg equilibrium}.
        \item {\color{o-Sun}Recombination} tends to restore the population to {\color{o-Sun}linkage equilibrium}.
    \end{itemize}
    \subsubsection{Maintenance of Sex}
    \begin{itemize}
        \item More attention is given to the maintenance of sex via its effects on linkage disequilibrium. 
        \item Sex {\color{o-Sun}purges deleterious mutations}.
            \begin{itemize}
                \item \textbf{Genetic load}: the burden imposed by the accumulating mutations in a population.
                \item \textbf{Muller's ratchet}: mutations should accumulate in asexual lineages.
                    \begin{itemize}
                        \item Lack of ability to purge mutations means that drift has higher chance of removing smaller proportion of mutation free individuals.
                        \item Thus, number deleterious mutations increases in individuals remaining, overtime lowering fitness of lineage. 
                    \end{itemize}
                \item A higher mutation rate selects for more frequent outcrossing; thus more males (more sexual reproduction).
                \item More frequent outcrossing results in more recombination, which opposes genetic drifts tendency of creating linkage disequilibrium.
                    \begin{itemize}
                        \item Recombination recreates lost genotypes caused by random chance due to drift.
                    \end{itemize}
                \item Even if two individuals mate carrying deleterious mutations, then there is still a chance that they can produce mutation free offsping.
                \item Genes responsbile for sex are maintained because they help create zero-deleterious-mutation genotypes. 
            \end{itemize}
        \item Sex {\color{o-Sun}accelerates adaptive evolution}.
            \begin{itemize}
                \item Natural selection also helps creates linkage disequilibrium, creating chances at losing advantageous genotype.
                    \begin{itemize}
                        \item Sex again introduces recombination allowing for recovery of lost advantageous genotypes.
                        \item Recombination can be can also disadvantageous once optimal genotypes arise, raising question how it remains useful.
                    \end{itemize}
                \item \textbf{Red Queen hypothesis}: sex remains beneficial indefintely in populations subject to ever-changing selection.
                    \begin{itemize}
                        \item Results in coevolutionary arms races.
                        \item Genotypes in lower-than-equilibrium frequencies due to previous disadvantage may soon become advantageous.
                    \end{itemize}
                \item Thus, sex is continually selected for as recombination is continually favored in changing conditions.
                    \begin{itemize}
                        \item Creates an accelrated rate of adaptive evolution in genes that take advantage of changing fitness of genotypes relative eachother.
                    \end{itemize}
                \item Genes for sex are maintained in populations due to higher frequencies of genetic hitchhiking in genotypes they create.
            \end{itemize}
    \end{itemize}
\end{itemize}
%\endgroup
%%%%%%%%%%%%%%%%%%%%%%%%%%%%% Chapter 8 %%%%%%%%%%%%%%%%%%%%%%%%%%%%%

%%%%%%%%%%%%%%%%%%%%%%%%%%%%% Chapter 9 %%%%%%%%%%%%%%%%%%%%%%%%%%%%%
%\begingroup
\clearpage
\section{Quantitative Genetics}\phantomsection
\subsection{The Nature of Quantitative Traits}
\begin{itemize}
    \item \textbf{Quantitative genetics}: deals with phenotypes that vary continuosly due to multilocus traits and environmental factors.
        \begin{itemize}
            \item Allows for the prediction of how a population will respond to selection, even when we do not know the genetic basis of the trait.
        \end{itemize}
    \item \textbf{Qualitative traits}: descrete traits that can be assigned individual categories by observation of simple genetic test.
    \item \textbf{Quantitative traits}: continuous traits determined by the combined influence of the genotype at multiple loci and the environment; the focus of quantitative genetics.
        \begin{itemize}
            \item Study of continuous distribution requires many other statistical methods such as the effect size, mean, and variance, to link phenotypes to genotypes.
        \end{itemize}
\end{itemize}

\subsection{Heritable Variation}
\begin{itemize}
    \item \textbf{Heritability}: the degree of {\color{o-Sun}variation} in a phenotypic trait in a population that is due to genetic variation betwen individuals in that population.
    \item \textbf{Phenotypic variance (V\(_{P}\))}: the genetic variance (V\(_{G}\)) combined with the environmental variance (V\(_{E}\)).
        \begin{itemize}
            \item {\color{o-Sun}\(V_P = V_G + V_E + V_{GE}\)}
            \item \(V_{GE}\) represents variance associated with {\color{o-Sun}intereaction} of genetic and environmental factors. 
        \end{itemize}
    \subsubsection{Genetic Variance}
    \begin{itemize}
        \item \textbf{Additive genetic variance (V\(_{A}\))}: how much the phenotypic trait is influenced by traits that show an additive effect on the quantitative traits.
            \begin{itemize}
                \item Measures the magnitude to which individual phenotypic differences can be predicted due to additive effects of allelic substitutions.
                \item The greater the additive genetic variation for the trait, the greater is response to selection can be.
            \end{itemize}
        \item \textbf{Dominance genetic variance (V\(_{D}\))}: associated with the dominant gene actions which cover the influence of the recessive alleles at the particular locus.
        \item \textbf{Epistasis (V\(_{I}\))}: occurs due to statistical interaaction among loci, i.e., gene-by-gene modification.
        \item  {\color{o-Sun}\(V_G = V_A + V_D + V_I\)}
    \end{itemize}
    \item \textbf{Broad-sense heritability (H\(^{2}\))}: {\color{o-Sun}all genetic contributions} to a populations phenotypic variance, including additive, dominant, and epistatic, and maternal/paternal effects. 
        \begin{itemize}
            \item \(H^2 = \dfrac{V_G}{V_P}\)
        \end{itemize}
    \item \textbf{Narrow-sense heritability (h\(^{2}\))}: proportion of the total phenotypic variance that is due to the {\color{o-Sun}additive effects} of genes.
        \begin{itemize}
            \item \(h^2 = \dfrac{V_A}{V_P}\)
            \item Allows prediction of how a population will respond to selection.
            \item The greater the additive genetic variation for the trait, the greater its response to selection can be.
        \end{itemize}
    \subsubsection{Estimating Heritability}
    \begin{itemize}
        \item Heritability is often estimated by measuring correlations between parents and offspring.
        \item \textbf{Midparnet}: the average of the trait value between parents.
            \begin{itemize}
                \item The mother's value often has a scaling value applied to help account for developmental sex differences.
            \end{itemize}
        \item \textbf{Midoffspring}: the average value of a trait among offsping.
        \item Correlations are determind through a line of best fit.
            \begin{itemize}
                \item Heritability of the trait is measued between 0 and 1 determind by the slope of the best fit line plotted against midparant and midoffsping values.
                \item Using a method of {\color{o-Sun}least-squares linear regression}, which minimizes the sum of the squared vertical distances between points, the heritability slope represents {\color{o-Sun}narrow-sense heritability}.
            \end{itemize}
        \item If a trait is heritable, monozygotic twins will resemble each other more than dizygotic twins.
    \end{itemize}
    \subsubsection{Predicting Evolutionary Responses}
    \begin{itemize}
        \item \textbf{Selection differential (S)}: the difference between the mean of the selected individuals ($\bar{x}$) and the mean of the entire population ($\mu$)
            \begin{itemize}
                \item \(S=\bar{x}-\mu\)
                \item Can be used to create a statical selection gradient, which is used to represent the strength of selection.
            \end{itemize}
        \item \textbf{Breeder's equation}: evolutionary response (R) can be predicted using narrow-sense heritability and the selection differential.
            \begin{itemize}
                \item {\color{o-Sun}\(R=h^2S\)}
                \item The greater the heritability and selection differential, the greater the response to selection.
            \end{itemize}
        \item When characters are genetically correlated, selection for one can drag the other along through genetic hitchhiking.
        \item Selection on the alleles at any single locus affecting a quantitative is often very weak; substantial genetic variation may persist at equilibrium between mutation and selection.
        \item {\color{o-Sun}Low heritability} often represents populations effected by {\color{o-Sun}strong selection}, as selection removes variation.
        \item {\color{o-Sun}Higher heritability} represents {\color{o-Sun}high variance} within the population. 
    \end{itemize}
\end{itemize}
%\endgroup
%%%%%%%%%%%%%%%%%%%%%%%%%%%%% Chapter 9 %%%%%%%%%%%%%%%%%%%%%%%%%%%%%

%%%%%%%%%%%%%%%%%%%%%%%%%%%%% Chapter 11 %%%%%%%%%%%%%%%%%%%%%%%%%%%%%
%\begingroup
\setcounter{section}{10}
\clearpage
\section{Sexual Selection}\phantomsection
\subsection{Sexual Dimorphism and Sex}
\begin{itemize}
    \item \textbf{Sexual Dimorphism}: when two sexes of the same species exhibit different characteristics beyond differences in sexual organs.
        \begin{itemize}
            \item \textbf{Primary sexual characteristics}: differences that contribute directly to reproduction and serve no purpose in courtship.
            \item \textbf{Secondary sex characteristics}: traits that give advantage over rivals in courtship or competition and are not directly realted to reproduction.
            \item Males and females often have different equilibrium strategies, due in large part to relative parental investment.
        \end{itemize}
    \subsubsection{Asymmetries Between the Sexes}
        \begin{itemize}
            \item \textbf{Parental investment}: the energy and time expended constructing and caring for the offspring.
                \begin{itemize}
                    \item Typically higher in females.
                    \item Ultimately the investment is measured in fitness.
                \end{itemize}
            \item Females produce eggs which often require substantial investment; often limiting mating availability.
            \item Males produce many small gametes; allowing for frequent and multiple mating.
            \item Variation in mating success is predictied to be highest in males, as females are often the limiting factor and have no trouble reproducing.
        \end{itemize}
\end{itemize}

\subsection{Sexual Selection}
\begin{itemize}
    \item \textbf{Sexual selection}: differences in reproductive success due to variation among individuals in success at getting mates.
    \begin{itemize}
        \item \textbf{Intersexual selection}: when members of one sex choose mates of the other sex.
        \item \textbf{Intrasexual selection}: when members of same sex compete for access to members of the opposite sex.
    \end{itemize}
    \item If sexual selection is strong for one sex, and weak for the other, then we predict that:
    \begin{itemize}
        \item members of the sex subject to strong sexual selection will be competitive (intersexual).
        \item members of the sex subject to weak sexual selection will be choosy (interasexual).
    \end{itemize}
    \subsubsection{Selection on Males: Competition}
    \begin{itemize}
        \item Male-male competition results in mainly combat, alternate mating strategies, sperm competition, and infanticide.
        \item \textbf{Combat}: fighting over access to mates that favours morphological traits such as body size, weaponry, armor, and tactical cunning.
            \begin{itemize}
                \item Typically leads to a select few dominant males mating with majority of females.
            \end{itemize}
        \item \textbf{Alternative strategies}: other strategies that are often employed by the less combat prone males.
            \begin{itemize}
                \item Variety of examples, such as pretendding to be female, sneaking, false displays of strength, and more.
                \item Generally less successful, as it still requires the fooling of females or avoidance of dominant males and only works well in small relative frequency.
            \end{itemize}
        \item \textbf{Sperm competition}: competition between gametes can continue after physical mating is complete.
            \begin{itemize}
                \item Large quantities of ejaculation, guarding of mates, prolonged copulations, deposit of a copulatory plug, applications of disadvantage pheromones, and more.
            \end{itemize}
        \item \textbf{Infanticide}: improvement of males reproductive success at cost of females due to killing of previous males offspring.
    \end{itemize}
    \subsubsection{Selection on Males: Female Choice}
    \begin{itemize}
        \item Intrasexual selection that occurs when males cannot monopolize access to females, allowing for females to choose mates.
        \item Models for female preference include: good genes, direct benefits, preexisting sensory bias, and runaway sexual selection.
        \item \textbf{Good genes}: traits that code of ornamented displays can indicate higher fitness.
            \begin{itemize}
                \item Often causes a handicap, which may be due to preexisting sensory bias and runaway sexual selection, or just a further dispaly of fitness.
                \item Sometimes displays end up being directly beneficial to reproduction or competition by chance.
                \item Other times it's just clear evidence of direct fitness advantages for partners.
            \end{itemize}
        \item \textbf{Preexisting bias}: preexisting preference, which arose before male trait, selecting for further evolution of the trait.
            \begin{itemize}
                \item Different biases can help lead to speciation.
            \end{itemize}
        \item \textbf{Runaway sexual selection}: when a locus for a trait in males becomes linked to a trait that codes for mating preferences in females.
            \begin{itemize}
                \item Can be direct preferences, or indirect. Can lead to overrespresentions of a trait that does nothing, or even harms the fitness of males.
            \end{itemize}
        \item Gender roles can be reversed, but female choice is more common.
    \end{itemize}
    \subsubsection{Selection on Females}
    \begin{itemize}
        \item Selection can occur when:
            \begin{itemize}
                \item a single insemination delivers vastly more sperm than a female needs,
                \item mating exposes females to diseases and other dangers,
                \item when traits that females posses increases fitness of offspring,
                \item when males provide parental care, 
                \item or when traits can attract more fit males that females can choose from.
            \end{itemize}
        \item \textbf{Polyandry}: multiple mating by females.
            \begin{itemize}
                \item Monogamy appears common, indicating little selection on females, but it turns out it's more common than once thought.
            \end{itemize}
    \end{itemize}
    \item \textbf{Sex allocation theory}: when individuals can change sex when the reproductive value of the other sex exceeds that of its present sex. 
        \begin{itemize}
            \item Has occurred in a variety of species, and is common in fish.
        \end{itemize}
\end{itemize}

%\endgroup
%%%%%%%%%%%%%%%%%%%%%%%%%%%%% Chapter 11 %%%%%%%%%%%%%%%%%%%%%%%%%%%%%

%%%%%%%%%%%%%%%%%%%%%%%%%%%%% Chapter 12 %%%%%%%%%%%%%%%%%%%%%%%%%%%%%
%\begingroup
\clearpage
\section{Evolution of Social Behavior}\phantomsection
\subsection{Four Kinds of Social Behavior}
\begin{itemize}
    \item \textbf{Mutally beneficial (cooperation)}: when both the recipient's and actor's fitness is increased due to the interaction between them.
        \begin{itemize}
            \item Often has the greatest overall benifits over long term.
        \end{itemize}
    \item \textbf{Selfish}: when the actor benifits, often to a greater degree, at the cost of the recipient.
        \begin{itemize}
            \item Greatest benifits over short term, and the what is expected to be selected for.
        \end{itemize}
    \item \textbf{Alturistic}: when the actor scarfices, or loses, on behalf of the recipient's success (selfish + cooperation).
    \item \textbf{Spiteful}: when the actor suffers a loss in order to impose a penalty on the recipient (selfish + selfish).
\end{itemize}

\subsection{Kin Selection and Costly Behavior}
\begin{itemize}
    \item \textbf{Hamiton's rule}: a rule centered around a parameters of relatedness (\(r\)) that shows how altrustic behavior, and thus cooperation, can spread.
        \begin{itemize}
            \item Altruism will spread when: \(Br - C > 0\)
            \item \(B\): benefit to the recipient, and \(C\): cost to the actor; both measured in units of surviving offspring.
            \item Indicates that altruism will spread when the benefits (\(Br\) to the recipeint are great, and the cost (\(C\)) is low.
        \end{itemize}
    \item Relatedness values:
        \begin{itemize}
            \item Parents-offspring: 0.5
            \item Full siblings: 0.5
            \item Half-siblings: 0.25
            \item Cousins: 0.125
        \end{itemize}
    \item \textbf{Inclusive fitness}: the sum of direct and indirect fitness of an individual.
        \begin{itemize}
            \item \textbf{Direct fitness}: reproduction and individual achieves on its own. 
            \item \textbf{Indirect fitness}: additional reproduction by relatives made possbile by individual's actions.
        \end{itemize}
    \item \textbf{Kin selection}: natural selection that leads to the spread of genes that increase the indrect componenet of inclusive fitness.
\end{itemize}
%\endgroup
%%%%%%%%%%%%%%%%%%%%%%%%%%%%% Chapter 12 %%%%%%%%%%%%%%%%%%%%%%%%%%%%%

%%%%%%%%%%%%%%%%%%%%%%%%%%%%% Chapter 14 %%%%%%%%%%%%%%%%%%%%%%%%%%%%%
%\begingroup
\clearpage
\section{Evolution of Human Health}\phantomsection
\subsection{}
\begin{itemize}
    \item 
\end{itemize}
%\endgroup
%%%%%%%%%%%%%%%%%%%%%%%%%%%%% Chapter 14 %%%%%%%%%%%%%%%%%%%%%%%%%%%%%

%%%%%%%%%%%%%%%%%%%%%%%%%%%%% Chapter 16 %%%%%%%%%%%%%%%%%%%%%%%%%%%%%
%\begingroup
\clearpage
\setcounter{section}{15}
\section{Mechanisms of Speciation}\phantomsection
\subsection{}
\begin{itemize}
    \item 
\end{itemize}
%\endgroup
%%%%%%%%%%%%%%%%%%%%%%%%%%%%% Chapter 16 %%%%%%%%%%%%%%%%%%%%%%%%%%%%%
%\endgroup

%%%%%%%%%%%%%%%%%%%%%%%%%%%%% Chapter 17 %%%%%%%%%%%%%%%%%%%%%%%%%%%%%
%\begingroup
\clearpage
\section{Origin of Life; Major Transitions}\phantomsection
\subsection{}
\begin{itemize}
    \item 
\end{itemize}
%\endgroup
%%%%%%%%%%%%%%%%%%%%%%%%%%%%% Chapter 17 %%%%%%%%%%%%%%%%%%%%%%%%%%%%%

%%%%%%%%%%%%%%%%%%%%%%%%%%%%% Chapter 18 %%%%%%%%%%%%%%%%%%%%%%%%%%%%%
%\begingroup
\clearpage
\section{Fossil Record and Extinctions}\phantomsection
\subsection{}
\begin{itemize}
    \item 
\end{itemize}
%\endgroup
%%%%%%%%%%%%%%%%%%%%%%%%%%%%% Chapter 18 %%%%%%%%%%%%%%%%%%%%%%%%%%%%%
%\endgroup

%%%%%%%%%%%%%%%%%%%%%%%%%%%%% Chapter 20 %%%%%%%%%%%%%%%%%%%%%%%%%%%%%
%\begingroup
\clearpage
\setcounter{section}{19}
\section{Human Evolution}\phantomsection
\subsection{}
\begin{itemize}
    \item 
\end{itemize}
%\endgroup
%%%%%%%%%%%%%%%%%%%%%%%%%%%%% Chapter 20 %%%%%%%%%%%%%%%%%%%%%%%%%%%%%

%%%%%%%%%%%%%%%%%%%%%%%%%%%%% Chapter 15 %%%%%%%%%%%%%%%%%%%%%%%%%%%%%
%\begingroup
\clearpage
\setcounter{section}{14}
\section{Genome Evolution}\phantomsection
\subsection{}
\begin{itemize}
    \item 
\end{itemize}
%\endgroup
%%%%%%%%%%%%%%%%%%%%%%%%%%%%% Chapter 15 %%%%%%%%%%%%%%%%%%%%%%%%%%%%%
\end{document}