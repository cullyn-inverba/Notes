\documentclass[12pt,a4paper]{article}
\usepackage{inverba}

\newcommand{\userName}{Cullyn Newman} 
\newcommand{\class}{[Subject]} 
\newcommand{\institution}{[Institution]} 
\newcommand{\theTitle}{\color{B-Cold} [Subject Title]}

\begin{document}
%%%%%%%%%%%%%%%%%%%%%%%%%%%%%%%%%%%%%%%%%%%%%%%%%%%%%%%%%%%%%%%%%%%%%
\tableofcontents
\cleardoublepage
\fancyhead{}
\fancyhead[R]{\hyperlink{home}{\nouppercase\leftmark}}
%%%%%%%%%%%%%%%%%%%%%%%%%%%%%%%%%%%%%%%%%%%%%%%%%%%%%%%%%%%%%%%%%%%%%
\clearpage
\section*{Week 8}\phantomsection
\addcontentsline{toc}{section}{\textbf{Week 8}}
\fancyhead[R]{\hyperlink{home}{Week 8}}

\fancyhead[L]{\hyperlink{home}{Friday, November 20 - Quiz 20}}
\subsection{Friday, November 20 - Quiz 20}
\begin{enumerate}
    {\color{G-Moon}\item What is the rate law for an \(S_N1\) reaction of an alkyl halide?
    \begin{itemize}
        \item rate = \(k[\text{alkyl halide}]^2\)
        \item {\color{o-Sun}\textbf{ rate = \(k[\text{alkyl halide}]\)}}
        \item rate = \(k[\text{nucleophile}]\)
        \item rate = \(k[\text{alkyl halide}]\)
    \end{itemize}}
        \begin{itemize}
            \item Unimolecular: \(S_N1\) and E2 reactions that are linearly dependent on the concentration of only one compound (the substrate).
            \item A first order rate: \(v_0=k[\text{substrate}]\) 
                \begin{itemize}
                    \item The substrate varies, and can be more than just an alkyl halide, but alkyl halides are very common.
                \end{itemize}
        \end{itemize}
    {\color{G-Moon}\item What is the rate limiting step for an \(S_N1\) reaction?
    \begin{itemize}
        \item {\color{o-Sun}\textbf{formation of carbocation}}
        \item loss of an \ch{H+} ion form the nucleophile
        \item backside attack of the nucleophile
    \end{itemize}}
        \begin{itemize}
            \item The alkyl halide often acts as the rate-determini step for \(S_N1\) reactions reactions.
                \begin{itemize}
                    \item I think the better answer would be that the loss of the alkyl halide (leaving group) acts as the rate-determining step, but the loss of the leaving group is what forms the carbocation.
                \end{itemize}
            \item \(S_N1\) reactions are often just two-steps (formation of carbocation $\rightarrow$ nucleophilic attack), but the transfer of an \ch{H+} can occur if the nucleophile is uncharged, which is done by a solvent molecule. 
        \end{itemize}
    {\color{G-Moon}\item Which reaction proceeds with an inversion of configuration?
    \begin{itemize}
        \item \(S_N1\)
        \item {\color{o-Sun}\textbf{\(\bm{S_N2}\)}}
        \item acid base reactions
        \item addition reactions
    \end{itemize}}
        \begin{itemize}
            \item \(S_N2\) reactions always invert, but technically both \(S_N1\) and \(S_N2\) reactions can do so as well.
            \item If the halogen in a \(S_N1\) reaction is bonded to an asymmetric center then it produces a pair of enantiomers, one of which will be inverted.
                \begin{itemize}
                    \item This is because the nucleophile can attack the carbocation in either direction. 
                    \item Note: observation evidence shows a there is sometimes a slight preference for the inverted product, probably due to the type of ion pairs created.
                \end{itemize}
        \end{itemize}
    {\color{G-Moon}\item Which conditions are most favorable an \(S_N1\) reaction?
    \begin{itemize}
        \item a bulky nucleophile and a \ang{1} alkyl halide
        \item a bulky nucleophile and a \ang{2} alkyl halide
        \item {\color{o-Sun}\textbf{a bulky nucleophile and a \ang{3} alkyl halide}}
        \item a small unhindered nucleophile and a \ang{1} alkyl halide
    \end{itemize}}
        \begin{itemize}
            \item Primary and secondary alkyl halides don't undergo \(S_N1\) reactions reactions. The carbocation needs to be stabilized (hyperconjugation), and can need the most stable form (\ang{3}) in oder to do so.
        \end{itemize}
    {\color{G-Moon}\item For either an \(S_N1\) or an \(S_N2\) reaction, which is the best leaving group?
    \begin{itemize}
        \item {\color{o-Sun}\textbf{\ch{I-}}}
        \item \ch{F-}
        \item \ch{Cl-}
        \item \ch{Br-}
    \end{itemize}}
        \begin{itemize}
            \item Alkyl iodides are the most reactive, while flourides are the least.
            \begin{itemize}
                \item \(I^->Br^->Cl^->F^-\)
            \end{itemize}
            \item This is in large part determined by {\color{o-Sun}polarizability}, which {\color{o-Sun}increases with size of atom} since electrons occupy more space around the atom and thus have a higher chance of randomly forming differences in charge due to uneven distribution of electrons.
        \end{itemize}
    {\color{G-Moon}\item Why is a tertiary carbocation so much more stable than a primary carbocation?
    \begin{itemize}
        \item steric bulk helps stabilize the tertiary carbocation
        \item {\color{o-Sun}\textbf{hyperconjugation provides more stability in a tertiary carbocation than in primary carbocation.}}
        \item a tertiary carbocation is more stabilized by the leaving group
        \item a tertiary carbocation is more ideal trigonal planar geometry
    \end{itemize}}
        \begin{itemize}
            \item \textbf{Hyperconjugation}: nearby electron sharing of electrons with adjacent empty, or partially filled orbitals, that give rise to more stability.
            \item A primary carbocation has no such orbitals to share with, so the instability is increased.
        \end{itemize}
    {\color{G-Moon}\item According to the Hammond Postulate, the transition state for a reaction that proceeds with a large activation energy 
    \begin{itemize}
        \item will be more likely to go by an \(S_N2\) pathway
        \item {\color{o-Sun}\textbf{looks more like the products}}
        \item will be mor likely to go by an \(S_N1\) pathway
        \item looks more like the reactants
    \end{itemize}}
        \begin{itemize}
            \item A higher energy transition state (larger activation energy) means it takes more energy to make any sort of change at all, which means change occurs rapidly once it finally can---creating transition states that look more like the products.
                \begin{itemize}
                    \item This is why they are {\color{o-Sun}endothermic}; more energy is required, resulting in  more energy {\color{o-Sun}entering} the system from the outside.
                    \item This also explains why the products are less stable; more energy remains in the system due to these products.
                \end{itemize}
        \end{itemize}
    {\color{G-Moon}\item The choice of a solvent affects the rate of the reaction by
    \begin{itemize}
        \item[A] changing the stability of the reactants
        \item[B] changing the stability of the transition state
        \item[C] changing the stability of the products
        \item all of the above
        \item {\color{o-Sun}\textbf{A and B}}
    \end{itemize}}
        \begin{itemize}
            \item The choice of solvent, often protic or aprotic, changes the stability of the added nucleophiles (increasing/decreasing respectively). 
            \item Stronger, less stable nucleophiles increase the potential energy in the reaction, which decreases the activation energy. The inverse is true for weaker, more stable 
            \item A change the activation energy often results in different products being formed, which usually differ in stability, but this does not affect the rate.
        \end{itemize}
\end{enumerate}
\end{document}