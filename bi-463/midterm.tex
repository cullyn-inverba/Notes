\documentclass[basic]{inVerba-notes}

\newcommand{\userName}{Cullyn Newman}
\newcommand{\class}{BI:\@ 463}
\newcommand{\theTitle}{Neurophysiology Midterm}
\newcommand{\institution}{Portland State University}

% chktex-file 36

\begin{document}
\begin{enumerate} 
  \minimal{\item[1-A.] Acetylcholinesterase is a crucial enzyme that is responsible for clearing acetylcholine from the synaptic cleft after presynaptic release. Some patients have a mutation that renders this enzyme only partially active. Please explain which branch of the autonomic nervous system (sympathetic vs.\ parasympathetic) would be MORE adversely affected by this mutation. Explain your reasoning. (4 pts)}

  Acetylcholinesterase (ACHE) is found mostly in neuromuscular junctions,  
  
  \minimal{\item[1-B.] Additionally, please list 3 symptoms (related to the autonomic nervous system) that an individual with this mutation may present with. (3 pts)}
  
  \minimal{\item[2-A.] Imagine an individual has gotten an injury resulting in a cut down the midline of the ENTIRE LUMBAR segment of the spinal corD. Upon clinical examination would this person be able to lift their right leg? Why or why not? (3 pts)}

  \minimal{\item[2-B.] Would this person be able to lift their left leg? Why or why not? (3 pts)}

  \minimal{\item[3-A.] Now imagine that a patient has a spinal cord injury resulting in major damage to one half of the CERVICAL spinal corD. Would this patient be able to lift their right leg? Why or why not? (3 pts)}
  
  \minimal{\item[3-B.] Would this person be able to lift their left leg? Why or why not? (3 pts)}


  \minimal{\item[4-A.] You have a patient that has developed symptoms of persistent muscle cramping in which muscles such as their bicep enters states of contraction that are hard to relax most times they try to pick something up. The muscle DOES EVENTUALLY RELAX, although it takes time. Please go into DETAIL describing TWO LIKELY cellular/molecular processes in muscle that COULD be causing these tetanic responses in the muscle. (6 pts)}

  \minimal{\item[4-B.] Now imaging you could measure the action potential from the bicep in this individual. Whereas a normal AP in muscle is quite broad (left), the AP you measure is more narrow (right). What two channels are more likely to have potentially caused the change in AP shape? How? (3 pts)}
  
  \minimal{\item[5-A.] In contrast to the cytoplasm (which has 100 nanomolar calcium), the inside of the sarcoplasmic reticulum in muscle fibers has a calcium concentration of 500 micromolar. Assume room temperature conditions. Please calculate the equilibrium potential of calcium given the concentrations above. (3 pts)}

  \minimal{\item[5-B.] If you increase the concentration of calcium within the sarcoplasmic reticulum will this increase or decrease the capacity of a given muscle fiber to contract? Explain why? (4 pts)}

  \minimal{\item[5-C.] The EXTRACELLULAR concentration of calcium is approximately 2 millimolar. If we decrease this calcium concentration to 500 micromolar, will this increase or decrease the capacity for an effective muscle contraction. Please show why mathematically with a short description of why you arrived at the answer you diD. (4 pts)}

  \minimal{\item There is a time delay between the action potential produced in muscle and the actual muscle contraction as depicted bellow:}

  \minimal{\item[6-A] In your own words please give a detailed description of why this delay exists including all steps between the AP produced through the END of the muscle contraction. (8 pts)}

  \minimal{\item In your own words, please explain the difference between how the sympathetic vs.\ parasympathetic nervous system affect the HEART during:}
  \begin{itemize}
    \minimal{\item[A.] A bear chasing you in Olympic national park (4 pts):}
    \minimal{\item[B.] Hanging out at home after taking an exam in neurophysiology (4 pts):}
  \end{itemize}

  \minimal{\item[7-C.] Briefly, how does this differ from what happens at the smooth muscle of the bladder wall during the A and B scenario (4 pts)?}

  \minimal{\item The following is a representative image of a neuromuscular junction.}
  
  \minimal{\item[8-A.]What kind of ion channel(s) produce the changes in membrane potential shown at the top right? (2 pt)}

  \minimal{\item[8-B.] What kind of ion channel(s) produce the changes in current shown at the bottom right? (2 pt)}

  \minimal{\item[8-C.] Where is the cell body from the motor neuron innervating the muscle located? (Please circle the correct answer) (2 pt)}

  \minimal{\item[8-D.] Where does the motor neuron exit the spinal cord? (Please circle the correct answer) (2 pt)}

  \minimal{\item The following image shows a current (red arrow) produced by a sensory receptor in the periphery.}
  
  \minimal{\item[9-A.] Assuming the resistance of this receptor is 5 megaohms, what is the driving force for the ions that flow through this channel? (4 pts)}

  \minimal{\item[9-B.] Imagine this receptor is only permeable to sodium and the original resting membrane potential of this cell was -50 mV prior to channel opening. What is the equilibrium potential for sodium? (Use the driving force calculated above)? Show your work. (4 pts)}

  \minimal{\item Please answer to the following questions. In the bellow graph assume an ENa+ = +60mV and EK+ = -90 mV. Assume room temperature for all experiments.}

  \minimal{\item[10-A.] Please draw a new graph on top of the graph below if we change out Na+ concentrations such that Na+ outside the cell = 5mM and Na+ inside the cell = 1mM\@? Explain your new graph. (2 pts)}

  \minimal{\item[10-B.] Please draw a new graph on top of the graph below if we change out K+ concentrations such that K+ outside the cell = 150mM and K+ inside the cell = 5mM\@? Explain your new graph. (2 pts)}

  \minimal{\item What ion channel is responsible for the depolarization phase of the AP\@? (2 pts)}
  
  \minimal{\item Which branch of the autonomic nervous system has a long preganglionic axon? (2 pts)}
  
  \minimal{\item Which ion is most crucial for the process of vesicular fusion/synaptic release? (2 pts)}
  
  \minimal{\item Please read the attached PDF\@.}
  
  \minimal{\item[14-A.] What questions did the authors wanted to address in this paper? (3 pts)}
  
  \minimal{\item[14-B.] What techniques did they use to accomplish the experiments presented? (3 pts)}

  \minimal{\item[14-C.] What were the main findings of the paper? In detail, discuss how this relates to the motor unit we learned about in class. What implications do their findings have for control of motor movement? (6 pts)}

  \minimal{\item[14-D.] For patients suffering from this disease, why are acetylcholinesterase inhibitors prescribed? (3 pts)}
\end{enumerate}
\end{document}