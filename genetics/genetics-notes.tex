\documentclass[12pt,a4paper]{article}
\usepackage{inverba}
\newcommand{\userName}{Cullyn Newman} 
\newcommand{\class}{BI 341} 
\newcommand{\institution}{Portland State University} 
\newcommand{\thetitle}{\hypertarget{home}{Genetics Notes}}
\rfoot{\hyperlink{home}{\thepage}}

\begin{document}
%%%%%%%%%%%%%%%%%%%%%%%%%%%%%%%%%%%%%%%%%%%%%%%%%%%%%%%%%%%%%%%%%%%%%
\tableofcontents
\cleardoublepage
\fancyhead{}
\fancyhead[R]{\hyperlink{home}{\nouppercase\leftmark}}
%%%%%%%%%%%%%%%%%%%%%%%%%%%%%%%%%%%%%%%%%%%%%%%%%%%%%%%%%%%%%%%%%%%%%
\setcounter{section}{7}
%%%%%%%%%%%%%%%%%%%%%%%%%%%%% Chapter 8 %%%%%%%%%%%%%%%%%%%%%%%%%%%%%
%\begingroup
\clearpage
\section{Chapter}
\subsection{Section}
\begin{itemize}
    \item 
\end{itemize}
%\endgroup
%%%%%%%%%%%%%%%%%%%%%%%%%%%%% Chapter 8 %%%%%%%%%%%%%%%%%%%%%%%%%%%%%
\setcounter{section}{9}
%%%%%%%%%%%%%%%%%%%%%%%%%%%% Chapter 10 %%%%%%%%%%%%%%%%%%%%%%%%%%%%%
%\begingroup
\clearpage
\section{Genomics, Proteomics, and Genetic Engineering}
\subsection{Genome Sequencing}
\begin{itemize}
    \item In most methods, short, doubled stranded \textbf{adapter molecules}, which are complementary to oligonucleotide primers allowing for PCR amplification.
    \item \textbf{Sequencing by synthesis}: shear DNA, spread out over a flat surface, and subjected to PCR amplification. Result is many small clusters of PCR products, that are analyzed with reversible terinators, with 3' end blocked, forcing addition of one base as well as a floresent tags.
    \item \textbf{Ion torrent sequencing}: a method of DNA sequencing based on the detection of hydrogen ions that are released during the polymerization of DNA. This technology differs from other sequencing-by-synthesis technologies in that no modified nucleotides or optics are used.
    \item \textbf{Single-molecule sequencing}: A single DNA polymerase enzyme is affixed at the bottom of a ZMW with a single molecule of DNA as a template. The ZMW is a structure that creates an illuminated observation volume that is small enough to observe only a single nucleotide of DNA being incorporated by DNA polymerase.
    \item \textbf{Nanopore sequencing}: a single molecule of DNA or RNA can be sequenced without the need for PCR amplification or chemical labeling of the sample. Has higher rate of error, up to 5-10\%.
    \item \textbf{Comparative genomics}: a field of research where organisms are compared, mostly by alignment of genome sequences and checking extent of conservation.
    \item Comparative genomics can also help target regulatory motifs, which are hard to find due to their relatively short sequences and ability to change position. 
\end{itemize}

\subsection{Genomics and Proteomics}
\begin{itemize}
    \item \textbf{Functional genomics}: a dynamic focus on genome-wide paterns of gene expression and coordination instead of just DNA structures.
    \item \textbf{DNA mircoarray}: a collection of microscopic DNA spots attached to a solid surface that allow for the measurement of the expression levels of large numbers of genes simultaneously or to genotype multiple regions of a genome. 
\end{itemize}

\subsection{Recombinant DNA}
\begin{itemize}
    \item \textbf{Recombinant DNA}: isolated DNA is cut into fragments by one or more restircion enzyme, joined back together in a new combination, and then reintroduced into a cell or organism.
    \item Restriction enzymes have the same sticky ends regardless of organism as log as they were produced by same enzyme.
    \item Most useful vectors are easily introduced, contain a replication origin, and allow the growth of a cell on a solid selective medium.
    \item Some vectors can accept large DNA fragments, up to 100-200 kb, and are called \textit{artificial chromosomes}. 
    \item Most common are bacterial artificial chromosomes (BACs).
    \item The sticky ends are joined together using DNA ligase.
    \item Reverse transcriptase can synthesize a complementary strand of DNA called cDNA.
    \item The 3' end of cDNA can fold back on itself, creating a hairpin primer for second-strand synthesis.
    \item \textit{Polylinker}, or multiple cloning site (MCS), in a vectors makes directional cloning possible. 
\end{itemize}


%\endgroup
%%%%%%%%%%%%%%%%%%%%%%%%%%%% Chapter 10 %%%%%%%%%%%%%%%%%%%%%%%%%%%%%
\setcounter{section}{12}
%%%%%%%%%%%%%%%%%%%%%%%%%%%% Chapter 13 %%%%%%%%%%%%%%%%%%%%%%%%%%%%%
%\begingroup
\clearpage
\section{Molecular Genetids of the Cell Cycle and Cancer}
\subsection{The Cell Cycle is Under Genetic Control}
\begin{itemize}
    \item \textbf{Centrosome}: a small region of clear cytoplasm near the interphase nucleus, typically (in most eukaryotes) made up of a pair of \textbf{centrioles}. 
    \item The function of both are microtubule-organizing centers and a regulators of the cell cycle.
    \item Genes encoding proteins that are needed in the cell cycle are typically transcribed right before they are needed.
    \item \textbf{cyclin-CDK complexes}: regulator of progression in the early stages of the cell cycle.
    \item Protein degradation also helps regulate the cell cycle.
\end{itemize}
\subsection{Cell Checkpoints}
\begin{itemize}
    \item Checkpoints help maintain the correct steps as the cell cycle progresses, causing the cycle to pause until correction is done.
    \item Three principal checkpoints: DNA damage, centrosome duplication, and spindle checkpoints.
    \item Failure to stop may lead to aneuploidy(spindle), polyploidy(centrosome), of increased number of mutations (DNA; translocation,deletion, of amplification).
    \item \textbf{p53 transcription factor}: key proteins that come in slightly different forms that respond to stress and DNA damage.
    \item In norman cells, p53 is low and is removed by Mdm2 for degradation. Damaged DNA results in phosphorylation and and inability of Mdm2 to export it. 
    \item Increased p53 turns on/off transcription of other genes that halt the cell cycle and other cellular properties.
    \item \textbf{Oncogenes}: genes associated with cancers, which can interfer with apoptosis.
    \item Shortage of oxygen, DNA damage, or shortage of nucleoside triphosphates can increase p53 activity.
    \item Apopotsis, angiogenesis/metastasis, or arrest/repair pathways may all be activated by activated/deactivated genes.
    \item Centrosome duplication checkpoint is one that monitors the formation of the spindle and coordinate entry into mitosis.
    \item The spindle assembly checkpoint may work with the centrosome checkpoint, but it also monitors spindle attachment of kinetochores (spindle fiber attachment site on the chromosome). 
    \item Improper spindle assembly blocks the separation of sister chromatides, preventing anaphase.
\end{itemize}

\subsection{Cancer Cell Mutations}
\begin{itemize}
    \item \textbf{Familial}: clear evidence for segregagion of a gene that prediposes cells to progress to the cancerous state.
    \item \textbf{Sporadic}: cancer resulting in genetic changes in somatic cells that is not the result of a familial case.
    \item Six attributes of cancer cells:
        \begin{itemize}
            \item loss of growth-factor dependence.
            \item Insensitivity to anti-growth signals.
            \item Evasion of apoptosis.
            \item No cell senescence.
            \item Ability to metastasize and invade other tissues.
            \item Sustained angiogenesis (formation of blood vessels)
        \end{itemize}
    \item \textit{proto-oncogenes}: promote cell division of prevent apoptosis.
    \item \textit{tumor-suppressor}: prevent dell division or promote apoptosis.
\end{itemize}

%\endgroup
%%%%%%%%%%%%%%%%%%%%%%%%%%%% Chapter 13 %%%%%%%%%%%%%%%%%%%%%%%%%%%%%

%%%%%%%%%%%%%%%%%%%%%%%%%%%% Chapter 14 %%%%%%%%%%%%%%%%%%%%%%%%%%%%%
%\begingroup
\clearpage
\section{Molecular Evolution and Population Genetics}
\subsection{Evolutionary Relationships Among Species}
\begin{itemize}
    \item \textbf{Molecular evolution}: study of how the sequences of macromolecules change through time.
    \item \textbf{Molecular phylogenetics}: study of evolutionary relationships among species.
    \item a mutant allele is \textbf{fixed} if it replaces all other allels in the population.
    \item \textbf{Gene tree}: a methoad to estimate the pattern of evolutionary relationships in different species based on a single gene.
    \item Rates of evolution can differ dramatically from one protein to another.
    \item \textbf{Synonymous substitution}: a modification in which the amino acid is not affected.
    \item \textbf{Nonsynonymous substitution}: a modification that does result in an amino acid replacement.
    \item Nonsynonymous sites occurs primarily at the first and second codon positions.
    \item \textbf{Pseudogenes}: duplicate genes that have lost their function due to mutations.
    \item \textit{fourfold degenerate site}: a synonymous nucleotide site that produces same amino acid regardless of nucleotide.
    \item \textit{Twofold degenerate site}: encoded amino acid only is affected depending on if it is a pyrimidine or purine.
    \item \textbf{Subfunctionalization}: specialization of paralogs accompanying loss of functional capabilities.
\end{itemize}

\subsection{Genotypes Frequency in Populations}
\begin{itemize}
    \item \textbf{Genotype frequency}: proportion of a population that have a particular genotype e.g., amounts of AA.
    \item \textbf{Allele frequency}: proportion of of all alleles that of specified type e.g., amount of A.
\end{itemize}

\subsection{Random Mating} 
\begin{itemize}
    \item One important implication ot the Hardy-Weinberg principle is that the allele frequencies remain constant from generation to generation.
    \item Hardy-Weinberg depends on several assumptions:
        \begin{itemize}
            \item Mating is random.
            \item Allele frequencies are the same in males and females.
            \item All genotypes have equal survivability and fertility.
            \item Mutation does not occur.
            \item Migration is absent.
            \item The population is sufficiently large to avoid chance of sudden major allele frequency changes.
        \end{itemize}
    \item For a rare allele, the frequency of heterozygotes far exceeds the frequency of the rare homozygote.
    \item 
\end{itemize}

\subsection{Highly Polymorphic Sequences and DNA Typing}
\begin{itemize}
    \item \textbf{Polymorphic}: two or more allele are common in the population.
    \item \textbf{DNA typing}: the use of polymorphisms to identify an individuals DNA characteristics.
    \item \textbf{Simple tandem repeat (STR)}: a polymorphism that contains units of DNA repeated in tandem.
    \item Most people are heterozygous for STR alleles that produce restriction fragments of different sizes.
    \item The restriction fragments from different people cover a wide range of sizes-- indicating the population as a whole contains many STR allels.
\end{itemize}
%\endgroup
%%%%%%%%%%%%%%%%%%%%%%%%%%%% Chapter 14 %%%%%%%%%%%%%%%%%%%%%%%%%%%%%

%%%%%%%%%%%%%%%%%%%%%%%%%%%% Chapter 15 %%%%%%%%%%%%%%%%%%%%%%%%%%%%%
%\begingroup
\clearpage
\section{Chapter}
\subsection{Section}
\begin{itemize}
    \item 
\end{itemize}
%\endgroup
%%%%%%%%%%%%%%%%%%%%%%%%%%%% Chapter 15 %%%%%%%%%%%%%%%%%%%%%%%%%%%%%
\end{document}