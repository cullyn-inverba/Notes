\documentclass[12pt,a4paper]{article}
\usepackage{inverba}

\newcommand{\userName}{Cullyn Newman} 
\newcommand{\class}{CH 334} 
\newcommand{\institution}{Portland State University} 
\newcommand{\theTitle}{\color{B-Cold} Organic Chemisty}

\begin{document}
%%%%%%%%%%%%%%%%%%%%%%%%%%%%%%%%%%%%%%%%%%%%%%%%%%%%%%%%%%%%%%%%%%%%%
\tableofcontents
\cleardoublepage
\fancyhead{}
\fancyhead[R]{\hyperlink{home}{\nouppercase\leftmark}}
%%%%%%%%%%%%%%%%%%%%%%%%%%%%%%%%%%%%%%%%%%%%%%%%%%%%%%%%%%%%%%%%%%%%%

\clearpage
\fancyhead[L]{Midterm 1}
%\begingroup
%%%%%%%%%%%%%%%%%%%%%%%%%%%%% Chapter 1 %%%%%%%%%%%%%%%%%%%%%%%%%%%%%
%\begingroup
\clearpage
\section{General Chemistry Review}\phantomsection
\subsection{Electrons, Bonds, and Lewis Structures}
\begin{itemize}
    \item \textbf{Covalent bond}: two atoms sharing a pair of electrons.
    \item \textbf{Octet rule}: \textit{main group elements} that tend to bond in a way that each atom has {\color{o-Sun}eight} electrons in it's valence shell.
        \begin{itemize}
            \item Atoms that do not have eight will share electrons with other elements in order to maintain a stable state.
        \end{itemize}
    \item \textbf{Main group elements}: sometimes called representative elements, are groups 1, 2 and 13--18 in periodic table.
        \begin{itemize}
            \item Some elements in group 3 and 12 share properties between transition metals and the main group.
        \end{itemize}
    \item The lowest energy (most stable) state of two atoms is determined both by bond length and bond strength.
    \item Valence electrons are determined by the group, 1A--8A, of the periodic table.
    \item \textbf{Lone pair}: unshared, or nonbonding, electrons.
    \item \textbf{Lewis structures}: 2D model that represnets covalent bonds as straight lines and lonpairs as dots.
    \item Examples: \ch{COF2}, \ch{H2O}, \ch{NO3-}, \ch{N2O}:
        \begin{align*}
            \chemfig{C(=[2,.8]O)(-[5]F)(-[7]F)}
            \hspace{1cm}
            \chemfig{[:40]H-\lewis{13,O}-[::-80]H} 
            \hspace{1cm}
            \left[\chemfig{N(=[2,.8]O)(-[5]O)(-[7]O)}\right]^{-1} 
            \hspace{1cm}
            \chemfig{N(~[4]N)(-[0]O)}
        \end{align*}
    \item \textbf{Resonance structures}: a set of two or more Lewis structures that collectively describe the electronic bonding of a single polyatomic species, including fractional bonds.
\end{itemize}

\subsection{Identifying Formal Charges}
\begin{itemize}
    \item \textbf{Formal charge}: any atom that does not exhibit the appropriate number of valance electrons.
    \item Determing formal charge:
        \begin{itemize}
            \item Formula: {\color{o-Sun}\(FC = V - N - \dfrac{B}{2}\)}
            \item V = valance electrons of element
            \item N = lone pair electrons
            \item B = bonded electrons
        \end{itemize}
    \item {\color{pos}Less} than expected number of valence electrons results in a {\color{pos} positive} charge.
    \item {\color{neg}More} than expected number of valence electrons results in a {\color{neg}negative} charge.
    \item The lower the {\color{o-Sun}magnitude} of formal charge, the {\color{o-Sun}greater the stability} of the whole molecule.
    \item Atoms that are {\color{neg}more electronegative} hold {\color{neg}negative} formal charges better, which results in {\color{o-Sun}greater stability} vs when the negative charge is spread on less electronegative elements in a polyatomic species.
        \begin{itemize}
            \item The dominant resonance structure will be that of the greatest stability. 
        \end{itemize}
\end{itemize}

\subsection{Induction and Polar Covalent Bonds}
\begin{itemize}
    \item Bonds can classified into three categories: covalent, polar covalent, and ionic.
    \item The categories emerge from the electronegativity values of the atoms sharing a bond.
    \item \textbf{Electronegativity}: a measure of the ability of an atom to attract electrons.
        \begin{itemize}
            \item Electronegativity generally {\color{o-Sun}increases left to right}, and from the {\color{o-Sun}bottom to top} of the periodic table.
        \end{itemize}
    \item \textbf{Covalent bond}: when the difference in electronegativity is {\color{o-Sun}less than 0.5}.
    \item \textbf{Polar covalent bond}: when the difference in electronegativity is {\color{o-Sun}between 0.5 and 1.9}, then the electrons are not equally shared and become polar.
    \item \textbf{Induction}: the withdrawl of electrons towards to more electronegative atom. {\color{pos}\ch{$\delta$^+}} represnets partial positive charged gained when electrons are pulled away, while {\color{neg}\ch{$\delta$^-}} represnets the partial negative charge pulled closer.
    \item \textbf{Ionic bond}: when the difference in electronegativity is {\color{o-Sun}greater than 1.9}.
        \begin{itemize}
            \item Electrons are not shared in this case, and attraction is insetsad just the result of oppositely charged ions.
        \end{itemize}
\end{itemize}

\subsection{Atomic Orbitals}
\begin{itemize}
    \item \textbf{Atomic orbital (AO)}: standing quantum wave (excitation in electron field) around an atom.
    \begin{itemize}
            \item More energy leads to higher orbtails levels.
                \begin{itemize}
                    \item Gives principle quantum number, \(n\), as is associated with distance from nucleus.
                \end{itemize}
            \item Orbital levels: s(1 pair), p(3 pairs), d(5 pairs), f(7 pairs). 
                \begin{itemize}
                    \item Angular momentum quantum number that describes three-dimensional region of space that the electron density occupies.
                \end{itemize}
            \item Magnetic quantum number descrices orientation in space of electron density.
                \begin{itemize}
                    \item \(m_l=0\); s orbital
                    \item \(m_l=-1, 0, 1\); p\(_{x}\), p\(_{y}\), p\(_{z}\) orbitals.
                \end{itemize}
            \item Locations where $\psi$ (quantum wave function) is zero are called \textbf{nodes}.
                \begin{itemize}
                    \item The {\color{o-Sun}more nodes} that an orbital has, the {\color{o-Sun}greater} it's energy.
                \end{itemize}
            \item \textit{Spin}: allows an orbital to contain only two electrons, \(\pm \frac{1}{2}\)
        \end{itemize}
    \item \textbf{Degenerate orbitals}: orbitals with the same energy level.
    \item Order in which orbitals are filled is determined by three principles:
        \begin{itemize}
            \item \textbf{Aufbau principle}: lowest energy orbital is filled first.
            \item \textbf{Pauli exclusion principle}: each orbital can accommodate a maximum of two electrons that have opposite spin.
            \item \textbf{Hund's rule}: electrons are placed in each degenerate orbital before being paired up.
        \end{itemize}
    \item Describing the nature of atomic orbital is done with two commoly used theories: \textit{Valence Bond Theory} and \textit{Molecular Orbital Theory}.
    \item The commonly used theories give a deeper understanding of covalent bonds, which is essentially just the {\color{o-Sun}overlap of atomic orbitals}.
    \item \textbf{Constructive/destructive interference}: the result of two waves that approach each other, or overlap.
        \begin{itemize}
            \item Constructive interference produces a wave with the vector sum of both waves.
            \item Destructive interference cancel each other out and produes a node.
        \end{itemize}
\end{itemize}

\subsection{Valence Bond Theory}
\begin{itemize}
    \item \textbf{Valence bond theory}: the sharing of electron density between two atoms is a result of the constructive interference of their atomic orbitals.
    \item \textit{Bond axis}: the line that can be drawn between two hydrogen atoms.
    \item \textbf{Sigma bond (\(\bm{\sigma}\))}: a particular type of covalent bond that has circular symmetry with respect to the bond axis.
        \begin{itemize}
            \item All single bonds are $\sigma$ bonds.
            \item The strongest type of covalent bond.
        \end{itemize}
    \item \textbf{Pi bond ($\bm{\pi}$)}: covalent bonds where two lobes of an orbital overlap with two lobes of another atom. 
        \begin{itemize}
            \item Each atomic orbital has zero electron density at a shared nodal plane, passing through the two bonded nuclei.
            \item $\pi$ bonds form double ($\sigma + \pi$) and triple bonds ($\pi + \sigma + \pi$).
            \item Individual $\pi$ bonds are weaker than $\sigma$ bonds.
        \end{itemize}
\end{itemize}

\subsection{Molecular Orbital Theory}
\begin{itemize}
    \item \textbf{Molecular orbital theory (MO)}: uses linear combinations of atomic orbitals to model and explore the consequences of orbital overlap.
        \begin{itemize}
            \item The newly described orbitals are called {\color{o-Sun}molecular orbitals} accroding to MO theory.
        \end{itemize}
    \item Atomic orbitals refer to an individual atom, while molecular orbitals is associated with an entire molecular.
    \item In other words, MO theory states that atomic orbitals cease to exist when they overlap. Instead they are replaced with multiple molecular orbitals which span the entire molecule.
    \item Molecular orbitals are more stable (lower energy) since electrons are attracted by both nuclei.
    \item When there are {\color{o-Sun}nodes} between the nuclei, then the resulting $\sigma^*$ orbitals become {\color{o-Sun}antibonding}, as they {\color{o-Sun}destabilize} (increase the energy) of a molecular orbital.
    \item Best used to produce a quantitative picture of bonding.
        \begin{itemize}
            \item Describes strength, order, and polarity of bonds.
            \item Allows for the presence of paired or unpaired electrons.
            \item Has spectroscopic preperties.
        \end{itemize}
\end{itemize}

\subsection{Hybridized Atomic Orbitals}
\begin{itemize}
    \item \textbf{sp\(\bm{^3}\)-hybridized orbitals}: produced by averaging one \textit{s} orbital and {\color{o-Sun}three} \textit{p} orbitals.
        \begin{itemize}
            \item Hybridized orbitals explains to geomtry of methane, which results form the {\color{o-Sun}now four degenerate} orbitals pushing apart to achieve tetrahedral geometry.
            \item Hybridized orbitals become {\color{o-Sun}unsymmetrical}, producing a larger front lobe that is more efficient than standard \textit{p} orbitals in the ability to form bonds.
            \item All bonds in are {\color{o-Sun}$\sigma$ bonds}, and thus can be individually represented by the overlap of atomic orbitals.
        \end{itemize}
    \item \textbf{sp\(\bm{^2}\)-hybridized orbitals}: produced by averaging the \textit{s} orbital with only {\color{o-Sun}two} of \textit{p} orbitals.
        \begin{itemize}
            \item The remaining \textit{p} orbital is unaffected, and free multiple p orbitals results in a $\pi$ bond.
            \item This is done to expain geometry of compounds bearing a double bond.
            \item A double bond if formed from one $\sigma$ bond and one $\pi$ bond.
            \item Associated with \textit{trigonal planar geometry}.
        \end{itemize}
    \item \textbf{sp-hybridized orbitals}: produced by averaging of one \textit{s} orbital and {\color{o-Sun}one} \textit{p} orbital.
        \begin{itemize}
            \item Leaves two \textit{p} orbitals and resulting in two $\pi$ bonds.
            \item A triple bond is formed with the addition of one $\sigma$ bond due to the overlap of the sp orbitals.
            \item Geometry of a triple bond has \textit{linear geometry}.
        \end{itemize}
    \item Finding the hybridization of any atom can be done simply: 
        \begin{itemize}
            \item[1.] Look at the central item.
            \item[2.] Determin groups (number of bonds, $\pi$ bonds count as 1, and lone pairs attached) of atom.
                    \begin{itemize}
                        \item groups aka regions of electron density.
                    \end{itemize}
            \item[3.] For groups 1-4: sp\(^{x}\); x = groups - 1  
            \item[4.] For groups 5-6: sp\(^{3}\)d\(^{x}\); x = groups - 4 
        \end{itemize}
    \item Bond Strength and Bond Length:
        \begin{itemize}
            \item Bond length {\color{neg}decreases} with more bonds.
            \item Bond strength {\color{pos}increases} with more bonds.
            \item The more {\color{o-Sun}\textit{s} character}, the {\color{pos}shorter} and {\color{pos}stronger} the bond, and the {\color{pos}larger} the bond angle.
                \begin{itemize}
                    \item \textit{s-character}: contribution of the $\sigma$ bond in a hybridization.
                        \begin{itemize}
                            \item e.g. sp = 50\%, sp\(^{2}\) = 33\%, sp\(^{3}\) = 25\%
                        \end{itemize}
                    \item sp-sp bond is the strongest, sp\(^{3}\)-sp\(^{3}\) is the weakest.
                \end{itemize}
        \end{itemize}
\end{itemize}

\subsection{Molecular Geometry}
\begin{itemize}
    \item \textbf{Valence shell electron pair repulsion (VSEPR) theory}: enables the {\color{o-Sun}prediction of molecular geometry} due to the pressumption that all electron pairs repel each other; resulting in a three-dimensional space that {\color{o-Sun}maximizes distance} from each other.
    \item \textbf{Steric number}: the total number of electron pairs in a molecule. Can be bonds or lone pairs.
    \item \textbf{Tetrahedral geometry}: result of four $\sigma$ bonds and zero lone pairs. 
        \begin{itemize}
            \item produces a tetrahendron with bond angles of \ang{109.5}.
        \end{itemize}
    \item \textbf{Trigonal pyramidal geometry}: three $\sigma$ bonds and one lone pair.
        \begin{itemize}
            \item The lone pair occupy more space than bonded electron pairs, so the remaining angles are slightly less than a tetrahedral, at \ang{107}.
            \item The lone pair sits atop the base forming a pyramid like structure.
        \end{itemize}
    \item \textbf{Bent geometry}: two $\sigma$ bonds and two lone pairs.
        \begin{itemize}
            \item VSEPR predicts the lone pairs to be in two corners of the tetrahedral, producing bond angles of \ang{105}.
            \item VSEPR predicts geometry \ch{H2O} correctly, but for wrong reasons.
                \begin{itemize}
                    \item The lone pairs in \ch{H2O} have different energy levels, suggesting one pair occupies a \textit{p} orbital with the other in a lower-energy hybridized orbital.
                \end{itemize}
        \end{itemize}
    \item VSEPR theory is best used for a first approximation and is mostly accurate for most small molecules.
    \item \textbf{Trigonal planar geometry}: three electron pairs forming three bond angles of \ang{120} and lie on the same plan. 
    \item \textbf{Linear geometry}: two electron pairs that oppose each other at \ang{180}, forming a linear structure.
    \item General method of determining structure:
    \begin{itemize}
        \item[1.] Count steric number---the total number of electron pairs in a molecule. Can be bonds or lone pairs.
        \item[2.] Determine predicted geomterical structure predicted (EDG) by VSEPR using steric number.
            \begin{itemize}
                \item Octahedral:6, Bipyramid:5, Tetrahedral:4, Trigonal:3, Linear:2
            \end{itemize}
        \item[3.] Determin impact (the MG) of lone pairs; more lone pairs results in less space between bonded pairs. Shape depends on EDG.
    \end{itemize} 
\end{itemize}

\subsection{Dipole Moments and Molecular Polarity}
\begin{itemize}
    \item \textbf{Dipole moment} ($\bm{\mu}$): defined as the amount of partial charge, $\delta$, on on either end of the dipole multiplied by the distance separtion, d:
        \begin{itemize}
            \item {\color{o-Sun}\(\mu=\delta d\)}
            \item $\mu$ generally has an order of magnitude of {\color{o-Sun}\SI{e-18}{esu.cm}} due to general partial charge (esu) and distance (cm) values.
            \item 1 {\color{o-Sun}debye (D)} = \SI{e-18}{esu.cm} 
        \end{itemize}
    \item \textbf{Molecular dipole moment}: the vector sum of the individual dipole moments.
        \begin{itemize}
            \item Lone pairs have significant effect on the molecular dipole moment.
            \item Also called the net dipole moment.
        \end{itemize}
\end{itemize}

\subsection{Intermolecular Forces and Physical Properties}
\begin{itemize}
    \item \textbf{Intermolecular forces}: the attractive forces between individual molecules that determed the physical properties of a compound.
    \item \textit{Electrostatic}: forces that occur as a result of the attraction between opposite charges. 
    \item Electrostatic interactions for neutral molecules (no formal charge) are often classified as into the following categories:
        \begin{itemize} 
            \item \textbf{Dipole-dipole interaction}: Compounds with net dipole moments. 
                \begin{itemize}
                    \item In {\color{o-Sun}solid} space these intereactions either {\color{o-Sun}repel or attract} each other.
                    \item In {\color{o-Sun}liquid} space these interactions tend to {\color{o-Sun}attract more often}, raising melting/boiling point.
                \end{itemize}
            \item \textbf{Hydrogen bonding}:
                \begin{itemize}
                    \item Not actually a bond, just an interaction.
                    \item When hydrogen bonds to a electronegative atom, then the hydrogen will have a {\color{pos}$\delta^+$}.
                    \item \textbf{F, O, N, Cl} (Br, I). Most electronegative elements, from left to right, that hydrogen most often bonds too.
                    \item Hydrogen bonding is strong due to size of hydrogen atom, resulting in very close partial charge interactions.
                    \item The {\color{o-Sun}more} hydrogen bonds, the {\color{o-Sun}higher} the boiling point tends to be.
                \end{itemize}
            \item \textbf{Fleeting dipole-dipole interactions}:
                \begin{itemize}
                    \item Electrons are considered to be in constant motion, which restult in the center of negative charge to vary.
                    \item On average, the dipole moment is zero, though can experience transient dipole moments, initiating fleeting attraction/repulsion.
                    \item Heavier hydrocarbons generally experience a stronger force due to increased surface area, and thus greater chance for non-zero dipole moments, which results in higher boling points.
                    \item Branched hydrocarbons generally have decreased surface area, decreasing boiling point relative to others of similar weight.
                \end{itemize}
        \end{itemize}
    \item When comparing boling points of compounds, look for following factors:
        \begin{itemize}
            \item Any dipole-dipole interactions?
            \item Formation of hydrogen bonds?
            \item Number of carbon atoms. (surface area)
            \item Degree of branching of compound. (surface area)
        \end{itemize}
\end{itemize}

\subsection{Structural Theory of Matter}
\begin{itemize}
    \item \textbf{Constitutional isomers}: aka structural isomers; same {\color{o-Sun}chemical formula}, but different in the way the {\color{o-Sun}atoms are connect}, i.e. their constitution is different.
        \begin{itemize}
            \item Consistenet with the octet rule.
            \item Each element forms a predictable number of bonds, from one to four.
            \item The number of {\color{pos}possbile constitutional isomers increases} as the number of {\color{pos}carbon atoms increases}
        \end{itemize}
    \item \textbf{Stereoisomers}: isomers that differ in {\color{o-Sun}spatial arrangement} of atoms, rather than connectivity.
        \begin{itemize}
            \item \textbf{Geometric isomerism}: aka cis--trans; {\color{o-Sun}locked into spatial positions} due to double bonds or a ring structure.
                \begin{itemize}
                    \item Cis indicates functional groups that are on the same side of the carbon chain.
                    \item Trans indicates functional groups on opposite sides of the carbon chain.
                \end{itemize}
            \item \textbf{Enantiomers}: aka optical isomers; mirror images of each other that are non-superposable.
                \begin{itemize}
                    \item Human hands are a macroscopic analogy.
                \end{itemize}
        \end{itemize}
    \item \textit{More detail will be covered in later sections.}
\end{itemize}
%\endgroup
%%%%%%%%%%%%%%%%%%%%%%%%%%%%% Chapter 1 %%%%%%%%%%%%%%%%%%%%%%%%%%%%%


%%%%%%%%%%%%%%%%%%%%%%%%%%%%% Chapter 2 %%%%%%%%%%%%%%%%%%%%%%%%%%%%%
%\begingroup
\clearpage
\section{Molecular Representations}\phantomsection
\subsection{Molecular Representations}
\begin{itemize}
    \item ({\color{o-Sun}CH\(_{3}\)})\(_{3}\){\color{true}CC}H\(_{2}\){\color{true}C}H({\color{o-Sun}CH\(_{3}\)}){\color{true}C}H({\color{o-Sun}CH\(_{3}\)}){\color{true}C}H({\color{o-Sun}CH\(_{3}\)})\(_{2}\)
    \item $\chemfig{{\color{true}C}(-[4]{\color{o-Sun}CH_3})(-[6]{\color{o-Sun}CH_3})(-[2]{\color{o-Sun}CH_3})
            (-[0]{\color{true}C}(-[6]H)(-[2]H)
            (-[0]{\color{true}C}(-[6]{\color{o-Sun}CH_3})(-[2]H)
            (-[0]{\color{true}C}(-[6]{\color{o-Sun}CH_3})(-[2]H)
            (-[0]{\color{true}C}(-[6]{\color{o-Sun}CH_3})(-[2]H)(-[0]{\color{o-Sun}CH_3})
            ))))}$
\end{itemize}

\subsection{Bond-Line Structures}
\begin{itemize}
    \item 
\end{itemize}
%\endgroup
%%%%%%%%%%%%%%%%%%%%%%%%%%%%% Chapter 2 %%%%%%%%%%%%%%%%%%%%%%%%%%%%%
%\endgroup
\end{document}