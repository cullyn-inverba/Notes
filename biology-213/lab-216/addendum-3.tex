\documentclass[12pt,letterpaper]{article}
\usepackage{inverba}
\newcommand{\userName}{Cullyn Newman} 
\newcommand{\class}{BI 216} 
\newcommand{\institution}{Portland State University} 
\newcommand{\thetitle}{\hypertarget{home}{Lab 3 Addendum: Water Quality}}
\rfoot{\hyperlink{home}{\thepage}}

\begin{document}

\section*{Part A: Smithsonian Ocean}
\begin{enumerate}[font=\bfseries, wide]
    \item What does the figure just below the  “A More Acidic Ocean” figure illustrate? Please summarize the data in the figure (remember to account for all of the variables!) (2pts)
    \item What does “ocean acidification” mean? (1pt)
    \item How does the level of acidity in the ocean impact:
    \begin{enumerate}
        \item  A bivalve’s shell development? (0.5pts)
        \item  A coral’s structure? (0.5pts)
    \end{enumerate}
    \item Follow the links, and work to find the actual article on the research on black-finned clownfish which was described in the In the Lab section
    \begin{enumerate}
        \item What is the name of the journal in which it was published, and what was the (print) publication date and year (0.25pts)?
        \item The study was novel because it showed that fish have the ability to respond to predator sounds when played underwater:  TRUE or FALSE (circle one, 0.25pts)
    \end{enumerate}
\end{enumerate}

\section*{Part B: Measuring Salinity in Estuaries}
\subsection*{Level 3: Measuring Salinity in Estuaries}
\begin{enumerate}[font=\bfseries, wide, resume]
    \item Question \#5 from the activity: Which statement represents a valid conclusion based on the graph?  Enter the correct letter and the statement (0.5pts)\par

    C. A rainstorm on Oct 25 may have caused the decrease in salinity on Oct 27

    \item What may have caused Delta Smelt to be found outside of their normal range? (0.5pts)\par 

    There was a signicant amount of rainfall during the times the salinity was higher than 2, and salinity also spiked when ever rainfail increased. 
\end{enumerate}

\subsection*{Level 4 - Research Question: Predicting the Return of the Atlantic Sturgeon}
\begin{enumerate}[font=\bfseries, wide, resume]
    \item To get started, use the online Fact Sheet to select an estuary where Atlantic Sturgeon are found. Record the estuary name and location here:  (0.5pts) \par 

    The locaiton we chose was Chesapeake Bay, MD. 

    \item Write your research question in the space below. (1pt)\par 

    How does dissolved oxygen and temperature change over 

    \item Complete the table (1pt)\par
    \begin{table}[h]
        \centering
        \caption[]{Caption}
        \begin{tabular}{m{3.5cm}m{3cm}m{4cm}m{5cm}}
            \toprule
            Location & Water Quality \par Parameter & Range of dates & Notes \\
            \midrule
            Otter Point Creek & Temp, Salinity & & \\
             & & & \\
             & & & \\
             & & & \\
             & & & \\
             & & & \\
            \bottomrule
            \end{tabular}
    \end{table}
    \item Can you identify a time period when the water temperature is within the range for the sturgeon to return? (0.5pts)
    \item What is the range of the other water quality parameters during that time period? (0.5pts)
    \item Can you identify a time period when all the conditions look right for the sturgeon to return to spawn? (0.5pts)
    \item Do the same conditions occur around the same time, year after year? (0.5pts)
\end{enumerate}

\subsection*{Level 5: Work as a team to develop your own investigation}
\begin{enumerate}[font=\bfseries, wide, resume]
    \item Read through Level 5 on your own, and then work with your team to develop your research question. State your research question here: (1 pt)
    \item State your hypothesis: (1 pt)
    \item \textit{Make a Plan:} Make a lis tbelow of the specific data you will need to answer the question (1 pt) \par
    \begin{table}[h]
        \centering
        \caption[]{Caption}
        \begin{tabular}{m{3.5cm}m{3cm}m{4cm}m{5cm}}
            \toprule
            Location & Water Quality \par Parameter & Range of dates & Notes \\
            \midrule
             & & & \\
             & & & \\
             & & & \\
             & & & \\
            \bottomrule
            \end{tabular}
    \end{table}
    \item Other than the data listed above, what other information (if any) will you need to answer your question? (1 pt)
    \item Insert figure here (1 pt)
    \item \textit{Interpret the data:} What does your data show? Be specific and descriptive. Does the data suppport your hypothesis? (1 pt)
    \item \textit{Draw a Conclusion:} What is the answer to your question? Use evidence and data to support your conclusion. (1 pt)
    \item Give a specific example of why it would be biologically relevant to measure \textbf{temperature} in an aqautic envrionment. (1 point)
    \item Give a specific example of why it would be biologically relevant to measure \textbf{dissolved oxygen} in an aqautic envrionment. (1 point)
    \item Give a specific example of why it would be biologically relevant to measure \textbf{carbon dioxide} in an aqautic envrionment. (1 point)
\end{enumerate}
\end{document}