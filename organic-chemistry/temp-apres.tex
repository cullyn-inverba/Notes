\documentclass[12pt,a4paper]{article}
\usepackage{inverba}

\newcommand{\userName}{Cullyn Newman} 
\newcommand{\class}{[Subject]} 
\newcommand{\institution}{[Institution]} 
\newcommand{\theTitle}{\color{B-Cold} [Subject Title]}

\begin{document}
%%%%%%%%%%%%%%%%%%%%%%%%%%%%%%%%%%%%%%%%%%%%%%%%%%%%%%%%%%%%%%%%%%%%%
\tableofcontents 
\cleardoublepage{}
\fancyhead{}
\fancyhead[R]{\hyperlink{home}{\nouppercase\leftmark}}
%%%%%%%%%%%%%%%%%%%%%%%%%%%%%%%%%%%%%%%%%%%%%%%%%%%%%%%%%%%%%%%%%%%%%
\clearpage
\section*{Week 10}\phantomsection{}
\addcontentsline{toc}{section}{\textbf{Week 10}}
\fancyhead[R]{\hyperlink{home}{Week 10}}

\fancyhead[L]{\hyperlink{home}{Friday, November 30 - Quiz 21}}
\subsection{Friday, November 30 - Quiz 21}
\begin{enumerate}
    {\color{G-Moon}\item Which of the following is the most stable carbocation?
        \begin{itemize}
            \item \chemfig{-[:30](-[:90](-[6.3,0.9,,,draw=none]\scriptstyle\oplus))(-[:-90])-[:-30]-[:30]}
            \item \chemfig{-[:30](-[:90])(-[:-90])-[:-30]\chembelow[1ex]{}{\scriptstyle\oplus}-[:30]}
            \item {\color{o-Sun}\chemfig{-[:30](-[:90]\chembelow[6ex]{}{\scriptstyle\oplus})-[:-30](-[:-90])-[:30]}}
            \item \chemfig{-[:30](-[:90])(-[:-90])-[:-30]-[:30]{\scriptstyle\oplus}}
        \end{itemize}
    \item What mechanism is most likely for the following?
    
    \schemestart
    \chemfig{-[:30](-[:90])(-[:-90])-[:-30]-[:30]\ch{Br}}
    \hspace*{12pt}
    \+
    \hspace*{12pt}
    \chemfig{-(-[:90])(-[:-90])-O^{\ominus}}
    \arrow{->[][heat]}
    \schemestop
        \begin{itemize}
            \item \(S_N1\)
            \item \(S_N2\)
            \item E1
            \item {\color{o-Sun}E2}
        \end{itemize}
    \item Which mechanism is most likely for the following?
    
    \schemestart
    \chemfig{-[:30]-[:-30]-[:30]\ch{Br}}
    \hspace*{12pt}
    \+
    \hspace*{12pt}
    \chemfig{\ch{CH3O-}}
    \hspace*{6pt}
    \arrow{->}
    \schemestop
        \begin{itemize}
            \item \(S_N1\)
            \item {\color{o-Sun}\(S_N2\)}
            \item E1
            \item E2
        \end{itemize}
    \item What will the major product be from the following?
    
    \schemestart
    {\small\chemfig{-[:30]-[:-30](-[:-90]\ch{Br})-[:30]}}
    \hspace*{12pt}
    \+
    \hspace*{12pt}
    \chemfig{\ch{CH3O-}}
    \hspace*{6pt}
    \arrow{->[][heat]}
    \schemestop
        \begin{itemize}\small
            \item \chemfig{=[:-30](-[:-90])-[:30]}
            \item {\color{o-Sun}\chemfig{-[:30]=[:-30]-[:30]}}
            \item \chemfig{-[:30]-[:-30]=[:30]}
        \end{itemize}
    \item What is the major product from the following?
    
    \schemestart
    {\small\chemfig{-[:30](-[:-90]\ch{Br})(-[:90])-[:-30]-[:30]}}
    \hspace*{12pt}
    \+
    \hspace*{12pt}
    \chemfig{\ch{CH3OH}}
    \hspace*{6pt}
    \arrow{->[][heat]}
    \schemestop
        \begin{itemize}\small
            \item  {\color{o-Sun}\chemfig{-[:30](-[:90])=[:-30]-[:30]}}
            \item  \chemfig{=[:30](-[:90])-[:-30]-[:30]}
            \item  \chemfig{-[:30](-[:90])-[:-30]=[:30]}
        \end{itemize}
    \item What is the major product from the following?
    
    \schemestart
    {\small\chemfig{-[:30](-[:-90])(-[:90])-[:-30]-[:30](-[:90]\ch{Br})-[:-30]}}
    \hspace*{12pt}
    \+
    \hspace*{12pt}
    \chemfig{\ch{CH3OH}}
    \hspace*{6pt}
    \arrow{->[][heat]}
    \schemestop
        \begin{itemize}\small
            \small
            \item \chemfig{-[:30](-[:-90])(-[:90])-[:-30]=[:30]-[:-30]}
            \item {\color{o-Sun}\chemfig{-[:30](-[:90])=[:-30](-[:-90])-[:30]-[:-30]}}
            \item \chemfig{-[:30](-[:-90])(-[:90])-[:-30]-[:30]=[:-30]}
            \item \chemfig{-[:30](-[:-90])(-[:90])=[:-30]-[:30]-[:-30]}
        \end{itemize}
    \item What is the major product from the following?
    
    \schemestart
    {\small\chemfig{-[:-30](-[:-90])-[:30](-[:90])(-[:-90]\ch{Br})-[:-30]}}
    \hspace*{12pt}
    \+
    \hspace*{12pt}
    \chemfig{-[:30]-(-[:90]-[:15])(-[:-90]-[:-15])-O^{\ominus}}
    \hspace*{6pt}
    \arrow{->[][heat]}
    \schemestop
        \begin{itemize}\small
            \item {\color{o-Sun}\chemfig{-[:-30](-[:-90])-[:30](-[:90])=[:-30]}}
            \item \chemfig{=[:-30](-[:-90])-[:30](-[:90])-[:-30]}
            \item \chemfig{-[:-30](-[:-90])=[:30](-[:90])-[:-30]}
            \item \chemfig{-[:30]=[:-30]-[:30](-[:90])-[:-30]}
        \end{itemize}
    \item What mechanism is most likely under the following conditions? 

    \ang{3} alkyl halide; bulky weak base; polar protic solvent; heat
        \begin{itemize}
            \item \(S_N1\)
            \item {\color{o-Sun}E1}
            \item E2
            \item \(S_N2\)
        \end{itemize}
    \newpage
    \item What mechanism is most likely under the following conditions? 

    \ang{3} alkyl halide; small unhindered weak base; polar protic solvent; room temperature
        \begin{itemize}
            \item E2
            \item E1
            \item \(S_N2\)
            \item {\color{o-Sun}\(S_N1\)}
        \end{itemize}
    \item What mechanism is most likely under the following conditions? 

    \ang{3} alkyl halide; strong base; polar solvent; high temperature
    
        \begin{itemize}
            \item E1
            \item \(S_N1\)
            \item {\color{o-Sun}E2}
            \item \(S_N2\)
        \end{itemize}
    \item What mechanism is most likely under the following conditions? 

    \ang{1} alkyl halide; small unhindered nucleophile; polar solvent; room temperature
        \begin{itemize}
            \item \(S_N1\)
            \item {\color{o-Sun}\(S_N2\)}
            \item E2
            \item E1
        \end{itemize}
    }
\end{enumerate}
\end{document}