\documentclass[12pt,a4paper]{article}
\usepackage{inverba}

\newcommand{\userName}{Cullyn Newman} 
\cfoot{\hyperlink{home}{\color{B-Cold}\thepage}} 
\lfoot{}
\rfoot{}
\newcommand{\theTitle}{\color{B-Cold}Principles of Neural Science}

\begin{document}
%%%%%%%%%%%%%%%%%%%%%%%%%%%%%%%%%%%%%%%%%%%%%%%%%%%%%%%%%%%%%%%%%%%%%
\tableofcontents
\cleardoublepage
\fancyhead{}
\fancyhead[R]{\hyperlink{home}{\nouppercase\leftmark}}
%%%%%%%%%%%%%%%%%%%%%%%%%%%%%%%%%%%%%%%%%%%%%%%%%%%%%%%%%%%%%%%%%%%%%

\fancyhead[L]{Part I Overall Perspective}
%\begingroup
%%%%%%%%%%%%%%%%%%%%%%%%%%%%% Chapter 1 %%%%%%%%%%%%%%%%%%%%%%%%%%%%%
%\begingroup
\clearpage
\section{The Brain and Behavior}\phantomsection


\subsection{The Brain Has Distinct Functional Regions}
\begin{itemize}
    \item \textbf{Spinal cord}: most caudal part of the central nervous system. It is subdivided into cervical, thoracic, lumbar, and sacral regions.
    \item \textbf{Brain stem}: consists of the medulla oblongata, pons, and midrain. Relays input from the spinal cord and back, and controls input to and from the head.
    \item \textbf{Medulla oblongata}: rostral to spinal cord and includes several centers responsible for vital autonomic functions. 
    \item \textbf{Pons}: rostral to medulla and conveys information about movement.
    \item \textbf{Cerebellum}: lies behind pons, modulates force and range of movement, and involved in learning motor skills.
    \item \textbf{Diencephalon}: lies rostral to midrain and contains two structures, thalamus (processes information reaching cerebral cortex) and hypthalamus (regulates autonomic, endocrine, and visceral functions).
    \item \textbf{Cerebrum}: comprises two cerebral hemispheres, each consisting of wrinkled outer layer (the cerebral cortex), and three deep lying structures (basal ganglia, the hippocampus, and the amygdaloid nuclei).
    \item \textbf{Cerebral cortex}: divided into four distinct lobes--- frontal, parietal, occipital, and temporal. The frontal lobe is largely concerned with short-term memory and planning, as well as movement; the parietal lobe with somatic sensation, forming a body image, and relating it to extrapersonal space; the occipital lobe with vision; and the temporal lobe with hearing---combined with deeper structures---with learing, memory, and emotion.
\end{itemize}
%\endgroup
%%%%%%%%%%%%%%%%%%%%%%%%%%%%% Chapter 1 %%%%%%%%%%%%%%%%%%%%%%%%%%%%%

%%%%%%%%%%%%%%%%%%%%%%%%%%%%% Chapter 2 %%%%%%%%%%%%%%%%%%%%%%%%%%%%%
%\begingroup
\clearpage
\section{Nerve Cells, Neural Circuitry, and Behavior}\phantomsection
\subsection{The Nervous System Has Two Classes of Cells}
\begin{itemize}
\item There are two main classes of cells in the nervous system: nerve cells, or neurons, and glial cells, or glia.
    \item A neuron has four defined regions:
        \begin{itemize}
            \item \textbf{Cell body}: or \textbf{soma}, is the metabolic center of the cell, containing normal cell organelles.
            \item \textbf{Dendrites}: branch out in tree-like fashion and are main apparatue for receiving signals.
            \item \textbf{Axon}: extends some distance from a cell and carries signals to other neurons.
            \item \textbf{Presynaptic terminals}: specialized enlarged regions of it's axon's branches and is responsible for tranfer of signals.
        \end{itemize}
    \item \textbf{Principle of dynamic polarization}: electrical signals only forlow in one direction in neurons.
    \item \textbf{Connectional specificity}: nerve cells do not connect randomly with one another in these formation of networks.
    \item Neurons are classified into three groups:
        \begin{itemize}
            \item \textbf{Unipolar}: simpiliest due to single primary process, which gives rise to many branches. One branch as axon and others as receiving structures. These cells predominate invertebrates; they orccur in the autonomic nervous system in vertebrates. 
            \item \textbf{Bipolar}: oval soma that gives rise to two processes: a dendritic structure that receives signals and an axon that carries information towards the central nervous system. Many sensory cells are bipolar, and pain receptors are pseudo-unipolar. 
            \item \textbf{Multipolar}: predominate nervous system of vertebrates and vary greatly in shape; typically containing a single neuron and many dendritic points emerging from various points around the cell body.
        \end{itemize}
    \item Glial cells support nerve cells and greatly outnumber neurons. 
    \item Glial cells surround the cell bodies, axons, and neurons and can be divided into two major classes:
        \begin{itemize}
            \item \textbf{Microglia}: immune system cells that become phagocytes during injury, infection, or degenerative diseases. 
            \item There are three main types of \textbf{macroglia}: oligodendrocytes, Schwann cells, and astrocytes. About 80\% of all brain cells are macrogalia.
        \end{itemize}
\end{itemize}
%\endgroup
%%%%%%%%%%%%%%%%%%%%%%%%%%%%% Chapter 2 %%%%%%%%%%%%%%%%%%%%%%%%%%%%%
%\endgroup

\clearpage
\fancyhead[L]{Part II Molecular Biology of the Neuron}
%\begingroup
%%%%%%%%%%%%%%%%%%%%%%%%%%%%% Chapter 5 %%%%%%%%%%%%%%%%%%%%%%%%%%%%%
%\begingroup
\clearpage
\setcounter{section}{4}
\section{Ion Channels}\phantomsection
\subsection{Rapid Signaling in the Nervous System Depends on Ion Channels}
\begin{itemize}
    \item Up to 100 million ions can pass through a single channel each second, comaprable to the turnover rate of the fastest enzymes, catalase and carbonic anhydrase.
    \item Each channel allows only one or a few types on ions to pass.
    \item Many open and close, however, some remain open resulting in significant contribution to resting potential. 
    \item Ions pumps maintained gradients and are 100 to 100,000 times slower than channels.
    \item Questions for this chapter:
        \begin{itemize}
            \item Why do nerve cells have channels?
            \item How can channels conduct ions at such high rates adn still be selective?
            \item How are channels gated?
            \item How are properties of theses channels modified by various intrinsic and extrinsic conditions?
        \end{itemize}
\end{itemize}

\subsection{Ion Channels are Proteins That Span the Cell Membrane}
\begin{itemize}
    \item Cells have channels in order the transport ions across lipid bilayer easily and eliminate the need to be stripped of waters of hydration.
    \item The smaller the ion, the greater attraction to water, and the lower its mobility. This partially explains selection, but does how does the inverse selection, that selecting of lower mobility, occur?
    \item Some ions bind to proteins that can transport them, but this is far to slow for some cases.
    \item An extension of pore theory says that channels have narrow regions that act as molecular sieves, where the ion sheds most of it's water and only is let through by a binding to a specifically charged selectivity filter. 
\end{itemize}

\subsection{Ion Channels in ALl Cells Share Several Characteristics}
\begin{itemize}
    \item The opening and closing of a channel invole conformational changes.
    \item \textbf{Gating}: the transition of a channel between theses stable functional states.
    \item Three major gating mechanisms:
        \begin{itemize}
            \item Ligand: binding of chemical ligands known as agonists at either cellular site; transmitters on the extracelluar; others that activate signaling cascades; and more. 
            \item Voltage-gated: changes in electrochemical changes as often as temperature sensors.
            \item Mechanical stretch or physical changes in the membrane.
        \end{itemize}
\end{itemize}
%\endgroup
%%%%%%%%%%%%%%%%%%%%%%%%%%%%% Chapter 5 %%%%%%%%%%%%%%%%%%%%%%%%%%%%%
%\endgroup

\clearpage
\fancyhead[L]{Part III Synaptic Transimission}
%\begingroup
%%%%%%%%%%%%%%%%%%%%%%%%%%%%% Chapter 8 %%%%%%%%%%%%%%%%%%%%%%%%%%%%%
%\begingroup
\clearpage
\setcounter{section}{7}
\section{Overview of Synaptic Transmission}\phantomsection
\subsection{Synapses Are Either Electrical or Chemical}
\begin{itemize}
    \item Average neuron forms and receive several thousand synaptic connections each, with the Purkinje cell of the cerebellum receiving up to 100,000.
    \item Both forms of transmission can be enhanced or diminished by cellular activity.
    \item Electrial synapses are used to send rapid stereotyped depolarizing signals.
    \item Chemical synapses are capable of more complex behaviors due to vairable signaling. 
    \item Most synapses are chemical.
\end{itemize}
\subsection{Electrical Synapses Provide Instantaneous Signal Transmission}
\begin{itemize}
    \item Presynaptic terminals must be big enough for its membrane to contain many ion channels to trigger initial depolarization.
    \item Postsynaptic terminals must be relatively to small in due to Ohm's law.
    \item Even weak subthreshold depolarizing currents can be carried to the postsynaptic neuron and depolarize it.
    \item Electrical synapses have a specialized region of contact called the gap junction, with seperation of only 4 nm, bridged by gap-junction channels specialized to conduct ionic current.
    \item Electrical transmission can be used to orchestrate actions or large groups of neurons.
    \item Groups of electrically coupled cells allows for explosive reactions.
    \item Gap junctions are formed between glial cells as well as neurons.
\end{itemize}
\subsection{Chemical Synapses Can Amplify Signals}
\begin{itemize}
    \item Chemical synapses are used to amplify or inhibit signals. 
    \item The synaptic cleft is 20-40 nm wide and depend on the diffusion of neurotransmitters to carry out signaling.
    \item Neurotransmitters are clustered at specialized. regions called \textbf{active zones}, which allow for selective activate of nearby postsynaptic receptors, which lead to the opening or closing of ion channels.
    \item Chemical synapses can be as short as 0.3 ms but often last several ms.
    \item Weak activations of chemical synapses can activate larger electrial synapses. 
    \item The action of a transmitter depends on the properties of the postsynaptic receptor, not the chemical properties of itself.
    \item Neurotransmitters control the opening of ion channels in the postsynaptic cell either directly or indirectly.
    \item Indirect effects tend to last seconds to minutes and often modulate behavior due to alterations in the excitability of neurons and their synaptic connections.
\end{itemize}
%\endgroup
%%%%%%%%%%%%%%%%%%%%%%%%%%%%% Chapter 8 %%%%%%%%%%%%%%%%%%%%%%%%%%%%%

%%%%%%%%%%%%%%%%%%%%%%%%%%%%% Chapter 10 %%%%%%%%%%%%%%%%%%%%%%%%%%%%%
%\begingroup
\setcounter{section}{9}
\clearpage
\section{Synaptic Integration in CNS}\phantomsection
\subsection{Central Neurons Receive Excitatory and Inhibitory Inputs}
\begin{itemize}
    \item Generation of an action potential often requires the near-synchronous firing of a number of sensor neurons.
    \item Small inhibitory postsynaptic potential (ISPS), if strong enough, can counteract the sum of the excitatory actions and prevent membrane potential from reaching threshold potential.
    \item \textbf{Sculpting} function of synaptic inhibition that exerts control over action potentials in neurons that are spontaneously active due to intrinsic pacemaker channels, often completely shaping the firing patterns of cells.
    \item Most transmitters usually are inhibitory or excitatory despite being able to be either type. 
\end{itemize}

\subsection{Inhibitory Synaptic Action}
\begin{itemize}
    \item Inhibitory synapses play an essential role in the nervous system both by preventing too much excitation and by helping coordinate activity among networks of neurons. 
    \item Inhibitory inputs that hyperpolarize the cell perform subtraction on the excitatory inputs, where those that \textbf{shunt} perform division. 
    \item Adding or removing nonshunting inhibitory inputs results in summation, while the combination or excitatory with the removal of inhibitory shunt produces a multiplication.
\end{itemize}

%\endgroup
%%%%%%%%%%%%%%%%%%%%%%%%%%%%% Chapter 10 %%%%%%%%%%%%%%%%%%%%%%%%%%%%%

%%%%%%%%%%%%%%%%%%%%%%%%%%%%% Chapter 13 %%%%%%%%%%%%%%%%%%%%%%%%%%%%%
%\begingroup
\clearpage
\setcounter{section}{12}
\section{Neurotransmitters}\phantomsection
\subsection{Four Criteria of a Neurotransmitter}   
\begin{itemize}
    \item Four steps of synaptic transmission: 
        \begin{itemize}
            \item[1.] Synthesis and storage of a transmitter.
            \item[2.] Release of the transmitter.
            \item[3.] Interaction of the transmitter with receptors and postsynaptic membrane.
            \item[4.] removal of the tranmitter from the synaptic cleft.
        \end{itemize}
    \item First approximation of a neurotransmitter can be defined as a substance released by a neuron that affects a specific target in a specific manner.
    \item Neurotransmitters typically act on targets other thanthe releasing neuron itself, unlike autacoids.
    \item Neurotransmitter interaction with receptors is typically transient, lasting from milliseconds to minutes.
    \item General criteria for neurotransmitters:
        \begin{itemize}
            \item It is synthesized in the presynaptic neuron.
            \item It is present in the presynaptic terminal and is released in amounts sufficient to exert a defined action on the postsynaptic neuron or effector organ.
            \item When administered exogenously in a reasonable concentrations it mimics the action of the endogenous transmitter.
            \item A specific mechanism usually exists for removing the substance from the synaptic cleft.
        \end{itemize}
\end{itemize}

\subsection{Only a Few Small-Molecule Substances Act as Transmitters}
\begin{itemize}
    \item \textbf{Acetylcholine (ACh)}: 
        \begin{itemize}
            \item Only low weight amine transmitter substance that is not an amino acid or derived directly from one.
            \item Nervous tissue cannot synthesize choline, which limits ACh biosynthesis due to choline acetyltransferase being the only enzymatic reaction. 
            \item ACh is released by spinal motor neurons. 
            \item In the autacoids nervous system it is the transmitter for all preganglionic neurons and for parasympathetic postganglionic neurons as well.
            \item ACh is the principle neurotransmitter of the reticular activating system, which modulates arousal, sleep, wakefulness, and other critical aspects of human consciousness.
        \end{itemize}
    \item \textbf{Biogenic Amines}:
            \begin{itemize}
                \item \textbf{Catecholamine Transmitters}:
                    \begin{itemize}
                        \item \textbf{Dopamine} --- Tyrosine
                        \item \textbf{Norepinephrine} --- Tyrosine
                        \item \textbf{Epinephrine} --- Tyrosine
                        \item Tyrosine hydroxylase is the rate-limiting for synthesis of bothdopamine and norepinephrine.
                        \item $\beta$-hydroxylase converts dopamine to norepinephrine and is membrane-associated.
                        \item Norepinephrine is the only transmitter synthesized within vesicles.
                        \item In order for epinephrine to be formed, then its immediate precursor, norepinephrine, must exit from vesicles into the cytoplasm.
                        \item In order to be released, epinephrine must be taken up into vesicles.
                        \item Three of four dopaminergic nerve tracts arise in the midrain, with the last arising in the arcuate nucleus of the hypothalamus.
                        \item Synthesis of biogenic amines  is highly regulated and ca be rapidly increased.
                    \end{itemize}
                \item \textbf{Serotonin} --- Tryptophan
                \item \textbf{Melatonin} --- Serotonin
                    \begin{itemize}
                        \item Typtophan hydroxylase is the limiting reaction and the first enzyme in the pathway.
                        \item Cell bodies with serotonergic neurons are found around the midline raphe nuclei of the brain stem and are involved in regulating attention.
                        \item Productions of serotonergic cells are widely distrubted throughout the brain and spinal cord.
                        \item Antidepressant mediacation inhibit the uptake of serotonin, norepinephrine, and dopamine. 
                    \end{itemize}
                \item \textbf{Histamine} --- Histidine
                    \begin{itemize}
                        \item Long been recognized as a autacoid, active when released from mast calls in the inflammatory reaction.
                        \item Concentrated in the hypothalamus.
                    \end{itemize}
            \end{itemize}
        \item \textbf{Amino Acid Transmitters}:
            \begin{itemize}
                \item \textbf{Apartate} --- Oxaloacetate
                \item \textbf{$\gamma$-Aminobutyric acid (GABA)} --- Glutamine:
                    \begin{itemize}
                        \item Presnet at high concentrations thoughout the central nervous system and detectable in other tissues.
                    \end{itemize}
                \item \textbf{Glutamate} --- Glutamine:
                \begin{itemize}
                    \item Most frequently used at excitatory synapses throughout the central nervous system.
                \end{itemize}
                \item \textbf{Glycine} --- Serine:
                    \begin{itemize}
                        \item Major transmitter used by inhibitory interneurons of the spinal cord.
                    \end{itemize}
            \end{itemize}
        \item \textbf{ATP and Adenosine}
            \begin{itemize}
                \item Can act as transmitters at some synapses.
                \item Adenosine has an inhibitory effect in the central nervous system.
                \item Caffeine's stimulatory effect depends on inhibition of adenosine binding to its receptors.
                \item ATP released by tissue damage acts to transmit pain sensation in some cases.
            \end{itemize}
\end{itemize}

\subsection{Small-Molecule Transmitters Are Actively Taken up into Vesicles}
\begin{itemize}
    \item \textbf{Tranmitter} glutamate must be kept separate from \textbf{metabolic} glutamate; this is done through compartmentalization in synaptic vesicles.
    \item Drugs that are sufficiently similar to the normal transmitter substance can act as false transmitters, tho they often bind weakly decreasing efficacy of of transmission.
\end{itemize}

\subsection{Removal of Transmitter from the Synaptic Cleft Terminates Transimission}
\begin{itemize}
    \item If transmitter molecules released in one synaptic action were allowed to remain in the cleft after release, then they would prevent ner signlas from getting through, the synapse would become refractory due to desensitization.
    \item Transmitters are removed by three mechanisms: diffusion, enzymatic degradation, and reuptake.
    \item Degradation is only used by cholinergic synapses.
    \item Degradation allows for single use signaling and for the lingering choline to be reused.
\end{itemize}
%\endgroup
%%%%%%%%%%%%%%%%%%%%%%%%%%%%% Chapter 13 %%%%%%%%%%%%%%%%%%%%%%%%%%%%%

%\endgroup

\clearpage
\fancyhead[L]{Part IV Basis of Cognition}
%\begingroup
%%%%%%%%%%%%%%%%%%%%%%%%%%%%% Chapter 15 %%%%%%%%%%%%%%%%%%%%%%%%%%%%%
%\begingroup
\setcounter{section}{14}
\clearpage
\section{Organization of the Central Nervous System}\phantomsection
\subsection{The Central Nervous System Consists of the Spinal Cord and the Brain}
\begin{itemize}
    \item The spinal cord is divided into a core of central gray matter and surrounding white matter.
    \item The gray matter is divied into \textbf{dorsal} and \textbf{ventral} horns.
    \item \textbf{Dorsal horn}: contains orderly sensory relay neurons that receive input from periphery.
    \item \textbf{Ventral horn}: contains group of motor neurons and interneurons that regulate motor neural firing patterns.
    \item The brain stem(\textbf{Medulla, pons, and midbrain}) has five distinct functions:
        \begin{itemize}
            \item[1.] Spinal cord mediate sensation and motor control of trunk and limbs, but the brain stem control the head, neck and face.
            \item[2.] Site of entry for information from several specialized sites such as hearing, balance, and taste.
            \item[3.] Mediation of parasympathetic reflexes, such out cardiac output, pupil constriction, and more.
            \item[4.] Contains ascending and descending pathways that carry sensory and motor information to other parts of the CNS.
            \item[5.] Contains the \textbf{reticular formation}, which receives a summary of incoming sensory information and regulates alertness and arousal.    
        \end{itemize}
\end{itemize}
\subsection{The Major Functional Systems Are Similarly Organized}
\begin{itemize}
    \item The central nervous system consists of several functional systems that are relatively autonomous and much work together using numerous interconnected anatomical sites throughout the brain.
    \item Information is transformed at each synaptic relay, with the output rarely being the same as the input.
    \item Neurons at each synaptic relay are organized into a neural map of the body.
    \item Most sensory systems inputs are arranged topographically through out successive stages of processing.
    \item Each functional system is hierarchically organized.
    \item \textbf{Decussations}: crossing of second order fiber from the brain stem and the spinal cord.
\end{itemize}

\subsection{The Cerebral Cortex is Concerned with Cognition}
\begin{itemize}
    \item Increasing the surface area due to sulci and gyri allow for greater number of cortical neurons which provide a greater capacity for information processing.
    \item Neurons in the cerebral cortex are organized in layers and columns which helps computational efficiency.
    \item The neocortex receives inputs from the thalamus, other cortical regions on both sides of the brain, and other structures then output to other various regions.
    \item The input-output relation is organized into orderly layering of cortical neurons, with most containing six layers.
        \begin{itemize}
            \item Layer I: the \textbf{molecular layer}, is occupied by dendrites of cells located in deeper layers and axons that make connections to other areas of the cortex.
            \item Layer II: the \textbf{external granule cell layer}, one of two layers that contain small spherical neurons.
            \item Layer III: the \textbf{external pyramidal cell layer}, second layer of small spherical neurons, typically larger that layer II.
            \item Layer IV: the \textbf{internal granule cell layer}, contains much larger number of spherical neurons and is main recipient of sensory input from the thalamus.
            \item Layer V: the \textbf{internal pyramidal cell layer}, contains pyramidal neurons that are also larger than it's external layer. Theses neurons give rise to major output pathways of the cortex.
            \item Layer VI: the \textbf{multiform layer}, a blend neurons into white matter that forms the deep limit of the cortex and carries axons to and from areas of the cortex.
        \end{itemize}
    \item Thickness of the layers vary throughout the cortex.
    \item The cerebral cortex has a large variety of neurons, more than 40 different types based only on the distribution of their dentrites and axons.
    \item Most neurons are either principal (projection) neurons or local interneurons.
\end{itemize}

\subsection{Subcortical Regions of the Brain are Functionally Organized into Nuclei}
\begin{itemize}
    \item Three major structures lie deep within the cerebral hemisphere: the basal ganglia, the hippocampal foramtion, the amygdala, and the basal ganglia. These subcortical structures act to regulate the cortical activity. 
    \item Basal ganglia regulates movement and certain cognitive functions such as learning of motor skills.
    \item THe basal ganglia has five major functional subcomponents: the caudate nucleus, putamen, globus pallidus, subthalamic nucleus, and substantia nigra.
    \item The hippocampal formation includes the hippocampus, dentate gyrus, and subiculum. Together theses structures are responsible for the formation of long-term memories episodic memories, but not responsible for storage.
    \item The amygdala is involved in analying the emotional significance of sensory stimuli.
\end{itemize}

\subsection{Modulatory Systems Influence Motivation, Emotion, and Memory}
\begin{itemize}
    \item Some brain areas are neither sensory nor motor, but instead modify specific functions.
    \item Distinct modulatory systems within the brain stem modulate attention and arousal. 
\end{itemize}

\subsection{The Peripheral Nervous System is Anatomically Distinct from the Central Nervous System}
\begin{itemize}
    \item The peripheral nervous system supplies the central nervous system with a continuous stream of information about both externala and internal environments. It is split into two divisions.
    \item The \textbf{somotic division} includes the sensory neurons that receive information from skin, muscles, and joints and provide information about position and pressure. 
    \item The \textbf{autonomic division} mediates visceral sensation as well as motor control of the viscera, vascular, and exocrine glands. It consists of sympathetic (response to stress), parasympathetic (restores homeostasis), and enteric (controls smooth muscle of the gut) systems.
\end{itemize}
%\endgroup
%%%%%%%%%%%%%%%%%%%%%%%%%%%%% Chapter 15 %%%%%%%%%%%%%%%%%%%%%%%%%%%%%

%%%%%%%%%%%%%%%%%%%%%%%%%%%%% Chapter 16 %%%%%%%%%%%%%%%%%%%%%%%%%%%%%
%\begingroup
\clearpage
\section{Organization of Perception and Movement}\phantomsection
\subsection{Sensory Information Processing is Illustrated in the Somatosensory System}
\begin{itemize}
    \item Complex behaviors require the integrated action of several nuclei and cortical regions, processed in a hierarchical fashion, and becomes increasingly complex.
    \item Complex processing results in a light touch or painful prick in the skin being mediated by often very different pathways.
    \item Somatosensory information from the trunk and limbs is conveyed to the spinal cord.
    \item The spinal cord is divied into four major regions: cervical, thoracic, lumbar, and sacral.
    \item Spinal nevres at the cervical level are involved with sensory perceptions and motor function of the back of the head, neck, and arms.
    \item Thoracic nerves innervate the upper trunk.
    \item Lumbar and sacral nevres innervate the lower trunk, back, and legs.
    \item Each of the four regions of the spinal cord contains several segments, depsite the lack of appearance of segmentation of mature spinal cords.  
    \item The spinal cord varies in size and shape due to two organizational features.
    \item First, the relatively few sensory axons enter the cord at the sacral level, with number of entering axons increasing progressively at higher levels.
    \item Most descending axons from the brain terminate at cervical levels.
    \item Second, the variation in the size of the ventral and dorsal hons.
    \item The number of ventral motor neurons dedicated to the body region roughly parallels the dexterity of movements of that region.
    \item \textbf{lumbosaral} and \textbf{cervical enlargements}: regions of the spinal cord where fibers enter the cord due demends of sensory neurons for finer tactile discrimination in limbs.
    \item The primary sensory neurons of the trunk and limbs are clustered in the dorsal root ganglia. These neurons are pseudo-unipolar in shape and have bifurcated axon with central and peripheral branches.
    \item Local branches activate local reflex circuits while ascending branches carrying information to the brian that give rise the perception.
    \item The central axons of dorsal root ganglion neurons are arranged to produce a map of the body surface.
    \item Each somatic submodality if processed in a distinct subsystem from the periphery to the brain.
\end{itemize}

\subsection{The Thalamus is an Essential Link Between Sensory Receptors and the Cerebral Cortex for All Modalities Except Olfaction}
\begin{itemize}
    \item The thalamus conveys sensory input to the primary sensory areas of the cerebral cortex and additionally acts as a gatekeeper depending of behavioral state of the animal.
    \item The thalamus is a good example of a brain region made up of several well-defined nuclei.
    \item Some nuclei receive information specific to a sensory modality and projet to a specific area of the neocortex.
    \item The nuclei of the thalamus are most commonly classified into four groups:
        \begin{itemize}
            \item \textbf{anterior group}: recvies most input form the mammillary nuclei of the hypothalamus and presubiculum of the hippocampal formation. The role of this region is uncertain, but thought to be related to memory and emotion.
            \item \textbf{Medial group}: consists mostly of the mediodorsal nucleus. It receives input from the basal ganglia, amygdala, and midrain and is been implicated in memory.
            \item \textbf{ventral group}: important for motor control and carry information from basal ganglia and cerebellum to the motor cortex.
            \item \textbf{posterior group}: includes the medial and lateral geniculate nucleus (component of auditory system), lateral posterior nucleus (componenet of the retina and visual cortex in the occipital lobe), and the pulmonary(involved in the parietal-occipital-temporal cortex).
        \end{itemize}
    \item \textbf{Reticular nucleus}: a unique sheet-like structure covering the thalamus.
    \item Neurons ot the reticular nucleus are not interconnected with the neocortex, instead the axons terminate on the other nuclei of the thalamus.
    \item Thus, the reticular nucleus modulates activity in other thalamic nuclei based on its moitoring of the entirety of the thalamocortical stream.
    \item The thalamus not only relays information but is a crucial step and adds substantial degree of information processing.
\end{itemize}

\subsection{Sensory Information Processing Culminates in the Cerebral Cortex}
\begin{itemize}
    \item  Parts of the body are represented in the cortex somatotopically, but the area of the cortex is not proportional to it mass. Instad, it proportional t the density of innervation.
    \item The cortical areas involved in the early stages of sensory processing are concerned primarily with a single modality. 
    \item Unimodal association areas converge on multimodal association areas of the cortex concerned with combining sensory modalities.
    \item Multimodal associational areas are heavily interconnected with the hippocampus and appear to be important for unified percept and representation of the percept in memory.
    \item There is a close linkage between the somatosensory and motor functions of the cortex.
\end{itemize}

\subsection{Voluntary Movement is Mediated by Direct Connections Between the Cortex and Spinal Cord}
\begin{itemize}
    \item The human corticospinal tract consists of approximately one million axons, with 40\% originating from the motor cortex.
    \item Most of the corticospinal fibers cross the midline in the medulla at a location known as the pyramidal decussation.
    \item 10\% of those fibers do not cross until they reach the local where they terminate.
    \item THe motor information carried in the corticospinal tract is significantly modulated by the sensory information and information from other motor regions.
    \item The cerebellum is thought to be part of an error-correcting mechanism for movements because it can compare movement commands from the cortex with somatic sensory information about what actually happened.
\end{itemize}
%\endgroup
%%%%%%%%%%%%%%%%%%%%%%%%%%%%% Chapter 16 %%%%%%%%%%%%%%%%%%%%%%%%%%%%%

%%%%%%%%%%%%%%%%%%%%%%%%%%%%% Chapter 17 %%%%%%%%%%%%%%%%%%%%%%%%%%%%%
%\begingroup
\clearpage
\section{Representation of Space and Action}\phantomsection
\subsection{The Brain has an Orderly Representation of Personal Space}
\begin{itemize}
    \item Internal representation can be thought of as a certain pattern of neural activity that has at least two aspects:
        \begin{itemize}
            \item The pattern of activation within a particular population of neurons.
            \item the pattern of firing in individual cells.
        \end{itemize}
    \item The cortex has a map of the sensory receptive surface for each sensory modality.
    \item Cortical maps of the body are the basis of accurate clinical neurological examinations.
    \item The is a direct relationship between the anatomical organization of the functionalpathwaysin the brain and specific perceptual and motor behaviors.
\end{itemize}

\subsection{The Internal Representation of Personal Space is Modified by Experience}
\begin{itemize}
    \item Details of sensory maps vary considerably from one individual to another.
    \item Lost connections can be taken over by existing nearby connections.
\end{itemize}

\subsection{Is Consciousness Accessible to Neurobiological Analysis?}
\begin{itemize}
    \item John Searle and Thomas Nagel have defined three essential features of self-awareness:
        \begin{itemize}
            \item Subjectivity, or the awareness of a self that is the center of experience.
            \item Unity, or the fact that our experience of the world at any given moment is felt as a single unified experience.
            \item Intentionality, or the the experience that connets successive moments and the sense that the successive moments are directed to some goal.
        \end{itemize}
    \item Crick and Koch argue that our efforts should be focused on visual perceptionand in particular on two phenomena: binocular rivalry and selective attention.
    \item Sensory input alone does not give rise to consciousness; higher-level interpretation of that input is needed.
\end{itemize}
%\endgroup
%%%%%%%%%%%%%%%%%%%%%%%%%%%%% Chapter 17 %%%%%%%%%%%%%%%%%%%%%%%%%%%%%

%%%%%%%%%%%%%%%%%%%%%%%%%%%%% Chapter 18 %%%%%%%%%%%%%%%%%%%%%%%%%%%%%
%\begingroup
\clearpage
\section{Organization of Cognition}\phantomsection
\subsection{Functionally Related Areas of Cortex Lie Close Together}
\begin{itemize}
    \item The cortex of each cerebral hemisphere is a continuous sheet of gray matter.
    \item At the coarsest level, it consists of five lobes, with each lobe further subdivided.
    \item Functional areas are distinguished by cellular structure, connectivity, and the physiological response properties of neurons.
    \item Precepts that govern the organization of functional areas in the macaque (old world monkey) cerebral cortex:
        \begin{itemize}
            \item[1.] All areas fall into a few major functional groups.
            \item[2.] Areas in a given category occupy a discrete, continuous portion of the cortical sheet.
            \item[3.] Functionally related areas occupy neighboring sites. 
        \end{itemize}
\end{itemize}

\subsection{Sensory Information is Processed in the Cortex in Serial Pathways}
\begin{itemize}
    \item Cortical areas communicate with each other through bundles of axons traveling together in identifiable tracts.
    \item \textbf{Primary sensory areas} posses four properties characteristic of their role in the early stages of information processing:
        \begin{itemize}
            \item Inputs from thalamic sensory relay nuclei.
            \item Neurons in a primary sensory area have small receptive fields adn are arranged to form a precise somatotopic map of the sensory receptor surface.
            \item Injury to a part of the map causes a simple sensory loss confined to the corresponding part of the contralateral sensory receptor surface.
            \item Connections to other cortical areas are limited, mostly to nearby areas that process information in the same modality.
        \end{itemize}
    \item Higher-order sensory areas have a different set of properties important to their role in the later stages of information processing:
        \begin{itemize}
            \item Inputs arise from other thalamic nuclei and lower-order areas of sensory cortex instead of sensory relay nuclei.
            \item Large receptive fields and imprecise maps of the array of receptors in the periphery.
            \item Injury results in abnormalities of perception, but does not impair ability to detect sensory stimuli.
            \item Connected to distant areas in the frontal and limbid nodes as well as nearby unimodal areas.
        \end{itemize}
    \item Sensory information is processed serially, but not exclusively; higher-order areas project back to lower-order areas which can modulate the activity of neurons in lower-order areas.
    \item \textbf{Association cortex}: regions of the cortex where injury causes cognitive deficits that cannot be explained by impairment of sensory or motor function alone.
    \item Large regions of association cortex are contained within each of the four lobes:
        \begin{itemize}
            \item \textbf{parietal}: critical for sensory guidance of motor behavior and spatial awareness.
            \item \textbf{temporal}: recognition of sensory stimuli and for storage of semantic (factual) knowledge.
            \item \textbf{frontal}: key role in organizing behavior in working memory.
            \item \textbf{limbic}: complex functions related to emotion and episodic (autobiographical) memory.
        \end{itemize}
    \item Association areas have much more extensive input and output connections than do lower-order sensory and motor areas.
    \item All association lobes are densley interconnected network of pathways.
\end{itemize}

\subsection{Goal-Directed Motor Behavior Is Controlled in the Frontal Lobe}
\begin{itemize}
    \item All areas of th frontal lobe participate in the control of motor behavior and are connected in a series of functional hierarchy.
    \item Neuronal activity in the premotor cortex, adjacent to the primary motor cortex, reflect global aspects of motor behavior.
    \item Dorsolateral prefrontal cortex contributes to cognitive control of behavior.
    \item The orbital-ventromedial prefrontal corftex, connected to the dorsolateral prefrontal cortex (then premotor), is involved with emotional processes associated with executive control of behavior.
    \item Information flows from higher-order areas in the frontal lobe to primary order cortex, contrasting sensory's periphery first flow.
    \item Prefrontal cortex is important for the executive control of behavior.
    \item The orbital-ventromedial prefrontal cortex is linked strongly to the hypothalamus and amygdala, receives input from every sensory system, and projects to the dorsolateral prefrontal cortex. 
    \item Thus, the above pathway allows for response to emotional and sensory inputs and allows the trigger of appropriate behavior.
\end{itemize}

\subsection{Limbic Association Cortex is a Gateway to the Hippocampal Memory System}
\begin{itemize}
    \item The limbic (limbus---edge) association cortex forms a ring that is visible in the medial view of the hemisphere.
    \item Previously it was thought to make up an entire system in combinations with other areas, but some divisions of the limbic lobe have other functions, with some not yet well understood.
    \item The limbic association cortex does play an important role in long-term memory formation.
\end{itemize}

%\endgroup
%%%%%%%%%%%%%%%%%%%%%%%%%%%%% Chapter 18 %%%%%%%%%%%%%%%%%%%%%%%%%%%%%

%%%%%%%%%%%%%%%%%%%%%%%%%%%%% Chapter 19 %%%%%%%%%%%%%%%%%%%%%%%%%%%%%
%\begingroup
\clearpage
\section{Functions of the Premotor Systems}\phantomsection
\subsection{Direct Connections Between the Cerebral Cortex and Spinal Cord Play a Fundamental Role in the Organization of Voluntary Movements}
\begin{itemize}
    \item Individual muscles and joints are represented in the cortex multiple times in a complex mosaic.
    \item Each muscle joint is represented by a column of neurons whose axons branch and terminate in several functionally related spinal motor nuclei.
    \item Movement can also be elicited by stimulation of premotor areas.
    \item Neurons in the primary motor cortex fire in connection with a variety of goal-directed movements.
    \item There are three pathways from the premotor and motor areas to the motor neurons in the spinal cord: a direct corticospinal and two indirect, the medial and lateral brain stem systems.
    \item The pathways together make up the \textbf{corticospinal system}.
    \item \textbf{Medial brain stem system}: receives information from the cortex and other motor centers for the control of posture and locomotion.
    \item \textbf{Lateral brain stem system}: similar to medial but is involved in control of arm and hand movements.
    \item Reflex circuits can generate stereotyped movements without descending commands; new patterns can be generated through direct action on motor neurons.
    \item The cortical motor areas receive feedback from the cerebellum and basal ganglia in order for smooth execution of skilled movements in motor learning.
\end{itemize}

\subsection{The Four Premotor Areas of the Primate Brain Also Have Direct Connections in the Spinal Cord}
\begin{itemize}
    \item \textbf{Lateral ventral premotor area}: concerned the "what" in visual preception and controls mostly mouth and hand movements.
    \item \textbf{Lateral dorsal premotor area}: concerned with the "where" in visual preception and controls direction or movements.
    \item \textbf{Supplementary motor area}:
    \item \textbf{Cingulate motor areas}: a group of areas in the cingulate sulcus. 
    \item The four areas are connected to the primary motor cortex.
\end{itemize}

\subsection{Motor Circuits Involved in Voluntary Actions are Organized to Achieve Specific Goals}
\begin{itemize}
    \item The parietal lobe contains more than one representation of space and each one is dependent on motor activity.
    \item Neurons that respond to objects in peripersonal space are located mostly in the inferior parietal lobe where hand and mouth movements are represented, whereas neurons that respond to futher away objects are found where eye movements are represented.
\end{itemize}

\subsection{The Hand Has a Critical Role in Primate Behavior}
\begin{itemize}
    \item Investigation of the \textbf{anterior intraparietal area (AIP)} show that neurons fall into three main classes: motor-dominant, visual-dominant, and visual-motor combination.
    \item Futher suggestion shows that these neurons are involved in transfering sensory representations of objects into motor representations.
    \item \textbf{Canonical neurons}: neurons that fire in response to visual observation (and actual grasping) of graspable objects of certain size, shape, and orientation.
    \item Canonical neurons are thought to translate objects physical properties into \textit{potential motor acts}.
\end{itemize}

\subsection{The Join Activity of Neurons in the Parietal and Premotor Cortex Encodes Potential Motor Acts}
\begin{itemize}
    \item Studies of the parietal and premotor canonical neurons show that some neurons encode the possibilities for interaction with an object.
    \item \textbf{Mirror neurons}: discharge during specific motor acts, but also fire when the individual observation action being done by another.
    \item Mirror neurons may help us understand the intention of others.
    \item Potential motor acts are suppreseed or released by motor planing centers.
    \item Neurons in the supplementary motor area are involved in the planning, generation, and control of sequential motor actions.
    \item After long periods of practice, when the behavior becomes automatic, activity in the presupplementary motor area ceases.
\end{itemize}
%\endgroup
%%%%%%%%%%%%%%%%%%%%%%%%%%%%% Chapter 19 %%%%%%%%%%%%%%%%%%%%%%%%%%%%%

%%%%%%%%%%%%%%%%%%%%%%%%%%%%% Chapter 20 %%%%%%%%%%%%%%%%%%%%%%%%%%%%%
%\begingroup
\clearpage
\section{Functional Imaging of Cognition}\phantomsection
\subsection{Functional Imaging Reflects the Metabolic Demand of Neural Activity}
\begin{itemize}
    \item A large amount of neuron's total energy metabolism, about one-half, is devoted to mainting resting potential.
    \item The other half is for other biochemical processes, including all molecular reactions for normal function.
    \item fMRI responses have been highly correlated with neural spiking.
    \item Though, some fMRI measurements fail to capture subthreshold modulatory inputs.
\end{itemize}
%\endgroup
%%%%%%%%%%%%%%%%%%%%%%%%%%%%% Chapter 20 %%%%%%%%%%%%%%%%%%%%%%%%%%%%%
%\endgroup

\clearpage
\fancyhead[L]{Part V Perception}
%\begingroup

%%%%%%%%%%%%%%%%%%%%%%%%%%%%% Chapter 21 %%%%%%%%%%%%%%%%%%%%%%%%%%%%%
%\begingroup
\clearpage
\section{Sensory Coding}\phantomsection
\subsection{Psychophysics Relates the Physical Properties of Stimuli to Sensations}
\begin{itemize}
    \item \textbf{Psychophysics}: relationship between the physical characteristics of a stimulus and attributes of sensory experience.
    \item \textbf{Sensory physiology}: examination of neural consequences of a stimulus.
    \item \textbf{Sensory threshold}: the lowest stimulus strength a subject can detect.
    \item Sensory thresholds can be altered by emotional or psychological factors.
    \item Sensations are quantified using probabilistic statistics.
    \item Reaction times are correlated with cognitive processes.
\end{itemize}

\subsection{Physical Stimuli are Represented in the Nervous System by Means of the Sensory Code}
\begin{itemize}
    \item Neural coding of sensory information is better understood at the early stages than later.
    \item Sensory receptors are responsive to a single type of stimulus energy.
    \item \textbf{Stimulus transduction}: the time it takes to convert a stimulus response into an electrical signal.
    \item Multiple subclasses of sensory receptors are found in each sense organ.
    \item There are rapid and slowly adapting sensors that illustrate a major principal of decoding: \textit{contrast}.
    \item  The timing of action potentials between neurons has a profound effect on long-term potentiation and depression at synapses.
    \item The receptive field of a sensory neuron conveys spatial information.
    \item Fragmentation of a stimulus into componenets, each encoded by an individual neuron, is the initial step in sensory processing.
\end{itemize}

\subsection{Modality-Specific Pathways Extend to the Central Nervous System}
\begin{itemize}
    \item Activity of sensory neurons are more variable than that of neurons in the periphery.
    \item Central sensory neurons fire irregularly before and after stimulation, even during times of no stimulation.
    \item The variability is a result of: alertness, attention, previous experience, and recent activation by similar stimuli.
    \item The receptor surface is represented topographically in central nuclei.
    \item Feedback regulates sensory coding and top-down learing mechanisms influences sensory processing.
\end{itemize}
%\endgroup
%%%%%%%%%%%%%%%%%%%%%%%%%%%%% Chapter 21 %%%%%%%%%%%%%%%%%%%%%%%%%%%%%

%%%%%%%%%%%%%%%%%%%%%%%%%%%%% Chapter 22 %%%%%%%%%%%%%%%%%%%%%%%%%%%%%
%\begingroup
\clearpage
\section{Somatosensory System}\phantomsection
\begin{itemize}
    \item The somatosensory system serves three major functions:
        \begin{itemize}
            \item \textbf{Proprioception}: the sense of oneself. Skeletal muscle, joint capsules, and the skin allow for aweraness of our own body.
            \item \textbf{Exteroception}: the sense of direct interaction with the external world. Touch, contact, pressure, storking, temperature, pain, motion, vibration are used to identify objects.
            \item \textbf{Interoception}: sense of major organ systems of the body and it's internal state. Most information does not appear conscious, but plays a major role. Primarily consist of chemoreceptors.
        \end{itemize}
    \item All somatic senses are mediated by the dorsal root ganglion neurons.
\end{itemize}
\subsection{The Primary Sensory Neuron of the Somatosensory System are Clustered in the Dorsal Root Ganglia}
\begin{itemize}
    \item Dorsal root ganglion neurons are pseudo-unipolar cells.
    \item The central branches termintate in the spinal cord or brain stem, forming the first synapses in somatosensory pathways.
    \item \textbf{Primary afferent fiber}: the axon of each dorsol root ganglion cell serves as a single tranmission line from receptor to central nervous system.
    \item \textbf{Peripheral nerves}: individual primary afferent fibers group that are grouped together, and also include motor axons innervating nearby muscles, blood vessels, glands, or viscera. 
\end{itemize}

\subsection{Peripheral Somatosensory Nerve Fibers Conduct Action Potentials at Different Rates}
\begin{itemize}
    \item Difference in peripheral nerve's diameter and conduction velocity mediate somatic sensation.
    \item Larger diameter tends to relay faster, not accounting for degree myelinated fibers.
    \item Electrial stimulation of whole nerves is also used to classify peripheral nerve fibers.
    \item \textbf{Compound action potential}: summed action potential of all nerve fibers excited by a stimulus pulse and is roughly proportional to the total number of active nerve fibers.
    \item The conduction velocity throughout teh nervous system is correlated with the need to maintain synchrony.
\end{itemize}

\subsection{Many Specialized Receptors Are Employed by the Somatosensory System}
\begin{itemize}
    \item The receptor class expressed in the nerve terminal of a sensory neuron determines the type of stimulus detected.
    \item Mechanoreceptors mediate touch and proprioception.
    \item The skin has eight types of mechanoreceptors that are responsible for touch.
    \item Proprioceptors measure muscle activity and joint positions.
    \item Nociceptors mediate pain.
    \item Thermal receptors detect changes in skin temperature.
    \item Itch is a distinctive cutaneous sensation.
    \item Visceral sensations represnet the status of various interanal organs.
\end{itemize}
%\endgroup
%%%%%%%%%%%%%%%%%%%%%%%%%%%%% Chapter 22 %%%%%%%%%%%%%%%%%%%%%%%%%%%%%

%%%%%%%%%%%%%%%%%%%%%%%%%%%%% Chapter 23 %%%%%%%%%%%%%%%%%%%%%%%%%%%%%
%\begingroup
\clearpage
\section{Touch}\phantomsection
\subsection{Active and Passive Touch Evoke Similar Responses in Mechanoreceptors}
\begin{itemize}
    \item Active and passive modes of tactile stimulation excite the same population of receptors in the skin and evoke similar responses in afferent fibers.
    \item Passive touch is used for naming objects or describing sensations.
    \item Active touch is used when the hand manipulates objects.
\end{itemize}

\subsection{The Hand Has Four Major Types of Mechanoreceptors}
\begin{itemize}
    \item \textbf{Merkel cell}: tips of epidermal sweat ridges; detects edges, points; slow adaption to sustained indentation.
    \item \textbf{Meissener corpuscle}: close to skin surface; detects lateral motion; no adaption to sustained indentation.
    \item \textbf{Ruffini ending}: located in dermis; senses skin stretching; slow adaption to sustained indentation.
    \item \textbf{Pacinian corpuscle}: located deep in dermis; senses vibration; no adaption to sustained indentation. 
    \item Receptive fields define the zone of tactile sensitivity.
    \item There are two types of receptive fields: one with highly specialized fields; the other with broader fields with a central hotspot.
    \item Slowly adapting fibers detect object pressure and from.
    \item Rapidly adapting ribers detect motion and vibration.
    \item Combination of slow and rapidly adapting fibers contribute to grip control.
\end{itemize}

\subsection{Tactile Information is Processed in the Central Touch System}
\begin{itemize}
    \item Cortical receptive fields intergrate information from neighboring receptors.
    \item Neurons in the somatosensory cortex are organized into functionally specialized columns.
    \item All neurons within a column receive inputs from teh same local area of teh receptor sheet and respond to the same class(es) of receptors.
    \item Columns share a common center that is clearly evident in layer IV.
    \item Horizontal connections within layers II and III link neurons in neighboring columns, sharing information when activated by the same stimulus.
    \item Cortical columns are organized somatotopically.
    \item \textbf{Cortical magnification}: the amount of cortical area devoted to a unit of area of skin. This various by more than a hundredfold across differrent body surfaces.
\end{itemize}
%\endgroup
%%%%%%%%%%%%%%%%%%%%%%%%%%%%% Chapter 23 %%%%%%%%%%%%%%%%%%%%%%%%%%%%%

%%%%%%%%%%%%%%%%%%%%%%%%%%%%% Chapter 24 %%%%%%%%%%%%%%%%%%%%%%%%%%%%%
%\begingroup
\clearpage
\section{Pain}\phantomsection
\subsection{Noxious Insults Activate Nociceptors}
\begin{itemize}
    \item Most nociceptors are simply the free nerve endings of primary sensory neurons.
    \item There are three main classes of nociceptors:
        \begin{itemize}
            \item Thermal
            \item Mechanical
            \item Polymodal--- high-intensity mechanical, chemical, or thermal. 
        \end{itemize}
    \item There is a less understood fourth class: silent nociceptors.
    \item The three main classes are widely distributed in the skin and deep tissues and are oftening coactivated.
    \item Silent nociceptors are found in viscera; activated by inflammation and various chemical agents.
    \item \textbf{Allodynia}: pain in response to stimuli that re normally innocuous.
    \item \textbf{Hyperalgesia}: an exaggerated response to noxious stimuli, typically persistant even in absence of sensory stimulation. 
    \item \textbf{Nociceptive pain}: activation of nociceptors and normally from accompanying inflammation.
    \item \textbf{Neuropathic pain}: direct injury to nerves in peripheral or central nervous system and is accompanied by burining or electric sensation.
\end{itemize}
%\endgroup
%%%%%%%%%%%%%%%%%%%%%%%%%%%%% Chapter 24 %%%%%%%%%%%%%%%%%%%%%%%%%%%%%

%%%%%%%%%%%%%%%%%%%%%%%%%%%%% Chapter 25 %%%%%%%%%%%%%%%%%%%%%%%%%%%%%
%\begingroup
\clearpage
\section{Visual Processing}\phantomsection
\subsection{Visual Perception is a Constructive Processe}
\begin{itemize}
    \item The brain guesses at scene presented to the eyes based on past experience.
    \item The modern vew of perception is based on the gestalt psychology---the perceptual interpretation we make of any visual object depends not just on the properties of the stimulus, but also the context.
    \item An important step in object recognition separating figures from the background.
    \item The brain analyzes a scene at three levels:
        \begin{itemize}
            \item Low: local contrast, orientation, color, and movement.
            \item Intermediate: analysis of the layout, surfaces, parsing global contours, and depth.
            \item High: object recognition.
        \end{itemize}
    \item Motion, depth, form, and color occur in a unified percept due to interacting neural pathways.
\end{itemize}

\subsection{Form, Color, Motion, and Depth are Processed in Discrete Areas of the Cerebral Cortex}
\begin{itemize}
    \item Visual areas of the cortex can be differentiated either by a visuotopic map, of by functional properties of the neurons.
    \item Visual areas are organized into two hierarchical pathways: ventral, involved in object recognition; and dorsal, dedicated to the use of visual information guiding movements.
    \item Pathways are interconnected so that information is shared and each connection is reciprocal--- each area sends information back to areas from which it receives input.
    \item The shared connections provide information about cognitive functions, spatial attention, stimulus expectation, and emotional conetent, to earlier levels of visual processing.
\end{itemize}

\subsection{The Receptive Fields of Neurons at Successive Relays in an Afferent Pathway Provide Clues to How the Brain Analyzes Visual Form}
\begin{itemize}
    \item \textbf{On-center}: cells that fire when a spot of light is turned on within a circular central region.
    \item \textbf{Off-center}: cells that fire inversely to on-center.
    \item If both cells are stimulated with diffuse light, then there is little to no response. This allows them to distinguish borders and contours very well and leads to the encoding of contrast.
    \item \textbf{Eccentricity}: size of the retina's receptive field, which varies in relative to the fovea and the position of neurons along the visual pathway.
\end{itemize}

\subsection{The Visual Cortex is Organized into Columns of Specialized Neurons}
\begin{itemize}
    \item The dominant feature of the functional organization of the primary visual cortex is the visuotopic organization of the of its cells: the visual field is systematically represented across the surface of the cortex.
    \item Columins reflect the functional role of that area in vision.
    \item Orientation and ocular dominance columns have embedded clusters of neurons that have strong color preferences.
    \item These clusters specialize to provide information about surfaces rather than edges.
    \item \textbf{Serial processing}: processing in successive connections between cortical areas that run from the back of the brain forward.
    \item \textbf{Parallel processing}: occurs simultaneously in subsets of fibers that process different submodalities such as from, color, movement.
\end{itemize}

\subsection{Intrinsic Cortical Circuits Transform Neural Information}
\begin{itemize}
    \item Each area of the visual cortex transforms information gathered by the eyes.
    \item Principal input to the primary visual cortex comes from two parallel pathways that originate in the parvocellular and magnocellular layers of the lateral geniculate nucleus.
    \item Neurons in different layers have distinctive receptive-field properties, with superficial layers have smaller fields while deeper layers tend to have larger ones.
    \item Feedback projections are thought to provide a means where higher centers in a pathway can influence lower ones.
    \item Feedback projection is still largely unknown.
\end{itemize}
%\endgroup
%%%%%%%%%%%%%%%%%%%%%%%%%%%%% Chapter 25 %%%%%%%%%%%%%%%%%%%%%%%%%%%%%


%%%%%%%%%%%%%%%%%%%%%%%%%%%%% Chapter 26 %%%%%%%%%%%%%%%%%%%%%%%%%%%%%
%\begingroup
\clearpage
\section{Low-Level Visual Processing}\phantomsection
\subsection{The Photoreceptor Layer Samples the Visual Image}
\begin{itemize}
    \item Ocular optics limit the quality of the retinal image.
    \item The density of photoreceptors, bipolar cells, and ganglion cells is highest at the fovea(center of eye).
    \item There are two types of photoreceptors: rods and cones.
    \item Rods: very sensitive---low light, no color.
    \item Cones: less sensitive---for daylight, multiple types, faster response time.
    \item Central fovea has an absence of rods.
\end{itemize}

\subsection{Ganglion Cells Tranmist Neural Images to the Brain}
\begin{itemize}
    \item Optic nerve has only 1\% as many axons as there are receptor cells, so the retinal circuit must edit information before it is conveyed to the brain.
    \item The two major ganglion cells are binary, ON or OFF cells.
    \item Many ganglion cells fire regardless of current lighting condition, but ON cells fire more rapidly with increasing light, while OFF slows or stops. The inverse is also true when going form light to dark.
    \item Many ganglion cells respond strongly to edges in the image.
    \item Output produced by ganglion cells enhance regions of contrast, while reducing homogenous illumination.
    \item Ganglion output also emphasizes temporal changes in stimuli through transient (burst response) and sustained (steady) neurons.
    \item Retinal output emphasizes moving objects.
    \item Several ganglion cell types project to the brain through parallel pathways.
    \item About 20 ganglion cells have been described, which allow the optic nerve to convey about 20(?) different represnations of the world based on polarity (on/off, fine/coarse, sustained/transient, motion, spectral filtering... and more?)
\end{itemize}

\subsection{A Network of Interneurons Shapes the Retinal Output}
\begin{itemize}
    \item Parallel pathways originate in bipolar cells.
    \item Most retinal processing is accomplished with graded membrane potentials via the \textit{ribbon synapse}.
    \item Action potentials occur only in certain amacrine and ganglion cells.
    \item Stimulus represention in ganglion cell population originates in dedicated bipolar cell pathways that are differentiated by their selective connections to photoreceptors and postsynaptic targets.
    \item Spatial filtering is accomplished by lateral inhibition.
    \item Amacrine cells are axonless neurons with dendrites that ramify in the inner plexiform layer, generally producing an inhibitory network.
    \item Temporal filtering occurs in synapses and feedback circuits.
    \item Color vision begins in cone-selective circuits.
    \item Rod and cone circuits merge in the inner retina.
\end{itemize}
%\endgroup
%%%%%%%%%%%%%%%%%%%%%%%%%%%%% Chapter 26 %%%%%%%%%%%%%%%%%%%%%%%%%%%%%

%%%%%%%%%%%%%%%%%%%%%%%%%%%%% Chapter 27 %%%%%%%%%%%%%%%%%%%%%%%%%%%%%
%\begingroup
\clearpage
\section{Intermediate-Level Visual Processing}\phantomsection
\subsection{Internal Models of Object Geometry Help the Brain Analyze Shapes}
\begin{itemize}
    \item \textbf{Visual Primitives}: local features in a visual scene: contrast, line orientation, brightness, color, movement, and depth.
    \item In the visual cortex neurons respond selectively to lines of particular orientations, which reflects the arrangement of inputs from cells in the lateral geniculate.
    \item There are two types of orientation-selective neurons:
        \begin{itemize}
            \item \textbf{Simple cells}: receptive fields are divided into binary subregions, firing when an light enters the ON region.
            \item Simple cells are cells highly selective for the position of a line or edge in space.
            \item \textbf{Complex cells}: lack discrete binary subregions. Instead they fire continuously as line or edge stimulus traverses.
        \end{itemize}
    \item Visual cortex neurons do not respond to an image that is stabalized on the retina.
    \item \textbf{Contextual modulation}: visual cortex neuron that is modulated by stimuli outside the receptive field's core, allowing for complex selectivity.
\end{itemize}

\subsection{Depth Perception Helps Segregate Objects from Background}
\begin{itemize}
    \item \textbf{Plane of fixation}: the point where corresponding positions are displayed on both retinas.
    \item Visual cortex neurons can be selective for objects lying on, in front, or behind the plane of fixation.
    \item Depth is also determined through monocular cues such as size, perspective, occlusion, brightness, and movement. 
\end{itemize}

\subsection{Local Movement Cues Define Object Trajectory and Shape}
\begin{itemize}
    \item The primary visual cortex determines the direction of movement of objects.
    \item Determining the direction of motion requires resolving multiple cues.
    \item An important determinant of perceived direction is scene segmentation. 
\end{itemize}

\subsection{Context Determines the Perception of Visual Stimuli}
\begin{itemize}
    \item The visual system attempts to measure the surface characteristics of objects by comparing the light arriving from different parts fo the visual field. 
    \item Transient and variable stimulation of the retina construct representations of a stable, three-dimensional world.
    \item Color and brightness depend heavily on nearby contextual cues.
    \item Receptive-field properties also depend on context.
    \item Contextual influences are pervasive in contour integration, scene segmentation, and determination of object shape and surface properties.
\end{itemize}

\subsection{Cortical Connections, Functional Architecture, and Perception are Intimately Related}
\begin{itemize}
    \item Intermediate-level visual processing requires sharing information from throughout the visual field.
    \item Cortical circuits make horizontal connections that link distant locations in the visual field suggest connections between orientation columns of similar specificity have a role in countour integration over a large area of the visual cortex.
    \item Perceptual learning requires plasticity in cortical connections.
    \item Perceptual learning involves repeating a discrimination task many times, does not require error feedback, and improvement mainfests itself as decrease in threshold in discrimination of smaller differences.
    \item Visual search relies on the cortical representation of visual attributes and shapes.
    \item Certain objects pop out from others in a complex image due to visual system processes in parallel pathways---features of target and the features of surrounding distractors.
    \item Cognitive processes influence visual perception and segmentation, such as top-down spatial attention, visual expectation, or the particular perceptual task at hand.
\end{itemize}
%\endgroup
%%%%%%%%%%%%%%%%%%%%%%%%%%%%% Chapter 27 %%%%%%%%%%%%%%%%%%%%%%%%%%%%%

%%%%%%%%%%%%%%%%%%%%%%%%%%%%% Chapter 28 %%%%%%%%%%%%%%%%%%%%%%%%%%%%%
%\begingroup
\clearpage
\section{High-Level Visual Processing}\phantomsection
\subsection{The Inferior Temporal Cortex is the Primary Center for Object Perception}
\begin{itemize}
    \item Object perception is the nexus between vision and cognition, and is what high-level visual processing is concerned with.
    \item The hierarchy of synaptic relays in the cortical system extend from the primary visual cortex to the temporal lobe.
    \item Temporal lobe is a site of convergence of many types of visual information.
    \item Clinical evidence identifies the inferior temporal cortex as essential for object recognition.
    \item Neurons in the inferior temporal cortex encode complex visual stimuli, such as specific patterns or features.
    \item Neurons in the inferior temporal cortex are functionally organized in columns.
    \item Face-selective cells contitute a highly specialized class of neurons.
    \item Object recognition is intertwined with visual categorization, visual memory, and emotion; all of which the inferior temporal cortex contribute to.
\end{itemize}

\subsection{Object Recognition Relies on Perceptual Constancy}
\begin{itemize}
    \item For object recognition to take place, various invariant attributes must be represented independently of other image properties. Doing this correctly is termed \textbf{Perceptual constancy}
    \item Perceptual constancy has many forms, from size, position, rotation, viewing angle, likeness, and more.
    \item A more general type of constancy is the perception of objects as belonging to the same semantic category.
\end{itemize}

\subsection{Visual Memory is a Component fo High-Level Visual Processing}
\begin{itemize}
    \item Visual memory can influence the processing of new incoming visual information.
    \item Implicit visual learning leads to changes in the selectivity of neuronal responses.
    \item Explicit visual learning depends on linkage of the visual system and declarative memory formation.
    \item The hippocampus and neocortical areas of the medial temporal lobe are essential both for the acquisition of visual associative memories and for the functional plasticity of the inferior temporal cortex. 
\end{itemize}

\subsection{Associative Recall or Visual Memories Depends on Top-Down Activation of the Cortical Neurons that Process Visual Stimuli}
\begin{itemize}
    \item Sensory experience of an image and recall of the same image are subjectively similar. 
    \item Sensory experience depends on bottom-up flow of visual information, while the latter is top-down flow. 
    \item The distinction is important, but under normal conditions both signal pathways collaborate to yield visual experience.
\end{itemize}
%\endgroup
%%%%%%%%%%%%%%%%%%%%%%%%%%%%% Chapter 28 %%%%%%%%%%%%%%%%%%%%%%%%%%%%%

%%%%%%%%%%%%%%%%%%%%%%%%%%%%% Chapter 29 %%%%%%%%%%%%%%%%%%%%%%%%%%%%%
%\begingroup
\clearpage
\section{Visual Processing and Action}\phantomsection
\subsection{Successive Fixations Focus Our Attention in the Visual Field}
\begin{itemize}
    \item A saccade usually lasts less than 40 ms, occurs several times per second, and redirects center of sight in the visual field.
    \item Attention selects objects for further visual examination.
    \item Voluntary attention is closely linked to saccades.
    \item Activity in the parietal lobe correlates with attention paid to objects.
    \item The locus of attention can be ascertained only by examing the entire salience map and choosing its peak; it cannot be identified by monitoring activity at one point alone.
\end{itemize}

\subsection{The Visual Scene Can Remain Stable Despite Continual Shifts in the Retinal Image}
\begin{itemize}
    \item Neurons shift their receptive fields from one part of the visual field to another before the saccade occurs.
    \item The parietal neurons must have advanced information about the saccade; either from feedback from the peripheral proprioceptors in the eye muscles directly (unlikely) or from the motor system that controls the movement of the eyes.
\end{itemize}

\subsection{Vision Lapses During Saccades}
\begin{itemize}
    \item An object can be seen during a saccade if it is moving as fast and in the same direction of the eye.
    \item Visual scene is thought to be blurred, but not unconscious during a saccade.
    \item Visual masking may hide the lower contrast, blured image if two images are close enough, which would happen often during a saccade.
    \item Extravisual input, such as corollary discharge, must also be present. This would reduce sensitivity during a saccade if the input is not there (\textit{Saccadic supression}).
    \item Saccadic supression due to corollary discharge is most pronounced for stimuli with low spatial frequency and high contrast, which are most effective for activating neurons in the magnocellular pathway of the visual system. 
    \item There is evidence that visual masking and corollary discharge must act together to reduce disruption of vision during saccades.
\end{itemize}
%\endgroup
%%%%%%%%%%%%%%%%%%%%%%%%%%%%% Chapter 29 %%%%%%%%%%%%%%%%%%%%%%%%%%%%%

%%%%%%%%%%%%%%%%%%%%%%%%%%%%% Chapter 31 %%%%%%%%%%%%%%%%%%%%%%%%%%%%%
%\begingroup
\clearpage
\setcounter{section}{30}
\section{Auditory Central Nervous System}\phantomsection
\subsection{The Neural Representation of Sound Begins in the Cochlear Nuclei}
\begin{itemize}
    \item Acoustic information is processed from the ear, to the brain stem, through the midrain and thalamus, then to the cerebral cortex.
    \item The afferent auditory pathways from the periphery to higher brain regions include efferent feedback at many levels.
    \item The cochlear nerve imposes a tonotopic organization on the cochlear nuclei and distributes acoustic information into parallel pathways.
    \item The cochlear nerve contains two groups of fibers: a large contingent (95\%) of myelinated fibers that receive input from inner hair cells, and the remaining (5\%) unmyelinated fibers that receive input from outer hair cells.
    \item Low frequency, apical fibers terminate ventrally into the ventral and dorsal cochlear nuclei; high frequency, basal fibers, terminate dorsally.
    \item Each cochlear nerve fiber innervates several different area within the cochlear nuclei.
    \item The result is at least four parallel ascending pathways.
    \item The ventral cochlear nucleus extracts information about the temporal and spectral structure of sounds.
    \item The dorsal cochlear nucleus intergrates acoustic with somatosensory information in making use of spectral cues for localizing sounds.
\end{itemize}

\subsection{The Superior Olivary Complex of Mammals Contains Separate Circuits for Detecting Interaural Time and Intensity Differences}
\begin{itemize}
    \item The medial superior olive generates a map of interaural time differences.
    \item The lateral superior olive detects interaural intensity differences.
\end{itemize}

\subsection{Efferent Signals from the Superior Olivary Complex Help Provide Feedback to the Cochlea}
\begin{itemize}
    \item Brain stem pathways converge in the inferior colliculus.
    \item The inferior colliculus occupies a central position in the auditory pathway due to ascending pathways converging from the brain stem.
    \item \textbf{Precedence effect}: surpression of all but the earliest versions of sound; measurements in the inferior colliulus show that inhibition of reflected sounds occur there.
    \item Sound location information from the inferior colliculus creates a spatialmap of sound in the superior colliculus.
    \item The inferior colliculus is also a branch point for ascending or outflow pathways.
    \item The superior colliculus is critical for reflexive orientating movements of the head and eyes to acoustic and visual cues in space.
    \item Unlike visual and somatosensory maps, the auditory map is computed from a combination of cues that identify the specific position of sound source in space, not on peripheral receptor surface.
\end{itemize}

\subsection{Inferior Colliculus Transmits Auditory Information to the Cerebral Cortex}
\begin{itemize}
    \item The auditory cortex maps numerous aspect of sound.
    \item Ascending auditory pathways termintate in the auditory cortex, which includes multiple distinct areas on the dorsal surface of the temporal lobe.
    \item Auditory information is processed in multiple cortical areas that surround the primary auditory areas.
    \item As many as 7-10 secondary belt areas surround 3-4 primary core areas.
    \item Pure tones activate in core regions, whereas the belt areas prefer sounds such asa narrow-band noise bursts.
    \item A second sound-localization pathway from the inferior colliculus involves the cerebral cortex in gaze control.
    \item Auditory circuits in the cerebral cortex are segregated into separate processing streams.
\end{itemize}
%\endgroup
%%%%%%%%%%%%%%%%%%%%%%%%%%%%% Chapter 31 %%%%%%%%%%%%%%%%%%%%%%%%%%%%%
%\endgroup

\clearpage
\fancyhead[L]{Part VII Neural Processing}
%\begingroup

%%%%%%%%%%%%%%%%%%%%%%%%%%%%% Chapter 45 %%%%%%%%%%%%%%%%%%%%%%%%%%%%%
%\begingroup
\clearpage
\setcounter{section}{44}
\section{Sensory, Motor, and Reflex Functions}\phantomsection
\subsection{The Cranial Nerves Are Homologous to the Spinal Nerves}
\begin{itemize}
    \item Spinal nerves only reach as high as cervical vertebra, so the cranial nerves must provide somatic and visceral sensory and motor innervation for the head.
    \item Cranial nerves associate with one or more functions and may overlap with each other.
    \item Cranial nevres are numbered I through XII in rostrocaudal sequence.
    \item Cranial nerves mediate the sensory and motor functions of the face and head, as well as the autonomic functions of the body.
    \item Cranial nerves leave the skull in groups and often are injured together.
\end{itemize}

\subsection{Cranial Nerve Nuclei in the Brain Stem are Organized on the Same Basic Plan as are Sensory and Motor Regions of the Spinal Cord}
\begin{itemize}
    \item The nuclei of the brain stem are divided into general nuclei, which serve functions similar to those of the spinal cord laminae and special nuclei, which serve functions unique to the head.
    \item Adult cranial nerve nuclei have columnar organization.
    \item The brain stem nuclei are organized into six rostrocaudal columns, three sensory nuclei and three motor.
    \item \textbf{General Somatic Sensory Column}:
        \begin{itemize}
            \item Occupies the most lateral region of the alar plate and includes the trigeminal sensory nuclei.
            \item \textbf{Spinal trigeminal nucleus}: a continuation of the dorsal-most laminae of the spinal dorsal horn. It receives inputs from all cranial nerve sensory ganglia concerned with pain and temperature. 
            \item \textbf{Principal sensory trigeminal nucleus}: lies in the mid pons lateral to the trigeminal motor nucleus. It receives inputs concerned with position sense and fine touch discrimination. 
            \item \textbf{Mesencephalic trigeminal nucleus}: relays mechanosensory information from the muscles of mastication and periodontal ligaments. It also provides monosynaptic feedback to the jaw, critical for chewing.
        \end{itemize}
    \item \textbf{Special somatic sensory column}: receives inputs from the acoustic and vestibular nerves. 
    \item \textbf{Visceral sensory column}: concerned with special visceral information (taste) and general visceral information from the facial, glossopharyngeal, and vagus nerves. 
    \item Visceral sensory information from different afferent nerves produce a unified visceral sensory map of the body in the nucleus.
    \item \textbf{General Visceral Motor Column}:
        \begin{itemize}
            \item \textbf{Edinger-Westphal nucleus}: lies next to oculomotor complex beneath the floor of cerebral aqueduct; controls pupillary constriction and lens accommodation.
            \item \textbf{Superior salivatory nucleus}: lies dorsal to the facial motor nucleus. Innervates the sublingual and submandibular salivary glands.
            \item \textbf{Inferior salivatory nucleus}: innervates the parotid gland. 
            \item \textbf{Dorsal motor vagal nucleus}: innervate the gastrointestinal tract below the diaphragm.
            \item \textbf{Nucleus ambiguus}: innervate thoracic organs, including esophagus, heart, respiratory, and special visceral motor neurons that innervate the larynx and pharynx.
        \end{itemize}
    \item \textbf{Special Visceral Motor Column}: includes motor nuclei that innervate muscles derived from branchial arches.
        \begin{itemize}
            \item \textbf{Trigeminal motor nucleus}: innervates the muscles of mastication. 
            \item \textbf{Accessory trigeminal nuclei}: innervates tensor tympani, tensor veli palatini, mylohyoid muscles, and the anterior belly of the digastric muscle.
            \item \textbf{Facial motor nucleus}: ?
        \end{itemize}
    \item \textbf{General Somatic Motor Column}: contains the oculomotor nucleus, trochlear nucleus, abducens nucleus and the hypoglossal nucleus.
    \item Embryonic cranial nerve nuclei have a segmental organization.
    \item The organization of the brain stem and spinal cord differs in three important ways:
        \begin{itemize}
            \item The long ascending and descending sensory tracts that run along the outside of the spinal cord are incorporated within the interior of the brain stem. 
            \item In the brain stem, the cerebellum and its associated pathways form additional structures that are superimposed on the basic plan of the spinal cord.
            \item The spinal cord is not segmented during development, but the final pattern is; the inverse is true for hindbrain.
        \end{itemize}
\end{itemize}

\subsection{Neuronal Ensembles in the Brain Stem Reticular Formation Coordinate Reflexes and Simple Behaviors Necessary for Homeostasis and Survival}
\begin{itemize}
    \item A variety of reflexes and simple behaviors are mediated by the cranial nerves, from autonomic and motor responses, to facial expressions, breathing, and eating.
    \item Cranial nerve reflexes involve mono and polysynaptic brain stem relays.
    \item \textbf{Vestibulo-ocular reflexes} stabilize the image on the retina during head movement.
    \item \textbf{Corneal reflex}: closure of both eyelids as well as upward turning of the eyes.
    \item \textbf{Stapedial reflex}: contracts the stapedius muscle in response to a loud sound.
    \item A variety of gastrointestinal reflexes are controlled by multisynaptic brian stem relays.
    \item \textbf{Gag reflex}: protects the airway in response stimulation to posterior oropharynx. 
    \item Pattern generator neurons coordinate stereotypic and autonomic behaviors.
    \item A complex pattern generator regulates breathing.
    \item Respiratory activity can be generated by the medulla even when it is isolated from the rest of the nervous system.
\end{itemize}
%\endgroup
%%%%%%%%%%%%%%%%%%%%%%%%%%%%% Chapter 45 %%%%%%%%%%%%%%%%%%%%%%%%%%%%%


%%%%%%%%%%%%%%%%%%%%%%%%%%%%% Chapter 46 %%%%%%%%%%%%%%%%%%%%%%%%%%%%%
%\begingroup
\clearpage
\section{Modulatory Functions of the Brain Stem}\phantomsection
\subsection{Ascending Projections from the Brain Stem Maintain Arousal}
\begin{itemize}
    \item The portion of the brain stem from the midbrain to the medulla keeps the forebrain awake.
    \item Neurons in the brain stem fall into chemically distinct groups based on their monoamine neurotransmitter content; norepinephrine, dopamine, histamine, epinephrine, and serotonin.
    \item The monoaminergic and cholinergic neurons of the ascending arousal system have widespread projections, virtually to every part of the nervous system.
    \item Monoaminergic and cholinergic neurons in the brain stem are primarily found in four regions:
        \begin{itemize}
            \item Locus ceruleus; which contains noradrenergic neurons.
            \item Dorsal and median raphe nuclei, which contain serotonergic, but also some dopaminergic neurons.
            \item Pedunculopontine and and lateraldorsal tegmental nuclei, which contain cholinergic neurons.
            \item Tuberomammillary nucleus, which contains histaminergic neurons. 
        \end{itemize}
    \item Monoaminergic pathways alter specific cellular properties of the postsynaptic neurons in the thalamus and cerebral cortex, enhancing alterness and interaction with environmental stimuli.
\end{itemize}

\subsection{Monoaminergic and Cholinergic Neurons Share Many Properties}
\begin{itemize}
    \item Monomines are biochemical compounds with aromatic ring that are synthesized from aromatic amino acids.
    \item Most neurons that use monoamines as neurotransmitters share properties, such as continuous and spontaneous action potentials in a regular pattern.
    \item Many monoaminergic and cholinergic neurons are linked to the sleep-wake cycles.
    \item Monoaminergic and cholinergic neurons maintain arousal by modulations neurons in the thalamus and cortex.
\end{itemize}

\subsection{Monoamines Regulate Many Brain Functions Other Than Arousal}
\begin{itemize}
    \item Monoaminergic neurons also regulate cognitive performance during waking and affect a variety of other central nervous system functions.
    \item Cognitive performance is optimized by ascending projections from monoaminergic neurons.
    \item Monoamines and acetylcholine each induce arousal, but have different effects on cognitive function during waking.
    \item Neurons of the locus ceruleus play an important role in attention and have two differnet modes in alert monkeys:
        \begin{itemize}
            \item \textbf{Phasic mode}: baseline activity of the neurons is low to moderate, with brief excitement just before response to stimuli. Responsbile for selective attention.
            \item \textbf{Tonic mode}: baseline level is elevated, doesn't respond to external stimuli. Responsbile for disruption of attention and search for new goal when reward is low.
        \end{itemize}
    \item Monoamines are involved in autonomic regulation and breathing.
    \item Serotonin regulates many different autonomic functions including gastrointestinal peristalsis, thermoregulation, cardiovascular, and breathing.
    \item Pain and anti-nociceptive pathways are modulated by monoamines.
    \item Monoamines facilitate motor activity.
\end{itemize}
%\endgroup
%%%%%%%%%%%%%%%%%%%%%%%%%%%%% Chapter 46 %%%%%%%%%%%%%%%%%%%%%%%%%%%%%

%%%%%%%%%%%%%%%%%%%%%%%%%%%%% Chapter 47 %%%%%%%%%%%%%%%%%%%%%%%%%%%%%
%\begingroup
\clearpage
\section{Autonomic Motor System and Hypothalamus}\phantomsection
\subsection{The Autonomic Motor System Mediates Homeostasis}
\begin{itemize}
    \item All homeostatic behavior, inclding control of circulation, arises from neural modulation of the physiological properties of organ systems mediated by hypothalamic control of the autonomic motor system and the endocrine system.
    \item The peripheral components, the autonomic ganglia, connect with the spinal cord and brain stem and mediate simple reflexes.
\end{itemize}

\subsection{The Autonomic System Contains Visceral Motor Neurons that are Organized into Ganglia}
\begin{itemize}
    \item Overall, the nervous system has many more autonomic than somatic motor neurons.
    \item The autonomic system contains three divisions:
        \begin{itemize}
            \item Sympathetic
            \item Parasympathetic
            \item Enteric
        \end{itemize}
    \item All neurons in the sympathetic, and parasympathetic ganglia are controlled by preganglionic neurons, which synthesize and release ACh.
    \item ACh acts on nicotinic ACh  reptors in the postganglionic neurons, which produce fast excitatory potentials that end in effector cells in \textit{end-organs}
    \item The sympathetic and parasympathetic are distinguished by five criteria:
        \begin{itemize}
            \item The segmental organization of their preganglionic neurons in the spinal cord and brain stem.
            \item The peripheral locations of their ganglia.
            \item The types and locations of end-organs they innervate
            \item The effcts they produce on end-organs.
            \item The neurotransmitters employed by their postganglionic neurons.
        \end{itemize}
    \item Preganglionic neurons are localized in three regions along the brain stem and spinal cord.
    \item Sympathetic ganglia project to many targets throughout the body.
    \item The sympathetic motor system regulates systemic physiological parametes such as blood pressure and body temperature.
    \item Parasympathetic ganglia innervate single organs and lie near to or within the end organs they regulate.
    \item The parasympathetic does not influence skin or skeletal muscle, except in the head (vascular beds in jaw, lip, and tongue).
    \item The enteric ganglia regulate the gastrointestinal tract.
    \item The enteric ganglia is the largest and most complex division of the autonomic nervous system. 
    \item Unlike the other divisions, the enteric division contain interneurons and sensory neurons in addition to motor neurons.
    \item Autonomic behavior is the product of cooperation between all three autonomic divisions.
\end{itemize}

\subsection{Autonomic and Endocrine Functions are Coordinated by Central Autonomic Network Centered in the Hypothalamus}
\begin{itemize}
    \item The sympathetic and parasympathetic responses are coordinated by a central autonomic network, a network of brain regions that interacts with two other systems to support homeostasis.
    \item The two systems critical control elements are centered in the hypothalamus.
    \item General visceral sensory information reaches the central autonomic network through two cranial nerves, which end in the nucleus of the solitary tract. 
    \item The nucleus of the solitary tract has tro basic functions:
        \begin{itemize}
            \item Project to networks in the brain stem and spinal cord that control and coordinate automic reflexes.
            \item Contains ascending projections that integrate autonomic with neuroendocrine and behavioral responses.
        \end{itemize}
    \item The nucleus of the solitary tract has direct and indirect projections to the forebrain.
    \item \textbf{Parabrachial nucleus}: a major indirect target important for behavioral responses to visceral information as well as taste.
    \item The periaqueductal gray surrounds the cerebral aqueduct in the midbrain, which also receives inputs from most parts of the central autonomic network and projects to the medullary reticular formation to initiate integrated behavioral and autonomic responses.
    \item  Viscerosensory and gustatory information is relayed from the nucleus of the solitary tract and parabrachial nucleus in axons that end topographically in a specialized part of the thalamus.
    \item Visceral regions of cortex, along with many subcortical parts of the central autonomic network, interact with the amygdala.
\end{itemize}

\subsection{Hypothalamus Integrates Autonomic, Endocrine, and Behavioral Responses}
\begin{itemize}
    \item The functions of the hypothalamus can be enhanced or eliminated when particular site are experimentally manipulated.
    \item Magnocellular neuroendocrine neurons control the pituitary gland directly.
    \item Parvicellular neuroendocrine neurons control the pituitary gland indirectly.
    \item The hypothalamus integrates behavioral, autonomic, and neuroendocrine responses in six vital functions:
        \begin{itemize}
            \item Blood pressure and electrolyte composition.
            \item Energy metabolism.
            \item Reproductive (sexual and parental) behaviors.
            \item Body temperature.
            \item Defensive behavior.
            \item Sleep-wake cycle.
        \end{itemize}
\end{itemize}
%\endgroup
%%%%%%%%%%%%%%%%%%%%%%%%%%%%% Chapter 47 %%%%%%%%%%%%%%%%%%%%%%%%%%%%%


%%%%%%%%%%%%%%%%%%%%%%%%%%%%% Chapter 48 %%%%%%%%%%%%%%%%%%%%%%%%%%%%%
%\begingroup
\clearpage
\section{Emotions and Feelings}\phantomsection
\subsection{The Amygdala Emerged as a Critical Regulatory Site in Circuits of Emotions}
\begin{itemize}
    \item Studies in avoidance conditioning first implicated the amygdala in fear responses.
    \item Pavlovian conditioning is ued extensively to study the amygdala and learned fear.
    \item Animals with amygdala fail to learn association between conditioned stimuli and shock.
    \item Amygdala and the cortex are activated simultaneously, but the amygdala is able to respond to danger before the cortex can process the information.
    \item The central amygdala drives motor outputs and also part of the circuitry where fear associations are formed and stored.
    \item The amygdala also has been implicated in innate fears.
    \item The amygdala is also involved in positive emotions in animals and humans.
\end{itemize}

\subsection{Other Brain Areas Contribute to Emotional Processing}
\begin{itemize}
    \item The amygdala contribute to emotional processing as part of a larger circuit that includes regions of the hypothalamus and brain stem.
    \item Studies in humans have implicated the ventral region of the anterior cingulate cortex, the insular cortex, and the ventromedial prefrontal cortex in various aspects of emotional processing.
    \item Damage to some sectors of the prefrontal cortex impairs social emotions and related feelings.
    \item The prefrontal cortex in the ventromedial sector is particularly important; some of these areas project extensively to subcortical areas related to emotions: the amygdala, hypothalamus, and the periaqueductal gray region in the brain stem tegmentum.
    \item During emotional response, the ventromedial areas govern attention accorded to certain stimuli, shaping mental plans and altering cognitive processes.
\end{itemize}

\subsection{The Neural Correlates of Feeling Are Beginning to be Understood}
\begin{itemize}
    \item Damage to the right somatosensory cortex impairs social feelings such as empathy. Lesions in this area cause patients to fail to guess the feelings behind facial expressions.
    \item Damage to the human insular cortex, especially the left, can suspend addictive behaviors.
    \item Complete bilateral damage to the insular cortex does not preclude emotional feelings or body feelings, suggesting the somatosensory cortices and subcortical nuclei in the hypothalamus and brainstem are involved in generating feeling states.
\end{itemize}
%\endgroup
%%%%%%%%%%%%%%%%%%%%%%%%%%%%% Chapter 48 %%%%%%%%%%%%%%%%%%%%%%%%%%%%%

%%%%%%%%%%%%%%%%%%%%%%%%%%%%% Chapter 49 %%%%%%%%%%%%%%%%%%%%%%%%%%%%%
%\begingroup
\clearpage
\section{Homeostasis, Motivation, and Addiction}\phantomsection
\subsection{Drinking Occurs Both in Response to and in Anticipation of Dehydration}
\begin{itemize}
    \item \textbf{Primary drinking}: behaviors associated with the insistent discomfort of thirst.
    \item \textbf{Secondary drinking}: drinking in excess, even in absence of error signal. Often coincides with feeding.
    \item Body water is partitioned between intracellular and extracellular compartments, which are regulated differently.
    \item The principal extracellular cation is \ch{K^+}, whereas intracellular is \ch{Na^+}.
    \item Fluctuation is generally more pronounced in \ch{Na^+}.
    \item The intravascular compartment is monitored by parallel endocrine and neural sensors.
    \item The intracellular compartment is monitored by \textbf{osmoreceptors}, which are specialized neurons that translate cell shrinkage or swelling into membrane potential.
    \item Motivational systems anticipate the appearance and disappearance of error signals.
\end{itemize}

\subsection{Energy Stores are Precisely Regulated}
\begin{itemize}
    \item Leptin and insulin contribute to long-term energy balance.
    \item Lesions of the paraventicular nucleus increase food intake and body weight, whereas lesions in the lateral hypothalamic area produce opposite effects. 
    \item Both long and short term signals interact to control feeding.
\end{itemize}

\subsection{Motivational States Influence Goal-Directed Behavior}
\begin{itemize}
    \item Both internal and external stimuli contribute to motivational states.
    \item Internal inputs include physiological error signals and the circadian clock.
    \item External inputs include incentive stimuli, and vary widely. 
    \item Motivational states serve both regulatory and nonregulatory needs, with physiological errors driving force for nonregulatory (mating, socializing, exploritory) needs.
    \item Brain reward circuitry may provide a common logic for goal selection.
    \item Neural mechanisms for gaol selection must weigh anticipate risks, costs, and benifits of behaviors.
\end{itemize}
%\endgroup
%%%%%%%%%%%%%%%%%%%%%%%%%%%%% Chapter 49 %%%%%%%%%%%%%%%%%%%%%%%%%%%%%

%%%%%%%%%%%%%%%%%%%%%%%%%%%%% Chapter 50 %%%%%%%%%%%%%%%%%%%%%%%%%%%%%
%\begingroup
\clearpage
\section{Seizures and Epilepsy}\phantomsection
\subsection{Classification of Seizures and the Epilepsies is Important for Pathogenesis and Treatment} 
\begin{itemize}
    \item Seizures can be either focal (partial) or generalized.
    \item Focal seizures originate in a small group of neurons, and the symptoms depend of the location.
    \item \textbf{Auras}: preceding symptoms of a focal seizure, resulting in variety of abnormal sensenations depending on focal point.
    \item \textbf{Postictal period}: time between end of seizure and normalcy. 
    \item Focal seizures can escalate into \textit{secondary generalized seizures}.
    \item \textit{Primary generalized seizures} begin with out an aura and involve both hemispheres from the onset. They can be either convulsive or nonconvulsive.
    \item \textbf{Epilepsy} is the chronic condition of recurrent seizures.
\end{itemize}
%\endgroup
%%%%%%%%%%%%%%%%%%%%%%%%%%%%% Chapter 50 %%%%%%%%%%%%%%%%%%%%%%%%%%%%%

%%%%%%%%%%%%%%%%%%%%%%%%%%%%% Chapter 51 %%%%%%%%%%%%%%%%%%%%%%%%%%%%%
%\begingroup
\clearpage
\section{Sleep and Dreaming}\phantomsection
\subsection{Sleep Consists of Alternating REM and Non-REM Periods}
\begin{itemize}
    \item Non-rem sleep has four stages:
        \begin{itemize}
            \item Stage 1: transition between wake and sleep.
            \item Stage 2: marked in EEG by the onset of spindle waves; muscle tone decreases, eyes slowly roll back and forth, respiration slows and regularizes, and body temperature falls.
            \item Stage 3: marked by the appearance of a significant fraction of delta wave oscillations, with increased synchronization of cortical and thalamic activity.
            \item Stage 4: Deepest stage, with delta waves > 50\%; heart rate slows, muscles relax, temperature drifts even lower.
        \end{itemize}
    \item The first four stages take roughly 30 minutes, then enters REM sleep.
    \item REM sleep occupies appropriately 25\% of total sleep time.
    \item Dreams can occur in both REM and non-REM sleep, but differnet. With non-rem dreams being shorter, less visual, less emotional, and more conceptual.
\end{itemize}
%\endgroup
%%%%%%%%%%%%%%%%%%%%%%%%%%%%% Chapter 51 %%%%%%%%%%%%%%%%%%%%%%%%%%%%%
%\endgroup

\clearpage
\fancyhead[L]{Part VIII Development}
%\begingroup

%%%%%%%%%%%%%%%%%%%%%%%%%%%%% Chapter 52 %%%%%%%%%%%%%%%%%%%%%%%%%%%%%
%\begingroup
\clearpage
\section{Patterning the Nervous System}
    This chapter has been intentionally left blank. Some notes were taken initially, but were shallow and unconnected so they were omitted.
%\endgroup
%%%%%%%%%%%%%%%%%%%%%%%%%%%%% Chapter 52 %%%%%%%%%%%%%%%%%%%%%%%%%%%%%

%%%%%%%%%%%%%%%%%%%%%%%%%%%%% Chapter 53 %%%%%%%%%%%%%%%%%%%%%%%%%%%%%
%\begingroup
\clearpage
\section{Differentiation and Survival of Nerve Cells}\phantomsection
\subsection{The Proliferation of Neural Progenitor Cells Involves Symmetric and Asymmetric Modes or Cell Division}
\begin{itemize}
    \item Regulation of the proliferation of neural cells is a major driving force in shaping brain development.
    \item Asymmetric division: the progenitor produces one differentiated daughter and one retains stem cells properties.
    \item Symmetric: produces two stem cells, leading amplification of progenitor cells.
    \item Radial glial cells are the earliest morphologically distinguishable cell type to appear within the primitive neural epithelium.
    \item Radial glial cells serve as neural progenitors, that generate both neurons and astrocytes, which makes allows them to also be structural scaffolds for migration of neurons that emerge from the ventricular zone.
\end{itemize}

\subsection{Neuronal Migration Establishes the Layered Organization of the Cerebral Cortex}
\begin{itemize}
    \item The mammalian cerebral cortex develops in three main stages: preplate, a cortical plate, then into a mature pattern of layers.
    \item The laminar settling position correlates precisley with its \textit{birthday}, which is the time at which a dividing precursor undergoes its final round ofo cell division and becomes a postmitotic neuron.
    \item Cells that exit later migrate further, resulting in a inside-first, ourside-last rule.
\end{itemize}

\subsection{Central Neurons Migrate Along Glial Cells and Axons to Reach Their Final Settling Position}
\begin{itemize}
    \item Migration of neurons follows one of three programs:
        \begin{itemize}
            \item \textit{Radial}: movement along unbranched processes of radial glial cells.
            \item \textit{Tangentail}: use of axonal tracts as guides.
            \item \textit{Free}: occurs in peripheral without radial glia or axonal tracts.
        \end{itemize}
    \item Radial glial scaffolds are especially important in development of the primate cortex, where neurons are required to migrate over long distances as cerebral wall expands.
    \item Free migration requires significant cytoarchitectural and cell adhesive changes and differs from most of the migratory events in the central nervous system.
\end{itemize}

\subsection{The Neurotransmitter Phenotype of a Neuron is Plastic}
\begin{itemize}
    \item The final transmitter phenotype is determined in part by the the end organs they innervate.
    \item Transcription facts control the phenotype of a central neuron.
\end{itemize}

\subsection{The Survival of a Neuron is Regulated by Neurotrophic Signals from the Neurons's Target}
\begin{itemize}
    \item The target of a neuron is a key sourceo of factors essential for the neuron's survival.
    \item Approximately half of allmotor neurons generated in the spinal cord die during embryonic developmet.
    \item \textbf{Neurotrophic factor hypothesis}: cells at or near the target of a neuron secrete small amounts of an essential nutrient or trophic factor, and is needed for survival.
    \item Discovery of nerve growth factor (NGF) confrimed the hypothesis.
    \item We now know of dozens of factors, but neurotrophins are the most studied.
    \item Neurotrophic factors surpress apoptosis.
\end{itemize}
%\endgroup
%%%%%%%%%%%%%%%%%%%%%%%%%%%%% Chapter 53 %%%%%%%%%%%%%%%%%%%%%%%%%%%%%

%%%%%%%%%%%%%%%%%%%%%%%%%%%%% Chapter 54 %%%%%%%%%%%%%%%%%%%%%%%%%%%%%
%\begingroup
\clearpage
\section{Growth and Guidance of Axons}\phantomsection
\subsection{Differences in the Molecular Properties of Axons and Dendrites Emerge Early in Development}
\begin{itemize}
    \item Neuronal polarity is established through rearrangements of the cytoskeleton.
    \item Axonal specification is a key event in neuronal polarization and signals from newly formed axons both suppress generation of additional axons and promote dendrite fromation.
    \item Dendrites are patterned by intrinsic and extrinsic factors.
    \item Intrinsic transcriptional programs that specify subtype presumably encode neuronal shape.
    \item A second mechanism for establishing the pattern of dendritic arbors is the recognition of one dendrite by others in the same cell; one function repels growth of same cell dendrites, which helps ensure that dendrites expand rather than bunch together. 
    \item \textbf{Tiling}: covering of surface with minimal dendritic overlap due to inhibitory interactions with a particular class of neuron.
\end{itemize}

\subsection{The Growth Cone is a Sensory Transducer and a Motor Structure}
\begin{itemize}
    \item \textbf{Growth cone}: a specialized structure at the tip of an axon that is the key neuronal element responsible for axonal growth.
    \item Both axons and denrites use growth cones, but those linked toa axons have been studied closer.
    \item The growth cone is both a sensory structure that receives directional cues and a motor structure that drives axon elongation.
    \item The growth cone guides the axon by transducing positive and negative cues into signals that regulate the cytoskeleton.
    \item Growth cones have three main compartments:
        \begin{itemize}
            \item \textbf{Central core}: rich in microtubules, mitochondria, and other organelles.
            \item \textbf{Filopodia}: long slender extensions that project from the body of the growth cone and used to sense environment.
            \item \textbf{Lamelipodia}: lies between filopodia and are also motile and give the growth cone  a ruffled appearance.
        \end{itemize}
    \item Filopodia are tens of micrometers in some cases and very flexible, which allows sampling of environments far in advance of the central core while avoiding obstacles.
\end{itemize}

\subsection{Molecular Cues Guide Axons to Their Targets}
\begin{itemize}
    \item Guidance cues are presented on the cell surface, in extracellular matrix, or in soluble form.
    \item Cue act by either promoting or inhibiting outgrowth or the axon.
    \item Most receptors are embedded in the growth cone membrane; they have an extracellular domain that selectively binds to cognate ligand and an intracellular domain that couples to the cytoskeleton, either directly or indirectly.
    \item Axonal guidance is now viewed as the orderly consequence of protein-protein interactions that instruct the growth cone to grow, turn, or stop.
    \item Essentially, axonal guidance is explained by how and where ligands are presented and how the interactions are intergrated.
\end{itemize}
%\endgroup
%%%%%%%%%%%%%%%%%%%%%%%%%%%%% Chapter 54 %%%%%%%%%%%%%%%%%%%%%%%%%%%%%

%%%%%%%%%%%%%%%%%%%%%%%%%%%%% Chapter 55 %%%%%%%%%%%%%%%%%%%%%%%%%%%%%
%\begingroup
\clearpage
\section{Formation and Elimination of Synapses}\phantomsection
\subsection{Recognition of Synaptic Targets is Specific}
\begin{itemize}
    \item Recognition molecules promote selective synapse formatimon.
    \item Reestablishment of selectivity in adults after nerve damage shows that specificity does not emerge through peculiarrites of embryonic timing or neuronal positioning.
    \item Retinal ganglion neurons form layer-specific synapses.
    \item Olfactory receptors also help the axon to form appropriate synapses on target neurons.
    \item Different synaptic inputs are directed to discrete domains of the postsynaptic cell through specificity on molecular on their cell surface.
    \item Each neuronal subtype must express a variety of synaptic recognition molecules.
    \item Neural activity sharpens and refines synaptic specificity.
    \item Molecular cues control initial specificity, but neural activity sharpens and solidifies the connection.
    \item Neural activity can fix inappropriate connections into correct ones in some cases.
    \item \textbf{Motor unit homogeneity}: when branches of individual motor axon innervate muscle fibers of a single type, despite in situations with muscle fibers of a variety of types.
    \item It is possible that central axons can modify the properties of their synaptic targets, but not a major factor.
    \item Glial cells help promote synapse formation through both surface and secreted molecules. 
\end{itemize}

%\endgroup
%%%%%%%%%%%%%%%%%%%%%%%%%%%%% Chapter 55 %%%%%%%%%%%%%%%%%%%%%%%%%%%%%

%%%%%%%%%%%%%%%%%%%%%%%%%%%%% Chapter 56 %%%%%%%%%%%%%%%%%%%%%%%%%%%%%
%\begingroup
\clearpage
\section{Refinement of Synaptic Connections}\phantomsection
\subsection{Development of Human Mental Function is Influenced by Early Experience}
\begin{itemize}
    \item \textbf{Critical period}: Necessary periods of development where atypical use or deprivation may lead irreversible effects. 
    \item Early experience has lifelong effects on social behaviors.
    \item Lack of interaction and usage of early mental functions can cause lasting negative effects.
    \item Development of visual perception requires visual experience in early life; altered experience can affect structure and function of the visual cortex.
    \item Development of binocular circuits in the visual cortex depends on postnatal activity.
    \item Partial deprivations of sensory inputs in early life perturbs the function of segretating simultaneous inputs.
    \item Competition between two sets of afferent axons for the same population of cortical neurons drives their segregation into distinct target territories.
\end{itemize}

\subsection{Reorganization of Visual Circuits During a Critical Period Involves Alterations in Synaptic Connections}
\begin{itemize}
    \item Reorganization depends on a change in the balance of excitatory and inhibitory inputs.
    \item Loss of responsiveness to the deprived eye occurs only if the other eyes remains active, though the loss occurs due to circuit alteration rather than due to simple loss of input.
    \item Three reasons have been proposed to explain the observation above:
        \begin{itemize}
            \item Excitatory synapses may decrease in strength and undergo long-term depression(LTD).
            \item Inhibitory inputs may become stronger, leading to net decrease in the level of cortical neurons by inputs from the closed eye.
            \item An increase in inhibition within the cortex may alter in a more subtle way, pushing towards effects like LTD.
        \end{itemize}
    \item Balance of intrcortical excitation and inhibition is required for reorganization during the critical period in which monocular deprivation elicites changes.
    \item Postsynaptic structures are rearranged during the critical period.
    \item Spines, the small protrusions from denrites where excitatory synapses form, are dynamic structures. 
    \item Spine motility reflects formation and elimination of synapses and is associated with changes of behavior.
    \item Thalamic inputs may also be remodeled.
    \item Synaptic alterations precede large-scale axonal rearrangements, suggesting axonal remodeling may contribute making changes enduring and irreversible.
    \item Synaptic stabilization contributes to closing the critical period.
\end{itemize}

\subsection{Segregation of Retinal Inputs in the Lateral Geniculate Nucleus is Driven by Spontaneous Neural Activity in Utero}
\begin{itemize}
    \item The arbors of retinal ganglion cells from the two eyes are segregated into alternating layers, similar to alternating ocular dominance columns in the visual cortex.
    \item The segregation of inputs is completed before birth in the lateral geniculate nucleus; therefore vision does not drive segregation.
    \item Instead, retinal ganglion neurons are spontaneously firing and creating synchronous groups that strengthen synapses at the expense of other nearby synapses.
    \item Auditory maps are refined during a critical period similar to vision, though they are partially informed by vision.
    \item Distinct regions of the brain have different cirtical periods of development, even within the visual cortex.
\end{itemize}

\subsection{Activity-Dependent Refinement of Connections is a General Feature of Circuits in the Central Nervous System}
\begin{itemize}
    \item Critical periods are less defined than previously thought, and that they can be extended. 
    \item Behavioral context affects the ability of the nervous system to reorganize, though whether it's from increased sensory information, attention, arousal, motivation, reward, or a combination, need to be resolved.
    \item Some evidence shows that critical periods cn be reopened.
\end{itemize}
%\endgroup
%%%%%%%%%%%%%%%%%%%%%%%%%%%%% Chapter 56 %%%%%%%%%%%%%%%%%%%%%%%%%%%%%

%%%%%%%%%%%%%%%%%%%%%%%%%%%%% Chapter 57 %%%%%%%%%%%%%%%%%%%%%%%%%%%%%
%\begingroup
\clearpage
\section{Repairing the Damaged Brain}\phantomsection
\subsection{Damage to Axons Affects Neurons and Neighboring Cells}
\begin{itemize}
    \item Most injuries to the central or peripheral nervous system involve damage to axons.
    \item \textbf{Axotomy}: transection of an axon due to cutting or crushing.
    \item Axon degeneration is an active process.
    \item Expression a mutant \textit{Wlds} protein can significantly delay distal axonal loss and even extend the life span of mice with motor neuron diseases.
    \item \textbf{Chromatolytic reaction}: series of cellular and biochemical changes tht the proximal portion of the damaged axon.
    \item Changes due to Chromatolytic reaction are reversible if regeneration is successful.
    \item Axotomy causes glial cells the fragment and remove the myelin sheath that ensheths the distal nerve.
    \item Schwann cells then divide and secrete factors that recruit macrophages from the blood stream to dispose of debris, as well as produce growth factors that promot axon regeneration.
    \item \textbf{Microglia}: resident macrophages that dispose of myelin in events where they cannot be recruited due to blood-brain barrier, explains slower degeneration in the central nervous system.
    \item Postsynaptic neurons can die or beecome denervated in response to axotomy.
    \item Presynaptic neurons can undergo \textit{synaptic stripping}, which depresses activity can can impair recovery.
\end{itemize}

\subsection{Central Axons Regenerate Poorly After Injury}
\begin{itemize}
    \item Central and peripheral nerves differ substantially in their ability to regenerate after injury.
    \item Peripheral regeneration can often be reversed, but it isn't always perfect; fine movements in the motor system of usually impaired.
    \item Central nervous system in contrast is often poor, with long distance regeneration being rare.
\end{itemize}

\subsection{Therapeutic Interventions Promote Regeneration of Injured Central Neurons}
\begin{itemize}
    \item Axons from multiple regions that normally repair poorly were shown to repair well if provided with a suitable environment.
    \item Regrowth of central axons is intrinsically limited.
    \item Peripheral cells are shown to provide growth-promoting factors to injured areas. 
    \item Components of Schwann cell basal laminae and cell adhesion molecules of the immunoglobulin superfamily have been identified constituents for potent promoters of neurite outgrowth. 
    \item Denervated distal stumps produce neurotrophins and other trophic molecules that interact, nourish, and guide growing axons. 
    \item Components of myelin can inhibit neurite outgrowth.
    \item Injury-induced scarring hinders axonal regeneration.
    \item Even though central axons can regenerate in peripheral nerves, they still grow much less than peripheral in the same path; suggesting adult central axons may be less capable than peripheral in general. 
    \item Formation of new connections by intact axons can lead to functional recovery.
    \item Neurons in the injured brain can die, but new ones can be born.
\end{itemize}

%\endgroup
%%%%%%%%%%%%%%%%%%%%%%%%%%%%% Chapter 57 %%%%%%%%%%%%%%%%%%%%%%%%%%%%%
%\endgroup

\clearpage
\fancyhead[L]{Part IX Cognitive Behavior}
%\begingroup

%%%%%%%%%%%%%%%%%%%%%%%%%%%%% Chapter 60 %%%%%%%%%%%%%%%%%%%%%%%%%%%%%
%\begingroup
\setcounter{section}{59}
\clearpage
\section{Language}\phantomsection
\subsection{Language Acquisition}
\begin{itemize}
    \item Children initially exhibit universal patterns of speech perception and production that does not depend on culture or language.
    \item \textbf{Categorical perception}: ability to discern slight acoustic changes at boundaries between phonetic categories.
    \item Categorical perception has been shown to exist in chinchillas and monkeys, suggesting language development is strongly influenced by preexisting auditory structures.
    \item By end of the first year, infants fail to discriminate phonetic changes if not exposed frequently.
    \item Social interaction appears to play an essentail role in learning, whereas passive observation is significantly less effective.
    \item \textbf{Prosodic}: cues of pitch, duration, and loudness.
    \item Infants (7-8 months) using transitional probability predictions to identify continuous speech.
    \item \textbf{Neural commitment}: priority to learning distinctions in native language that interferes with secondary language learning and enhances primary learning.
\end{itemize}

\subsection{Cortical Regions Involved in Language Processing}
\begin{itemize}
    \item The left hemisphere is specialized for phonetic, word, and sentence processing.
    \item Prosodic information engages both right and left hemispheres depending on information conveyed, with emotional changes being best example of engaging the right hemisphere.
    \item Discourse and generalization of language into full meaning also takes place in the right hemisphere.
    \item New evidence suggests there are three large systems interconnected to process language:
        \begin{itemize}
            \item \textit{Implemenetation system}: analyzes incoming auditory signals to activate conceptual system and supports phonemic and grammatical construction, as well as speech production.
            \item \textit{Mediational system}: numerous separate regions in temporal, parietal, and frontal association areas that act as brokers between the implemenetation and conceptual system.
            \item \textit{Conceptual system}: collection of regions distrubted throughout the association areas.
        \end{itemize}
\end{itemize}
%\endgroup
%%%%%%%%%%%%%%%%%%%%%%%%%%%%% Chapter 60 %%%%%%%%%%%%%%%%%%%%%%%%%%%%%

%%%%%%%%%%%%%%%%%%%%%%%%%%%%% Chapter 65 %%%%%%%%%%%%%%%%%%%%%%%%%%%%%
%\begingroup
\clearpage
\setcounter{section}{64}
\section{Learning and Memory}\phantomsection
\subsection{Short-Term and Long-Term Memory Involve Different Neural Systems}
\begin{itemize}
    \item Working memory consits of at lest two subsystems, one for verbal and another for visuospatial information.
    \item The two subsystems are coordinated by a third system called the \textbf{executive control process}.
    \item Short-term memory is selectively transferred to long-term memory.
    \item Different subregions of the medial temporal lobe may not have equivalent roles. For example, some argue that the hippocampus may be relatively more important for spaital representation than for object recognition or vice versa depending on hemisphere.
    \item Long-term memory is either implicit (procedural) or explicit (declarative).
    \item Explicit is highly flexible, while implicit is highly connected to the original conditions under which the learning occurred.
\end{itemize}

\subsection{Explicit Memory Has Episodic and Semantic Forms}
\begin{itemize}
    \item The brain does not have a single long-term store of explicit memories, instead knowledge is widely distributed.
    \item Explicit memory is mediated by at least four related but distinct types of processing:
        \begin{itemize}
            \item \textit{Encoding}: process by which new information is attended and linked to existing information in memory. Conscious motivation can increase depth of encoding.
            \item \textit{Storage}: the neural mechanisms and sites by which memory is retained over time.
            \item \textit{Consolidation}: process that makes temporarily stored and still labile information more stable. 
            \item \textit{Retrieval}: process by which stored information is recalled. Retrieval is much like perception; it is a constructive process.
        \end{itemize}
    \item Retrieval is most effective when the cue is most similar to when the memory was encoded.
    \item Evidence suggests that the frontal lobe and medial temporal lobe processing contribute to encoding episodic memory.
    \item Semantic knowledge is stored in distinct association cortices and retrieval depends on the prefrontal cortex.
    \item Semantic knowledge is stored in distrubted manner in the nerocortex, including lateral and ventral temporal lobes.
\end{itemize}

\subsection{Implicit Memory Supports Perceptual Priming}
\begin{itemize}
    \item \textbf{Conceptual priming}: provides easier access to task-relevant semantic knowledge due to previous use.
    \item \textbf{Perceptual priming}: depends on cortical modules that operate on sensory information about the form and structure of objects.
    \item Most tasks include both types of priming; there is no sharp distinction between the two.
    \item Implicit memory also underlies Pavlovian associative conditioning. 
    \item Implicit memory can be associative or nonassociative.
    \item Nonassociative results in increased or decreased response to a single stimulus.
    \item Classical conditioning involves associating two stimuli.
    \item Operant conditioning can be considered as the formation of a predictive relationship between an action and outcome. 
\end{itemize}
%\endgroup
%%%%%%%%%%%%%%%%%%%%%%%%%%%%% Chapter 65 %%%%%%%%%%%%%%%%%%%%%%%%%%%%%

 %%%%%%%%%%%%%%%%%%%%%%%%%%%%% Chapter 67 %%%%%%%%%%%%%%%%%%%%%%%%%%%%%
%\begingroup
\setcounter{section}{66}
\clearpage
\section{Explicit Memory Storage}\phantomsection
\subsection{Working Memory Depends on Persistent Neural Activity in the Prefrontal Cortex}
\begin{itemize}
    \item Neurons in the prefrontal cortex fire persistently, presumably contributing to the neural representation of the image in working memory.
    \item Intrinsic membrane properties can generate persistant activity, from seconds to minutes after end of the stimulus. 
    \item Persistent firing is not affected by blockers of fast excitatory and inhibitory synaptic transmission. 
    \item Network connections are another mechanism for sustained activity.
    \item \textbf{Reverberatory}: process of firing through chained neural circuits that maintains activity in a population. 
    \item \textbf{Disinhibition}: a mechanism of positive feedback between reciprocal inhibitory synapses between two populations of neurons that leads to the sustained firing of the first population.
    \item Working memory depends on the activation of the D\(_{1}\) dopamine receptors.
    \item Working memory is most efficacious at intermediate levels of (D\(_{1}\)) activation.
\end{itemize}

\subsection{Explicit Memory in Mammals Involves Different Forms of Long-Term Potentiation in the Hippocampus}
\begin{itemize}
    \item Long-term storage of memory is thought to depend on long-lasting changes in the strength of synaptic connections.
    \item Information from the entorhinal cortex reaches CA1 along two excitatory pathways, one direct and one indirect, termed \textbf{preforant pathways}.
    \item CA1 neurons are thought to compare information from both pathways, as both are required to be functional in order for normal learning and memory.
    \item \textbf{Long-term potentiation (LTP)}: a long-lasting increase in the amplitude of the excitatory postsynaptic potentials in the dentate granule neurons.
    \item LTP is not a single form of synaptic plasticity, rather it comprises a family of processes that strengthen synaptic tranmission through distinct cellular and molecular mechanisms.
    \item Spatial memory depends on LTP in the hippocampus.
    \item Different subregions ot the hippocampus are required for pattern separation and completion.
\end{itemize}

\subsection{Memory Also Depends on Long-Term Depression of Synaptic Transmission}
\begin{itemize}
    \item \textbf{Long-term depression (LTD)}: an inhibitory mechanism that downregulates synaptic function to counteract LTP.
    \item LTD is important for motor learning.
    \item Much less is known about the behavioral role of LTD compared to LTP.
    \item LTD may be needed for behavioral flexibility, preventing LTP saturation and palying an active role in memory storage.
\end{itemize}

\subsection{Are There Molecular Building Blocks for Learning?}
\begin{itemize}
    \item Epigenetic changes in chromatin structure are important for long-term synatptic plasticity, learning, and memory.
    \item Three key finding from cellular studies:
        \begin{itemize}
            \item The molecular mechanisms of some associative forms of synaptic plasticity are based on those of nonassociative forms in the same cell.
            \item Molecular mechanisms of elementary forms of associative learning, implicit anc explicit, are similar.
            \item Despite differences, implicit and explicit memory storage both rely on common multi-component genetic switch elements that convert labile short-term memory into long-term.
        \end{itemize}
\end{itemize}
%\endgroup
%%%%%%%%%%%%%%%%%%%%%%%%%%%%% Chapter 67 %%%%%%%%%%%%%%%%%%%%%%%%%%%%%
%\endgroup
\end{document}