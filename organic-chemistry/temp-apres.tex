\documentclass[12pt,a4paper]{article}
\usepackage{inverba}

\newcommand{\userName}{Cullyn Newman} 
\newcommand{\class}{[Subject]} 
\newcommand{\institution}{[Institution]} 
\newcommand{\theTitle}{\color{B-Cold} [Subject Title]}

\begin{document}
%%%%%%%%%%%%%%%%%%%%%%%%%%%%%%%%%%%%%%%%%%%%%%%%%%%%%%%%%%%%%%%%%%%%%
\tableofcontents
\cleardoublepage
\fancyhead{}
\fancyhead[R]{\hyperlink{home}{\nouppercase\leftmark}}
%%%%%%%%%%%%%%%%%%%%%%%%%%%%%%%%%%%%%%%%%%%%%%%%%%%%%%%%%%%%%%%%%%%%%
\clearpage
\section*{Week x}\phantomsection
\addcontentsline{toc}{section}{\textbf{Week x}}
\fancyhead[R]{\hyperlink{home}{Week x}}

\fancyhead[L]{\hyperlink{home}{Tueday, November 17 - Quiz 19}}
\subsection{Tueday, November 17 - Quiz 19}
\begin{enumerate}
    {\color{G-Moon}\item In an \(S_N2\) reaction in which \ch{OH^-} is the nucleophile, in which solvent will the rate will the rate (be) the fastest?
    \begin{itemize}
        \item {\color{o-Sun}\textbf{a polar aprotic solvent}}
        \item a nonpolar solvent
        \item a polar protic solvent
    \end{itemize}
    }
    \begin{itemize}
        \item \textbf{Polar aprotic solvents}: protic solvents lacking a hydrogen atom connected to electronegative atoms.
        \begin{itemize}
            \item Lack of hydrogen {\color{o-Sun}fails to stabilize added nucleophiles}, leaving compounds with higher potential energy and thus a {\color{o-Sun}lower activation energy} for a reaction to take place.
            \item \(S_N2\) reactions are generally much {\color{o-Sun}faster in polar aprotic solvents}.
        \end{itemize}
    \end{itemize}
    {\color{G-Moon}\item The carbon center in a alkyl halide is
    \begin{itemize}
        \item an unsaturated center
        \item a nucleophile
        \item a cation
        \item {\color{o-Sun}\textbf{an electrophile}}
    \end{itemize}
    }
    \begin{itemize}
        \item \textbf{Alkyl halides}: compounds in which a halogen (Cl, Br, I) is connected to sn \(sp^3\) hybridized carbon atom.
        \begin{itemize}
            \item Alkyl halides are electrophiles since they contain such electron deficient halogens---making them easily accept new electrons.
        \end{itemize}
    \end{itemize}
    {\color{G-Moon}\item Why is the rate of reaction for an \(S_N2\) reactions so much slower for a \ang{3} alkyl halide than for a \ang{1} alkyl halide?
    \begin{itemize}
        \item the leaving group is more reactive in an \ang{1} alkyl halide
        \item steric crowding is much less in \ang{3} alkyl halides
        \item because \(\Delta G^\circ\) is much smaller for a \ang{3} alkyl halide
        \item {\color{o-Sun}\textbf{steric crowding is much greater in \ang{3} alkyl halides}}
    \end{itemize}
    }
    \begin{itemize}
        \item The more substituents, the more bonds that need to be broken/changed and the more steric intereactions there are during the transition state---leading to higher activation energy and thus a slower reaction.
            \begin{itemize}
                \item Steric crowding refers to the steric intereactions that act to increases such interference.
                \item \ang{3} refers to the number of of $\beta$ positions (max \ang{3}, min \ang{1})
                    \begin{itemize}
                        \item \textbf{$\bm{\alpha}$ position}: the position connected directly next to the halogen.
                        \item \textbf{$\bm{\beta}$ position}: positions connected to the $\alpha$ position.
                    \end{itemize}
            \end{itemize}
    \end{itemize}
    {\color{G-Moon}\item At the transition state of an \(S_N2\) reaction reaction
    \begin{itemize}
        \item {\color{o-Sun}\textbf{the \ch{C-Nu} bond is \textit{partially} formed and the \ch{C-LG} bond is \textit{partially} broken.}}
        \item the \ch{C-Nu} bond is \textit{partially} formed and the \ch{C-LG} bond is \textit{completely} broken.
        \item the \ch{C-Nu} bond is \textit{completely} formed and the \ch{C-LG} bond is \textit{partially} broken.
        \item the \ch{C-Nu} bond is \textit{partially} formed and the \ch{C-LG} bond is \textit{completely in tact}.
    \end{itemize}
    }
    \begin{itemize}
        \item \textbf{Transition states}: represent local maxima of the reaction.
        \begin{itemize}
            \item Cannot be isolated.
            \item Represents high-energy states where bonds are being simultanesously broken and formed.
                \begin{itemize}
                    \item I.e., partially broken and partially formed.
                \end{itemize}
        \end{itemize}
    \end{itemize}
    {\color{G-Moon}\item The rate law expression for an \(S_N2\) reactions reaction has the form
    \begin{itemize}
        \item rate = \(k[\text{electrophile}]\)
        \item rate = \(k[\text{electrophile}]^2\)
        \item rate = \(k[\text{nucleophile}]^2\)
        \item {\color{o-Sun}\textbf{rate = \(k[\text{electrophile}][\text{nucleophile}]\)}}
    \end{itemize}
    }
    \begin{itemize}
        \item \textbf{Kinetics of \(S_N2\) reactions}: a biomolecular (2) nucleophilic (N) substitution (S) reaction. 
        \begin{itemize}
            \item \textbf{Biomolecular}: a step that involves two chemical entities, such as when the alkyl halide and nucleophile collide during the reaction mechanism. 
            \item Rate: \(v_0 = k[\text{alkyl halide}][\text{nucleophile}]\)
                \begin{itemize}
                    \item As mentioned above, alkyl halides are electrophiles.
                \end{itemize}
        \end{itemize}
    \end{itemize}
    \newpage
    {\color{G-Moon}\item What is the meaning of \(S_N2\)?
    \begin{itemize}
        \item substitution nucleophilic two transition states
        \item substitution nucleophilic two two reagents
        \item {\color{o-Sun}\textbf{substitution nucleophilic two second order}}
        \item substitution nucleophilic two twice
    \end{itemize}
    }
    \begin{itemize}
        \item Rate =\(k[A]^x[B]^y\)
            \begin{itemize}
                \item \textbf{Rate order}: the sum of exponents of the reactants.
                \begin{itemize}
                    \item E.g., \(kA=\) first, \(kAB=\) second, \(kA^2B=\) third.
                \end{itemize}
            \end{itemize}
    \end{itemize}
    {\color{G-Moon}\item Which would you expect to be the best nucleophile?
    \begin{itemize}
        \item \ch{F-}
        \item \ch{(CH3)2CH-}
        \item {\color{o-Sun}\textbf{\ch{CH3-}}}
        \item \ch{(CH3)3C-}
    \end{itemize}
    }
    \begin{itemize}
        \item \textbf{Nucleophilicity}: the rate at which a nucleophile will attack a suitable electrophile.
        \begin{itemize}
            \item There are multiple factors that contribute, but three main factors at play here: the role electron density, electronegativity, and steric hinderance (crowding). 
                \begin{itemize}
                    \item \ch{(CH3)2CH-} (\ang{2}) and \ch{(CH3)3C-} (\ang{3}) have more $\beta$ branching, thus more hinderance.
                    \item \ch{F-} electron density is so small that it causes low polarizability (more sable), which reduces reactivity.
                    \item Nucleophilicity also decreases as electronegativity increases (\(F>O>N>C\)), so carbon will be have higher Nucleophilicity.
                \end{itemize}
        \end{itemize}
    \end{itemize}
    {\color{G-Moon}\item In substitution reactions with alkyl halides,
    \begin{itemize}
        \item the nucleophile is the leaving group
        \item a hydrogen becomes the leaving group
        \item {\color{o-Sun}\textbf{the halide is the leaving group}}
        \item the electrophile is the leaving group
    \end{itemize}
    }
    \begin{itemize}
        \item I an alkyl halide, the halogen serves two critical functions that render the alkyl halide reactive:
        \begin{itemize}
            \item The halogen withdraws electron density via {\color{o-Sun}induction}, rending the adjacent carbon atom electrophilic, and therefore subject to attack.
            \item The halogen can serve as the {\color{o-Sun}leaving group} for the compound, vital for substitution or an elimination to occur.
            \begin{itemize}
                \item Good leaving groups are conjugate bases of strong acids, i.e., good groups are weak bases.
                \item Generally an acid with a \(pK_a < 0\) generates a stable enough base to be a good leaving group, which is another reason why F (\(pK_a\)of \ch{HF} is 3.2) is not one, despite being a halogen.
            \end{itemize}
        \end{itemize}
    \end{itemize}
    {\color{G-Moon}\item Of the following, which is the better nucleophile?
    \begin{itemize}
        \item \ch{H2O}
        \item \ch{NH3}
        \item \ch{RNH-}
        \item {\color{o-Sun}\textbf{\ch{NH2-}}}
    \end{itemize}
    }
    \begin{itemize}
        \item This question has more to do with the solvent effects on nucleophilicity.
            \begin{itemize}
                \item \textbf{Protic solvents}: polar solvent that {\color{o-Sun}contains} a hydrogen atom connected directly to an electronegative atom. 
                \item \textbf{Polar aprotic solvents}: protic solvents {\color{o-Sun}lacking} a hydrogen atom connected to electronegative atoms.
                \item Both \ch{H2O} and \ch{NH3} have are protic solvents, with hydrogen on the electronegative oxygen and nitrogen.
            \end{itemize}
        \item Comes down to \ch{RNH-} and \ch{NH2-}.
            \begin{itemize}
                \item I believe the R is a functional group, which I assume adds a greater degree of polarizability due to spread of electron density. 
                \item Although, I'm not completely sure, and I welcome a better explanation.
            \end{itemize}
    \end{itemize}
\end{enumerate}
\end{document}