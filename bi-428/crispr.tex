\documentclass[plain,basic]{inVerba-notes}

\newcommand{\userName}{Cullyn Newman}
\newcommand{\class}{BI:\@ 428}
\newcommand{\theTitle}{CRISPR Example}
\newcommand{\institution}{Portland State University}

\begin{document}
    \section{He Jiankui Affair}

    On Nov. 25, 2018 by the revelation that He Jiankui had used clustered regularly interspaced short palindromic repeats (‘CRISPR’) to edit embryos—two of which had, sometime in October, become living babies.

    \textbf{\large Background}
    \begin{itemize}
        \item He Jiankui, worked the Southern University of Science and Technology (SUSTech) in Shenzhen, China where he aimed to help people with HIV-related fertility problems, specifically involving HIV-positive fathers and HIV-negative mothers.
        \item The first babies, known by their pseudonyms Lulu and Nana, are twin girls born in October 2018, and the second birth or the third baby born was in 2019.
        \item Although there were not any specific laws in China on gene editing in humans, He Jiankui violated the available guideline on handling human embryos and received widespread criticism, and included concern for the girls' well-being
        \item Chinese authorities announced that he was found guilty of forging documents that allowed for circumvention of usual guidelines, as well as unethical conduct; he was sentenced to three years in prison with a three-million-yuan fine (US\$430,000).
    \end{itemize}
    \vspace{4pt}
    
    \textbf{\large Scientific Basis}
    \begin{itemize}
        \item It is known that the C-C chemokine receptor type 5 (CCR5) is a protein essential for HIV infection of the white blood cells by acting as co-receptor to HIV\@.
        \item Mutation in the gene CCR5, (called CCR5Δ32 because the mutation is specifically a deletion of 32 base pairs in human chromosome 3) renders resistance to HIV\@. 
        \item Resistance is higher in homozygous alleles, with weak protection in heterozygous alleles, however not all homozygote individuals are completely resistant. 
        \item Available sources indicate that Lulu and Nana are carrying incomplete CCR5 mutations.
            \begin{itemize}
                \item Nana carries a homozygous mutant gene with a 4-bp deletion and a single base insertion, failing to achieve complete 32-bp deletion.
                \item Lulu carries a mutant CCR5 that has a 15-bp in-frame deletion only in one chromosome 3, making her heterozygous. 
            \end{itemize}
        \item Because the babies' mutations are different from the typical CCR5Δ32 mutation it is not clear whether they are prone to infection.
        \item There are concerns of off-target mutation and mosaicism, causing a variety of different and irregular cell types to develop in the same embryo. He Jiankui was criticized for failing to take precautions to avoid these possibilities.
        \item CCR5 is linked to improved memory function in mice, as well as enhanced recovery from strokes in humans, meaning Lulu and Nana may have potential enhancements. He Juankui stated he was against editing for enhancements, but acknowledged there were studies that may have linked to enhance memory function. 
    \end{itemize}

    \nocite*{}
    \bibliographystyle{apacite}
    \bibliography{crispr.bib}
\end{document}