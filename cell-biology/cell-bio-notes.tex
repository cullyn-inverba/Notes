\documentclass[12pt,a4paper]{article}
\usepackage{inverba}
\newcommand{\userName}{Cullyn Newman} 
\newcommand{\class}{BI 336} 
\newcommand{\institution}{Portland State University} 
\newcommand{\thetitle}{\hypertarget{home}{Cell Biology Notes}}
\rfoot{\hyperlink{home}{\thepage}}

\begin{document}
%%%%%%%%%%%%%%%%%%%%%%%%%%%%%%%%%%%%%%%%%%%%%%%%%%%%%%%%%%%%%%%%%%%%%
\tableofcontents
\cleardoublepage
\fancyhead{}
\fancyhead[R]{\hyperlink{home}{\nouppercase\leftmark}}
%%%%%%%%%%%%%%%%%%%%%%%%%%%%%%%%%%%%%%%%%%%%%%%%%%%%%%%%%%%%%%%%%%%%%

%%%%%%%%%%%%%%%%%%%%%%%%%%%%% Chapter 1 %%%%%%%%%%%%%%%%%%%%%%%%%%%%%
%\begingroup
\clearpage
\setcounter{section}{14}
\section{Cell Signaling}
\subsection{Principles of Cell Signaling}
\begin{itemize}
    \item Extracellular signaling \(\rightarrow\) Receptor protein \(\rightarrow\) Intracellular signaling proteins \(\rightarrow\) Effector proteins.
    \item Effector proteins consists transcription regulators, ion channels, or metabolic enzymes.a
    \item More than 1500 genes encode for receptor proteins, and variation in the proteins are increased even more due to alternative RNA splicing and post-translational modifications.
\end{itemize}

\subsubsection{Extra Signals Can Act Over Short or Long Distances}
\begin{itemize}
    \item \textbf{Contact dependent signaling}: many extracellular signals are bound to the surface of cells that require direct contact from other membrane bound signaling cells in order for activation.
    \item \textbf{Local mediators}: secreted molecules that generally only act on cells in the local environment.
    \item \textbf{Paracrine signaling}: result of signaling using local mediators, and generally acts on cells of different types.
    \item \textit{Autocrine signaling}: results of local mediators acting on the cells that secret them.
    \item \textbf{Endocrine cells}: signaling over long distance using hormones that travel through the bloodstream. 
\end{itemize}

\subsubsection{Classes of Cell-Surface Receptor Proteins}
\begin{itemize}
    \item Most extracellular signals do not enter the nucleus, instead they act as signal transducers of extracellular ligand-binding events.
    \item \textbf{Ion-channel-coupled receptors}: or ionotropic receptors, are involved in rapid synaptic signaling between nerve cells of other electronically excitable cells.
    \item \textbf{G-protein-coupled receptors}: indirectly regulating separate membrane-bound target proteins, usually using enzymes or ion channels, resulting in small intracellular signaling molecules to alter the behavior of different signaling proteins in the cell.
    \item \textbf{Enzyme-coupled receptors}: either function as or directly associate with enzymes they activate. Usually are single pass transmembrane proteins that have ligand-binding site outside the cell and the catalytic site inside.
    \item Majority of enzyme are either protein kinases of associate with protein kinases.
\end{itemize}

\subsubsection{Cell-Surface Receptors Relay Signals Via Intracellular Signaling Molecules}
\begin{itemize}
    \item \textbf{Second messengers}: small chemicals involved in intracellular signaling that are generated in large amounts in response to receptor activation and diffuse to spread signals to other parts of the cell.
    \item Either water soluble or lipid soluble, resulting in an extension of the signal by binding/altering the behavior of selected signaling or effector proteins.
    \item Most intracellular molecules are proteins that act like a chain of molecular switches to transmit signals.
    \item \textbf{Protein kinase}: covalently adds one or more phosphate groups.
    \item \textbf{Protein phosphatase}: removes phosphate groups.
    \item 30-50\% of human proteins contain covalently attached phosphate.
    \item Many intracellular proteins are protein kinases themselves, resulting in \textbf{kinase cascades}.
    \item \textit{trimeric GTP-binding proteins}: help relay signals from G-proteins-coupled receptors that activate them.
    \item \textbf{monomeric GTPases} help relay signals from many classes of cell-surface receptors.
    \item \textbf{GTPase-activating proteins (GAPs)}: drive proteins into an "off" state by increasing rate of hydrolysis of bound GTP. 
    \item \textbf{Guanine nucleotide exchange factors (GETs)}: promote the release of bound GDP, allowing new GTP to bind, which activates the proteins.
\end{itemize}

\subsubsection{Modular Interaction Domans Mediate Interactions Between Inracellular Signaling Proteins}
\begin{itemize}
    \item \textit{Induced proximity}: the result of a signal triggering the assembly of a signaling complex, which can activate other proteins simply due to proximity.
    \item \textbf{Interaction domains}: highly conserved domains that are found in many intracellular signaling proteins and highly affect assembly of proximity complexes.
    \item The modular function helped evolution because the new pathways can be inserted with out disturbing the proteins folding of function.
    \item \textbf{Adaptors}: two or more interaction domains that link two proteins together in a signaling pathway.
    \item \textbf{Primary cilium}: an example of how to bring receptors and intracellular signaling together by concentrating them in them specific region of the cell (an antenna like object that projects from the surface of most vertebrate cells).
\end{itemize}

\subsubsection{The Reelationship Between Signal and Response Varies in Different Signaling Pathways}
\begin{itemize}
    \item All signaling systems do not work in the same way and have a variety of response factors.    
    \item \textit{Response timing}: can range from milliseconds to hours or days.
    \item \textit{Sensitivity}: controlled by changes to numbers or affinity of the receptors on the target cell and often further influenced by amplification factors.
    \item \textit{Dynamic range}: related to sensitivity. More focused on the broader range sensitivity, responding to the changes in sensitivity itself, which is know as \textbf{adaptation} to the responsiveness according to prevailing amounts of signal.
    \item \textit{Persistance}: how long signaling is active, regulated by numerous mechanisms, including positive feedback.
    \item \textit{Signal processing}: converting simple signals into complex responses; regulated by switch like or oscillatory responses.
    \item \textit{Integration}: allows a response to be governed by multiple inputs; equivalent to AND gates.
    \item \textit{Coordination}: multiple responses due to a single extracellular signal. Depends on mechanisms for distributing a signal to multiple effectors, even sometimes modulating the strength of a response to other signals.
\end{itemize}
\subsection{Signaling Through G-Protein-Coupled Receptors}
\begin{itemize}
    \item \textbf{G-protein-coupled receptors (GPCRs)}: the largest family of cell-surface receptors and the mediators of most responses to cell signaling.
    \item The same signal molecule can activate many different GPCR family members.
    \item GPCRs consists of a single polypeptide chain that threads across the membrane layer seven times, forming a cylindrical structure, often with a ligand binding site in the center.
\end{itemize}

\subsubsection{Trimeric G Proteins Relay Signals From GPCRs}
\begin{itemize}
    \item \textbf{trimeric GTP-binding protein (G protein)}: activated when a receptor undergoes a conformational change and couples the receptor to enzymes or ion channels in the membrane.
    \item The G protein can be associated with the receptors before of after activation depending on the type.
    \item G proteins are composed of three subunits ---$\alpha$, $\beta$, and $\gamma$.
    \item Unstimulated state: $\alpha$ subunit has GDP bound and G protein is inactive.
    \item When GPCR is activated it acts like GEF and induces $\alpha$ subunit to release bound GDP.
    \item GTP binding induces conformational change in G$\alpha$ subunit, releasing G protein from receptor and dissociation of GTP-bound G$\alpha$ from G$\beta$$\gamma$ pair.
    \item \textbf{regulator of G protein signaling (RGS)}: proteins that act as a $\alpha$-subunit-specific GAPs and help shut off G-protein-mediated responses.
\end{itemize}
\subsubsection{Some G Proteins Regulate the Production of Cyclic AMP}
\begin{itemize}
    \item \textbf{Cyclic AMP (cAMP)}: acts as a second messenger in some pathways and helps with signal transduction.
    \item cAMP is synthesized from ATP by \textbf{adenylyl cyclase} and destroyed by \textbf{cyclic AMP phosphodiesterase}.
    \item Many extracellular signals work by increasing cAMP.
\end{itemize}

\subsubsection{Cyclic-AMP-Dependent Protein Kinase (PKA) Mediates Most of the Effects of Cyclic AMP}
\begin{itemize}
    \item \textbf{Cyclic-AMP-dependent protein kinase (PKA)}: activated by cAMP and used to regulates selected proteins through phosphorylating specific serines of threonines.
    \item PKA consists of a complex of two catalytic subunits and two regulatory subunits.
    \item Binding of cAMP causes the regulatory subunits to dissociate and are free to phosphorylate specific target proteins.
\end{itemize}

\subsubsection{Some G protins Signal Via Phospholipids}
\begin{itemize}
    \item \textbf{Phospholipase C-$\beta$ (PLC$\beta$)}: plasma membrane bound enzyme that many GPCRs exert their effects through.
    \item 
\end{itemize}
\subsection{Signaling Through Enzyme-Coupled Receptors}
\begin{itemize}
    \item 
\end{itemize}

\subsection{Alternative Signaling Routes in Gene Regulation}
\begin{itemize}
    \item 
\end{itemize}
%\endgroup
%%%%%%%%%%%%%%%%%%%%%%%%%%%%% Chapter 1 %%%%%%%%%%%%%%%%%%%%%%%%%%%%%

%%%%%%%%%%%%%%%%%%%%%%%%%%%%% Chapter 2 %%%%%%%%%%%%%%%%%%%%%%%%%%%%%
%\begingroup
\clearpage
\section{Chapter}
\subsection{Section}
\begin{itemize}
    \item \textbf{Word}: the part of the cell that separates the exterior and the interior of a cell with a semipermeable lipid bilayer. The plasma membrane regulates import and export of materials for the cell and includes various proteins that interact with other cells. 
    \begin{itemize}
        \item relates to word
    \end{itemize}
    \item \textit{Also Important}: the resulting structure of the spontaneous alignment of mostly amphiphilic phospholipids. 
\end{itemize}
%\endgroup
%%%%%%%%%%%%%%%%%%%%%%%%%%%%% Chapter 2 %%%%%%%%%%%%%%%%%%%%%%%%%%%%%
\end{document}