\documentclass[12pt,a4paper]{article}
\usepackage{inverba}

\newcommand{\userName}{Cullyn Newman} 
\newcommand{\class}{[Subject]} 
\newcommand{\institution}{[Institution]} 
\newcommand{\theTitle}{\color{B-Cold} [Subject Title]}

\begin{document}
%%%%%%%%%%%%%%%%%%%%%%%%%%%%%%%%%%%%%%%%%%%%%%%%%%%%%%%%%%%%%%%%%%%%%
\tableofcontents
\cleardoublepage
\fancyhead{}
\fancyhead[R]{\hyperlink{home}{\nouppercase\leftmark}}
%%%%%%%%%%%%%%%%%%%%%%%%%%%%%%%%%%%%%%%%%%%%%%%%%%%%%%%%%%%%%%%%%%%%%
\clearpage
\section*{Week x}\phantomsection
\addcontentsline{toc}{section}{\textbf{Week x}}
\fancyhead[R]{\hyperlink{home}{Week x}}

\fancyhead[L]{\hyperlink{home}{Tueday, November 17 - Quiz 19}}
\subsection{Tueday, November 17 - Quiz 19}
\begin{enumerate}
    {\color{G-Moon}\item In an \(S_N2\) reaction in which \ch{OH^-} is the nucleophile, in which solvent will the rate will the rate (be) the fastest?
    \begin{itemize}
        \item {\color{o-Sun}\textbf{a polar aprotic solvent}}
        \item a nonpolar solvent
        \item a polar protic solvent
    \end{itemize}
    }
    \begin{itemize}
        \item 
    \end{itemize}
    {\color{G-Moon}\item The carbon center in a alkyl halide is
    \begin{itemize}
        \item an unsaturated center
        \item a nucleophile
        \item a cation
        \item {\color{o-Sun}\textbf{an electrophile}}
    \end{itemize}
    }
    \begin{itemize}
        \item 
    \end{itemize}
    {\color{G-Moon}\item Why is the rate of reaction for an \(S_N2\) reactions so much slower for a \ang{3} alkyl halide than for a \ang{1} alkyl halide?
    \begin{itemize}
        \item the leaving group is more reactive in an \ang{1} alkyl halide
        \item steric crowding is much less in \ang{3} alkyl halides
        \item because \(\Delta G^\circ\) is much smaller for a \ang{3} alkyl halide
        \item {\color{o-Sun}\textbf{steric crowding is much greater in \ang{3} alkyl halides}}
    \end{itemize}
    }
    \begin{itemize}
        \item 
    \end{itemize}
    {\color{G-Moon}\item At the transition state of an \(S_N2\) reaction reaction
    \begin{itemize}
        \item {\color{o-Sun}\textbf{the \ch{C-Nu} bond is \textit{partially} formed and the \ch{C-LG} bond is \textit{partially}broken.}}
        \item the \ch{C-Nu} bond is \textit{partially} formed and the \ch{C-LG} bond is \textit{completely} broken.
        \item the \ch{C-Nu} bond is \textit{completely} formed and the \ch{C-LG} bond is \textit{partially} broken.
        \item the \ch{C-Nu} bond is \textit{partially} formed and the \ch{C-LG} bond is \textit{completely in tact}.
    \end{itemize}
    }
    \begin{itemize}
        \item 
    \end{itemize}
    {\color{G-Moon}\item The rate law expression for an \(S_N2\) reactions reaction has the form
    \begin{itemize}
        \item rate = \(k[\text{electrophile}]\)
        \item rate = \(k[\text{electrophile}]^2\)
        \item rate = \(k[\text{nucleophile}]^2\)
        \item {\color{o-Sun}\textbf{rate = \(k[\text{electrophile}][\text{nucleophile}]\)}}
    \end{itemize}
    }
    \begin{itemize}
        \item 
    \end{itemize}
    {\color{G-Moon}\item What is the meaning of \(S_N2\)?
    \begin{itemize}
        \item substitution nucleophilic two transition states
        \item substitution nucleophilic two two reagents
        \item {\color{o-Sun}\textbf{substitution nucleophilic two second order}}
        \item substitution nucleophilic two twice
    \end{itemize}
    }
    \begin{itemize}
        \item 
    \end{itemize}
    {\color{G-Moon}\item Which would you expect to be the best nucleophile?
    \begin{itemize}
        \item \ch{F-}
        \item \ch{(CH3)2CH-}
        \item {\color{o-Sun}\textbf{\ch{CH3-}}}
        \item \ch{(CH3)3C-}
    \end{itemize}
    }
    \begin{itemize}
        \item 
    \end{itemize}
    {\color{G-Moon}\item In substitution reactions with alkyl halides,
    \begin{itemize}
        \item the nucleophile is the leaving group
        \item a hydrogen becomes the leaving group
        \item {\color{o-Sun}\textbf{the halide is the leaving group}}
        \item the electrophile is the leaving group
    \end{itemize}
    }
    \begin{itemize}
        \item 
    \end{itemize}
    {\color{G-Moon}\item Of the following, which is the better nucleophile?
    \begin{itemize}
        \item \ch{H20}
        \item \ch{NH3}
        \item \ch{RNH-}
        \item {\color{o-Sun}\textbf{\ch{NH2-}}}
    \end{itemize}
    }
    \begin{itemize}
        \item 
    \end{itemize}
\end{enumerate}
\end{document}