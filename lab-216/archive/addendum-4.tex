\documentclass[12pt,a4paper]{article}
\usepackage{inverba}
\newcommand{\userName}{Cullyn Newman} 
\newcommand{\class}{BI 216} 
\newcommand{\institution}{Portland State University} 
\newcommand{\thetitle}{\hypertarget{home}{Lab 4: Diffusion, Osmosis, Tonicity}}
\rfoot{\hyperlink{home}{\thepage}}

\begin{document}
\section*{Part A: Diffusion}
\begin{enumerate}[font=\bfseries, wide]
    {\color{under}\item Describe two factors that could speed up the rate of diffusion. \textbf{(0.5 pt)}}

    Increasing temperature or type of solute (smaller sized) used.

    {\color{under}\item At what point does net diffusion end? \textbf{(0.5 pt)}}

    When equilibrium is reached, or more specifically, when there is no net change between concentrations. 

    {\color{under}\item  The video discusses five factors that can affect the rate of diffusion. List one and explain it using your own words. \textbf{(1 pt)}}

    Temperature is a factor. This is because increased temperature increasing molecular movement, which increases the chances and thus ability for the solute to diffuse. 
    
    {\color{under}\item Identify a biological situation in which efficient diffusion of a solute from one region to another would be a matter of life or death for an organism. Please describe that situation, and what factors would influence the diffusion. \textbf{(1 pt)}}

    One example is our red blood cells and IV fluids. You need to have concentration of electrolytes otherwise the red blood cells will burst. This is really osmosis, where the water is moving into the cell, but this is still due to concentration differences.

\end{enumerate}

\section*{Part B: Osmosis and Tonicity; Simulation}
\begin{enumerate}[font=\bfseries, wide, resume]
    {\color{under}\item Explain one way that osmosis differs from diffusion, and one wya that it is similar. \textbf{(1 pt)}}

    Osmosis references the movement of water, rather then the movement of solutes like in diffusion. Both diffusion and osmosis occur from differences in solute concentration, and often happen at the same time, but osmosis takes over when solutes cannot diffuse across the barrier. 
    \newpage 
\subsection*{Simulation Questions}
    {\color{under}\item How long will the dialysis tubes remain in each beaker? Why does this matter? \textbf{(1 pt)}}

    24 hours. We need to allow sufficient time for osmosis to occur and reach equilibrium.

    \begin{table}[h]
        \centering
        \caption{Comparative analysis of osmosis between dialysis tube and beaker with various concentrations of added sugar as solute. }
        \begin{tabular}{r|ccccc}
            \toprule
            Trial & A & B & C & D & E\\
            \midrule
            Beaker \% Sugar 
            & 0  &  0& 5 & 10 & 15\\
            Dialysis Tube \% Sugar
            & 0 & 10 & 10 & 10 & 10 \\
            Initial Mass (g)
            & 17.59 & 8.75 & 11.24 & 10.71 & 18.05 \\
            Final Mass (g) 
            & 17.66 & 10.42 & 12.10 & 10.57 & 15.60 \\
            $\Delta$ Mass (g)
            & 0.07 & {\color{Fresh1}1.67} & {\color{Fresh1}0.86} & 0.14 & {\color{Sun1}-2.45}\\
            \bottomrule
            \end{tabular}
    \end{table}
    

    {\color{under}\item Name two variables that remained constant throughout the lab. \textbf{(1 pt)}}

    Same amount of water, same pressure on water, and same type of solute, and same amount of sugar in dialysis tube.

    {\color{under}\item Name the \textit{dependent} variable. \textbf{(1 pt)}}

    The amount of mass gained or loss of each dialysis tube.

    {\color{under}\item Name the \textit{independent} variable.\textbf{(1 pt)}}

    Concentration of solute in the dialysis tubes.

    {\color{under} The percent concentration of both fluids is different. There is {\color{darkmodetext}\textbf{0}\%} sugar in the beaker and {\color{darkmodetext}\textbf{100}\%} sugar in the dialysis tube. Now, think of it from the water's point of  view. There is {\color{darkmodetext}\textbf{90.2}\%} water in the beaker and {\color{darkmodetext}\textbf{9.8}\%} in the dialysis tube}

    {\color{under}\item Answer the following for B:}
    \begin{enumerate}
        {\color{under}\item Did the beaker have a higher/lower concentration of water than the dialysis tube? \textbf{0.5 pt}}

        Lower.
\newpage 
        {\color{under}\item Did the water flow in or out of the tube and what type of diffusion is this? \textbf{0.5 pt}}

        Into the tube and passive diffusion.
    \end{enumerate}

    {\color{under}\item Answer the following for C-E:}
    \begin{enumerate}
        {\color{under}\item Which beaker had a higher concentration of sugar solution fluid outside than inside? \textbf{0.5 pt}}

        E

        {\color{under}\item How will the water flow in this situation? If the tube was a cell, what type of solution was the cell placed in? \textbf{0.5 pt}}

        Water will flow \textbf{out}; \textbf{hypertonic}. 
    \end{enumerate}

    {\color{under}\item Answer the following for C-E:}
    \begin{enumerate}
        {\color{under}\item Which beaker had a percent concentration that was equal on both inside/outside the cell? \textbf{0.5 pt}}

        D

        {\color{under}\item How will the water flow in this situation? What type of solution was the cell (tube) placed in? \textbf{0.5 pt}}

        \textbf{Zero} net change in water flow; \textbf{isotonic}.
    \end{enumerate}

    {\color{under}\item Answer the following for C-E:}
    \begin{enumerate}
        {\color{under}\item Which beaker has a higher concentration of sugar solution inside the cell than outside?  \textbf{0.5 pt}}

        C 

        {\color{under}\item How will the water flow in this situation? What type of solution was the cell (tube) placed in? \textbf{0.5 pt}}

        Water will flow \textbf{in}; \textbf{hypotonic}.
    \end{enumerate}

    \newpage 

    {\color{under}\item Graphically represent the results of today’s simulation.  \textbf{(2 pts)}}

    \begin{figure}[h]
        \begin{tikzpicture}
            \begin{axis}[
                width=0.9\linewidth,
                bar width= 14pt,
                ybar,
                enlargelimits=0.2,
                ylabel={Percent Sugar / Change in Mass},
                legend style={at={(0.5,-0.15)},
                legend style={fill=none, draw=none, },
                anchor=north,legend columns=-1},
                symbolic x coords={Control, B, C, D, E},
                xtick=data,
                nodes near coords,
                every node near coord/.append style={font=\tiny},
                nodes near coords ={\pgfmathprintnumber[fixed] \pgfplotspointmeta},
                nodes near coords align={auto},
                ]
                \addplot [bottom color=Winter2, top color=Winter1,draw=none,] coordinates {(Control,0) (B,0) (C,5) (D,10) (E, 15)};
                \addplot [bottom color=Haze1, top color=Haze2,draw=none,]coordinates {(Control,0) (B,10) (C,10) (D,10) (E, 10)};
                \addplot [bottom color=Sun1, top color=Sun2,draw=none,]coordinates {(Control,0.07) (B,1.7) (C,0.86) (D,0.1) (E, -2.5)};
                \legend{\% Sugard in beaker~~~,\% Sugar in dialysis tube~~~, $\Delta$ Mass (g)}
            \end{axis}
        
            \end{tikzpicture}
        \caption{Graphical representation of data from Table 1; a comparative analysis of osmosis between dialysis tube and beaker with various concentrations of added sugar as solute. }
    \end{figure}
    
    
\end{enumerate}
    
\section*{Part C: Regulatory Mechanisms}
\begin{enumerate}[font=\bfseries, wide, resume]
    {\color{under}\item Describe one challenge inherent to being a marine organism.  \textbf{(1 pt)}}

    Marine organisms must control the gain of large concentrations of ions and loss of water.

    {\color{under}\item Describe one challenge inherent to being a freshwater organism.  \textbf{(1 pt)}}

    Freshwater fish must compensate for a high amount of water flowing into the gills in order for gas exchange to occur, which results in loss of ions.

    {\color{under}\item What mechanisms have evolved to compensate for these challenges for both types of organisms?  \textbf{(1 pt)}}

    Fresh water fish deal with their problem by producing large quantities of \textbf{diluted} urine to deal with the increased water and active transport ion pumps located in the gills to pumps ions \textbf{into} the blood.

    Marine fish solve their problem by drink large amounts of seawater and excrete highly \textbf{concentrated} urine and deal with increased ions by by actively transporting ions \textbf{out} of the blood. 

    {\color{under}\item Describe mechanisms that have evolved to facilitate organisms that switch from freshwater to marine environments. \textbf{(2 pts)}}

    Organisms such as salmon start of living in freshwater, but then switch the marine environment through a process called smoltification, which is where they develop specialized cells called ionocytes that pump out excess salt. Expression of related binding proteins at different times help control the expression of IGF which supports smoltification and regulates the transition.

\end{enumerate}

\end{document}