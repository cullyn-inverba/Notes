\documentclass[12pt,letterpaper]{article}
\usepackage{inverba}
\newcommand{\userName}{Cullyn Newman}
\newcommand{\class}{BI 336} 
\newcommand{\institution}{Portland State University} 
\newcommand{\thetitle}{\hypertarget{home}{Cellular Biology}}
\rfoot{\hyperlink{home}{10 --- \thepage}}

\begin{document}

%%%%%%%%%%%%%%%%%%%%%%%%%%%%%%%%%%%%%%%%%%%%%%%%%%%%%%%%%%%%%%%%%%%%%%%%%%%%%%%%%%%%%%%%%
%                               %   %   %   %   %   %   %                               %
%                           %   %   %   %   %   %   %   %   %                           %
%                       %   %                               %   %                       %
%   %   %   %   %   %   %   %   O   U   T   L   I   N   E   %   %   %   %   %   %   %   %
%                       %   %                               %   %                       %
%                           %   %   %   %   %   %   %   %   %                           %
%                               %   %   %   %   %   %   %                               %
%%%%%%%%%%%%%%%%%%%%%%%%%%%%%%%%%%%%%%%%%%%%%%%%%%%%%%%%%%%%%%%%%%%%%%%%%%%%%%%%%%%%%%%%%

\begin{chapbox}{Cellular Biology}{ 
\begin{enumerate}[font=\bfseries, wide]
    \setcounter{enumi}{9}
    \item \hyperlink{10}{\textbf{Membrane Structure}}
    \begin{itemize}
        \item \hyperlink{10.1}{The Lipid Bilayer}
        \item \hyperlink{10.2}{Membrane Proteins}
    \end{itemize}
    \item \hyperlink{11}{\textbf{Transport Across Membrane}}
    \begin{itemize}
        \item
    \end{itemize}
    \item \hyperlink{12}{\textbf{Intracellular Transport}}
    \begin{itemize}
        \item 
    \end{itemize}
    \item \hyperlink{13}{\textbf{Vesicular Trafficking, Secretion, \& Endocytosis}}
    \begin{itemize}
        \item
    \end{itemize}
    \item \hyperlink{14}{\textbf{Energy Conversion: Mitochondria and Chloroplasts}}
    \begin{itemize}
        \item 
    \end{itemize}
    \item \hyperlink{15}{\textbf{Cellular Communication}}
    \begin{itemize}
        \item 
    \end{itemize}
    \item \hyperlink{16}{\textbf{The Cytoskeleton}}
    \begin{itemize}
        \item 
    \end{itemize}
    \item \hyperlink{17}{\textbf{The Cell Cycle}}
    \begin{itemize}
        \item 
    \end{itemize}
    \item \hyperlink{18}{\textbf{Apoptosis}}
    \begin{itemize}
        \item 
    \end{itemize}
    \item \hyperlink{19}{\textbf{Cell Interactions}}
    \begin{itemize}
        \item 
    \end{itemize}
    \item \hyperlink{20}{\textbf{Cancer}}
    \begin{itemize}
        \item 
    \end{itemize}    
    \item[22.] \hyperlink{22}{\textbf{Stem Cells and Tissue Renewal}}
    \begin{itemize}
        \item 
    \end{itemize}
    \item[24.] \hyperlink{24}{\textbf{The Innate and Adaptive Immune System}}
    \begin{itemize}
        \item 
    \end{itemize}
\end{enumerate}
}\end{chapbox}

%%%%%%%%%%%%%%%%%%%%%%%%%%%%%%%%%%%%%%%%%%%%%%%%%%%%%%%%%%%%%%%%%%%%%%%%%%%%%%%%%%%%%%%%%
%                               %   %   %   %   %   %   %                               %
%                           %   %   %   %   %   %   %   %   %                           %
%                       %   %                               %   %                       %
%   %   %   %   %   %   %   %       N   O   T   E   S       %   %   %   %   %   %   %   %
%                       %   %                               %   %                       %
%                           %   %   %   %   %   %   %   %   %                           %
%                               %   %   %   %   %   %   %                               %
%%%%%%%%%%%%%%%%%%%%%%%%%%%%%%%%%%%%%%%%%%%%%%%%%%%%%%%%%%%%%%%%%%%%%%%%%%%%%%%%%%%%%%%%%

%%%%%%%%%%%%%%%%%%%%%%%%%%%%%%%%%%%%%%%%%%%%%%%%%%%%%%%%%%%%%%%%%%%%%%%%%%%%%%%%%%%%%%%%%
\clearpage

\renewcommand{\thetitle}{\hypertarget{10}{The Genetic Code of Genes
and Genomes}}
\rfoot{\hyperlink{10}{10 --- \thepage}}
\hypertarget{10}{} 

%%%%%%%%%%%%%%%%%%%%%%%%%%%%%%%%%%%%%%%%%%%%%%%%%%%%%%%%%%%%%%%%%%%%%%%%%%%%%%%%%%%%%%%%%

\begin{chapbox}{\hyperlink{home}{Chapter 10: The Cell Membrane}}
    \begin{enumerate}
        \item \hyperlink{10.1}{The Lipid Bilayer}
        \item \hyperlink{10.2}{Membrane Proteins}
    \end{enumerate}
\end{chapbox}

%%%%%%%%%%%%%%%%%%%%%%%%%%%%%%%%%%%%%%%%%%%%%%%%%%%%%%%%%%%%%%%%%%%%%%%%%%%%%%%%%%%%%%%%%%
%  vvvvvvvvvvvvvvvvvvvvvvvvvvvvvvvvv   Section 10.1   vvvvvvvvvvvvvvvvvvvvvvvvvvvvvvvvv  %
\hypertarget{10.1}{}
\begin{secbox}{\hyperlink{10}{The Lipid Bilayer}}{
    \subsection*{Phosphoglycerides, Sphingolipids, and Sterols Are the Major
    Lipids in Cell Membranes}
    \begin{itemize}
        \item \textbf{Plasma membrane}: the part of the cell that separates the exterior and the interior of a cell with a semipermeable lipid bilayer. The plasma membrane regulates import and export of materials for the cell and includes various proteins that interact with other cells. 
        \item \textbf{Lipid bilayer}: 
        \item \textbf{Amphiphilic}
        \item \textbf{Hydrophobic}
        \item \textbf{Hydrophilic}
        \item \textbf{Phospholipids}
        \item \textbf{Phosphoglycerides}
        \item \textbf{Cholesterol}
    \end{itemize}

    \subsection*{The Lipid Bilayer Is a Two-dimensional Fluid}
    \begin{itemize}   
        \item \textbf{Liposomes} 
    \end{itemize} 

    \subsection*{Despite Their Fluidity, Lipid Bilayers Can Form Domains of Different Compositions}
    \begin{itemize}
        \item \textbf{Lipid raft}
    \end{itemize}

    \subsection*{Lipid Droplets Are Surrounded by a Phospholipid Monolayer}
    \begin{itemize}
        \item \textbf{Lipid droplets}
    \end{itemize}

    \subsection*{The Asymmetry of the Lipid Bilayer Is Functionally Important}
    \begin{itemize}
        \item 
    \end{itemize}
    
    \subsection*{Glycolipids Are Found on the Surface of All Eukaryotic Plasma Membranes}
    \begin{itemize}
        \item \textbf{Glycolipids}
        \item \textbf{Gangliosides}
    \end{itemize}

    \begin{probbox}{The Lipid Bilayer: Summary}
        Biological membranes consist of a continuous double layer of lipid molecules in which membrane proteins are embedded. This lipid bilayer is fluid, with individual lipid molecules able to diffuse rapidly within their own monolayer. The membrane lipid molecules are amphiphilic. When placed in water, they assemble spontaneously into bilayers, which form sealed compartments. Although cell membranes can contain hundreds of different lipid species, the plasma membrane in animal cells contains three major classes—phospholipids, cholesterol, and glycolipids. Because of their different backbone structure, phospholipids fall into two subclasses—phosphoglycerides and sphingolipids. The lipid compositions of the inner and outer monolayers are different, reflecting the different functions of the two faces of a cell membrane. Different mixtures of lipids are found in the membranes of cells of different types, as well as in the various membranes of a single eukaryotic cell. Inositol phospholipids are a minor class of phospholipids, which in the cytosolic leaflet of the plasma membrane lipid bilayer play an important part in cell signaling: in response to extracellular signals, specific lipid kinases phosphorylate the head groups of these lipids to form docking sites for cytosolic signaling proteins, whereas specific phospholipases cleave certain inositol phospholipids to generate small intracellular signaling molecules.
    \end{probbox}
}\end{secbox}
%  ^^^^^^^^^^^^^^^^^^^^^^^^^^^^^^^^^   Section 10.1   ^^^^^^^^^^^^^^^^^^^^^^^^^^^^^^^^  %  
%%%%%%%%%%%%%%%%%%%%%%%%%%%%%%%%%%%%%%%%%%%%%%%%%%%%%%%%%%%%%%%%%%%%%%%%%%%%%%%%%%%%%%%%%%
%  vvvvvvvvvvvvvvvvvvvvvvvvvvvvvvvvvv  Section 10.2   vvvvvvvvvvvvvvvvvvvvvvvvvvvvvvvv  % 
\hypertarget{10.2}{}
\begin{secbox}{\hyperlink{10}{Membrane Proteins}}{
    \subsection*{Membrane Proteins Can Be Associated with the Lipid Bilayer in Various Ways}
    \begin{itemize}
        \item \textbf{Transmembrane protein}
        \item \textbf{Glycosylphosphatidylinositol (GPI) anchor}
        \item \textbf{Membrane-associated proteins}
    \end{itemize}
    
    \subsection*{Lipid Anchors Control the Membrane Localization of Some Signaling Proteins}
    \begin{itemize}
        \item
    \end{itemize}
    
    \subsection*{In Most Transmembrane Proteins, the Polypeptide Chain Crosses the Lipid Bilayer in an \(\bm{\alpha}\)-Helical Conformation}
    \begin{itemize}
        \item \textbf{Single pass transmembrane proteins}
        \item \textbf{Multi-pass transmembrane proteins}
    \end{itemize}
    
    \subsection*{Transmembrane \bfg{\alpha} Helices Often Interact with One Another}
    \begin{itemize}
        \item 
    \end{itemize}

    \subsection*{Some \bfg{\beta} Barrels Form Large Channels}
    \begin{itemize}
        \item \textbf{Lumen}
    \end{itemize}

    \subsection*{Many Membrane Proteins Are Glycosylated}
    \begin{itemize}
        \item \textbf{Carbohydrate layer}
        \item \textbf{Lectins}
    \end{itemize}

    \subsection*{Membrane Proteins Can Be Solubilized and Purified in Detergents}
    \begin{itemize}
        \item \textbf{Detergents}
    \end{itemize}

    \subsection*{Bacteriorhodopsin Is a Light-driven Proton H\bfg{^+} Pump That
    Traverses the Lipid Bilayer as Seven \bfg{\alpha} Helices}
    \begin{itemize}
        \item \textbf{Bacteriorhodopsin}
    \end{itemize}

    \subsection*{Some \bfg{\beta} Barrels Form Large ChannelsThe Cortical Cytoskeleton Gives Membranes Mechanical Strength and Restricts Membrane Protein Diffusion}
    \begin{itemize}
        \item \textbf{Spectrin}
        \item \textbf{Cortex}
    \end{itemize}

    \subsection*{Membrane-bending Proteins Deform Bilayers}
    \begin{itemize}
        \item \textbf{Membrane bending proteins}
    \end{itemize}
}\end{secbox}
%  ^^^^^^^^^^^^^^^^^^^^^^^^^^^^^^^^^   Section 10.2   ^^^^^^^^^^^^^^^^^^^^^^^^^^^^^^^  %  
%%%%%%%%%%%%%%%%%%%%%%%%%%%%%%%%%%%%%%%%%%%%%%%%%%%%%%%%%%%%%%%%%%%%%%%%%%%%%%%%%%%%%%%%%%
\end{document}