\documentclass[12pt,a4paper]{article}
\usepackage{inverba}

\newcommand{\userName}{Cullyn Newman} 
\newcommand{\class}{Cohen} 
\newcommand{\institution}{\color{r-Sun}Udemy} 
\newcommand{\theTitle}{\color{r-Sun}Statistics}
\begin{document}
%%%%%%%%%%%%%%%%%%%%%%%%%%%%%%%%%%%%%%%%%%%%%%%%%%%%%%%%%%%%%%%%%%%%%
\tableofcontents
\cleardoublepage
\fancyhead{}
\fancyhead[R]{\hyperlink{home}{\nouppercase\leftmark}}
%%%%%%%%%%%%%%%%%%%%%%%%%%%%%%%%%%%%%%%%%%%%%%%%%%%%%%%%%%%%%%%%%%%%%


%%%%%%%%%%%%%%%%%%%%%%%%%%%%% Chapter 1 %%%%%%%%%%%%%%%%%%%%%%%%%%%%%
%\begingroup
\clearpage
\section{Data}\phantomsection
\subsection{Data Basics}
\begin{itemize}
    \item Frequent types of data in statistics:
        \begin{itemize}
            \item \textbf{Interval}: numeric scale with meaningful intervals, e.g. temperature in celsius.
            \item \textbf{Ratio}: numeric but with a meaningful zero, e.g. height.
            \item \textbf{Discrete}: numeric with with no arbitrary precision, e.g. population.
            \item \textbf{Ordinal}: sortable and discrete, e.g. education level.
            \item \textbf{Nominal}: non-sortable and discrete, e.g. genre.
        \end{itemize}
    \item \textbf{Sample data}: Data from \textit{some} members of a group.
    \item \textbf{Population data}: Data from \textit{all} members of a group.
    \item Sample population sometimes uses hat notation, e.g. \(\hat{\beta},~\hat{\sigma}\), or other slight ambiguities.
    \item Most often sample data is used in statistics.
\end{itemize}

\subsection{Visualizing Data}note-workspace
\begin{itemize}
    \item 
\end{itemize}
%\endgroup
%%%%%%%%%%%%%%%%%%%%%%%%%%%%% Chapter 1 %%%%%%%%%%%%%%%%%%%%%%%%%%%%%

%%%%%%%%%%%%%%%%%%%%%%%%%%%%% Chapter 2 %%%%%%%%%%%%%%%%%%%%%%%%%%%%%
%\begingroup
\clearpage
\section{}\phantomsection
\subsection{}
\begin{itemize}
    \item 
\end{itemize}
%\endgroup
%%%%%%%%%%%%%%%%%%%%%%%%%%%%% Chapter 2 %%%%%%%%%%%%%%%%%%%%%%%%%%%%%
\end{document}