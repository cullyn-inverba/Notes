\documentclass[12pt,a4paper]{article}
\usepackage{inverba-dark}

\newcommand{\userName}{Cullyn Newman} 
\newcommand{\class}{[Subject]} 
\newcommand{\institution}{[Institution]} 
\newcommand{\theTitle}{\color{B-Cold} [Subject Title]}

% Dark Mode: #212121 → #dedede → #525252
\definecolor{dark}{HTML}{1f1f1f} 
\definecolor{darkText}{HTML}{e0e0e0} 


% {dark} → {light} to change mode --- might break colors!
\pagecolor{dark} 
\color{darkText} 


%%%%%%%%%%%%%%%%%%%%%%%%%%%%%%%%%
%              Base             %
%%%%%%%%%%%%%%%%%%%%%%%%%%%%%%%%%

% Background: #1f1f1f
    \definecolor{back}{HTML}{1f1f1f} 

% Text: #e0e0e0 #949494 #94f4f4f #f5b17a #ff9e4f
    \definecolor{text}{HTML}{e0e0e0}  
    \definecolor{minor}{HTML}{949494} 
    \definecolor{minimal}{HTML}{4f4f4f} 
    \definecolor{emph}{HTML}{f5b17a} 
    \definecolor{extra-emph}{HTML}{ff9e4f} 

% Titles: #7abef5 #b0d0eb
    \definecolor{ssec}{HTML}{7abef5} 
    \definecolor{sssec}{HTML}{b0d0eb} 

%%%%%%%%%%%%%%%%%%%%%%%%%%%%%%%%%
%         Complementary         %
%%%%%%%%%%%%%%%%%%%%%%%%%%%%%%%%%

% Charge: #f06767-> #406de3
    \definecolor{pos}{HTML}{f06767}
    \definecolor{neg}{HTML}{406de3}

% Boolean: #36cfb8 -> #ff6b8b
    \definecolor{true}{HTML}{36cfb8}
    \definecolor{false}{HTML}{ff6b8b}

% Relative: #bf88eb → #ffdb7b
    \definecolor{sys}{HTML}{bf88eb}   
    \definecolor{surr}{HTML}{ffdb7b}  


%%%%%%%%%%%%%%%%%%%%%%%%%%%%%%%%%
%             Triadic           %
%%%%%%%%%%%%%%%%%%%%%%%%%%%%%%%%%

%%%%%%%%%%%%%%%%%%%%%%%%%%%%%%%%%
%            Tetradic           %
%%%%%%%%%%%%%%%%%%%%%%%%%%%%%%%%%

%%%%%%%%%%%%%%%%%%%%%%%%%%%%%%%%%
%            Analogous          %
%%%%%%%%%%%%%%%%%%%%%%%%%%%%%%%%%

%%%%%%%%%%%%%%%%%%%%%%%%%%%%%%%%%
%             Neutral           %
%%%%%%%%%%%%%%%%%%%%%%%%%%%%%%%%%

\begin{document}
%%%%%%%%%%%%%%%%%%%%%%%%%%%%%%%%%%%%%%%%%%%%%%%%%%%%%%%%%%%%%%%%%%%%%
\tableofcontents
\cleardoublepage
\fancyhead{}
\fancyhead[R]{\hyperlink{home}{\nouppercase\leftmark}}
%%%%%%%%%%%%%%%%%%%%%%%%%%%%%%%%%%%%%%%%%%%%%%%%%%%%%%%%%%%%%%%%%%%%%

\clearpage
\fancyhead[L]{[Part]}
%\begingroup
%%%%%%%%%%%%%%%%%%%%%%%%%%%%% Chapter 1 %%%%%%%%%%%%%%%%%%%%%%%%%%%%%
%\begingroup
\clearpage
\section{Big Chapter}
\subsection{First Section}
\subsubsection{Various points}
\subsubsection{Various points}
\subsection{First Section}
\subsubsection{Various points}
\subsubsection{Various points}
\subsubsection{Various points}
\subsection{First Section}
Lorem ipsum dolor sit {\color{o-Sun}ametaee}, consectetur adipiscing elit. Maecenas consequat nulla sit amet mi dictum molestie. Fusce urna est, tempus eu mauris sit amet, interdum egestas sapien. Maecenas egestas varius elit, vitae vulputate risus ornare eu. Integer vel mollis eros. Etiam sed nunc ut turpis euismod auctor vitae quis enim. Cras ullamcorper auctor laoreet. Nunc porta malesuada elementum.

Duis dolor nibh, pretium eu magna vel, maximus pharetra arcu. Mauris non dapibus risus, at hendrerit nibh. Class aptent {\color{minor} taciti sociosqu ad litora torquent per conubia nostra, per inceptos himenaeos. Suspendisse potenti. Etiam }id nisi justo. Morbi vel ex nec enim pulvinar tincidunt quis sed nibh. Maecenas varius nisl magna, ac varius justo egestas faucibus. Nulla blandit metus eu nulla malesuada, a malesuada diam elementum. Etiam dignissim sem in egestas efficitur. Morbi malesuada velit non orci malesuada scelerisque. Lorem ipsum dolor sit amet, consectetur adipiscing {\color{minimal} elit. Nulla porttitor nulla ac metus interdum, nec tincidunt est finibus.}

{\color{emph} Mauris id\textbf{ nisi }rutrum, porta enim eget,  {\color{extra-emph} imperdiet luctus. Cras \textbf{lacinia} leo} vulputate purus. Proin sit amet mollis neque, sit} amet aliquet lectus. In id mauris et enim ullamcorper sagittis. In in lacinia risus. Phasellus viverra fringilla magna  nec quam porta, vitae fringilla felis aliquet. Nulla in diam dolor. Curabitur pulvinar nibh a eros malesuada, sit amet pharetra felis luctus. Fusce turpis quam, pulvinar sed mi sagittis, cursus vehicula mauris. Integer vel euismod nibh, at facilisis libero. Sed erat urna, iaculis at lacinia iaculis, consequat et libero. Cras faucibus, felis sed dapibus fermentum, nunc leo vulputate mauris, ac interdum mi leo vitae magna.

Quisque ut arcu neque. Pellentesque non ante quam. Aenean purus sem, tincidunt at velit eget, tristique fringilla risus. Curabitur et pellentesque augue, et faucibus sem. Suspendisse potenti. Nulla sed nulla est. Morbi maximus neque et quam finibus facilisis. Ut metus dui, interdum a euismod vel, vulputate eget quam. Sed sed vehicula nulla, quis lobortis nibh. Etiam sed pellentesque nisl, id lacinia felis. Phasellus rutrum accumsan sem ac mollis. Donec arcu sapien, pulvinar vitae faucibus et, pellentesque id ex. Sed iaculis euismod metus vel volutpat. Suspendisse commodo leo eget nulla tristique, at dignissim dui accumsan.

Praesent lobortis turpis ac ligula congue dictum. Suspendisse sit amet varius tellus, sed venenatis velit. In consectetur massa consequat diam venenatis, tempor fringilla neque aliquet. Aliquam auctor ex vitae porttitor tincidunt. Mauris et diam tempus, auctor justo ut, dignissim velit. Interdum et malesuada fames ac ante ipsum primis in faucibus. Sed in finibus nunc, a blandit magna. 

\begin{itemize}
    \item Praesent ut semper urna, nec posuere odio
        \begin{itemize}
            \item  Cras faucibus gravida nibh.
            \item  Morbi felis nulla, gravida non blandit non, imperdiet ac purus.
            \item  In ac erat non est rutrum sodales eget nec nunc.
                \begin{itemize}
                    \item  Morbi felis nulla, gravida non blandit non, imperdiet ac purus.
                    \item  In ac erat non est rutrum sodales eget nec nunc.
                \end{itemize}
            \item  Praesent tincidunt malesuada purus, sit amet tincidunt ex tincidunt at.
            \item  Sed tincidunt malesuada ipsum, ac dapibus nisl gravida quis.
            \item  Ut ex metus, tempor nec quam non, hendrerit iaculis ex.
        \end{itemize}
    \item Aenean vestibulum vestibulum augue, quis volutpat tortor aliquam sit amet
    \item Praesent diam turpis, semper non dignissim sit amet, lacinia at neque
    \item Morbi efficitur cursus nisl at lacinia
    \item Phasellus eu imperdiet turpis
    \item Duis neque nisl, molestie eget nunc ac, convallis commodo metus
    \item Cras eget posuere mauris, sed sagittis tellus
    \item Nunc tempus dictum lacinia. 
\end{itemize}

\subsubsection{\texorpdfstring{\(S_N1\) Mechanism}{Lg}}
    \begin{itemize}
        \item \textbf{Solvolysis}: when a alkyl halide undergoes ionization in a polar solvent (hydrogen connects to electronegative atom), and the solvent functions as a {\color{o-Sun}nucleophile} which attacks the intermediate carbocation, resulting in a two-step substitution.
            \begin{itemize}
                \item The concentration of the {\color{o-Sun}alkyl halide} acts as the {\color{o-Sun}rate-determining} step, since the loss of the leaving group, and the formation of a the carbocation, represents the highest energy transition state of the multi-step process.
                    \begin{itemize}
                        \item Loss of leaving group (formation of carbocation) $\rightarrow$ nucleophilic attack
                    \end{itemize}
            \end{itemize}
        \item If the nucleophile is uncharged, which is often the case for \(S_N1\) reactions, then there will be an additional step at the end of the mechanism in which the extra proton is removed by a solvent molecule.
    \end{itemize}
    \subsubsection{E1 Mechanism}
    \begin{itemize}
        \item When a \ang{3} alkyl halide undergoes ionization in a polar solvent and can function as a {\color{o-Sun}base} and depronate the intermediate carbocation, resulting in an two-step elimination.
            \begin{itemize}
                \item The {\color{o-Sun}loss of the leaving group} also is the {\color{o-Sun}rate-determining} step for eliminations.
            \end{itemize}
        \item Often the second rate-limiting step is either the substitution or elimination.
            \begin{itemize}
                \item Usually substituent is favored due to the lower energy transition state requirement.
                \item Elimination is favored when potential energy is high enough for the more stable product of elimination to have greater determining effect.
                    \begin{itemize}
                        \item E.g, when the resulting alkene is tri- or Tetrasubstituted; more stability in the end product is favored.
                    \end{itemize}
                \item This is why a mixture of products is often observed.
            \end{itemize}
    \end{itemize}
    \subsubsection{Solevent and Substrate Effects on Ionization Rates}
    \begin{itemize}
        \item Ionization process occurs most readily in {\color{o-Sun}protic solvents}, since both ionization products are wll solvated in protic solvents.
            \begin{itemize}
                \item Carbocations are stabliized by intereacting with lone pairs of the oxygen atoms.
                \item Chloride ions are stabilized via hydrogen bonds.
                \item Aprotic cannot stabilize both equally, as there is little free hydrogen, thus unfavorable for solvolysis reactions.
            \end{itemize}
        \item To contrast, \(S_N2\) reactions are enhanced by the {\color{o-Sun}strength of the nucleophile}, which allows them to overcome activation energy of the reaction.
            \begin{itemize}
                \item Lack of stabilization increases energy of nucleophiles.
            \end{itemize}
        \item Protic solvents stabilize the ionic intermediates and transition states, allowing for smaller activation energy barrier due to loss of leaving group being the rate-limiting step.
        \item \textbf{Effect of Substrate}
            \begin{itemize}
                \item Alkyl iodides are the most reactive, while flourides are the least.
                    \begin{itemize}
                        \item \(I^->Br^->Cl^->F^-\)
                    \end{itemize}
            \end{itemize}
    \end{itemize}

\subsection{Predicting Products: Substitution vs Elimination}
\begin{itemize}
    \item Often elimination and substitution are in competition with each other, with sometimes one product dominanting, or reactions resulting in multiple products.
    \item Main steps in determining products: 
        \begin{itemize}
            \item Determine the function of the reagent.
            \item Analyze the substrate and determine the expected machanism(s).
            \item Consider any relevant regiochemical and stereochemical requirements.
        \end{itemize}
    \subsubsection{Determining the Function of the Reagent}
    \begin{itemize}
        \item To recap: substitution occurs when the reagent functions as a nucleophile, while an elimination reactions occurs when the reagent functions as a base.
        \item Major factors (not all) that Determine nucleophilicity: presence of high electron density and polarizability, while basicity is determined by base stability.
            \begin{itemize}
                \item Strong acids have weaker conjugate bases. THe weaker the base, then the more stability the molecule has, and the less likely to act as base rather than nucleophile.
                \item Often bases bear negative charges, allowing them to also function as nucleophiles.
                \item Small size often decreases polarizability, which makes some weak bases unable to be nucleophiles.
            \end{itemize}
        \item Common reagents can be classified into four catagories, with two-dimensions based on base and nucleophile strength.
            \begin{itemize}
                \item \textbf{Strong base}, {\color{darklc}\textbf{weak nucleophile}}: \ch{NaH}, \ch{DBN}, \ch{DBU}
                \begin{itemize}
                    \item Strong tenency to give up hydrogens leads reagents functioning almost exclusively as bases, leading to the elimination reaction.
                \end{itemize}
                \item \textbf{Strong base}, \textbf{strong nucleophile}: {\color{neg}\ch{HO^-}}, {\color{neg}\ch{MeO^-}}, {\color{neg}\ch{EtO^-}}\begin{itemize}
                    \item Can act as both, often producing mixture of products.
                    \item Such reagents are generally used for biomolecular processes.
                \end{itemize}
                \item {\color{darklc}\textbf{Weak base}}, \textbf{strong nucleophile}: {\color{neg}\ch{I^-}}, {\color{neg}\ch{Br^-}}, {\color{neg}\ch{Cl^-}}, {\color{neg}\ch{RS^-}}, {\color{neg}\ch{HS^-}}, \ch{RSH}, \ch{H2S}
                    \begin{itemize}
                        \item Reagents mainly function as nucleophiles due to high polarizability, despite being weak bases, leading to substitution reactions.
                    \end{itemize}
                \item {\color{darklc}\textbf{Weak base}}, {\color{darklc}\textbf{weak nucleophile}}: \ch{H2O}, \ch{MeOH}, \ch{EtOH}
                    \begin{itemize}
                        \item Ban act as both a nucleophiles and base.
                        \item Mainly used in unimolecular reactions. 
                    \end{itemize}
            \end{itemize}
    \end{itemize}
    \subsubsection{Determining the Expected Mechanism(s)}
    \begin{itemize}
        \item After determining catagories, then the next step is to identify likely mechansisms, which is done by analyzing the substrate (\ang{3}, \ang{2}, \ang{1}).
        \begin{table}[h]
            \centering
            \caption{Expected Mechanism(s)\strut}
            \label{tab:Expected Mechanisms}
            \begin{tabular}{c|cccc}
                \toprule
                & \makecell{Strong Base\\{\color{darklc}Weak Nucleo}} 
                & \makecell{Strong Base\\Strong Nucleo}
                & \makecell{{\color{darklc}Weak Base}\\Strong Nucleo}
                & \makecell{{\color{darklc}Weak Base}\\{\color{darklc}Weak Nucleo}} \\
                \midrule
                \ang{1}& E2 &{\tiny{\(E2\)}}, \(S_N2\) &  \(S_N2\) & --- \\
                \ang{2}& E2 & E2, {\tiny{\(S_N2\)}}& \(S_N2\) & --- \\
                \ang{3}& E2 & E2 & \(S_N1\) & E1, \(S_N1\) \\
                \bottomrule
                \end{tabular}
        \end{table}
        \item The identify of the substrate plays a vital role in determining reactions when the reagent is a strong nucleophile and a strong base.
            \begin{itemize}
                \item If both are weak, then substitution and elimination can only occur in \ang{3} due the need of carbocation intermediate (some rare exceptions with\ang{2} allylic and benzylic compounds).
            \end{itemize}
        \item Elimination is not hindered by steric interactions, so generally it will prevail long as substitution isn't favored.
            \begin{itemize}
                \item Except in unimolecular, where E1 is favored when the products is tri- or tetrasubstituted, and \(S_N1\) with mono- and disubstituted.
            \end{itemize}
    \end{itemize}
    \subsubsection{Considering Regiochemical and Stereochemical Outcomes}
    \begin{itemize}
        \item Much of follow information was already discussed, but will be recapped here.
        \item \(S_N2\):
            \begin{itemize}
                \item Regiochemical: the nucleophile attacks the $\alpha$ position where the leaving group is connected.
                \item Stereochemical: the nucleophile replaces the leaving group with inversion of configuration. 
            \end{itemize}
        \item E2: 
            \begin{itemize}
                \item Regiochemistry: The Zaitsev product (more hydrogen-dense) is generally favored over Hofmann product (less hydrogen-dense), unless sterically hindered, then Hofmann will be favored.
                \item Stereochemical: stereoselective, with preference for \textit{trans} over \textit{cis} in disubstituted alkene.
                    \begin{itemize}
                        \item Also stereospecific: when the $\beta$ position has only one proton, then the stereoisomeric alkene resulting from the anti-periplanar elimination will be obtained.
                    \end{itemize}
            \end{itemize}
        \item \(S_N1\):
            \begin{itemize}
                \item Regiochemical: nucleophile attacks the carbocation, which is where leaving group is origanally connected, unless carbocation rearrangement took place.
                \item Stereochemical: nucleophile replaces the leaving group, giving a nearly racemic mixture of inverted and retained configurations.
                    \begin{itemize}
                        \item Often though there is slight preference on inversion as a result of the ion pairs effect.
                    \end{itemize}
            \end{itemize}
        \item E1:
            \begin{itemize}
                \item Regiochemical: Zaitsev product will always be favored over the Hofmann product.
                \item Stereochemical: Stereoselective, with the \textit{trans} disubstituted alkene often being favored.
            \end{itemize}
    \end{itemize}
\end{itemize}
%\endgroup
%%%%%%%%%%%%%%%%%%%%%%%%%%%%% Chapter 1 %%%%%%%%%%%%%%%%%%%%%%%%%%%%%
%\endgroup

%%%%%%%%%%%%%%%%%%%%%%%%%%%%% Chapter 2 %%%%%%%%%%%%%%%%%%%%%%%%%%%%%
%\begingroup
\clearpage
\section{Colors}\phantomsection
\subsection{Colors}
\begin{itemize}
    \item Yes
\end{itemize}
%\endgroup
%%%%%%%%%%%%%%%%%%%%%%%%%%%%% Chapter 2 %%%%%%%%%%%%%%%%%%%%%%%%%%%%%
\end{document}