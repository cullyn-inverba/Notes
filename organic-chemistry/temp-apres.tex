\documentclass[12pt,a4paper]{article}
\usepackage{inverba}

\newcommand{\userName}{Cullyn Newman} 
\newcommand{\class}{[Subject]} 
\newcommand{\institution}{[Institution]} 
\newcommand{\theTitle}{\color{B-Cold} [Subject Title]}

\begin{document}
%%%%%%%%%%%%%%%%%%%%%%%%%%%%%%%%%%%%%%%%%%%%%%%%%%%%%%%%%%%%%%%%%%%%%
\tableofcontents
\cleardoublepage
\fancyhead{}
\fancyhead[R]{\hyperlink{home}{\nouppercase\leftmark}}
%%%%%%%%%%%%%%%%%%%%%%%%%%%%%%%%%%%%%%%%%%%%%%%%%%%%%%%%%%%%%%%%%%%%%
\clearpage
\section*{Week 6}\phantomsection
\addcontentsline{toc}{section}{\textbf{Week 6}}
\fancyhead[R]{\hyperlink{home}{Week 6}}

\fancyhead[L]{\hyperlink{home}{Monday, November 9 - Quiz 16}}
\subsection{Monday, November 9 - Quiz 16}
\begin{enumerate}
    {\color{G-Moon}\item Which of the following molecules is not capable of existing in cis and trans isomeric forms?

    \chemfig{=[:30](-[:90])-[:-30]-[:30]}
    \hspace*{14pt}
    \chemfig{-[:30](-[:90])=[:-30]-[:30]}
    \hspace*{14pt}
    \chemfig{-[:30](-[:90])-[:-30]=[:30]}
    }
        \begin{itemize}
            \item {\color{o-Sun}\textbf{All of the above}} 
            \begin{itemize}
                \item Option 1 and 3 both have a $\pi$ bonded carbon (sp\(^{2}\)), which has two hydrogens as the substituents---compounds with same substituents on both sides are unable to be cis--trans.
                \item Likewise, option 2, has \ch{CH3} on both sides of one end of the $\pi$ bond, making them the have same substituents. 
            \end{itemize}
        \end{itemize}
    {\color{G-Moon}\item Which of the following molecules corresponds to a cis isomer?
    
    \chemfig{\ch{Cl}-[:90]-[:30](-[:90])-[:-30](-[-90]\ch{Cl})=[:30]}
    \hspace*{10pt}
    \chemfig{-[:30](-[:90])(-[:-90]\ch{Cl})-[:-30](-[-90]\ch{Cl})=[:30]}
    \hspace*{10pt}
    \chemfig{-[:30](-[:90]-[:30]\ch{Cl})(-[:-90]\ch{Cl})-[:-30]=[:30]}
    \hspace*{10pt}
    \chemfig{-[:30](-[:90])-[:-30](-[:-90]\ch{Cl})=[:30]-[:-30]}
    }
        \begin{itemize}
            \item {\color{o-Sun}\textbf{D}} 
            \begin{itemize}
                \item A, B, and C all cannot be cis--trans isomers due to double hydrogen substituents.
            \end{itemize}
        \end{itemize}
    {\color{G-Moon}\item What condition causes a carbon center to be classified as asymmetric?}
        \begin{itemize}
            \item {\color{o-Sun}\textbf{The C must have four different groups bonded to it}} 
            \begin{itemize}
                \item Asymmetric carbon center = \textbf{chiral center}, i.e., a tetrahedral carbon that bears four different groups.
            \end{itemize}
        \end{itemize}
    {\color{G-Moon}\item Suppose a sample of 2-methyl-1-butanol (see lecture notes), when placed in plane polarize light, showed a rotation of \ang{-4.32}. What is the enantiomeric excess of the enantiomer that rotates light to the left?}
        \begin{itemize}
            \item {\color{o-Sun}\textbf{75.1\%}} 
            \begin{itemize}
                \item \(\%~ee = \dfrac{|\text{observed}~(\alpha)|}{|\text{specific}~[\alpha]|} \times 100\%\)
                \item (from slides) 2-methyl-1-butanol: specific $[\alpha]^{\SI{20}{\celsius}}_D = \pm5.75$; observed \((\alpha) = -4.32\).  
                \item \(ee = \dfrac{\ang{4.32}}{\ang{5.75}} \times 100\% = 75.1\%\)
            \end{itemize}
        \end{itemize}
    {\color{G-Moon}\item While nature, i.e., enzymes, synthesize molecules with chiral centers in 100\% enantiomeric purity, that often proves very difficult for synthetic organic chemists to do.  What most often results in the lab is a mixture containing equal concentrations of both enantiomers. What term is used to describe this mixture?}
        \begin{itemize}
            \item {\color{o-Sun}\textbf{Racemic mixture}} 
            \begin{itemize}
                \item (from notes) \textbf{Racemic mixtrue}: a solution containing equal amounts of both enantiomers, resulting in an optically inactive appearance.
            \end{itemize}
        \end{itemize}
    {\color{G-Moon}\item What physical properties distinguish the R enantiomer from the S enantiomer of a molecule?}
        \begin{itemize}
            \item {\color{o-Sun}\textbf{they rotate plane polarized light in equal, but opposite, directions}} 
            \item Relevant notes:
                \begin{itemize}
                    \item Specific rotation for enantiomers are {\color{o-Sun}equal in magnitude} but {\color{o-Sun}opposite in direction}.
                        \begin{itemize}
                            \item {\color{pos}\textbf{dextrorotaory}}: a compound exhibiting {\color{pos}positive} rotation.
                            \item {\color{neg}\textbf{levorotatory}}: a compound exhibiting {\color{neg}negative} rotation.
                            \item No direct relationship between R/S system of nomenclature, as that is independent of conditions, but dependent on observation angle.
                            \item The {\color{o-Sun}direction} of polarized light, however, is {\color{o-Sun}dependent on conditions}, and can change based on temperature or wavelength even with the same given configuration.
                        \end{itemize}
                \end{itemize}
        \end{itemize}
\end{enumerate}


\end{document}