\documentclass[basic]{inVerba-notes}

\newcommand{\userName}{Cullyn Newman}
\newcommand{\class}{BI:\@ 463}
\newcommand{\theTitle}{\color{link}{Split-Brain Presentation Script}}
\newcommand{\institution}{Portland State University}

\begin{document}
    
\textbf{Current State of Research}

Give outline of presentation, introducing who will talk about what

\textbf{Overview}

Introduce main source of information, as well as brief overview of the key topics.

\textbf{Current Paradigm}

Give overview of the current paradigm

\textbf{Classical View}

This general conclusion emerge from the observations centered around white matter lesions that gave rise to cognitive, behavioral, and psychological dysfunctions. These dysfunctions are termed disconnection syndromes. The callosal syndrome relates specific dysfunctions of split-brain patients, which we as we previously describe and causes many issues.

The classical view is built on the study of callosal syndrome, and with somewhat limited means of study, lead to the conclusion that the callosal fibers simply allow for the exchange between two hemispheres.

\textbf{Integrative View}

Recently however, it has become clear plays a role in functional asymmetries. 

(read first point)

Symptoms following callosal disconnection are not simply due to loss of information transfer, but due to loss of distributed balance mediated by the callosal fibers together with the other cortical and subcortical nerve connections between hemispheres.

There's a notion of an equilibrating role, due to evidence that the

(read second point)

This observation suggests that callosal syndromes might be due to the result of the poor responses from the ``uniformed'' hemispheres, as information still may be transferred by other subcortical pathways, but not processed correctly due to improper integration of bilateral transfer. 

Also, transfer time was found to be different depending on the direction. This observation was attributed to other observations that show the right hemisphere having a greater number of fast-conducting, myelinated fibers. This is significant because this asymmetry helps support the prevalent, more integrative interpretation by demonstrating another factor responsible for dysfunctional asymmetries that arise in connection syndromes.

To sum it all up and put it a little more succinctly: the prevalent interpretation is one that views the interhemispheric communication taking place both by white matter and by bilateral subcortical projections, which makes the old view of the corpus callosum just being used for information transfer much weaker.

\textbf{Hemispheric Asymmetry For Faces}

Details\dots

\textbf{Social Perception and Cognition}

Details \dots 

\textbf{Subliminal Emotional Specialization}

Details \dots 


\end{document}