\documentclass[plain,basic]{inVerba-notes}

\newcommand{\userName}{Cullyn Newman}
\newcommand{\class}{BI:\@ 428}
\newcommand{\theTitle}{Bioethics Panel Summary}
\newcommand{\institution}{Portland State University}

\begin{document}

Our group decided on case 9, with the 35-year-old female with HD and a history of suicide. We concluded that there could be a high chance she was seeking out the test just to give justification of ending her life. Overall, this lead to a discussion on assisted suicide and the ethics of providing the test that would influence such decision and potentially cause unjust harm. (\textbf{Non-maleficence})

We don't know her family situation and many other contextual features. So our recommendation was potential therapy to confirm that preemptive suicidal tendencies would not unjustly influence the decision, and if this was a desired option, then making sure any external responsibilities and lives of others would not be drastically effected. (\textbf{Beneficence}) Overall, if she wanted the test and wanted to end her life earlier than when the disease would take it, then that would be okay (\textbf{Autonomy}), as long as appropriate discussion and potential precautions were considered first. (\textbf{Justice})

\textbf{Rreflection on this learning exercise}: we had about 20 minutes left and had a much more fruitful discussion centered around ethics of bioengineering human's. More people were actively engaged and a more lively discussion took place. We ended up going 20 minutes over allocated time and lead to a much more timely and relevant discussion. 

Perhaps future exercises could give a more freedom and autonomy to us during the exercise. That being said, there definitely was an attempt to do so, don't get me wrong. However, making the conversation is relevant is hard to do a fine line to walk. Hopefully this feedback is useful.

Anyways, I'd say attempting to heavily constrain it by following some exercise formula in order to fit in the class, then providing a plethora amount of questions defeats the purpose of behind discussion. We need to be the one to generate such questions---that's the exercise and skill being built here. The goal is noble, but the execution is backwards and limits growth with too much hand holding in the wrong direction by assuming we won't have a useful discussion without following some sort of predefined worksheet; at a certain point guidance is a hindrance to progresses. Essentially what I am describing is a prime example of how less can be more. 

Honestly without this optionally side conversation, then this whole exercise would have been a shallow waste of time bogged down by the specifics of the worksheet. Perhaps I am overly critical, but I am genuinely disappointed in this class; I'm trying to provide honest feedback in an attempt to improve it. Whether you care, or think I'm misguided, is up to you and I assume your overall judgment in better than my potentially biased stance. 




\end{document}