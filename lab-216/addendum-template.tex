\documentclass[12pt,a4paper]{article}
\usepackage{inverba}
\newcommand{\userName}{Cullyn Newman} 
\newcommand{\class}{BI 216} 
\newcommand{\institution}{Portland State University} 
\newcommand{\thetitle}{\hypertarget{home}{Lab x Addendum: title}}
\rfoot{\hyperlink{home}{\thepage}}

\begin{document}
\section*{Part A: }
\begin{enumerate}[font=\bfseries, wide]
    {\color{under}\item Question \textbf{(1 point)}}\par
    >> Ion channels form aqueous pores across the lipid bilayer and allow inorganic ions of appropriate size and charge to cross the membrane down their electrochemical gradients at rates about 1000 times greater than those achieved by any known transporter. The channels are “gated” and usually open transiently in response to a specific perturbation in the membrane, such as a change in membrane potential (voltagegated channels), or the binding of a neurotransmitter to the channel (transmitter-gated channels). K + -selective leak channels have an important role in determining the resting membrane potential across the plasma membrane in most animal cells. Voltage-gated cation channels are responsible for the amplification and propagation of action potentials in electrically excitable cells, such as neurons and skeletal muscle cells. Transmitter-gated ion channels convert chemical signals to electrical signals at chemical synapses. Excitatory neurotransmitters, such as acetylcholine and glutamate, open transmittergated cation channels and thereby depolarize the postsynaptic membrane toward the threshold level for firing an action potential. Inhibitory neurotransmitters, such as GABA and glycine, open transmittergated Cl– or K+ channels and thereby suppress firing by keeping the postsynaptic membrane polarized. A subclass of glutamate-gated ion channels, called NMDA-receptor channels, is highly permeable to Ca2+, which can trigger the long-term changes in synapse efficacy (synaptic plasticity) such as LTP and LTD that are thought to be involved in some forms of learning and memory. Ion channels work together in complex ways to control the behavior of electrically excitable cells. A typical neuron, for example, receives thousands of excitatory and inhibitory inputs, which combine by spatial and temporal summation to produce a combined postsynaptic potential (PSP) at the initial segment of its axon. The magnitude of the PSP is translated into the rate of firing of action potentials by a mixture of cation channels in the initial segment membrane.
    {\color{under}\item Question \textbf{(1 point)}}\par
    >> Answer
    {\color{under}\item Question \textbf{(1 point)}}\par
    >> Answer
\end{enumerate}

\section*{Part B: }
\begin{enumerate}[font=\bfseries, wide, resume]
    {\color{under}\item Question \textbf{(1 point)}}\par
    >> Answer
    {\color{under}\item Question \textbf{(1 point)}}\par
    >> Answer
    {\color{under}\item Question \textbf{(1 point)}}\par
    >> Answer
\end{enumerate}
    
\section*{Part C: }
\begin{enumerate}[font=\bfseries, wide, resume]
    {\color{under}\item Question \textbf{(1 point)}}\par
    >> Answer
    {\color{under}\item Question \textbf{(1 point)}}\par
    >> Answer
    {\color{under}\item Question \textbf{(1 point)}}\par
    >> Answer
    {\color{under}\item Question \textbf{(1 point)}}\par
    >> Answer
    {\color{under}\item Question \textbf{(1 point)}}\par
    >> Answer
    {\color{under}\item Question \textbf{(1 point)}}\par
    >> Answer
    {\color{under}\item Question \textbf{(1 point)}}\par
    >> Answer
\end{enumerate}

\end{document}