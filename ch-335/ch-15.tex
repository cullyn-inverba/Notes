\chapter{15: Alkynes}\label{15: Alkynes}
\section{Alkynes Basics/Review}\label{Alkynes Basics/Review}
\begin{itemize}
  \item \ddd{Alkynes}: an unsaturated hydrocarbon containing at least one carbon—carbon triple bond.
    \begin{itemize}
      \item Contains two unhybridized p-orbitals (sp) on each carbon.
      
      \medskip
      \schemestart{}
        \chemfig{\cadp{H}-\cadn{C}~[,,,,bbb]\cadn{C}-\cadp{H}}
      \schemestop{}

      \item Large amount of \(s\) character (\bbb{sp: 50\% \(s\)}) makes the hydrogens very acidic and the carbons very basic. 
      \item 1 mole of base takes off 1 hydrogen (1 equivalent, equiv, eq), which can be done twice to take off both (2 equivalent).  
      \item Traditionally known as acetylenes, though acetylene referers specifically to \ch{C2H2} (ethyne)
    \end{itemize}
    
  \subsection{Acetylides}\label{Acetylides}
  \item \ddd{Acetylide}: an alkyne with a negatively charged carbon on one end, e.g.,
    \begin{itemize}
      \item[]
      
    \medskip
    \schemestart{}
      \chemfig{H-C~C-H}
      \arrow{->[\bbb{:B}]}
      \chemfig{H-\bbb{C}~[,,,,bbb]\bbb{C^\ani}}
      \+ 
      \rrr{\chemabove[3pt]{BH}{\cat}}
    \schemestop{}
    \medskip
    
    \item \bbb{:B} is a name of generic base, typical bases used:
      \begin{itemize}
        \item \bbb{\ch{NaNH2}} \to~sodamide, or \bbb{\ch{NaH}} \to~sodium hydride.
      \end{itemize}
    \item Acetylide acts as a \emph{nucleophile} when it reacts with an epoxide, \ang{1}-halide, or a methyl-halide.
    
    \begin{center}
      \hspace{-30pt}
      \medskip
      \schemestart{}
        \chemname[-60pt]{\chemfig{?-[:60]O-[:-60]?}}{epoxide}
        \qquad
        \chemname[-60pt]{\chemfig{-[:30]\chemabove[3pt]{}{\ang{1}}-[:-30]X}}{\ang{1}-halide}
        \qquad
        \chemname[-60pt]{\chemfig{CH3-X}}{metyhl-halide}
      \schemestop{}
      \bigskip
    \end{center}
      \begin{itemize}
        \item Is a \(S_N2\) reaction.
        \item Can be used for chain extension, e.g.,

        {\footnotesize
        \medskip
        \schemestart{}
          \chemfig{-[:-30]-~\bbb{C}@{h}\rrr{H}}
          \arrow{->[\chemfig{@{b}\bbb{\ch{NaH2}}}][\chemfig{-[:30,.8]@{cc}-[@{sb}:-30,.8]Br@{br}}]}[0,1.1]
          \chemfig{-[:-30]-~\bbb{C^\ani}@{c}}
          \arrow{->}
          \chemfig{-[:-30]-~--[:-30]}
        \schemestop{}
        \bigskip
        \chemmove[dash pattern= on 1pt off 1pt]{
          \draw(h)..controls +(north:1cm) and +(north:1cm).. (b);
          \draw(c)..controls +(south:0.8cm) and +(250:2cm).. (cc);
          \draw(sb)..controls +(south:0.5cm) and +(230:0.5cm).. (br);
          }
        }

        \item Example two:
        
        {\footnotesize
        \medskip
        \schemestart{}
          \chemfig{-[:-30]-~\bbb{C}\rrr{H}}
          \arrow{->[\bbb{NaH}][\chemfig{?@{b2}-[:60]O-[:-60.5]?(-[:-30,.5])}]}
          \chemfig{-[:-30]-~--[:30](-[:90]OH)-[:-30]}
        \schemestop{}
        \bigskip
        \chemmove[dash pattern= on 1pt off 1pt]{
          \draw(c)..controls +(south:1cm) and +(110:3cm).. (b2);}
        }
      \end{itemize}

      \newpage
      \item Acetylide acts as a \emph{base} when it reacts with \ang{2}-halide or a \ang{3}-halide due to steric crowding.
        \begin{itemize}
          \item Is an  E2 reaction.
        \end{itemize}
      
      \medskip
      \schemestart{}
        \chemfig{-~}
        \arrow{->[\bbb{NaH}][\tiny\chemfig{*6(---(-I)---)}]}
        \chemfig{-~@{bc}\bbb{C^\ani}}
        \arrow{->}
        {\scriptsize\chemfig{*6(---(-[@{sb}]I@{i})=(-@{kh}H)--)}}
        \arrow{->}
        {\scriptsize\chemfig{*6(---=--)}}
      \schemestop{}
      \chemmove[dash pattern= on 1pt off 1pt]{
        \draw(bc)..controls +(north:1cm) and +(west:1cm).. (kh);
        \draw(sb)..controls +(south:0.5cm) and +(south:0.5cm).. (i);
        }
      \bigskip
      
      \medskip
      \schemestart{}
        \chemfig{-~}
        \arrow{->[\bbb{NaH}][{\tiny\chemfig{*6(---(-[:45])(-[:-15]I)---)}}]}
        \chemfig{-~\bbb{C^\ani}}
        \arrow(--[braces]){->}
        \chemname[-50pt]{{\scriptsize\chemfig{*6(---(=)---)}}}{Hoffman}
        \+
        \chemname[-50pt]{{\scriptsize\chemfig{*6(---(-)=--)}}}{Zaitsev}
      \schemestop{}
      \bigskip
      
      \item The Zaitsev product is more likely to be the major product due to the thermodynamics (more energetically stable) than the Hofmann product; a bulkier base would likely increase the Hofmann product. 

      \subsection{Regioselectivity and Stereospecificity of E2 Reactions}
        \item \ddd{Regiochemistry}: when a chemical reaction is said to produce two different regiochemical outcomes.
            \begin{itemize}
                \item \ddd{Regiochemical}: preference of chemical bonding or breaking \emph{direction}. 
            \end{itemize}
        \item \ddd{Regioselective}: when there is a preference in products of a regiochemical reaction.
        \item \ddd{Zaitsev product}: name of the \emph{more substituted} alkene that is generally observed to be the major product.
        \item \ddd{Hofmann product}: name of the \emph{less substituted} alkene.
        \item There ratio between the Zaitsev and Hofmann product is dependent on a number of factors and often difficult to predict.
            \begin{itemize}
                \item Steric hindrance of the base often plays a major role, often increasing the Hofmann product.
                \item The outcome of E2 reactions can often be carefully \emph{ controlled by choosing the base}, despite difficultly in overall prediction.
            \end{itemize}
        \item \ddd{Stereospecific}: when the stereoisomeric product of the E2 process depends on the configuration of the starting stereoisomeric substrate.
            \begin{itemize}
                \item The stereospecificity is only relevant when the \emph{\(\beta \) position has only one proton}.
            \end{itemize}
        \item \ddd{Stereoselective}: when the substrate itself is not necessarily stereoisomeric, but can yield two stereoisomeric products, one of which that often has a higher yield.
            \begin{itemize}
                \item Occurs when there are \emph{more than one proton in the \(\beta \) position}.
            \end{itemize}
        \item \ddd{Coplanar}: when the proton in the \(\beta \) position, the leaving group, and the two carbons atoms that form a double bond lie on the same plane.
            \begin{itemize}
                \item Often leads only one stereoisomer product being formed, thus more often stereospecific.
                    \begin{itemize}
                        \item If it stereoselective, then usually \textit{trans} conformation is favored.
                    \end{itemize}
                \item \ddd{Periplanar}: when the proton and leaving group are \emph{nearly} coplanar; often is used in place of coplanar to incorporate both situations.
            \end{itemize}
        \item \ddd{Syn-coplanar}: when the proton and the leaving group are \emph{ eclipsed} in a coplanar conformation.
            \begin{itemize}
                \item Elimination in this state involves a higher energy transition state due to eclipsed geometry, and is \emph{slower} than anti-coplanar arrangement.
            \end{itemize}
        \item \ddd{Anti-coplanar}: when the proton and the leaving group are \emph{staggered} in a coplanar conformation.
            \begin{itemize}
                \item Elimination in this state involves a lower energy transition state due to staggered geometry, leading to a \emph{faster} reaction relative to syn-coplanar.
            \end{itemize}
    \end{itemize}
\end{itemize}

\clearpage
\section{Alkyne Nomenclature}\label{Alkyne Nomenclature}
\begin{itemize}
    \item 
\end{itemize}

\clearpage
\section{Addition Reactions of Alkynes}\label{Addition Reactions of Alkynes}
\begin{itemize}
    \item 
\end{itemize}