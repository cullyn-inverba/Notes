\documentclass[12pt,a4paper]{article}
\usepackage{inverba}
\newcommand{\userName}{Cullyn Newman} 
\newcommand{\class}{BI 216} 
\newcommand{\institution}{Portland State University} 
\newcommand{\thetitle}{\hypertarget{home}{Lab 7 Addendum: COVID-19}}
\rfoot{\hyperlink{home}{\thepage}}
\usepackage{hyperref}
\hypersetup{
        colorlinks=true,
        linktoc=all,     
        linkcolor=liblue,
        urlcolor=liorange,
    }
\begin{document}
\section*{Part A: Spread of the Coronavirus}
\begin{enumerate}[font=\bfseries, wide]
    {\color{under}\item This video was posted in early March. Does the video’s model prediction align with the cases recorded since posting? What source(s) did you use to answer this question? How do you know that the source is credible? (2 pts)}
    
    Yes, \href{https://youtu.be/cEvgcoyZvB4?t=765}{a video about loarithms by the same creator} shows the crude model being very close 30 days later while teaching a lesson about logorithms. He does not list the source he got the info for, but he regularly sites sources for more significant claims and has credibility that I think is sufficient.

    {\color{under}\item What is exponential growth and how does it apply to COVID-19? (1 pt)}
    
    A way a quantity increases over time, one that is defined when the derivative of the quantity is proportional to the quantity itself. In other words, the more the quantity grows the faster it will grow as long as it grows at any rate greater than 1.0.

    {\color{under}\item Explain what the ‘growth factor’ is. (0.5 pt)}

    The ratio between the number of new cases one day, and the number of new cases the previous day. Or the ratio between two successive changes.

    {\color{under}\item What happens when a growth factor approaches ‘1’? (0.5 pt)}

    Approaching 1 means you are approaching the inflection point.
\subsection*{Epidemiology Lab}

    {\color{under}\item Spend a moment playing with the sliders for $\beta$ and $\gamma$. Explain how raising and lowering these parameters affect the shapes of the graphs of: S, I, and R?  (1 pt)}
    
    As $\beta$ increases, then the infectious population increases at a higher rate and the susceptible population decreases at a faster rate. As $\beta$ decreases the inverse is true.

    As $\gamma$ increases, then the infectious population maximum is decreased and the susceptible population flattens out faster. With a low enough $\gamma$, then eventually the entire population will become infected.

    {\color{under}\item  In the effort to curb COVID-19, both stay-at-home orders/lockdowns and the use of facemasks while in public have been widely employed worldwide. What parameter(s) in the SIR model would you alter in order to account for these behavior changes? Explain. (1 pt)}
    
    I would decrease the $\beta$ value, since said measures help decrease the chance of infection.

    {\color{under}\item Our SIR model makes a number of simplifying assumptions about the world in order to make it mathematically and conceptually tractable. State two of these assumptions, and explain why they are not perfect descriptions of reality. (1 pt)}
    
    The model assumes the transmission, susceptibility, and recovery are static and the same everywhere. Biology is extremely variable, and ace to preventive measures and health care is different for everyone. Differences in nations show how drastically different these effects can be.

    Another assumption is that recovered people are not susceptible or infectious still, or at least aiding in transmission.

    {\color{under}\item The SIR model is useful, but it does not explicitly predict the number of deaths from a disease, something we obviously care a great deal about. How would you alter the model to include both deaths and recoveries? Draw a diagram for your altered model like we did for the SIR in the PowerPoint (slide 20, with the boxes, arrows, and equations) and explain your reasoning for the alterations. (1 pt)}
    
    The recovery just lumps deaths and recoveries together. You would need information about the mortality rate and then factor that into two separate groups instead of just recoveries.

    {\color{under}\item  Describe how R$_e$ varies as the disease progresses. When does it equal 1, and when is it at its maximum? What causes R $_e$ to change over time? (1 pt)}
    
    R$_e$ varies based on several factors: transmission, susceptible, population, recovery time, and average number of contacts per day. If R$_e$ is equal to one, then it is considered a endemic, where there is no growth of decline. If it's greater than 1 than it produces a epidemic, and if it's less than one than it means the disease is declinging.
    
\end{enumerate}
\newpage
\section*{Part B: Climate \& COVID-19}
\begin{enumerate}[font=\bfseries, wide, resume]
    {\color{under}\item Chosen Articles (4 pts)}
        \begin{enumerate}[label=\alph*.]
            \item \textbf{Article 1}: How Is the Coronavirus Pandemic Affecting Climate Change?
            \item Author(s): Matt Simon C
            \item Publisher: wired.com
            \item Date: Date published: 04.21.2020 07:00 AM
            \item Bias Scale: Slight left leaning bias, but still stuck to the facts for the most part.
            \item Take home message: The coronavirus has slowed the flow of travel which has resulted in a decrease in energy consumption. This cause has had rehabilitation effect on the environment. The authors do mention that sulfur emitted by shipping boats reflect UV radiation. This effect protects the environment. 
            \vspace{15pt}
            \item \textbf{Article 2}: COVID-19 Is Driving A Dramatic Greenhouse Gas Decline, But How Is Renewable Energy Faring?
            \item Author(s): Mary Kate McCoy
            \item Publisher: npr; Wisconsin Public Radio.
            \item Date: Monday, April 27, 2020, 9:35am
            \item Bias Scale: Low, not pushing any agenda, just stating status of recent findings.
            \item Take home message: Green house gases are declining, but this one event is not enough to fix climate change. Although, the renewable energy is a growing sector, but might be still slowed down despite success due to the pandemic.
        \end{enumerate}
    {\color{under}\item Which article did you read? Please explain one thing that you thought that the article did well, and one thing that you would have liked to have seen done differently. (1 pt)}
    
    I read article 2 and I thought it was pretty good. To the point, didn't linger with opinions or speculation, instead just stating where we are at. Did a good job addressing my questions as it came up. Not really sure what they could have done better. Maybe even more concise? I don't care much for people quotes. Maybe more questions left for me to investigate if the didn't have full info? 
    
\end{enumerate}
    
\section*{Part C: So Many Questions!}
\begin{enumerate}[font=\bfseries, wide, resume]
    {\color{under}\item What is your COVID-related question? (1 pt)}
    
    I would love to know more accurate information on state of China and how their recovery has played out. Or possible a better analysis on economic effects to make a more solid argument, but that information seems way too speculative and hard to get right.
    
    {\color{under}\item What did you do to try to find answers to this question? (1 pt)}
    
    I've listened to several podcasts with credible experts in their fields (virologists, econmics), and keep an eye out on new research papers on preprint servers regarding said questions.
    
    {\color{under}\item Did you find any solid answers? Please explain. (1 pt)}
    
    No, not really. Lots of facts seemed hard to verify, or not super confident it reproducibility. And it's hard to trust anything China says due to how poorly they handled the initial outbreak and the major censoship they often exercise.

    {\color{under}\item What information would you need, or what experiments would need to be done to answer your questions adequately. Are these feasible? (1 pt)}

    Quantitative mortality rates and reinfection rates from China would be the best, since they are futher along. Our relationship is so poor with them though that I doubt it would be feasible.
    
\end{enumerate}

\end{document}