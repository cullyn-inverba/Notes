\documentclass[12pt,a4paper]{article}
\usepackage{inverba}

\newcommand{\userName}{Cullyn Newman} 
\newcommand{\class}{Cohen} 
\newcommand{\institution}{Udemy} 
\newcommand{\theTitle}{\color{r-Sun}Statistics}
\begin{document}
%%%%%%%%%%%%%%%%%%%%%%%%%%%%%%%%%%%%%%%%%%%%%%%%%%%%%%%%%%%%%%%%%%%%%
\tableofcontents
\cleardoublepage
\fancyhead{}
\fancyhead[R]{\hyperlink{home}{\nouppercase\leftmark}}
%%%%%%%%%%%%%%%%%%%%%%%%%%%%%%%%%%%%%%%%%%%%%%%%%%%%%%%%%%%%%%%%%%%%%


%%%%%%%%%%%%%%%%%%%%%%%%%%%%% Chapter 1 %%%%%%%%%%%%%%%%%%%%%%%%%%%%%
%\begingroup
\clearpage
\section{Data}\phantomsection
\subsection{Data Basics}
\begin{itemize}
    \item Frequent types of data in statistics:
        \begin{itemize}
            \item \textbf{Interval}: numeric scale with meaningful intervals, e.g. temperature in celsius.
            \item \textbf{Ratio}: numeric but with a meaningful zero, e.g. height.
            \item \textbf{Discrete}: numeric with with no arbitrary precision, e.g. population.
            \item \textbf{Ordinal}: sortable and discrete, e.g. education level.
            \item \textbf{Nominal}: non-sortable and discrete, e.g. genre.
        \end{itemize}
    \item \textbf{Sample data}: Data from \textit{some} members of a group.
    \item \textbf{Population data}: Data from \textit{all} members of a group.
    \item Sample population sometimes uses hat notation, e.g. \(\hat{\beta},~\hat{\sigma}\), or other slight ambiguities. Sample data is used more often than population in statistics.
\end{itemize}

\subsection{Visualizing Data}
\begin{itemize}
    \item \textbf{Bar plots}: used to represnet {\color{o-Sun}categorical} (nominal and ordinal) and {\color{o-Sun}discrete numerical} data.
    \item \textbf{Box plots}: collection of a data that is split into separate quatiles in order to illustrate {\color{o-Sun}overall distribution} of data and its potential outliers.
    \item \textbf{Histograms}: similar to bar plots, but with binned continuous data on the x-axis. {\color{o-Sun}Shape} and {\color{o-Sun}order} is meaningful.
        \begin{itemize}
            \item Histograms of \textbf{counts}: 
                \begin{itemize}
                    \item Often more meaningful interpretation of raw data.
                    \item Difficult to compare across datasets.
                    \item Does not need to sum up to 1.
                    \item Usually better for {\color{o-Sun}qualitative} inspection.
                \end{itemize}
            \item Histograms of \textbf{proportion}:
                \begin{itemize}
                    \item Can be more difficult to relate to raw data.
                    \item Easier to compare across datasets.
                    \item Illustrates proportion of dataset.
                    \item Usually better for {\color{o-Sun}quantitative} analysis.
                \end{itemize}
        \end{itemize}
    \item Translating from counts to proportions: \(bin_i = 100\,(bin_i \,/\, sum(bins))\)
    \item \textbf{Pie charts}: representation of nominal, ordinal, or discrete data that must sum up to 1.
\end{itemize}

%\endgroup
%%%%%%%%%%%%%%%%%%%%%%%%%%%%% Chapter 1 %%%%%%%%%%%%%%%%%%%%%%%%%%%%%

%%%%%%%%%%%%%%%%%%%%%%%%%%%%% Chapter 2 %%%%%%%%%%%%%%%%%%%%%%%%%%%%%
%\begingroup
\clearpage
\section{Descriptive Statistics}\phantomsection
\subsection{Descriptive vs. Inferential}
\begin{itemize}
    \item \textbf{Descriptive}:
        \begin{itemize}
            \item The point is to obtain individual numbers that describe a dataset.
            \item Mean, median, mode, variance, kurtosis, skew, distribution, spectrum.
            \item No relation to population; no generalization to other datasets of groups.
        \end{itemize}
    \item \textbf{Inferential}: 
        \begin{itemize}
            \item Use features of sample data set to make generalizations about a population.
            \item P-value, T/F/chi-square value.
            \item Confidence intervals.
            \item Hypothesis testing.
        \end{itemize}
\end{itemize}

\subsection{Accuracy, Precision, Resolution}
\begin{itemize}
    \item \textbf{Accuracy}: the relationship between measurement and the actual truth. Inversely related to {\color{o-Sun}bias}.
    \item \textbf{Precision}: the certainty of each measurement. Inversely related to {\color{o-Sun}variance}.
    \item \textbf{Resolution}: the number of data points per unit measurement.
\end{itemize}

\subsection{Data Distribution}
\begin{itemize}
    \item \textbf{Data Distribution}: a function that lists values or intervals of data, and how often each value occurs. 
    \item Common distributions include power-law, gaussian (bell curve), t, F, and Chi-squared.
    \item Most statistical procedures are based on assumptions about distributions.
    \item Data distributions provide insights into nature and often used to model physical and biological systems.
\end{itemize}

\subsection{Measures of Central Tendency}
\begin{itemize}
    \item \textbf{Central tendency}: the center of typical value for a probability distribution. 
    \item Common measures of central tendency: {\color{o-Sun}mean, median, mode}.
    \item Mean, aka average or arithmetic mean: 
    \begin{itemize}
        \item Formula: {\color{o-Sun}\(\bar{x}=n^{-1}\sum{x_i}\)}.
        \item Alternate notations for mean: \(\mu,\,\mu_x\).
        \item The mean is most suitable for normally distributed interval and ratio data.
        \item Discrete and ordinal data can be useful, but must be carefully interpreted.
    \end{itemize}
    \item Median:
        \begin{itemize}
            \item {\color{o-Sun}\(x_i,~i=\dfrac{n+1}{2}\)}
            \item Most suitable for unimodal distributed interval and ratio data.
        \end{itemize}
    \item Mode: the most common value that is suitable for any distribution and data type, though mostly used for nominal.
\end{itemize}
%\endgroup
%%%%%%%%%%%%%%%%%%%%%%%%%%%%% Chapter 2 %%%%%%%%%%%%%%%%%%%%%%%%%%%%%
\end{document}